\documentclass[12pt,a4paper]{ltjsarticle}

\usepackage[margin=20truemm]{geometry}
\usepackage{url}	% required for `\url' (yatex added)

% secsion 番号の再定義
\renewcommand{\thesection}{第\arabic{section}}
% enumirate enumii の再定義
\renewcommand{\labelenumii}{(\arabic{enumii})}

\begin{document}

Coinhive事件 最高裁 判決文

\url{https://www.courts.go.jp/app/files/hanrei_jp/869/090869_hanrei.pdf}

\hrulefill

\begin{center}
 \fbox{
 \begin{minipage}{300pt}
  令和2年(あ)第457号 不正指令電磁的記録保管被告事件

  令和4年1月20日 第一小法廷判決
 \end{minipage}
 }
\end{center}

\begin{center}
 \textbf{\LARGE 主文}

 \begin{minipage}{130pt}
  原判決を破棄する。

  本件控訴を棄却する。
 \end{minipage}
\end{center}

\begin{center}
 \textbf{\LARGE 理由}
\end{center}


弁護人平野敬の上告趣意のうち,刑法168条の2第1項にいう「人が電子計算
機を使用するに際してその意図に沿うべき動作をさせず,又はその意図に反する動
作をさせるべき不正な指令」の文言が漠然不明確であるとして憲法21条1項,3
1条違反をいう点は,同文言が不明確であるとはいえないから,前提を欠き,その
余は,憲法違反,判例違反をいう点を含め,実質は単なる法令違反,事実誤認の主
張であって,刑訴法405条の上告理由に当たらない。

しかしながら,所論に鑑み,職権をもって調査すると,原判決は,刑訴法411
条1号,3号により破棄を免れない。その理由は,以下のとおりである。

%第1 事案の概要及び事実関係
 \section{事案の概要及び事実関係}
 \begin{enumerate}
  \item
       本件公訴事実(訴因変更後のもの)の要旨は,「被告人は,インターネット
        上のウェブサイト『X』の運営者であるが,X閲覧者が使用する電子計算機の中央
        処理装置に同閲覧者の同意を得ることなく仮想通貨モネロの取引履歴の承認作業等
        の演算を行わせてそれによる報酬を取得しようと考え,正当な理由がないのに,人
        の電子計算機における実行の用に供する目的で,平成29年10月30日から同年
        11月8日までの間,X閲覧者が使用する電子計算機の中央処理装置に前記演算を
        行わせるプログラムコードが蔵置されたサーバコンピュータに同閲覧者の同意を得
        ることなく同電子計算機をアクセスさせ同プログラムコードを取得させて同電子計
        算機に前記演算を行わせる不正指令電磁的記録であるプログラムコード(以下「本
        件プログラムコード」という。)を,サーバコンピュータ上のXを構成するファイ
        ル内に蔵置して保管し,もって人が電子計算機を使用するに際してその意図に反す
        %%%- 2 -
        る動作をさせるべき不正な指令を与える電磁的記録を保管した」というものであ
        る。

        仮想通貨(暗号資産)の取引履歴の承認作業等の演算は,仮想通貨の信頼性を確
        保するために行われ,その演算のために電子計算機の機能を提供した者に対して,
        報酬として仮想通貨が発行される仕組みになっている。承認作業等の演算を行って
        仮想通貨を得ることを「マイニング」と称するところ,本件当時,ウェブサイトの
        収入源として,閲覧者の同意を得ることなくその電子計算機を使用してマイニング
        を行わせるCoinhiveというウェブサービス(以下「コインハイブ」とい
        う。)が,CoinhiveTeamという事業者(以下「コインハイブチーム」
        という。)により提供されていた。

        本件は,被告人が,Xの収入源としてコインハイブによるマイニングの仕組みを
        導入するために本件プログラムコードをサーバコンピュータに保管した行為につい
        て,不正指令電磁的記録保管罪に問われた事案であり,主な争点は,本件プログラ
        ムコードが,刑法168条の2第1項(以下「本件規定」という。)にいう「人が
        電子計算機を使用するに際してその意図に沿うべき動作をさせず,又はその意図に
        反する動作をさせるべき不正な指令を与える電磁的記録」に当たるか否かである
        (以下,「その意図に沿うべき動作をさせず,又はその意図に反する動作をさせる
        べき」という要件を「反意図性」といい,「不正な」という要件を「不正性」とい
        う。)。

  \item 第1審判決及び原判決の認定並びに記録によると,本件の事実関係は,以下
        のとおりである。

        被告人は,平成29年9月当時,音声合成ソフトウェアを用いて作られた楽曲の
        情報を共有するウェブサイト「X」を運営していた。

        コインハイブは,平成29年9月,コインハイブチームにより提供が開始された
        ウェブサービスである。その内容は,登録したウェブサイトの運営者(以下「登録
        者」という。)に対し,ウェブサイト閲覧者が閲覧中に使用する電子計算機の中央
        %%%- 3 -
        処理装置に同閲覧者の同意を得ることなく仮想通貨Monero(モネロ)の取引
        台帳へ取引履歴を追記する承認作業等の演算を行わせ,その演算が成功すると,報
        酬として仮想通貨の取得が可能になるというマイニングを実行するプログラムコー
        ド(以下「本体プログラム」という。)を取得するためのプログラムコードを提供
        し,報酬の7割を登録者に分配し,3割をコインハイブチーム側が取得するという
        ものであり,登録者が,提供された前記プログラムコードをウェブサイト内に設置
        すると,閲覧者の電子計算機によりマイニングが実行され,登録者が報酬の分配を
        得ることができるというものであった。

        コインハイブによるマイニングの仕組みは,前記プログラムコードが設置された
        ウェブサイトを閲覧すると,同プログラムコードの指令により閲覧者の電子計算機
        が自動的に本体プログラムが蔵置されたサーバコンピュータに接続され,本体プロ
        グラムが読み込まれてマイニングを指令され,その指令により閲覧者の電子計算機
        の中央処理装置が演算を行い,演算結果が同サーバコンピュータに送信されるとい
        うものであり,閲覧を終了するとマイニングも終了するというものであった。

        被告人は,X閲覧を通じて利益を得るため,平成29年9月21日,コインハイ
        ブに登録し,提供されたプログラムコードに,被告人に割り当てられたサイトキー
        を記述したもの(本件プログラムコード)を,サーバコンピュータ上のX内に設置
        し,本件公訴事実の期間中,Xを構成するファイル内に蔵置して保管した。本件当
        時,一般の使用者に,ウェブサイトの収益方法として閲覧者の電子計算機にマイニ
        ングを行わせるという仕組みは認知されていなかったが,被告人は,Xに,閲覧中
        にマイニングが行われることについて同意を得る仕様を設けたり,マイニングに関
        する説明やマイニングが行われていることの表示をしたりすることなく,本件プロ
        グラムコードを保管していた。

        被告人は,本件プログラムコードにおいて,閲覧者の電子計算機の中央処理装置
        使用率を調整する値を0.5と設定した。この数値の場合,マイニングを実行する
        と,閲覧者の電子計算機の消費電力が若干増加したり中央処理装置の処理速度が遅
        %%%- 4 -
        くなったりするが,極端に遅くはならず,これらの影響の程度は,閲覧者が気付く
        ほどではなく,また,一般的なウェブサイトで広く実行されている広告を表示する
        プログラム(以下「広告表示プログラム」という。)と有意な差異はなかった。
 \end{enumerate}

 %第2 第1審判決及び原判決
 \section{第1審判決及び原判決}
 \begin{enumerate}
  \item 第1審判決は,本件プログラムコードの不正指令電磁的記録該当性につい
        て,要旨,次のとおり判断して,被告人に無罪を言い渡した。
        \begin{enumerate}
         \item 反意図性は,当該プログラムの機能につき一般に認識すべきと考えられると
               ころを基準として判断するのが相当であるところ,Xにはマイニングに関する説明
               はなく,閲覧中にマイニングが行われることについて同意を得る仕様にもなってい
               なかったこと,ウェブサイトの収益方法として閲覧者の電子計算機にマイニングを
               行わせるという仕組みは一般の使用者に認知されておらず,マイニングによる電子
               計算機への負荷の程度に照らして一般の使用者がその実行に気付くことはないとい
               えることなどからすると,一般の使用者が,X閲覧者の電子計算機にマイニングを
               行わせるという本件プログラムコードの機能について認識すべきとはいえないか
               ら,反意図性が認められる。
         \item 不正性は,ウェブサイトの運営者及び閲覧者等にとっての有用性や必要性,
               使用者への影響や弊害等の事情を考慮し,当該プログラムの機能の内容が社会的に
               許容し得るものであるか否かという観点から判断するのが相当であるところ,①本
               件プログラムコードの実行により運営者が得る利益は,ウェブサイトの質の維持向
               上のための資金源になり得るから,閲覧者にとって利益となる面があること,②本
               件プログラムコードの実行により生ずる閲覧者の電子計算機の処理速度の低下等
               は,広告表示プログラム等の場合と大差ない上,X閲覧中に限定されることなどか
               らすると,本件プログラムコードが社会的に許容されていなかったとはいえず,不
               正性は認められない。
        \end{enumerate}
  \item 検察官が控訴し,第1審判決には本件規定の解釈適用の誤りや事実誤認があ
        ると主張した。原判決は,本件プログラムコードの不正指令電磁的記録該当性につ
        %%%- 5 -
        いて,要旨,次のとおり判断し,第1審判決は本件規定の解釈を誤って事実を誤認
        したものであるとして,第1審判決を破棄し,被告人を罰金10万円に処した。
        \begin{enumerate}
         \item 反意図性は,当該プログラムの機能について一般に認識すべきと考えられる
               ところを基準とした上で,一般の使用者の意思に反しないものと評価できるかとい
               う観点から規範的に判断すべきであり,一般の使用者が事前に機能を認識した上で
               実行することが予定されていないプログラムについては,その機能の内容そのもの
               を踏まえ,一般の使用者が機能を認識しないまま当該プログラムを使用することを
               許容していないと規範的に評価できる場合に反意図性を肯定すべきである。

               本件プログラムコードは,X閲覧者の電子計算機にマイニングを行わせるという
               機能を有するものであり,閲覧することによりマイニングが行われることの表示は
               予定されておらず,マイニングにより生じた報酬を閲覧者が得ることは予定されて
               いない。一般に,閲覧者は,閲覧に必要なプログラムを実行することは承認してい
               ると考えられるが,本件プログラムコードによるマイニングは閲覧に必要ではな
               い。その上,本件プログラムコードによるマイニングは閲覧者の電子計算機に一定
               の負荷を与えるものであるのに,閲覧者には利益がもたらされないし,閲覧者にマ
               イニングによって電子計算機が使用されていることを知る機会やマイニングを拒絶
               する機会も保障されていない。

               このような本件プログラムコードは,使用者に利益をもたらさない上,使用者に
               無断で電子計算機を使用して利益を得ようとするものであり,一般の使用者が許容
               しないことは明らかであるから,反意図性を認めた第1審判決の結論は正当である。

         \item 不正性は,反意図性のあるプログラムであっても,使用者として想定される
               者における当該プログラムを使用すること自体に関する利害得失や,使用者に生じ
               得る不利益に対する注意喚起の有無などを考慮した場合,プログラムに対する信頼
               保護や電子計算機による適正な情報処理という観点からみて,社会的に許容される
               ことがあるので,そのような場合を規制の対象から除外する趣旨の要件である。

               本件プログラムコードは,閲覧者に利益を生じさせない一方で一定の不利益を与
               %%%- 6 -
               えるものである上,不利益に関する表示等もされないから,プログラムに対する信
               頼保護という観点から社会的に許容すべき点はない。また,X閲覧中に,閲覧者の
               電子計算機を,閲覧者以外の利益のために無断で使用するものであり,電子計算機
               による適正な情報処理の観点からも,社会的に許容されるということはできない。

               第1審判決は,前記1 ①②等の事情を挙げて不正性を否定するが,①につい
               て,そのような利益は,意に反するプログラムの実行を使用者が気付かないような
               方法で受忍させた上で実現されるべきものではないし,②について,広告表示プロ
               グラムは閲覧に付随して実行され実行結果も表示されるものが一般的であり,その
               点で本件プログラムコードとは大きな相違があるから比較検討になじまない上,本
               件は,意図に反し電子計算機が使用されるプログラムであることが主な問題である
               から,処理速度の低下等が使用者の気付かない程度であったとしても不正性を左右
               しない。
        \end{enumerate}
        これらによれば,本件プログラムコードは,その機能を中心に検討すると,反意
        図性もあり不正性も認められる。
 \end{enumerate}

 %第3 当裁判所の判断
 \section{当裁判所の判断}
 \begin{enumerate}
  \item 不正指令電磁的記録に関する罪は,電子計算機において使用者の意図に反し
        て実行される不正プログラムが社会に被害を与え深刻な問題となっていることを受
        け,電子計算機による情報処理のためのプログラムが,「意図に沿うべき動作をさ
        せず,又はその意図に反する動作をさせるべき不正な指令」を与えるものではない
        という社会一般の信頼を保護し,ひいては電子計算機の社会的機能を保護するため
        に,反意図性があり,社会的に許容し得ない不正性のある指令を与えるプログラム
        の作成,提供,保管等を,一定の要件の下に処罰するものである。

        このような本件規定の趣旨及び保護法益に照らせば,プログラムの反意図性及び
        不正性については,次のとおり解するのが相当である。

        すなわち,
        \textbf{反意図性は,当該プログラムについて一般の使用者が認識すべき動作
        と実際の動作が異なる場合に肯定されるものと解するのが相当であり,一般の使用
        %%%- 7 -
        者が認識すべき動作の認定に当たっては,当該プログラムの動作の内容に加え,プ
        ログラムに付された名称,動作に関する説明の内容,想定される当該プログラムの
        利用方法等を考慮する必要がある。}

        \textbf{また,不正性は,電子計算機による情報処理に対する社会一般の信頼を保護し,
        電子計算機の社会的機能を保護するという観点から,社会的に許容し得ないプログ
        ラムについて肯定されるものと解するのが相当であり,その判断に当たっては,当
        該プログラムの動作の内容に加え,その動作が電子計算機の機能や電子計算機によ
        る情報処理に与える影響の有無・程度,当該プログラムの利用方法等を考慮する必
        要がある。}

  \item 本件プログラムコードの動作は,Xの閲覧中,閲覧者の電子計算機を使用し
        てマイニングを行わせるというものである。

        一般的なウェブサイトにおいて,運営者が閲覧を通じて利益を得る仕組みとして
        広告表示プログラムが広く実行されている実情に照らせば,一般の使用者におい
        て,ウェブサイト閲覧中に,閲覧者の電子計算機を一定程度使用して運営者が利益
        を得るプログラムが実行され得ることは,想定の範囲内であるともいえる。

        しかしながら,そのようなプログラムとして,本件プログラムコードの動作を一
        般の使用者が認識すべきといえるか否かについてみると,Xは,閲覧中にマイニン
        グが行われることについて同意を得る仕様になっておらず,マイニングに関する説
        明やマイニングが行われていることの表示もなかったこと,ウェブサイトの収益方
        法として閲覧者の電子計算機にマイニングを行わせるという仕組みは一般の使用者
        に認知されていなかったことといった事情がある。これらの事情によれば,本件プ
        ログラムコードの動作を一般の使用者が認識すべきとはいえず,反意図性が認めら
        れる。

  \item 本件プログラムコードは,Xの運営者である被告人が,X閲覧を通じて利益
        を得るため,閲覧者の同意を得ることなく,その電子計算機においてマイニングを
        行わせるために保管したものである。
        %%%- 8 -

        確かに,原判示のとおり,本件プログラムコードによるマイニングは,閲覧者の
        同意を得ることなくその電子計算機に一定の負荷を与え,これに関する報酬を閲覧
        者が取得することができないものであるのに,閲覧者にマイニングの実行を知る機
        会やこれを拒絶する機会が保障されていないなど,プログラムに対する信頼という
        観点から,より適切な利用方法等が採り得たものである。

        しかしながら,前記1の保護法益に照らして重要な事情である電子計算機の機能
        や電子計算機による情報処理に与える影響は,X閲覧中に閲覧者の電子計算機の中
        央処理装置を一定程度使用することにとどまり,その使用の程度も,閲覧者の電子
        計算機の消費電力が若干増加したり中央処理装置の処理速度が遅くなったりする
        が,閲覧者がその変化に気付くほどのものではなかったと認められる。

        また,ウェブサイトの運営者が閲覧を通じて利益を得る仕組みは,ウェブサイト
        による情報の流通にとって重要であるところ,被告人は,本件プログラムコードを
        そのような収益の仕組みとして利用したものである上,本件プログラムコードは,
        そのような仕組みとして社会的に受容されている広告表示プログラムと比較して
        も,閲覧者の電子計算機の機能や電子計算機による情報処理に与える影響において
        有意な差異は認められず,事前の同意を得ることなく実行され,閲覧中に閲覧者の
        電子計算機を一定程度使用するという利用方法等も同様であって,これらの点は社
        会的に許容し得る範囲内といえるものである。

        さらに,本件プログラムコードの動作の内容であるマイニング自体は,仮想通貨
        の信頼性を確保するための仕組みであり,社会的に許容し得ないものとはいい難い。

        以上のような,本件プログラムコードの動作の内容,その動作が電子計算機の機
        能や電子計算機による情報処理に与える影響,その利用方法等を考慮すると,本件
        プログラムコードは,社会的に許容し得ないものとはいえず,不正性は認められな
        い。

  \item \textbf{以上のとおり,本件プログラムコードは,反意図性は認められるが,不正性
        は認められないため,不正指令電磁的記録とは認められない。}
        %%%- 9 -
 \end{enumerate}
 原判決は,不正指令電磁的記録の解釈を誤り,その該当性を判断する際に考慮す
 べき事情を適切に考慮しなかったため,重大な事実誤認をしたものというべきであ
 り,これらが判決に影響を及ぼすことは明らかであって,原判決を破棄しなければ
 著しく正義に反すると認められる。

 よって,刑訴法411条1号,3号により原判決を破棄することとし,上記の検
 討によれば,本件プログラムコードの不正指令電磁的記録該当性を否定して被告人
 を無罪とした第1審判決は是認することができ,本件規定の解釈適用の誤りや事実
 誤認を主張する検察官の控訴は理由がないことに帰するから,同法413条ただし
 書,414条,396条によりこれを棄却することとし,裁判官全員一致の意見
 で,主文のとおり判決する。

 検察官 清野 憲一,同 古賀 栄美,同 山内 由光 公判出席

 (
 裁判長裁判官 山口 厚\
 裁判官 深山 卓也\
 裁判官 安浪 亮介\
 裁判官 岡 正晶\
 裁判官 堺 徹
   )




\end{document}


%
%- 1 -
%令和2年(あ)第457号 不正指令電磁的記録保管被告事件
%令和4年1月20日 第一小法廷判決
%主 文
%原判決を破棄する。
%本件控訴を棄却する。
%理 由
%弁護人平野敬の上告趣意のうち,刑法168条の2第1項にいう「人が電子計算
%機を使用するに際してその意図に沿うべき動作をさせず,又はその意図に反する動
%作をさせるべき不正な指令」の文言が漠然不明確であるとして憲法21条1項,3
%1条違反をいう点は,同文言が不明確であるとはいえないから,前提を欠き,その
%余は,憲法違反,判例違反をいう点を含め,実質は単なる法令違反,事実誤認の主
%張であって,刑訴法405条の上告理由に当たらない。
%しかしながら,所論に鑑み,職権をもって調査すると,原判決は,刑訴法411
%条1号,3号により破棄を免れない。その理由は,以下のとおりである。
%第1 事案の概要及び事実関係
%1 本件公訴事実(訴因変更後のもの)の要旨は,「被告人は,インターネット
%上のウェブサイト『X』の運営者であるが,X閲覧者が使用する電子計算機の中央
%処理装置に同閲覧者の同意を得ることなく仮想通貨モネロの取引履歴の承認作業等
%の演算を行わせてそれによる報酬を取得しようと考え,正当な理由がないのに,人
%の電子計算機における実行の用に供する目的で,平成29年10月30日から同年
%11月8日までの間,X閲覧者が使用する電子計算機の中央処理装置に前記演算を
%行わせるプログラムコードが蔵置されたサーバコンピュータに同閲覧者の同意を得
%ることなく同電子計算機をアクセスさせ同プログラムコードを取得させて同電子計
%算機に前記演算を行わせる不正指令電磁的記録であるプログラムコード(以下「本
%件プログラムコード」という。)を,サーバコンピュータ上のXを構成するファイ
%ル内に蔵置して保管し,もって人が電子計算機を使用するに際してその意図に反す
%- 2 -
%る動作をさせるべき不正な指令を与える電磁的記録を保管した」というものであ
%る。
%仮想通貨(暗号資産)の取引履歴の承認作業等の演算は,仮想通貨の信頼性を確
%保するために行われ,その演算のために電子計算機の機能を提供した者に対して,
%報酬として仮想通貨が発行される仕組みになっている。承認作業等の演算を行って
%仮想通貨を得ることを「マイニング」と称するところ,本件当時,ウェブサイトの
%収入源として,閲覧者の同意を得ることなくその電子計算機を使用してマイニング
%を行わせるCoinhiveというウェブサービス(以下「コインハイブ」とい
%う。)が,CoinhiveTeamという事業者(以下「コインハイブチーム」
%という。)により提供されていた。
%本件は,被告人が,Xの収入源としてコインハイブによるマイニングの仕組みを
%導入するために本件プログラムコードをサーバコンピュータに保管した行為につい
%て,不正指令電磁的記録保管罪に問われた事案であり,主な争点は,本件プログラ
%ムコードが,刑法168条の2第1項(以下「本件規定」という。)にいう「人が
%電子計算機を使用するに際してその意図に沿うべき動作をさせず,又はその意図に
%反する動作をさせるべき不正な指令を与える電磁的記録」に当たるか否かである
%(以下,「その意図に沿うべき動作をさせず,又はその意図に反する動作をさせる
%べき」という要件を「反意図性」といい,「不正な」という要件を「不正性」とい
%う。)。
%2 第1審判決及び原判決の認定並びに記録によると,本件の事実関係は,以下
%のとおりである。
%被告人は,平成29年9月当時,音声合成ソフトウェアを用いて作られた楽曲の
%情報を共有するウェブサイト「X」を運営していた。
%コインハイブは,平成29年9月,コインハイブチームにより提供が開始された
%ウェブサービスである。その内容は,登録したウェブサイトの運営者(以下「登録
%者」という。)に対し,ウェブサイト閲覧者が閲覧中に使用する電子計算機の中央
%- 3 -
%処理装置に同閲覧者の同意を得ることなく仮想通貨Monero(モネロ)の取引
%台帳へ取引履歴を追記する承認作業等の演算を行わせ,その演算が成功すると,報
%酬として仮想通貨の取得が可能になるというマイニングを実行するプログラムコー
%ド(以下「本体プログラム」という。)を取得するためのプログラムコードを提供
%し,報酬の7割を登録者に分配し,3割をコインハイブチーム側が取得するという
%ものであり,登録者が,提供された前記プログラムコードをウェブサイト内に設置
%すると,閲覧者の電子計算機によりマイニングが実行され,登録者が報酬の分配を
%得ることができるというものであった。
%コインハイブによるマイニングの仕組みは,前記プログラムコードが設置された
%ウェブサイトを閲覧すると,同プログラムコードの指令により閲覧者の電子計算機
%が自動的に本体プログラムが蔵置されたサーバコンピュータに接続され,本体プロ
%グラムが読み込まれてマイニングを指令され,その指令により閲覧者の電子計算機
%の中央処理装置が演算を行い,演算結果が同サーバコンピュータに送信されるとい
%うものであり,閲覧を終了するとマイニングも終了するというものであった。
%被告人は,X閲覧を通じて利益を得るため,平成29年9月21日,コインハイ
%ブに登録し,提供されたプログラムコードに,被告人に割り当てられたサイトキー
%を記述したもの(本件プログラムコード)を,サーバコンピュータ上のX内に設置
%し,本件公訴事実の期間中,Xを構成するファイル内に蔵置して保管した。本件当
%時,一般の使用者に,ウェブサイトの収益方法として閲覧者の電子計算機にマイニ
%ングを行わせるという仕組みは認知されていなかったが,被告人は,Xに,閲覧中
%にマイニングが行われることについて同意を得る仕様を設けたり,マイニングに関
%する説明やマイニングが行われていることの表示をしたりすることなく,本件プロ
%グラムコードを保管していた。
%被告人は,本件プログラムコードにおいて,閲覧者の電子計算機の中央処理装置
%使用率を調整する値を0.5と設定した。この数値の場合,マイニングを実行する
%と,閲覧者の電子計算機の消費電力が若干増加したり中央処理装置の処理速度が遅
%- 4 -
%くなったりするが,極端に遅くはならず,これらの影響の程度は,閲覧者が気付く
%ほどではなく,また,一般的なウェブサイトで広く実行されている広告を表示する
%プログラム(以下「広告表示プログラム」という。)と有意な差異はなかった。
%第2 第1審判決及び原判決
%1 第1審判決は,本件プログラムコードの不正指令電磁的記録該当性につい
%て,要旨,次のとおり判断して,被告人に無罪を言い渡した。
%反意図性は,当該プログラムの機能につき一般に認識すべきと考えられると
%ころを基準として判断するのが相当であるところ,Xにはマイニングに関する説明
%はなく,閲覧中にマイニングが行われることについて同意を得る仕様にもなってい
%なかったこと,ウェブサイトの収益方法として閲覧者の電子計算機にマイニングを
%行わせるという仕組みは一般の使用者に認知されておらず,マイニングによる電子
%計算機への負荷の程度に照らして一般の使用者がその実行に気付くことはないとい
%えることなどからすると,一般の使用者が,X閲覧者の電子計算機にマイニングを
%行わせるという本件プログラムコードの機能について認識すべきとはいえないか
%ら,反意図性が認められる。
%不正性は,ウェブサイトの運営者及び閲覧者等にとっての有用性や必要性,
%使用者への影響や弊害等の事情を考慮し,当該プログラムの機能の内容が社会的に
%許容し得るものであるか否かという観点から判断するのが相当であるところ,①本
%件プログラムコードの実行により運営者が得る利益は,ウェブサイトの質の維持向
%上のための資金源になり得るから,閲覧者にとって利益となる面があること,②本
%件プログラムコードの実行により生ずる閲覧者の電子計算機の処理速度の低下等
%は,広告表示プログラム等の場合と大差ない上,X閲覧中に限定されることなどか
%らすると,本件プログラムコードが社会的に許容されていなかったとはいえず,不
%正性は認められない。
%2 検察官が控訴し,第1審判決には本件規定の解釈適用の誤りや事実誤認があ
%ると主張した。原判決は,本件プログラムコードの不正指令電磁的記録該当性につ
%- 5 -
%いて,要旨,次のとおり判断し,第1審判決は本件規定の解釈を誤って事実を誤認
%したものであるとして,第1審判決を破棄し,被告人を罰金10万円に処した。
%反意図性は,当該プログラムの機能について一般に認識すべきと考えられる
%ところを基準とした上で,一般の使用者の意思に反しないものと評価できるかとい
%う観点から規範的に判断すべきであり,一般の使用者が事前に機能を認識した上で
%実行することが予定されていないプログラムについては,その機能の内容そのもの
%を踏まえ,一般の使用者が機能を認識しないまま当該プログラムを使用することを
%許容していないと規範的に評価できる場合に反意図性を肯定すべきである。
%本件プログラムコードは,X閲覧者の電子計算機にマイニングを行わせるという
%機能を有するものであり,閲覧することによりマイニングが行われることの表示は
%予定されておらず,マイニングにより生じた報酬を閲覧者が得ることは予定されて
%いない。一般に,閲覧者は,閲覧に必要なプログラムを実行することは承認してい
%ると考えられるが,本件プログラムコードによるマイニングは閲覧に必要ではな
%い。その上,本件プログラムコードによるマイニングは閲覧者の電子計算機に一定
%の負荷を与えるものであるのに,閲覧者には利益がもたらされないし,閲覧者にマ
%イニングによって電子計算機が使用されていることを知る機会やマイニングを拒絶
%する機会も保障されていない。
%このような本件プログラムコードは,使用者に利益をもたらさない上,使用者に
%無断で電子計算機を使用して利益を得ようとするものであり,一般の使用者が許容
%しないことは明らかであるから,反意図性を認めた第1審判決の結論は正当である。
%不正性は,反意図性のあるプログラムであっても,使用者として想定される
%者における当該プログラムを使用すること自体に関する利害得失や,使用者に生じ
%得る不利益に対する注意喚起の有無などを考慮した場合,プログラムに対する信頼
%保護や電子計算機による適正な情報処理という観点からみて,社会的に許容される
%ことがあるので,そのような場合を規制の対象から除外する趣旨の要件である。
%本件プログラムコードは,閲覧者に利益を生じさせない一方で一定の不利益を与
%- 6 -
%えるものである上,不利益に関する表示等もされないから,プログラムに対する信
%頼保護という観点から社会的に許容すべき点はない。また,X閲覧中に,閲覧者の
%電子計算機を,閲覧者以外の利益のために無断で使用するものであり,電子計算機
%による適正な情報処理の観点からも,社会的に許容されるということはできない。
%第1審判決は,前記1 ①②等の事情を挙げて不正性を否定するが,①につい
%て,そのような利益は,意に反するプログラムの実行を使用者が気付かないような
%方法で受忍させた上で実現されるべきものではないし,②について,広告表示プロ
%グラムは閲覧に付随して実行され実行結果も表示されるものが一般的であり,その
%点で本件プログラムコードとは大きな相違があるから比較検討になじまない上,本
%件は,意図に反し電子計算機が使用されるプログラムであることが主な問題である
%から,処理速度の低下等が使用者の気付かない程度であったとしても不正性を左右
%しない。
%これらによれば,本件プログラムコードは,その機能を中心に検討すると,反意
%図性もあり不正性も認められる。
%第3 当裁判所の判断
%1 不正指令電磁的記録に関する罪は,電子計算機において使用者の意図に反し
%て実行される不正プログラムが社会に被害を与え深刻な問題となっていることを受
%け,電子計算機による情報処理のためのプログラムが,「意図に沿うべき動作をさ
%せず,又はその意図に反する動作をさせるべき不正な指令」を与えるものではない
%という社会一般の信頼を保護し,ひいては電子計算機の社会的機能を保護するため
%に,反意図性があり,社会的に許容し得ない不正性のある指令を与えるプログラム
%の作成,提供,保管等を,一定の要件の下に処罰するものである。
%このような本件規定の趣旨及び保護法益に照らせば,プログラムの反意図性及び
%不正性については,次のとおり解するのが相当である。
%すなわち,反意図性は,当該プログラムについて一般の使用者が認識すべき動作
%と実際の動作が異なる場合に肯定されるものと解するのが相当であり,一般の使用
%- 7 -
%者が認識すべき動作の認定に当たっては,当該プログラムの動作の内容に加え,プ
%ログラムに付された名称,動作に関する説明の内容,想定される当該プログラムの
%利用方法等を考慮する必要がある。
%また,不正性は,電子計算機による情報処理に対する社会一般の信頼を保護し,
%電子計算機の社会的機能を保護するという観点から,社会的に許容し得ないプログ
%ラムについて肯定されるものと解するのが相当であり,その判断に当たっては,当
%該プログラムの動作の内容に加え,その動作が電子計算機の機能や電子計算機によ
%る情報処理に与える影響の有無・程度,当該プログラムの利用方法等を考慮する必
%要がある。
%2 本件プログラムコードの動作は,Xの閲覧中,閲覧者の電子計算機を使用し
%てマイニングを行わせるというものである。
%一般的なウェブサイトにおいて,運営者が閲覧を通じて利益を得る仕組みとして
%広告表示プログラムが広く実行されている実情に照らせば,一般の使用者におい
%て,ウェブサイト閲覧中に,閲覧者の電子計算機を一定程度使用して運営者が利益
%を得るプログラムが実行され得ることは,想定の範囲内であるともいえる。
%しかしながら,そのようなプログラムとして,本件プログラムコードの動作を一
%般の使用者が認識すべきといえるか否かについてみると,Xは,閲覧中にマイニン
%グが行われることについて同意を得る仕様になっておらず,マイニングに関する説
%明やマイニングが行われていることの表示もなかったこと,ウェブサイトの収益方
%法として閲覧者の電子計算機にマイニングを行わせるという仕組みは一般の使用者
%に認知されていなかったことといった事情がある。これらの事情によれば,本件プ
%ログラムコードの動作を一般の使用者が認識すべきとはいえず,反意図性が認めら
%れる。
%3 本件プログラムコードは,Xの運営者である被告人が,X閲覧を通じて利益
%を得るため,閲覧者の同意を得ることなく,その電子計算機においてマイニングを
%行わせるために保管したものである。
%- 8 -
%確かに,原判示のとおり,本件プログラムコードによるマイニングは,閲覧者の
%同意を得ることなくその電子計算機に一定の負荷を与え,これに関する報酬を閲覧
%者が取得することができないものであるのに,閲覧者にマイニングの実行を知る機
%会やこれを拒絶する機会が保障されていないなど,プログラムに対する信頼という
%観点から,より適切な利用方法等が採り得たものである。
%しかしながら,前記1の保護法益に照らして重要な事情である電子計算機の機能
%や電子計算機による情報処理に与える影響は,X閲覧中に閲覧者の電子計算機の中
%央処理装置を一定程度使用することにとどまり,その使用の程度も,閲覧者の電子
%計算機の消費電力が若干増加したり中央処理装置の処理速度が遅くなったりする
%が,閲覧者がその変化に気付くほどのものではなかったと認められる。
%また,ウェブサイトの運営者が閲覧を通じて利益を得る仕組みは,ウェブサイト
%による情報の流通にとって重要であるところ,被告人は,本件プログラムコードを
%そのような収益の仕組みとして利用したものである上,本件プログラムコードは,
%そのような仕組みとして社会的に受容されている広告表示プログラムと比較して
%も,閲覧者の電子計算機の機能や電子計算機による情報処理に与える影響において
%有意な差異は認められず,事前の同意を得ることなく実行され,閲覧中に閲覧者の
%電子計算機を一定程度使用するという利用方法等も同様であって,これらの点は社
%会的に許容し得る範囲内といえるものである。
%さらに,本件プログラムコードの動作の内容であるマイニング自体は,仮想通貨
%の信頼性を確保するための仕組みであり,社会的に許容し得ないものとはいい難い。
%以上のような,本件プログラムコードの動作の内容,その動作が電子計算機の機
%能や電子計算機による情報処理に与える影響,その利用方法等を考慮すると,本件
%プログラムコードは,社会的に許容し得ないものとはいえず,不正性は認められな
%い。
%4 以上のとおり,本件プログラムコードは,反意図性は認められるが,不正性
%は認められないため,不正指令電磁的記録とは認められない。
%- 9 -
%原判決は,不正指令電磁的記録の解釈を誤り,その該当性を判断する際に考慮す
%べき事情を適切に考慮しなかったため,重大な事実誤認をしたものというべきであ
%り,これらが判決に影響を及ぼすことは明らかであって,原判決を破棄しなければ
%著しく正義に反すると認められる。
%よって,刑訴法411条1号,3号により原判決を破棄することとし,上記の検
%討によれば,本件プログラムコードの不正指令電磁的記録該当性を否定して被告人
%を無罪とした第1審判決は是認することができ,本件規定の解釈適用の誤りや事実
%誤認を主張する検察官の控訴は理由がないことに帰するから,同法413条ただし
%書,414条,396条によりこれを棄却することとし,裁判官全員一致の意見
%で,主文のとおり判決する。
%検察官清野憲一,同古賀栄美,同山内由光 公判出席
%( 裁 判 長 裁 判 官 山 口 厚 裁 判 官 深 山 卓 也 裁 判 官 安 浪 亮 介 裁 判 官
%岡 正晶 裁判官 堺 徹)
