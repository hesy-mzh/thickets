\documentclass[10pt,b5paper]{ltjsarticle}

\usepackage[margin=15truemm]{geometry}
\pagestyle{empty}

\usepackage{amssymb}
\usepackage{amsmath}	% required for `\align' (yatex added)

\begin{document}

文章を読み問題に答える場合、
問題文に先に目を通してから文章を読む方法があります。
早く解くためには効果はあると思います。

ただ、きちんと解くためには内容を理解する必要があります。
概ね次のようなことが書かれています。
\begin{itemize}
 \item[第1段落] 現代科学の過去からの発展
 \item[第2段落] 東洋の科学
 \item[第3段落] 老子と荘子の思想
 \item[第4段落] 老子と荘子の思想について著者の考え
\end{itemize}
全体を通して著者は
「西洋の科学が主流だが東洋の思想からも科学が生まれてもおかしくない」
ことを書いています。

\begin{enumerate}\renewcommand{\theenumi}{(\arabic{enumi})}
 \item \textbf{主客転倒}

       主人と客の立場が逆になること。転じて、人の立場・順序・軽重などが逆になること。
 \item ヨーロッパで発達してきた

       文節に分ける場合、単語に分け単語だけで意味が通じる物(自立語)を見つけます。

       \textbf{ヨーロッパ} \slash\ \textbf{で} \slash\ \textbf{発達して}\
       \slash\ \textbf{きた}

       意味の通じない部分(付属語) ``\textbf{で}''はその前の
       ``\textbf{ヨーロッパ}''につなげて文節を作ります。

       ``\textbf{発達して}''は
       ``発達する''という動詞が変化したものです。
       ``発達'' + ``して'' とは分けません。
       ``して''は``する''という動詞ですが、
       名詞 + ``する'' で動詞となり
       動詞はそれだけで一つの単語になるので、
       ここで文節を分けることはありません。
%       この動詞は ``~をする'' という形で使われます。
%       主語の対し ``する''だけが来ることはありません。

       \underline{
       \textbf{ヨーロッパで} \slash\ \textbf{発達して}\
       \slash\ \textbf{きた}
       }

 \item 歴史的に正しいであろう

       言葉の意味としては「今までは正しい」ことを意味しています。

       解答欄は「~の話としては正しいと感じられる」となっているので、
       空欄には「ヨーロッパの科学」や「過去から今まで」といった意味が入ると
       考えられます。

       問題に8文字で探すように指示されているので、
       該当箇所を探します。
       多くの場合は前方にありますが、
       そうでない場合は同じ主張をもう一度している場所を探します。

       この場合、第2段落の先頭が同じようなことに言及していますから
       解答する8文字は「\underline{\textbf{過去か}}ら現在まで」となります。

 \item 荘子を読む効用

       第3段落の後半に荘子の文章について説明してあり、
       この中で読むとどんな効果があるかが書いてある1文を探すと
       次のものが見つかります。

       ``
       \underline{\textbf{読む方}}の頭の働きを刺激し、
       活発にしてくれるものが非常に多い気がする。
       ''

 \item 科学の発達のもとになりうるのがギリシャの思想だけとは限らない

       ギリシャの思想が現在の科学に根付いているという話から
       科学のもとになるのがギリシャの思想だけとは限らないのは
       ``仮想の話''か``未来の話''どちらかになります。

       第2段落に未来の話が書かれていて、
       そこに同じ様な意味があります。

       ``
       何もギリシャの思想だけが科学の発達の母胎となる唯一のもの
       とは限らないだろう。
       ''

       問題は32文字で探すように指示されていて、
       文章は1行32文字で書かれているから
       2行の区切りが同じ高さになる場所を探すと
       次のことばが抜き取れます。

       \underline{\textbf{ギリシ}}ャの思想だけが
       科学の発達の母胎となる唯一のものとは限らない

 \item そこの指し示す内容

       ``そこ''とは人間にとって不愉快なことであるので
       人にとって都合の悪い内容を探します。

       問題文には「~とみなすような老子や荘子の考え方」とあるので、
       老子や荘子の考え方を探します。

       該当する内容は第4段落の中盤です。
       
       老子や荘子は、自然の力は圧倒的に強く、
       人間の力ではどうにもならない自然の中で、
       人間はただ右へ左へ振り回せ荒れているだけだと考えた。
       著者は 中学時代に極端な考えであるとおもい、
       高校時代に人間は無力だという考え方が我慢ならなかった。

       \underline{\textbf{圧倒的な自然の力には人間は無力である}}(18文字)

 \item 内容とあっているもの

       \textbf{インドに古くからあった思想がギリシャ思想のもとになった}

       この文章にはインドについて触れていない。
       このもととなった文章には書かれているかもしれないし
       全く書かれていないかもしれない。
       この解答が正しいとも間違いともいえないが、
       今回の内容には含まれていない。

       \textbf{中国の古代哲学から科学が生まれていないというのは誤り}

       第2段落に``中国の古代哲学から科学は生まれなかった''とあるので、
       この解答は間違っています。

       \textbf{老子や荘子の思想はギリシャ思想とは違って合理主義}

       第3段落の終わりにこれについて書かれています。

       ``それはそれで一種の徹底した合理主義的な考え''

       この文章から著者は「ギリシャ思想は合理的であり、
       老子や荘子の思想は(異質ではあるが)合理的」と考えています。

       解答の文章は
       ギリシャ思想は合理的ではなく、老思想史は合理的だということなので
       間違いです。

       \textbf{ギリシャ思想は人間が自力で理想を実天する余地がある}

       第4段落の最初にこれについて書かれています。

       選択肢アは正しいか間違っているかはわからない内容でこの文章にはかかれていない。
       選択肢イとウは文章の内容とは異なっている。
       答えは \underline{\textbf{エ}}

 \item 段落の説明

       第1段落は現在の科学の現状について書かれている。

       第2段落は中国の思想がこれからの科学に影響を与えるかもしれないこと
       を書いている。

       第3段落は老子、荘子の思想について書いている。

       第4段落は老子、荘子について著者の考えが書いてある。

       解答の選択肢は\underline{\textbf{イ}}が最も適当
\end{enumerate}




\end{document}
