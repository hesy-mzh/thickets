\documentclass[12pt,b5paper]{ltjsarticle}

%\usepackage[margin=15truemm, top=5truemm, bottom=5truemm]{geometry}
%\usepackage[margin=10truemm,left=15truemm]{geometry}
\usepackage[margin=10truemm]{geometry}

\usepackage{amsmath,amssymb}
%\pagestyle{headings}
\pagestyle{empty}

%\usepackage{listings,url}
%\renewcommand{\theenumi}{(\arabic{enumi})}

%\usepackage{graphicx}

%\usepackage{tikz}
%\usetikzlibrary {arrows.meta}
%\usepackage{wrapfig}	% required for `\wrapfigure' (yatex added)
%\usepackage{bm}	% required for `\bm' (yatex added)

% ルビを振る
\usepackage{luatexja-ruby}	% required for `\ruby'

%% 核Ker 像Im Hom を定義
%\newcommand{\Img}{\mathop{\mathrm{Im}}\nolimits}
%\newcommand{\Ker}{\mathop{\mathrm{Ker}}\nolimits}
%\newcommand{\Hom}{\mathop{\mathrm{Hom}}\nolimits}

%\DeclareMathOperator{\Rot}{rot}
%\DeclareMathOperator{\Div}{div}
%\DeclareMathOperator{\Grad}{grad}
%\DeclareMathOperator{\arcsinh}{arcsinh}
%\DeclareMathOperator{\arccosh}{arccosh}
%\DeclareMathOperator{\arctanh}{arctanh}



%\usepackage{listings,url}
%
%\lstset{
%%プログラム言語(複数の言語に対応,C,C++も可)
%  language = Python,
%%  language = Lisp,
%%  language = C,
%  %背景色と透過度
%  %backgroundcolor={\color[gray]{.90}},
%  %枠外に行った時の自動改行
%  breaklines = true,
%  %自動改行後のインデント量(デフォルトでは20[pt])
%  breakindent = 10pt,
%  %標準の書体
%%  basicstyle = \ttfamily\scriptsize,
%  basicstyle = \ttfamily,
%  %コメントの書体
%%  commentstyle = {\itshape \color[cmyk]{1,0.4,1,0}},
%  %関数名等の色の設定
%  classoffset = 0,
%  %キーワード(int, ifなど)の書体
%%  keywordstyle = {\bfseries \color[cmyk]{0,1,0,0}},
%  %表示する文字の書体
%  %stringstyle = {\ttfamily \color[rgb]{0,0,1}},
%  %枠 "t"は上に線を記載, "T"は上に二重線を記載
%  %他オプション:leftline,topline,bottomline,lines,single,shadowbox
%  frame = TBrl,
%  %frameまでの間隔(行番号とプログラムの間)
%  framesep = 5pt,
%  %行番号の位置
%  numbers = left,
%  %行番号の間隔
%  stepnumber = 1,
%  %行番号の書体
%%  numberstyle = \tiny,
%  %タブの大きさ
%  tabsize = 4,
%  %キャプションの場所("tb"ならば上下両方に記載)
%  captionpos = t
%}



\begin{document}

\hrulefill
\textbf{Theorem}
\hrulefill

\textbf{Fermat's little theorem}

$a \in\mathbb{Z}$、$p$が素数とする。
$a$と$p$が互いに素である時、次の式が成り立つ。
\begin{equation}
 a^{p-1}\equiv 1 \pmod{p}
\end{equation}
これを\ruby{Fermat}{フェルマー}の小定理という。


\textbf{Euler's theorem}

$a\in\mathbb{Z}$とし、
$n\in\mathbb{N}$は$a$と互いに素であるとする。
この時、次の式が成り立つ。
\begin{equation}
 a^{\varphi(n)}\equiv 1 \pmod{n}
\end{equation}
$\varphi(n)$は\ruby{Euler}{オイラー}関数で、$n$未満の$n$と互いに素な自然数の個数を表す。
これを\ruby{Euler}{オイラー}の定理という。


\hrulefill
\textbf{問題}
\hrulefill

%\begin{equation}
% 999^{\left( 999^{\left( 999^{\left( 999^{999} \right)} \right)} \right)}
%\end{equation}

$999^{\left( 999^{\left( 999^{\left( 999^{999} \right)} \right)} \right)}$
を$11$で割った余りを求めよ。

\dotfill

問題の式を$a_0$とし、
指数部分を順に$a_1,a_2,a_3$を次のように置く。
\begin{equation}
 a_0 = 999^{\left( 999^{\left( 999^{\left( 999^{999} \right)} \right)} \right)},\
  a_1 = 999^{\left( 999^{\left( 999^{999} \right)} \right)},\
  a_2 = 999^{\left( 999^{999} \right)},\
  a_3 = 999^{999}
\end{equation}

$999$と$11$は互いに素であるので、
フェルマーの小定理より次が成り立つ。
\begin{equation}
 999^{11-1} = 999^{10} \equiv 1 \pmod{11}
  \label{fermat}
\end{equation}

つまり、$a_{0}$の指数部分$a_{1}$を
$10$で割った余り$r_{1}$に置き換えたものと$a_{0}$は合同である。
\begin{equation}
 a_0 = 999^{a_1}= 999^{10q_{1}+r_{1}} \equiv 999^{r_{1}} \pmod{11}
\end{equation}

そこで$a_1$を$10$で割った余り$r_{1}$を求める。
\begin{equation}
 a_1=10q_{1}+r_{1}
  \label{exp1}
\end{equation}

$999 = 10^3-1\equiv -1\pmod{10}$であるので、
$2$乗したものが$1$と合同となる。
\begin{equation}
 999^2 \equiv (-1)^2 \equiv 1 \pmod{10}
\end{equation}

つまり、$a_{1}$の指数部分$a_{2}$を
$2$で割った余り$r_{2}$に置き換えたものと$a_{1}$は合同である。
\begin{equation}
 a_1 = 999^{a_2}= 999^{2q_{2}+r_{2}} \equiv 999^{r_{2}} \pmod{2}
\end{equation}
である。


そこで$a_2$を$2$で割った余り$r_{2}$を求める。
\begin{equation}
 a_2=2q_{2}+r_{2}
  \label{exp2}
\end{equation}

$999 \equiv 1 \pmod{2}$であるので、
$a_{2}$が次のようになる。
\begin{equation}
 a_2 = 999^{a_3} \equiv 1^{a_3} \equiv 1 \pmod{2}
\end{equation}

つまり、式(\ref{exp2})の余りが$1$になる。
\begin{equation}
 a_{2}=2q_{2}+1
\end{equation}

これにより$a_{1}$は次のようになる。
\begin{equation}
 a_{1}= 999^{a_{2}}
  = 999^{2q_{2}+1}
  = \left( 999^{2} \right)^{q_{2}}\times 999^{1}
  = 999\left( 999^{2} \right)^{q_{2}}
\end{equation}



$\mod{10}$において
$999\equiv 9$ と $999^2\equiv 1$であるから
上の式は次のようになる。
\begin{equation}
 a_{1}= 999\left( 999^{2} \right)^{q_{2}}
  \equiv 9(1)^{q_{2}} \equiv 9\pmod{10}
\end{equation}

これを用いて式(\ref{exp1})の$r_{1}$を求める。
\begin{equation}
 a_{1}=10q_{1}+9
\end{equation}

これより$a_{0}$は次のようになる。
\begin{equation}
 a_{0}= 999^{a_1}
  = 999^{10q_{1}+9}
  = 999^{9} \left( 999^{10} \right)^{q_{1}}
\end{equation}

フェルマーの小定理 (\ref{fermat})より
$999^{10}\equiv 1\pmod{11}$
であるので上の式は次のようになる。
\begin{equation}
 a_{0}= 999^{9} \left( 999^{10} \right)^{q_{1}}
  \equiv 999^{9} (1)^{q_{1}}
  \equiv 999^{9} \pmod{11}
\end{equation}

つまり、$a_{0}$を$11$で割った余りと
$999^9$を$11$で割った余りは一致する。



$999 \equiv 9 \equiv -2 \pmod{11}$であるので、
$999^9 \equiv (-2)^9 \equiv -512$
となる。
$-512=11\times (-47) + 5$より
$a_{0}\equiv 5\pmod{11}$となる。
\begin{equation}
 999^{\left( 999^{\left( 999^{\left( 999^{999} \right)} \right)} \right)}
  \equiv 5 \pmod{11}
\end{equation}


\hrulefill


フェルマーの小定理(\ref{fermat})より
$999^{10}\equiv 1\pmod{11}$としたが、
$999^{5}\equiv 1\pmod{11}$であるので
これを用いたほうが少しだけ式が単純になるかもしれない。

\hrulefill



\end{document}
