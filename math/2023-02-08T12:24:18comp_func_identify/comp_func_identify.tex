\documentclass[12pt,b5paper]{ltjsarticle}

%\usepackage[margin=15truemm, top=5truemm, bottom=5truemm]{geometry}
%\usepackage[margin=10truemm,left=15truemm]{geometry}
\usepackage[margin=10truemm]{geometry}

\usepackage{amsmath,amssymb}
%\pagestyle{headings}
\pagestyle{empty}

%\usepackage{listings,url}
%\renewcommand{\theenumi}{(\arabic{enumi})}

\usepackage{graphicx}

%\usepackage{tikz}
%\usetikzlibrary {arrows.meta}
%\usepackage{wrapfig}
%\usepackage{bm}

% ルビを振る
%\usepackage{luatexja-ruby}	% required for `\ruby'

%% 核Ker 像Im Hom を定義
%\newcommand{\Img}{\mathop{\mathrm{Im}}\nolimits}
%\newcommand{\Ker}{\mathop{\mathrm{Ker}}\nolimits}
%\newcommand{\Hom}{\mathop{\mathrm{Hom}}\nolimits}

%\DeclareMathOperator{\Rot}{rot}
%\DeclareMathOperator{\Div}{div}
%\DeclareMathOperator{\Grad}{grad}
%\DeclareMathOperator{\arcsinh}{arcsinh}
%\DeclareMathOperator{\arccosh}{arccosh}
%\DeclareMathOperator{\arctanh}{arctanh}



%\usepackage{listings,url}
%
%\lstset{
%%プログラム言語(複数の言語に対応,C,C++も可)
%  language = Python,
%%  language = Lisp,
%%  language = C,
%  %背景色と透過度
%  %backgroundcolor={\color[gray]{.90}},
%  %枠外に行った時の自動改行
%  breaklines = true,
%  %自動改行後のインデント量(デフォルトでは20[pt])
%  breakindent = 10pt,
%  %標準の書体
%%  basicstyle = \ttfamily\scriptsize,
%  basicstyle = \ttfamily,
%  %コメントの書体
%%  commentstyle = {\itshape \color[cmyk]{1,0.4,1,0}},
%  %関数名等の色の設定
%  classoffset = 0,
%  %キーワード(int, ifなど)の書体
%%  keywordstyle = {\bfseries \color[cmyk]{0,1,0,0}},
%  %表示する文字の書体
%  %stringstyle = {\ttfamily \color[rgb]{0,0,1}},
%  %枠 "t"は上に線を記載, "T"は上に二重線を記載
%  %他オプション:leftline,topline,bottomline,lines,single,shadowbox
%  frame = TBrl,
%  %frameまでの間隔(行番号とプログラムの間)
%  framesep = 5pt,
%  %行番号の位置
%  numbers = left,
%  %行番号の間隔
%  stepnumber = 1,
%  %行番号の書体
%%  numberstyle = \tiny,
%  %タブの大きさ
%  tabsize = 4,
%  %キャプションの場所("tb"ならば上下両方に記載)
%  captionpos = t
%}



\begin{document}

\hrulefill

複素関数$f(z)\in\mathbb{C}[z]$が正則であるとする。

線分$z=t+it \ (t\in\mathbb{R}, \ 0 \leq t \leq 1)$上で
$f(z)=it^2$となる関数$f(z)$を全て求めよ。

\dotfill

一致の定理より
複素平面上で正則な関数は
線分$z=t+it$上で一致すれば全体でも一致する。

$f(z)$は正則であるので、
Taylor展開を考える。

$f^{\prime}(z)$を計算する。
\begin{align}
 f^{\prime}(z)
  =& \lim_{\Delta z \to 0} \frac{f(z+ \Delta z)-f(z)}{\Delta z}
  = \lim_{h \to 0} \frac{f((t+it)+(h+ih))-f(t+it)}{h+ih}\\
  =& \lim_{h \to 0} \frac{i(t+h)^2-it^2}{h(1+i)}
  = \lim_{h \to 0} \frac{i(2t+h)}{1+i}
  = \frac{2ti}{1+i}
  = (1+i)t
\end{align}
同様に
$f^{\prime\prime}(z),f^{\prime\prime\prime}(z)$
も求める。
\begin{align}
 f^{\prime\prime}(z)
  =& \lim_{\Delta z \to 0}
  \frac{f^{\prime}(z+ \Delta z)-f^{\prime}(z)}{\Delta z}
 = \lim_{h\to 0} \frac{(1+i)(t+h) - (1+i)t}{h+ih}
% =& \lim_{h\to 0} \frac{(1+i)(t+h) - (1+i)t}{h+ih}
 =1
\end{align}

$f^{\prime\prime}(z)=1$となり、
3次の導関数以降
$f^{\prime\prime\prime}(z)=\cdots =0$となる。


これらを用いて
$z=0$を中心としたTaylor展開を行う。
\begin{align}
 f(z)
  =& \sum_{k=0}^{\infty}\frac{f^{(k)}(0)}{k!}z^k
  = f(0) + f^{\prime}(0)z + \frac{f^{\prime\prime}(0)}{2!}z^{2} + \frac{f^{\prime\prime\prime}(0)}{3!}z^{3} +\cdots\\
  =& 0 + 0z + \frac{1}{2}z^2 + 0z^3 + \cdots
  = \frac{1}{2}z^2
\end{align}

一致の定理より線分上で一致する正則関数は
$f(z)=\frac{1}{2}z^2$のみとなる。

\hrulefill

\end{document}
