\documentclass[12pt,b5paper]{ltjsarticle}

%\usepackage[margin=15truemm, top=5truemm, bottom=5truemm]{geometry}
%\usepackage[margin=10truemm,left=15truemm]{geometry}
\usepackage[margin=10truemm]{geometry}

\usepackage{amsmath,amssymb}
%\pagestyle{headings}
\pagestyle{empty}

%\usepackage{listings,url}
%\renewcommand{\theenumi}{(\arabic{enumi})}

%\usepackage{graphicx}

%\usepackage{tikz}
%\usetikzlibrary {arrows.meta}
%\usepackage{wrapfig}
%\usepackage{bm}

% ルビを振る
\usepackage{luatexja-ruby}	% required for `\ruby'

%% 核Ker 像Im Hom を定義
%\newcommand{\Img}{\mathop{\mathrm{Im}}\nolimits}
%\newcommand{\Ker}{\mathop{\mathrm{Ker}}\nolimits}
%\newcommand{\Hom}{\mathop{\mathrm{Hom}}\nolimits}

%\DeclareMathOperator{\Rot}{rot}
%\DeclareMathOperator{\Div}{div}
%\DeclareMathOperator{\Grad}{grad}
%\DeclareMathOperator{\arcsinh}{arcsinh}
%\DeclareMathOperator{\arccosh}{arccosh}
%\DeclareMathOperator{\arctanh}{arctanh}



%\usepackage{listings,url}
%
%\lstset{
%%プログラム言語(複数の言語に対応,C,C++も可)
%  language = Python,
%%  language = Lisp,
%%  language = C,
%  %背景色と透過度
%  %backgroundcolor={\color[gray]{.90}},
%  %枠外に行った時の自動改行
%  breaklines = true,
%  %自動改行後のインデント量(デフォルトでは20[pt])
%  breakindent = 10pt,
%  %標準の書体
%%  basicstyle = \ttfamily\scriptsize,
%  basicstyle = \ttfamily,
%  %コメントの書体
%%  commentstyle = {\itshape \color[cmyk]{1,0.4,1,0}},
%  %関数名等の色の設定
%  classoffset = 0,
%  %キーワード(int, ifなど)の書体
%%  keywordstyle = {\bfseries \color[cmyk]{0,1,0,0}},
%  %表示する文字の書体
%  %stringstyle = {\ttfamily \color[rgb]{0,0,1}},
%  %枠 "t"は上に線を記載, "T"は上に二重線を記載
%  %他オプション:leftline,topline,bottomline,lines,single,shadowbox
%  frame = TBrl,
%  %frameまでの間隔(行番号とプログラムの間)
%  framesep = 5pt,
%  %行番号の位置
%  numbers = left,
%  %行番号の間隔
%  stepnumber = 1,
%  %行番号の書体
%%  numberstyle = \tiny,
%  %タブの大きさ
%  tabsize = 4,
%  %キャプションの場所("tb"ならば上下両方に記載)
%  captionpos = t
%}



\begin{document}

\hrulefill

\textbf{有限加法族}

集合$X$の部分集合族$\mathcal{F}$が
\textbf{有限加法族}である
とは次を満たすときをいう。
\begin{enumerate}
 \item $\emptyset \in \mathcal{F}$
 \item $A \in \mathcal{F} \Rightarrow X\backslash A \in\mathcal{F}$
 \item $A,B\in\mathcal{F}
       \Rightarrow A\cup B \in\mathcal{F}$
\end{enumerate}


\textbf{有限加法的測度}

集合$X$上の有限加法族$\mathcal{F}$について、
$m:\mathcal{F}\to [0,\infty]$が
$(X,\mathcal{F})$上の
\textbf{有限加法的測度}であるとは、
次の2つの条件を満たすときをいう。
\begin{enumerate}
 \item $m(\emptyset) =0$
 \item $A,B\in\mathcal{F}$が互いに素である時、
       $m(A\cup B) = m(A) + m(B)$
\end{enumerate}


\textbf{外測度}

$X$を集合とする。
$\Gamma : 2^{X}\to [0,\infty]$が
$X$上の\textbf{外測度}であるとは、
次の3つの条件を満たすときをいう。
\begin{enumerate}
 \item
      $\Gamma (\emptyset) = 0$
 \item
      $A,B \subset X$が
      $A\subset B$を満たす時、
      $\Gamma(A)\leq \Gamma(B)$
 \item
      $X$の任意の部分集合列$\{A_{n}\}_{n=1}^{\infty}$
      に対し、
      $\Gamma(\bigcup_{n=1}^{\infty}A_{n}) \leq \sum_{n=1}^{\infty}\Gamma(A_{n})$
\end{enumerate}



\textbf{$\Gamma$-可測}

$X$を集合とする。
$\Gamma : 2^{X}\to [0,\infty]$を
$X$上の外測度とする。

集合$E\subset X$が\textbf{$\Gamma$-可測}
(または \ruby{Carath\'eodory}{カラテオドリ}の意味で可測)
とは、
任意の$A\subset X$に対し次を満たすときをいう。
\begin{equation}
 \Gamma(A\cap E) + \Gamma(A\cap (X\backslash E)) = \Gamma(A)
\end{equation}

また、$\Gamma$-可測集合全体を$\mathcal{M}_{\Gamma}$と表す。




\hrulefill

\begin{enumerate}
 \item
      $X=\{ 1,2,3\}$とする。
      $X$上の有限加法族を全て挙げよ。

      \dotfill

      $X$の部分集合
      \begin{equation}
       \emptyset,\{1\},\{2\},\{3\},
        \{1,2\},\{2,3\},\{1,3\},X
      \end{equation}

      $\emptyset$を含む最小の有限加法族
      \quad $\{ \emptyset,X\}$

      $\emptyset,\{1\}$を含む最小の有限加法族
      \quad $\{ \emptyset,\{1\},\{2,3\},X\}$

      $\emptyset,\{2\}$を含む最小の有限加法族
      \quad $\{ \emptyset,\{2\},\{1,3\},X\}$

      $\emptyset,\{3\}$を含む最小の有限加法族
      \quad $\{ \emptyset,\{3\},\{1,2\},X\}$

      $\emptyset,\{1\},\{2\}$を含む最小の有限加法族
      \quad $\{ \emptyset,\{1\},\{2\},\{3\},\{1,2\},\{1,3\},\{2,3\},X\}$

      $\emptyset,\{1\},\{3\}$や
      $\emptyset,\{1\},\{1,2\}$や
      $\emptyset,\{1\},\{1,3\}$を含む最小の有限加法族
      は上の加法族と同じであり、
      $\emptyset,\{1\},\{2,3\}$を含む最小の有限加法族
      は
      $\emptyset,\{1\}$を含む最小の有限加法族
      と同じである。

      よって、以上の5種類が$X$上の有限加法族である。

      \hrulefill

 \item
      $X$を集合とし、
      $\Gamma : 2^{X}\to[0,\infty]$を
      $X$上の外測度とする。
      $E\subset X$が$\Gamma$-可測であることと
      次が成り立つことは同値であることを示せ。
      \begin{center}
       任意の$A\subset X$に対し、
       $\Gamma(A\cap E) + \Gamma(A\cap (X\backslash E)) \leq \Gamma(A)$
      \end{center}

      \dotfill

      \begin{enumerate}
       \item \label{enu:kasoku}
             $E\subset X$が$\Gamma$-可測である
       \item \label{enu:prop}
             任意の$A\subset X$に対し、
             $\Gamma(A\cap E) + \Gamma(A\cap (X\backslash E)) \leq \Gamma(A)$
      \end{enumerate}

      \eqref{enu:kasoku}
      $\Rightarrow$
      \eqref{enu:prop}

      $E\subset X$が$\Gamma$-可測であるとする。
      定義より、
      任意の$A\subset X$に対し
      $\Gamma(A\cap E) + \Gamma(A\cap (X\backslash E)) = \Gamma(A)$
      である。

      $\Gamma : 2^{X}\to[0,\infty]$であるので、
      $\Gamma(A)\in[0,\infty]$である。
      つまり、$\Gamma(A) \leq \Gamma(A)$ である。

      よって、
      任意の$A\subset X$に対し、
      $\Gamma(A\cap E) + \Gamma(A\cap (X\backslash E)) \leq \Gamma(A)$
      となる。
      

      \eqref{enu:kasoku}
      $\Leftarrow$
      \eqref{enu:prop}

      任意の$A\subset X$に対し、
      $\Gamma(A\cap E) + \Gamma(A\cap (X\backslash E)) \leq \Gamma(A)$
      とする。

      $\Gamma$は外側度であるので、
      $\Gamma(\bigcup_{n=1}^{\infty}A_{n}) \leq \sum_{n=1}^{\infty}\Gamma(A_{n})$である。

      これより、
      次の式が得られる。
      \begin{equation}
       \Gamma( (A\cap E) \cup (A\cap (X\backslash E) )
        \leq
       \Gamma(A\cap E) + \Gamma(A\cap (X\backslash E))
      \end{equation}

      $(A\cap E) \cup (A\cap (X\backslash E) =A$
      であるので、
      \begin{equation}
       \Gamma(A)
        \leq
       \Gamma(A\cap E) + \Gamma(A\cap (X\backslash E))
       \leq
       \Gamma(A)
      \end{equation}
      であり、
      \begin{equation}
       \Gamma(A\cap E) + \Gamma(A\cap (X\backslash E))
       =
       \Gamma(A)
      \end{equation}
      となる。

      つまり、
      $E$が$\Gamma$-可測であることになる。



      \hrulefill

\end{enumerate}

\hrulefill

\end{document}
