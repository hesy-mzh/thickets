\documentclass[12pt,b5paper]{ltjsarticle}

%\usepackage[margin=15truemm, top=5truemm, bottom=5truemm]{geometry}
%\usepackage[margin=10truemm,left=15truemm]{geometry}
\usepackage[margin=10truemm]{geometry}

\usepackage{amsmath,amssymb}
%\pagestyle{headings}
\pagestyle{empty}

%\usepackage{listings,url}
%\renewcommand{\theenumi}{(\arabic{enumi})}

%\usepackage{graphicx}

%\usepackage{tikz}
%\usetikzlibrary {arrows.meta}
%\usepackage{wrapfig}	% required for `\wrapfigure' (yatex added)
%\usepackage{bm}	% required for `\bm' (yatex added)

% ルビを振る
%\usepackage{luatexja-ruby}	% required for `\ruby'

%% 核Ker 像Im Hom を定義
%\newcommand{\Img}{\mathop{\mathrm{Im}}\nolimits}
%\newcommand{\Ker}{\mathop{\mathrm{Ker}}\nolimits}
%\newcommand{\Hom}{\mathop{\mathrm{Hom}}\nolimits}

%\DeclareMathOperator{\Rot}{rot}
%\DeclareMathOperator{\Div}{div}
%\DeclareMathOperator{\Grad}{grad}
%\DeclareMathOperator{\arcsinh}{arcsinh}
%\DeclareMathOperator{\arccosh}{arccosh}
%\DeclareMathOperator{\arctanh}{arctanh}



%\usepackage{listings,url}
%
%\lstset{
%%プログラム言語(複数の言語に対応,C,C++も可)
%  language = Python,
%%  language = Lisp,
%%  language = C,
%  %背景色と透過度
%  %backgroundcolor={\color[gray]{.90}},
%  %枠外に行った時の自動改行
%  breaklines = true,
%  %自動改行後のインデント量(デフォルトでは20[pt])
%  breakindent = 10pt,
%  %標準の書体
%%  basicstyle = \ttfamily\scriptsize,
%  basicstyle = \ttfamily,
%  %コメントの書体
%%  commentstyle = {\itshape \color[cmyk]{1,0.4,1,0}},
%  %関数名等の色の設定
%  classoffset = 0,
%  %キーワード(int, ifなど)の書体
%%  keywordstyle = {\bfseries \color[cmyk]{0,1,0,0}},
%  %表示する文字の書体
%  %stringstyle = {\ttfamily \color[rgb]{0,0,1}},
%  %枠 "t"は上に線を記載, "T"は上に二重線を記載
%  %他オプション:leftline,topline,bottomline,lines,single,shadowbox
%  frame = TBrl,
%  %frameまでの間隔(行番号とプログラムの間)
%  framesep = 5pt,
%  %行番号の位置
%  numbers = left,
%  %行番号の間隔
%  stepnumber = 1,
%  %行番号の書体
%%  numberstyle = \tiny,
%  %タブの大きさ
%  tabsize = 4,
%  %キャプションの場所("tb"ならば上下両方に記載)
%  captionpos = t
%}



\begin{document}

\hrulefill

\textbf{弧状連結}

位相空間$X$の任意の2点$x,y$に対し、
閉区間$I=[0,1]$から$X$への
連続写像$\sigma:I\to X$で、
$\sigma(0)=x,\sigma(1)=y$を満たすものが存在する時、
$X$は弧状連結であるという。

\dotfill

\textbf{連結}

位相空間$X$において、
開集合$A,B\subset X$が
$X=A\cup B,\ A\cap B=\emptyset$であるなら
$A=\emptyset$または
$B=\emptyset$である時、
$X$は連結であるという。

\hrulefill

\begin{enumerate}
 \item \textbf{--- 弧状連結の位相的性質 ---}

       弧状連結性は位相的性質であることを示せ。

       \dotfill

       位相空間$X,Y$を同相とし、$X$を弧状連結であるとする。
       このときの同相写像を$f:X\to Y$とする。

       任意の2点$y_1,y_2\in Y$に対して
       $f^{-1}(y_1),f^{-1}(y_2)\in X$が存在する。
       $X$は弧状連結空間であるから
       $f^{-1}(y_1),f^{-1}(y_2)$の間にはパスが存在する。
       $f$は連続写像であるので、
       $f^{-1}(y_1),f^{-1}(y_2)$を結ぶパスは
       $y_1,y_2$を結ぶパスに対応する。

       $X$が弧状連結であるから$Y$も弧状連結になるので、
       弧状連結製は位相的性質である。

       \hrulefill

 \item \textbf{--- 弧状連結なら連結 ---}

       弧状連結ならば連結であることを示せ。

       \dotfill

       位相空間$X$を弧状連結空間とする。
       点$x_0\in X$を一つ定める。
       任意の点$p\in X$について
       パス$\sigma:I=[0,1]\to X$が存在し
       $x_0=\sigma(0),\ p=\sigma(1)$である。
       つまり、$x_0$と$p$は$X$の連結な部分集合に含まれる。

       任意の点$p\in X$について言えるので、
       $X$のすべての点は$x_0\in X$と同じ連結な部分集合に含まれる。

       よって、$X$は連結な空間である。

       \hrulefill

\newpage

 \item \textbf{--- 2点集合 ---}

       2点集合$\{a,b\}$上の位相$\{ \emptyset , \{a\}, \{a,b\} \}$
       は弧状連結であることを示せ。

       \dotfill

       次の写像$f$が連続となるように定義できれば
       任意の$\{a,b\}$の点が弧状連結であるので、
       $\{a,b\}$が弧状連結であるといえる。
       \begin{equation}
        f: [0,1]\to \{a,b\},\quad
         f(0)=a,f(1)=b
       \end{equation}

       $f^{-1}(\emptyset)=\emptyset$、
       $f^{-1}(\{a,b\})=[0,1]$であるので、
       $f^{-1}(\{a\})$が開集合となるように$f$が定義できればよい。

       そこで、$f$を次のように定義する。
       \begin{equation}
        f(x)=
        \begin{cases}
         a & (0\leq x< \frac{1}{2})\\
         b & (\frac{1}{2} \leq x\leq 1)
        \end{cases}
       \end{equation}

       これにより
       $f^{-1}(\{a\})=\left[0,\frac{1}{2}\right)$
       となるので開集合となり、
       $f$は連続写像として定義できる。
       つまり、$f$により$a,b$は弧状連結である。

       $\{a,b\}$の任意の点が弧状連結であるため、
       $\{a,b\}$は弧状連結である。
       


\end{enumerate}

\hrulefill

\end{document}
