\documentclass[12pt,b5paper]{ltjsarticle}

%\usepackage[margin=15truemm, top=5truemm, bottom=5truemm]{geometry}
\usepackage[margin=15truemm]{geometry}

\usepackage{amsmath,amssymb}
%\pagestyle{headings}
\pagestyle{empty}

%\usepackage{listings,url}
%\renewcommand{\theenumi}{(\arabic{enumi})}

\usepackage{graphicx}

\usepackage{tikz}
\usetikzlibrary {arrows.meta}
\usepackage{wrapfig}	% required for `\wrapfigure' (yatex added)
\usepackage{bm}	% required for `\bm' (yatex added)
\usepackage{luatexja-ruby}	% required for `\ruby'
%% 像Im を定義
%\newcommand{\Img}{\mathop{\mathrm{Im}}\nolimits}

\begin{document}

東京農工

工学部
3年次編入試験

\hrulefill
2021年度
\hrulefill

\begin{enumerate}
 \item 2変数関数$f(x,y)$について次の問いに答えよ。
       \begin{equation}
        f(x,y) = \frac{2}{3}x^3 + 3x^2y - y^3 + 11x^2 +6xy + 3y^2 - 12x
       \end{equation}
       \begin{enumerate}
        \item $f_x(x,y) = f_y(x,y) = 0$を満たす点$(x,y)$を全て求めよ。
              \label{135442_20May22}
        \item $z=f(x,y)$の極値を求めよ。
              \label{110917_21May22}
       \end{enumerate}
 \item 次の2つの曲面で囲まれた図形$V$の体積を求めよ。
       \begin{gather}
        x^2+2y+z=0\\
        y^2-z-3=0
       \end{gather}
 \item $n$は自然数とする。
       次の行列$A$に対し、$A^n$を求めよ。
       \begin{equation}
        A=\begin{pmatrix} -2 & -1 \\ 4 & 3 \end{pmatrix}
       \end{equation}
 \item 次の微分方程式の解$y=y(x)$で、
       $y(0)=0, \frac{\mathrm{d}y}{\mathrm{d}x}(0)=1$
       を満たすものを求めよ。
       \begin{equation}
        \frac{\mathrm{d}^2y}{\mathrm{d}x^2}
         +2 \frac{\mathrm{d}y}{\mathrm{d}x}
         + 5y =5\sin{x}
       \end{equation}
\end{enumerate}

\dotfill

\textbf{偏微分の記号}
\begin{align}
 f_x(x,y) =& \frac{\partial}{\partial x}f(x,y)
  & f_y(x,y) =& \frac{\partial}{\partial y}f(x,y)\\
 f_{xy}(x,y) =& \frac{\partial}{\partial y}\frac{\partial}{\partial x}f(x,y)
  & f_{yx}(x,y) =& \frac{\partial}{\partial x}\frac{\partial}{\partial y}f(x,y)
\end{align}


\textbf{ヘッセ行列}

2変数関数$f(x,y)$のヘッセ行列$H(f)$は次のように定義される。
\begin{equation}
 H(f)=
  \begin{pmatrix}
   \frac{\partial^2 f}{\partial x\partial x} & \frac{\partial^2 f}{\partial x\partial y}\\
   \frac{\partial^2 f}{\partial y\partial x} & \frac{\partial^2 f}{\partial y\partial y}
  \end{pmatrix}
  =
  \begin{pmatrix}
   f_{xx} & f_{yx}\\
   f_{xy} & f_{yy}
  \end{pmatrix}
\end{equation}

2変数の偏微分を行う順序で$H(f)$の成分の場所が決まる。

ヘッセ行列の行列式を\ruby{Hessian}{ヘッシアン}という。
\begin{equation}
 \lvert H(f) \rvert = f_{xx}f_{yy}-(f_{xy})^2
\end{equation}


\textbf{極値}

関数$f(x,y)$の極値(極大値、極小値)の求め方

\begin{enumerate}
 \item $f_x(x,y)=f_y(x,y)=0$を満たす点$(a,b)$を求める。
 \item 上の条件を満たす点$(a,b)$について$\lvert H(f(a,b)) \rvert$を計算する。
       \begin{itemize}
        \item $\lvert H(f(a,b)) \rvert <0$ の時、$f(a,b)$は極値ではない。
        \item $\lvert H(f(a,b)) \rvert >0$ の時、$f(a,b)$は極値である。
       \end{itemize}
 \item $f_{xx}(a,b)$を計算する。
       \begin{itemize}
        \item $f_{xx}(a,b) >0$であれば、$f(a,b)$は極小値である。
        \item $f_{xx}(a,b) <0$であれば、$f(a,b)$は極大値である。
       \end{itemize}
\end{enumerate}
上記条件のどれかが0になる場合($\lvert H(f(a,b)) \rvert =0$や$f_{xx}(a,b) =0$)は
この方法では判定できない。


\textbf{ヤコビ行列}

重積分の座標変換に次のヤコビ行列の行列式を用いる。
ヤコビ行列式の事を\ruby{Jacobian}{ヤコビアン}という。
\begin{equation}
 \frac{\partial (x,y)}{\partial (r,\theta)}
  =
  \begin{pmatrix}
   \frac{\partial x}{\partial r} & \frac{\partial x}{\partial \theta}\\
   \frac{\partial y}{\partial r} & \frac{\partial y}{\partial \theta}
  \end{pmatrix}
\end{equation}

変数変換を$x=r\cos \theta,\ y=r\cos \theta$とするとヤコビ行列は次のようになる。
\begin{equation}
 \frac{\partial (x,y)}{\partial (r,\theta)}
  =
  \begin{pmatrix}
   \cos\theta & -r\sin\theta\\
   \sin\theta & r\cos\theta
  \end{pmatrix}
\end{equation}
ヤコビ行列式は以下の通りである。
\begin{equation}
 \left\lvert \frac{\partial (x,y)}{\partial (r,\theta)} \right\rvert
  =
  \begin{vmatrix}
   \cos\theta & -r\sin\theta\\
   \sin\theta & r\cos\theta
  \end{vmatrix}
  = r
\end{equation}


\textbf{微分方程式}

$2$階線形微分方程式の解法

\begin{equation}
 y^{\prime\prime}+ay^{\prime}+by=r(x)
  \quad (a,b:\text{定数},\ r(x):x\text{の関数})
  \label{smpl_diff_eq}
\end{equation}

まずはじめに右辺の関数がない
同次方程式について考える。
\begin{equation}
 y^{\prime\prime}+ay^{\prime}+by=0
\end{equation}
この方程式の解を$y_1=e^{\alpha x}$とすると上の式に当てはめて
\begin{gather}
 \alpha^2 e^{\alpha x} + a \alpha e^{\alpha x} + b e^{\alpha x} =0\\
 (\alpha^2 + a \alpha + b ) e^{\alpha x} =0
\end{gather}
が得られる。
$\alpha^2 + a \alpha + b=0$の時に同次方程式は成り立つので、
この$\alpha^2 + a \alpha + b=0$を\textbf{特性方程式}という。
特定方程式は2次式であるから解も2つ存在し、
これを$\alpha,\beta$とすると
同次方程式の解は$e^{\alpha x},e^{\beta x}$である。(基本解)
また、これらを用いてできる
$c_{\alpha}e^{\alpha x}+c_{\beta}e^{\beta x}$\ ($c_{\alpha},c_{\beta}$は定数)
も同次方程式の解となる。(一般解)

特性方程式が虚数解を持つ場合はオイラーの公式を用いて解を計算する。
\begin{equation}
 \text{(\ruby{Eular}{オイラー}の公式)} \qquad e^{i\theta} = \cos\theta +i\sin\theta
\end{equation}

式(\ref{smpl_diff_eq})の特殊解を一つ求める。
この式の右端 $\cdots+by=r(x)$ から $y$を予測し
それを式(\ref{smpl_diff_eq})に当てはめて成立するように係数などを求める。

この特殊解が求まれば
同次方程式の一般解と加えたものが
式(\ref{smpl_diff_eq})の一般解となる。







\newpage

\dotfill

\begin{enumerate}
 \item
\begin{enumerate}
 \item

%
% sagemath 用コード
% https://sagecell.sagemath.org/
%
%x , y=var('x,y')
%
%f=2*x^3/3+3*x^2*y-y^3+11*x^2+6*x*y+3*y^2-12*x
%
%factor(f.diff(y))
%print (solve([f.diff(x), f.diff(y)],x,y))
%
%a=0
%b=2
%print(f(x=a,y=b))
%A=matrix([[f.diff(x).diff(x)(x=a,y=b),f.diff(x).diff(y)(x=a,y=b)],
%          [f.diff(y).diff(x)(x=a,y=b),f.diff(y).diff(y)(x=a,y=b)]])
%print(A)
%det(A)




まず、偏微分した式を求める。
\begin{align}
 f_x(x,y) =& 2x^2+6xy + 22x + 6y -12\\
 f_y(x,y) =& 3x^2 -3y^2 + 6x + 6y\\
 =& 3(x+y)(x-y+2)
\end{align}
$f_x(x,y) = f_y(x,y) = 0$を満たす点は次の連立方程式を解くことで求まる。
\begin{equation}
 \begin{cases}
  2x^2+6xy + 22x + 6y -12 = 0\\
  3(x+y)(x-y+2) = 0
 \end{cases}
\end{equation}
2つ目の式より$x+y=0$の場合と$x-y+2=0$の場合に分けて考える。

$x+y=0$の場合

$y=-x$を$2x^2+6xy + 22x + 6y -12 = 0$に代入し解くと、
$(x,y)=(1,-1),(3,-3)$が得られる。

$x-y+2=0$の場合

同様に$y=x+2$を代入すると
$(x,y)=(-5,-3),(0,2)$が得られる。

よって、求めるべき点は次の4点である。
\begin{equation}
 (x,y)=(1,-1),(3,-3),(-5,-3),(0,2)
\end{equation}


 \item
$z=f(x,y)$の極値を求めよ。

上で求めた点のヘッセ行列を求め極値の判定を行う。

\begin{equation}
 H(f) =
  \begin{pmatrix}
   f_{xx} & f_{yx}\\
   f_{xy} & f_{yy}
  \end{pmatrix}
  =
  \begin{pmatrix}
   4x+6y+22 & 6x+6\\
   6x+6 & -6y+6
  \end{pmatrix}
\end{equation}

\begin{itemize}
 \item
点$(1,-1)$の場合

\begin{equation}
 \lvert H(f(1,-1)) \rvert =
  \begin{vmatrix}
   20 & 12\\
   12 & 12
  \end{vmatrix}
  = 96 >0
\end{equation}
であり、
$f_{xx}(1,-1)=20>0$より
$f(1,-1)=-\frac{16}{3}$は
極小値である。
 \item
点$(3,-3)$の場合

\begin{equation}
 \lvert H(f(3,-3)) \rvert =
  \begin{vmatrix}
   16 & 24\\
   24 & 24
  \end{vmatrix}
  = -192 <0
\end{equation}
であるので$f(3,-3)=0$は極値ではない。

 \item
点$(-5,-3)$の場合

\begin{equation}
 \lvert H(f(-5,-3)) \rvert =
  \begin{vmatrix}
   -16 & -24\\
   -24 & 24
  \end{vmatrix}
  = -960 <0
\end{equation}
であるので$f(-5,-3)=\frac{512}{3}$は極値ではない。

 \item
点$(0,2)$の場合

\begin{equation}
 \lvert H(f(0,2)) \rvert =
  \begin{vmatrix}
   34 & 6\\
   6 & -6
  \end{vmatrix}
  = -240 <0
\end{equation}
であるので$f(0,2)=4$は極値ではない。
\end{itemize}

以上により
$(x,y)=(1,-1)$の時、極小値$z=-\frac{16}{3}$となる。
\end{enumerate}


 \item
      \begin{equation}
       x^2+2y+z=0, \quad y^2-z-3=0
      \end{equation}
      この式は次のように$z$の式に変形する。
      \begin{equation}
       z=-x^2-2y, \quad z=y^2-3
      \end{equation}

      2つの曲面の共有点について調べる。
      \begin{gather}
       -x^2-2y = y^2-3\\
       x^2 + (y +1)^2 = 4
      \end{gather}
      共有点は中心$(0,-1)$、半径$2$の円周上にある。

      求めるべき体積は次の領域$D$上にある。
      \begin{equation}
       D= \{ (x,y)\in\mathbb{R}^2 \mid x^2 + (y +1)^2 \leq 4 \}
      \end{equation}

      $D$上では曲面$z=-x^2-2y$の方が曲面$z=y^2-3$より上にある。
      実際に差を求めると次のように$D$上では0以上になる。
      \begin{equation}
       (-x^2-2y)-(y^2-3) = -x^2 -(y+1)^2 +4 \geq0
      \end{equation}

      この為、求めるべき体積は次の積分を計算することで得られる。
      \begin{align}
       & \iint_{D} ((-x^2-2y)-(y^2-3))\mathrm{d}x\mathrm{d}y\\
       =& \iint_{D} (-x^2-(y+1)^2+4)\mathrm{d}x\mathrm{d}y
      \end{align}

      変数変換$x=r\cos\theta, \ y=r\sin\theta-1$とすると
      ヤコビ行列は次のようになる。
      \begin{equation}
       \frac{\partial (x,y)}{\partial (r,\theta)} =
        \begin{pmatrix}
         \cos\theta & -r\sin\theta\\
         \sin\theta & r\cos\theta
        \end{pmatrix}
      \end{equation}

      積分する領域$D$も次のように書き換えられる。
      \begin{align}
       D=& \{ (x,y)\in\mathbb{R}^2 \mid x^2 + (y +1)^2 \leq 4 \}\\
       =& \{ (r,\theta) \mid 0 \leq r \leq 2 ,\ 0\leq \theta \leq 2\pi \}
      \end{align}

      この為、積分は次のように計算される。
      \begin{align}
       & \iint_{D} (-x^2-(y+1)^2+4)\mathrm{d}x\mathrm{d}y\\
       =& \int_{0}^{2\pi}\!\!\!\!\int_{0}^{2}
       (-(r\cos\theta)^2-(r\sin\theta)^2+4)
          \begin{vmatrix}
           \cos\theta & -r\sin\theta\\
           \sin\theta & r\cos\theta
          \end{vmatrix}
       \mathrm{d}r\mathrm{d}\theta\\
       =& \int_{0}^{2\pi}\!\!\!\!\int_{0}^{2}(-r^2+4)r\mathrm{d}r\mathrm{d}\theta\\
       =& \int_{0}^{2\pi}\left[-\frac{r^4}{4}+2r^2\right]_{r=0}^{r=2}\mathrm{d}\theta\\
       =& \int_{0}^{2\pi}4\mathrm{d}\theta\\
       =& [4\theta]_{\theta=0}^{\theta=2\pi}=8\pi
      \end{align}

 \item
      $A^n$を求める。
       \begin{equation}
        A=\begin{pmatrix} -2 & -1 \\ 4 & 3 \end{pmatrix}
       \end{equation}

      とりあえず$A^2$を計算してみる。
       \begin{equation}
        A^2=\begin{pmatrix} 0 & -1 \\ 4 & 5 \end{pmatrix}
       \end{equation}
      あまり0が増えないので他の方法で求める。

      方針は対角化行列$P^{-1}AP$を求めこれを$n$乗する。

      まず、固有値と固有ベクトルを求める。
      \begin{equation}
       \begin{vmatrix}-2-\lambda & -1\\ 4 & 3-\lambda\end{vmatrix}=0
      \end{equation}
      これを解くと固有値$\lambda=-1,2$が求まる。

      $\lambda=-1$の時
      \begin{equation}
       \begin{pmatrix}-2-(-1) & -1\\ 4 & 3-(-1)\end{pmatrix}
       \begin{pmatrix}x\\y\end{pmatrix}
       =0
      \end{equation}
      これより固有ベクトル$\begin{pmatrix}1\\-1\end{pmatrix}$が得られる。

      $\lambda=2$の時
      \begin{equation}
       \begin{pmatrix}-2-2 & -1\\ 4 & 3-2\end{pmatrix}
       \begin{pmatrix}x\\y\end{pmatrix}
       =0
      \end{equation}
      これより固有ベクトル$\begin{pmatrix}1\\-4\end{pmatrix}$が得られる。

      よって行列$A$を対角化すると次のようになる。
      \begin{equation}
       P^{-1}AP=
        \begin{pmatrix}-1 & 0\\ 0 & 2\end{pmatrix}
        \qquad
         ただし\
         P=
         \begin{pmatrix}1 & 1\\ -1 & -4\end{pmatrix}
      \end{equation}

      対角化行列を$n$乗する。
      \begin{gather}
       (P^{-1}AP)^n=\begin{pmatrix}-1 & 0\\ 0 & 2\end{pmatrix}^n\\
       P^{-1}A^nP=\begin{pmatrix}(-1)^n & 0\\ 0 & 2^n\end{pmatrix}\\
       A^n=P\begin{pmatrix}(-1)^n & 0\\ 0 & 2^n\end{pmatrix}P^{-1}
      \end{gather}

      ここで、
      \begin{equation}
       P^{-1}=\frac{1}{3}\begin{pmatrix}4 & 1\\ -1 & -1\end{pmatrix}
      \end{equation}
      であるので、
      \begin{align}
       A^n =& P\begin{pmatrix}(-1)^n & 0\\ 0 & 2^n\end{pmatrix}P^{-1}\\
       =& \frac{1}{3}
       \begin{pmatrix}1 & 1\\ -1 & -4\end{pmatrix}
       \begin{pmatrix}(-1)^n & 0\\ 0 & 2^n\end{pmatrix}
       \begin{pmatrix}4 & 1\\ -1 & -1\end{pmatrix}\\
       =& \frac{1}{3}
       \begin{pmatrix}(-1)^n & 2^n\\ (-1)^{n+1} & -2^{n+2}\end{pmatrix}
       \begin{pmatrix}4 & 1\\ -1 & -1\end{pmatrix}\\
       =& \frac{1}{3}
       \begin{pmatrix}
       (-1)^n\cdot 4-2^n & (-1)^n-2^n\\
       (-1)^{n+1}\cdot 4+2^{n+2} & (-1)^{n+1}+2^{n+2}
       \end{pmatrix}
      \end{align}
      である。


 \item
      \begin{equation}
       \frac{\mathrm{d}^2y}{\mathrm{d}x^2}
        +2 \frac{\mathrm{d}y}{\mathrm{d}x}
        + 5y =5\sin{x}
        \label{prob_diff_eq}
      \end{equation}

      この方程式の特殊解を考える。

      $\cdots + 5y =5\sin{x}$とあり、
      微分を行うことから三角関数であると考え次のようにおく。
      \begin{equation}
       y=A\sin x + B\cos x
      \end{equation}
      ここから導関数を求める。
      \begin{equation}
       \frac{\mathrm{d}y}{\mathrm{d}x}
        = A\cos x -B\sin x
        , \qquad
        \frac{\mathrm{d}^2y}{\mathrm{d}x^2}
        = -A\sin x -B\cos x
      \end{equation}
      これを問題の式(\ref{prob_diff_eq})に当てはめる。
      \begin{gather}
       (-A\sin x -B\cos x) + 2(A\cos x -B\sin x)
       +5(A\sin x + B\cos x) = 5\sin x\\
       (-A-2B+5A-5)\sin x +(-B+2A+5B)\cos x =0
      \end{gather}
      $\sin x,\ \cos x$の係数が0になるように$A,\ B$を求める。
      \begin{equation}
       A= 1, \ B =-\frac{1}{2}
      \end{equation}

      これで特殊解は次のようになる。
      \begin{equation}
       y=\sin x -\frac{1}{2}\cos x
        \label{kai_toku}
      \end{equation}

      次に付随する同次方程式の一般解を求める。
      \begin{equation}
       \frac{\mathrm{d}^2y}{\mathrm{d}x^2}
        +2 \frac{\mathrm{d}y}{\mathrm{d}x}
        + 5y = 0
      \end{equation}
      特性方程式$\lambda^2+2\lambda+5=0$の解は
      $\lambda=-1\pm2i$であるので、
      基本解は次の2つ。
      \begin{equation}
       e^{(-1+2i)x}=e^{-x}(\cos 2x +i\sin 2x),
        \quad
        e^{(-1-2i)x}=e^{-x}(\cos 2x -i\sin 2x)
      \end{equation}

      ここから一般解は定数$c_{\alpha},\ c_{\beta}$を用いて次のようになる。
      \begin{equation}
       c_{\alpha}e^{(-1+2i)x} + c_{\beta}e^{(-1-2i)x}
        = (c_{\alpha}+c_{\beta})e^{-x}\cos 2x
        +(c_{\alpha}-c_{\beta})ie^{-x}\sin 2x
      \end{equation}

      定数部分を$C_1=c_{\alpha}+c_{\beta}, \ C_2=(c_{\alpha}-c_{\beta})i$とおくと
      同次方程式の一般解は次のようになる。
      \begin{equation}
       C_{1}e^{-x}\cos 2x + C_{2}e^{-x}\sin 2x
        \label{kai_ippan}
      \end{equation}


      これにより問題の方程式の解は特殊解(\ref{kai_toku})と
      同次式の一般解(\ref{kai_ippan})から
      \begin{equation}
       y=\sin x -\frac{1}{2}\cos x
        + C_{1}e^{-x}\cos 2x + C_{2}e^{-x}\sin 2x
      \end{equation}
      である。

      初期値
      $y(0)=0, \frac{\mathrm{d}y}{\mathrm{d}x}(0)=1$
      を満たすように$C_{1},\ C_{2}$を求める。

      $y(0)=0$より$C_{1}=\frac{1}{2}$が分かる。

      微分をするとこの式になるので、
      \begin{equation}
       y^{\prime}=\cos x + \frac{1}{2}\sin x
        + (-C_{1}+2C_{2})e^{-x}\cos 2x + (-2C_{1}-C_{2})e^{-x}\sin 2x
      \end{equation}
      これより$C_{2}=\frac{1}{4}$が得られる。

      よって求めるべき方程式の解は
      \begin{equation}
       y=\sin x -\frac{1}{2}\cos x
        + \frac{1}{2}e^{-x}\cos 2x + \frac{1}{4}e^{-x}\sin 2x
      \end{equation}


\end{enumerate}


\newpage
\hrulefill
2022年度
\hrulefill

\begin{enumerate}
 \item 2変数関数$f(x,y)$について次の問いに答えよ。
       \begin{equation}
        f(x,y) = 2x^3 + xy^2 + 9x^2 + y^2 - 2
       \end{equation}
       \begin{enumerate}
        \item $\displaystyle \frac{\partial f}{\partial x}(x,y) = \frac{\partial f}{\partial y}(x,y) = 0$
              を満たす点$(x,y)$を全て求めよ。
        \item $z=f(x,y)$の極値を求めよ。
       \end{enumerate}
 \item 以下の問いに答えよ。
       \begin{enumerate}
        \item 次の広義積分の値を求めよ。
              \begin{equation}
               \int_{0}^{\infty}\frac{x^2+x+1}{(x^2+1)^2}\mathrm{d}x
              \end{equation}
        \item 次の重積分の値を求めよ。
              \begin{equation}
               \iint_{D}(x-y)e^{x+y}\mathrm{d}x\mathrm{d}y,
                \
                D=\{ (x,y) \mid 0 \leq x-y \leq 2,\ 0\leq x+y \leq 3 \}
              \end{equation}
       \end{enumerate}
 \item $t$は実数とする。
       次の3次正方行列$A$とベクトル$\bm{v}_1, \ \bm{v}_2$に対し、
       $A\bm{v}_1 = \bm{v}_2$が成り立つ時、以下の問いに答えよ。
       \begin{equation}
        A=\begin{pmatrix} 3 & 1 & 1 \\ 0 & 2 & 1 \\ -1 & t & 0 \end{pmatrix},
        \quad
         \bm{v}_1=\begin{pmatrix} 2 \\ -5 \\ 1 \end{pmatrix},
         \quad
         \bm{v}_2=\begin{pmatrix} 2 \\ -9 \\ 3 \end{pmatrix}
       \end{equation}
       \begin{enumerate}
        \item $t$の値を求めよ。
        \item $A$の逆行列を求めよ。
        \item $A$の固有値のうち最小のものを$p$とする。
              $p$に属する固有ベクトルで
              $\begin{pmatrix} x \\ y \\ 1 \end{pmatrix}$
              の形のものを求めよ。
       \end{enumerate}
 \item 次の微分方程式の解$y=y(x)$で、
       $y(0)=0,\ \frac{\mathrm{d}y}{\mathrm{d}x}(0)=0$
       を満たすものを求めよ。
       \begin{equation}
        \frac{\mathrm{d}^2y}{\mathrm{d}x^2}
         -5 \frac{\mathrm{d}y}{\mathrm{d}x}
         + 6y =3x + e^{-x}
       \end{equation}
\end{enumerate}

\hrulefill
解答
\hrulefill

\begin{enumerate}
 \item

%
% SageMathCell
% https://sagecell.sagemath.org/
%
%x,y = var('x,y')
%f=2*x^3+x*y^2+9*x^2+y^2-2
%print(f.diff(x), "=", factor(f.diff(x)))
%print(f.diff(y), "=", factor(f.diff(y)))
%print (solve([f.diff(x), f.diff(y)],x,y))
%
%print("===ヘッセ行列===")
%A=matrix([[f.diff(x).diff(x),f.diff(x).diff(y)],
%          [f.diff(y).diff(x),f.diff(y).diff(y)]]) 
%print(A)
%
%a=-3
%b=0
%print("関数値", f(x=a,y=b)) 
%A=matrix([[f.diff(x).diff(x)(x=a,y=b),f.diff(x).diff(y)(x=a,y=b)],
%          [f.diff(y).diff(x)(x=a,y=b),f.diff(y).diff(y)(x=a,y=b)]]) 
%print("===ヘッセ行列===")
%print(A)
%print("ヘッセ行列式", det(A))
%

      \begin{enumerate}
       \item
            偏微分を行う。
            \begin{align}
             \frac{\partial f}{\partial x}
             =& 6x^2 + y^2 + 18x\\
             \frac{\partial f}{\partial y}
             =& 2xy + 2y = 2y(x+1)
            \end{align}
            この2つの式の連立方程式を解くと次の4つの解を得る。
            \begin{equation}
             (-3,0), \ (0,0), \ (-1,-2\sqrt{3}), \ (-1,2\sqrt{3})
            \end{equation}
       \item
            $z=f(x,y)$のヘッセ行列を求める。
            \begin{equation}
             H(f) =
              \begin{pmatrix}
               f_{xx} & f_{yx}\\
               f_{xy} & f_{yy}
              \end{pmatrix}
              =
              \begin{pmatrix}
               12x+18 & 2y\\
               2y & 2x+2
              \end{pmatrix}
            \end{equation}

            先程の4つの解を用いてヘッセ行列式を求める。
            \begin{itemize}
             \item $(x,y)=(-3,0)$

                   \begin{equation}
                    H(f(-3,0))
                     = \begin{vmatrix}-18&0\\0&-4\end{vmatrix}
                     = 72 >0
                   \end{equation}
                   $f_{xx}$の成分$-18$は負であるので、
                   $(-3,0)$では極大値となる。
             \item $(x,y)=(0,0)$

                   \begin{equation}
                    H(f(0,0))
                     = \begin{vmatrix}18&0\\0&2\end{vmatrix}
                     = 36 >0
                   \end{equation}
                   $f_{xx}$の成分$18$は正であるので、
                   $(0,0)$では極小値となる。
             \item $(x,y)=(-1,-2\sqrt{3})$

                   \begin{equation}
                    H(f(-1,-2\sqrt{3}))
                     = \begin{vmatrix}6&-4\sqrt{3}\\-4\sqrt{3}&0\end{vmatrix}
                     = -48 <0
                   \end{equation}
                   よって極値ではない。
             \item $(x,y)=(-1,2\sqrt{3})$

                   \begin{equation}
                    H(f(-1,2\sqrt{3}))
                     = \begin{vmatrix}6&4\sqrt{3}\\4\sqrt{3}&0\end{vmatrix}
                     = -48 <0
                   \end{equation}
                   よって極値ではない。
            \end{itemize}
            これより極値は次のよう求まる。
            \begin{align}
             \text{極大値}\ z&= 25 \quad (x,y)=(-3,0)\\
             \text{極小値}\ z&= -2 \quad (x,y)=(0,0)
            \end{align}
      \end{enumerate}
 \item
      \begin{enumerate}
       \item
%            広義積分は次のように有限区間の積分の極限を指す。
%            \begin{equation}
%             \int_{0}^{\infty}\frac{x^2+x+1}{(x^2+1)^2}\mathrm{d}x
%              = \lim_{s\rightarrow\infty}
%              \int_{0}^{s}\frac{x^2+x+1}{(x^2+1)^2}\mathrm{d}x
%            \end{equation}

%            まず、右辺の定積分を計算する。

            $x=\tan\theta$と置くことで積分区間を$(0,\infty)$から$(0,\frac{\pi}{2})$へと変える。
            また、$\frac{\mathrm{d}x}{\mathrm{d}\theta}=\frac{1}{\cos^2\theta}$より
            $\mathrm{d}x=\frac{1}{\cos^2\theta}\mathrm{d}\theta$と置き換える。
            \begin{align}
             \int_{0}^{\infty}\frac{x^2+x+1}{(x^2+1)^2}\mathrm{d}x
             =& \int_{0}^{\frac{\pi}{2}}
             \frac{\tan^2\theta+\tan\theta+1}{(\tan^2\theta+1)^2}
             \frac{1}{\cos^2\theta}\mathrm{d}\theta \label{tr_tan}\\
             =& \int_{0}^{\frac{\pi}{2}}
             (\sin^2\theta +\sin\theta\cos\theta+\cos^2\theta)
             \mathrm{d}\theta\\
             =& \int_{0}^{\frac{\pi}{2}}
             \left(\cos2\theta+\frac{1}{2}\sin2\theta\right)
             \mathrm{d}\theta\\
             =& \left[
             \frac{1}{2}\sin 2\theta - \frac{1}{4}\cos 2\theta
             \right]_{0}^{\frac{\pi}{2}}\\
             =& \frac{1}{2}
            \end{align}
       \item
            $X=x-y,\ Y=x+y$と変数変換する。
            この時のヤコビ行列式は次のようになる。
            \begin{equation}
             \begin{vmatrix}
              \frac{\partial X}{\partial x} & \frac{\partial X}{\partial y}\\
              \frac{\partial Y}{\partial x} & \frac{\partial Y}{\partial y}
             \end{vmatrix}
             =\begin{vmatrix}1&-1\\1&1\end{vmatrix}
             =2
            \end{equation}
            また、積分する領域$D$から積分する区間は$0\leq X\leq 2,\ 0\leq Y\leq 3$となる。
            \begin{align}
             \iint_{D}(x-y)e^{x+y}\mathrm{d}x\mathrm{d}y
             =& \int_{0}^{3} \!\!\! \int_{0}^{2}Xe^{Y} \cdot2\mathrm{d}X\mathrm{d}Y\\
             =& \int_{0}^{3} \left[ X^2e^{Y} \right]_{X=0}^{X=2}\mathrm{d}Y\\
             =& \int_{0}^{3} 4e^{Y} \mathrm{d}Y\\
             =& \left[ 4e^{Y} \right]_{0}^{3}\\
             =& 4(e^3-1)
            \end{align}
      \end{enumerate}
 \item
      \begin{enumerate}
       \item
            $t$に関わる部分だけ抜き出す。
            \begin{equation}
             \begin{pmatrix}-1&t&0\end{pmatrix}
             \begin{pmatrix}2\\-5\\1\end{pmatrix}
             =3
            \end{equation}
            これを解いて$t=-1$。
       \item
            逆行列は余因子行列を用いる方法と
            掃き出し法がある。
            \begin{align}
            (A \ E) =& \begin{pmatrix}
                       3&1&1& & 1&0&0\\
                       0& 2& 1& & 0& 1& 0\\
                       -1& -1& 0& & 0& 0& 1
                      \end{pmatrix}\\
             \stackrel{\text{1行目を3で割る}}{\longrightarrow}&
             \begin{pmatrix}
              1 & \frac{1}{3} & \frac{1}{3} & & \frac{1}{3} & 0 & 0\\
              0& 2& 1& & 0& 1& 0\\
              -1& -1& 0& & 0& 0& 1
             \end{pmatrix}\\
             \stackrel{\text{1行目を3行目に足す}}{\longrightarrow}&
             \begin{pmatrix}
              1 & \frac{1}{3} & \frac{1}{3} & & \frac{1}{3} & 0 & 0\\
              0& 2& 1& & 0& 1& 0\\
              0& -\frac{2}{3}& \frac{1}{3}& & \frac{1}{3}& 0& 1
             \end{pmatrix}\\
             \stackrel{\text{2行目を2で割る}}{\longrightarrow}&
             \begin{pmatrix}
              1 & \frac{1}{3} & \frac{1}{3} & & \frac{1}{3} & 0 & 0\\
              0& 1 & \frac{1}{2} & & 0& \frac{1}{2} & 0\\
              0& -\frac{2}{3}& \frac{1}{3}& & \frac{1}{3}& 0& 1
             \end{pmatrix}\\
             \stackrel{\text{2行目に1/3掛けて1行目から引く}}{\longrightarrow}&
             \begin{pmatrix}
              1 & 0 & \frac{1}{6} & & \frac{1}{3} & - \frac{1}{6} & 0\\
              0& 1 & \frac{1}{2} & & 0& \frac{1}{2} & 0\\
              0& -\frac{2}{3}& \frac{1}{3}& & \frac{1}{3}& 0& 1
             \end{pmatrix}\\
             \stackrel{\text{2行目に2/3掛けて3行目に足す}}{\longrightarrow}&
             \begin{pmatrix}
              1 & 0 & \frac{1}{6} & & \frac{1}{3} & - \frac{1}{6} & 0\\
              0& 1 & \frac{1}{2} & & 0& \frac{1}{2} & 0\\
              0& 0 & \frac{2}{3}& & \frac{1}{3} & \frac{1}{3}& 1
             \end{pmatrix}\\
             \rightarrow \text{(略)} \rightarrow &
             \begin{pmatrix}
              1&0&0& & \frac{1}{4} & - \frac{1}{4} & - \frac{1}{4}\\
              0&1&0& & - \frac{1}{4}& \frac{1}{4} & - \frac{3}{4}\\
              0&0&1& & \frac{1}{2} & \frac{1}{2}& \frac{3}{2}
             \end{pmatrix}
            \end{align}

            これにより逆行列$A^{-1}$は次のようになる。
            \begin{equation}
             A^{-1} =
             \begin{pmatrix}
              \frac{1}{4} & - \frac{1}{4} & - \frac{1}{4}\\
              - \frac{1}{4}& \frac{1}{4} & - \frac{3}{4}\\
              \frac{1}{2} & \frac{1}{2}& \frac{3}{2}
             \end{pmatrix}
             = \frac{1}{4}
             \begin{pmatrix}
              1 & -1 & -1\\
              -1 & 1 & -3\\
              2 & 2 & 6
             \end{pmatrix}
            \end{equation}
       \item
            固有値を$\lambda$とする。
            固有方程式$\lvert A - \lambda E \rvert =0$を計算する。
            \begin{align}
             \begin{vmatrix}
              3-\lambda & 1 & 1\\
              0 & 2-\lambda & 1\\
              -1 & -1 & -\lambda
             \end{vmatrix}
             =&(3-\lambda)
             \begin{vmatrix}
              2-\lambda & 1\\
              -1 & -\lambda
             \end{vmatrix}
             + (-1)
             \begin{vmatrix}
              1 & 1\\
              2-\lambda & 1
             \end{vmatrix}\\
             =& (3-\lambda)(-\lambda(2-\lambda)+1)+(-1)(1-(2-\lambda))\\
             =& -(\lambda - 1)(\lambda - 2)^2 =0
            \end{align}
            これにより固有値は$1,2$である。


            固有値が1の固有ベクトルを求める。
            \begin{equation}
             \begin{pmatrix} 3-1 & 1 & 1\\ 0 & 2-1 & 1\\ -1 & -1 & -1 \end{pmatrix}
             \begin{pmatrix}x_1\\x_2\\x_3\end{pmatrix}
              =\begin{pmatrix}0\\0\\0\end{pmatrix}
            \end{equation}
            この式を解くと次の結果が得られる。
            \begin{equation}
             x_1=0,\ x_2 + x_3 =0
            \end{equation}
            問題は$x_3 = 1$となる固有ベクトルを求めるので、
            $x_2=-1,\ x_3=1$である。
            よって、求めるべき固有ベクトルは
            \begin{equation}
             \begin{pmatrix}
              0\\-1\\1
             \end{pmatrix}
            \end{equation}

      \end{enumerate}
 \item

%
% sagemath
%
%x = var('x')            # 変数 t を定義
%y = function('y')(x)    # x を t の関数とする
%DE = diff(diff(y, x),x)-5*diff(y, x) + 6*y - 3*x-exp(-x)
%desolve(DE, [y,x])


      \begin{equation}
       \frac{\mathrm{d}^2y}{\mathrm{d}x^2}
        -5 \frac{\mathrm{d}y}{\mathrm{d}x}
        + 6y =3x + e^{-x}
      \end{equation}

      この方程式の特殊解を予測し次のように置く。
      \begin{equation}
       y=Ax+B+Ce^{-x} \quad (A,B,C:\text{定数})
      \end{equation}

      微分すると次のような式が得られる。
      \begin{equation}
       y^{\prime} = A-Ce^{-x}, \quad
        y^{\prime\prime} = Ce^{-x}
      \end{equation}
      この式を問題の式に代入し、定数$A,B,C$を求める。

      \begin{gather}
       Ce^{-x} -5(A-Ce^{-x}) +6(Ax+B+Ce^{-x}) = 3x-e^{-x}\\
       6Ax + (-5A+6B) +(C+5C+6C)e^{-x} = 3x-e^{-x}\\
       A=\frac{1}{2}, \ B= \frac{5}{12}, \ C=\frac{1}{12}
      \end{gather}

      よって特殊解は
      \begin{equation}
       y=\frac{1}{2}x + \frac{5}{12} + \frac{1}{12}e^{-x}
        \label{sol_sp}
      \end{equation}

      次に同次方程式の解を求める。
      \begin{equation}
       y^{\prime\prime} -5 y^{\prime} + 6y=0
      \end{equation}
      特性方程式$\lambda^2 -5\lambda +6=0$を解くと
      $\lambda=2,3$となるので
      同次方程式の一般解は$c_1e^{2x}+c_2e^{3x}$となる。

      これと、式(\ref{sol_sp})の特殊解と用いて
      一般解は次のようになる。
      \begin{equation}
       y=c_1e^{2x}+c_2e^{3x} + \frac{1}{2}x + \frac{5}{12} + \frac{1}{12}e^{-x}
      \end{equation}

      これに初期値条件$y(0)=0,\ y^{\prime}(0)=0$を考える。
      導関数$y^{\prime}$をまず計算する。
      \begin{equation}
       y^{\prime}
        =2c_1e^{2x}+3c_2e^{3x} + \frac{1}{2} - \frac{1}{12}e^{-x}
      \end{equation}
      初期値を満たすように$c_1,c_2$を求める。
      \begin{align}
       y(0) =& c_1 + c_2 + \frac{5}{12} + \frac{1}{12} =0\\
       y^{\prime}(0) =& 2c_1 + 3c_2 + \frac{1}{2} - \frac{1}{12} =0\\
       & c_1=-\frac{13}{12}, \ c_2=\frac{7}{12}
      \end{align}

      これにより方程式の解は次夜のようになる。
      \begin{equation}
       y= -\frac{13}{12}e^{2x}+ \frac{7}{12}e^{3x} + \frac{1}{2}x + \frac{5}{12} + \frac{1}{12}e^{-x}
      \end{equation}

\end{enumerate}


\hrulefill


定数係数の2階線形微分方程式
の特殊解について

方程式を解く上で特殊解を一つ見つける必要がある。
特殊解を見つける方法はいくつかあるが、
予測で大まかに決めるのが早い。

\begin{equation}
 y^{\prime\prime} + a y^{\prime} + by = r(x)
\end{equation}
この時の$r(x)$の式によって予測される式が変わる。

\begin{itemize}
 \item $r(x)$が多項式の場合

       $r(x)$と次数が同じ多項式
       \begin{equation}
        c_nx^n+c_{n-1}x^{n-1}+\cdots +c_1x + c_0 \quad(c_i:\text{定数})
       \end{equation}
 \item $r(x)=e^{kx}$の場合($k$は定数)

       微分することで変化があまりないので、定数倍した式
       \begin{equation}
        ce^{kx} \quad(c:\text{定数})
       \end{equation}

       付随する同次方程式の基本解と同じ形をしている場合は$x$をかけた式
       \begin{equation}
        cxe^{kx} \quad(c:\text{定数})
       \end{equation}
 \item $r(x)=\sin{kx}$又は$r(x)=\cos{kx}$の場合($k$は定数)

       微分により$\sin kx$と$\cos kx$が交互に現れるので
       これらを繋いだ式
       \begin{equation}
        c_1\sin kx + c_2\cos kx \quad(c_i:\text{定数})
       \end{equation}
 \item $r(x)$が上記3種類の組み合わせの場合

       特殊解もそれぞれを組み合わせた式にする。
\end{itemize}


\newpage

\textbf{計算の訂正}

式(\ref{tr_tan})以降の変形が間違っていました。

\begin{align}
 &  \frac{\tan^2\theta+\tan\theta+1}{(\tan^2\theta+1)^2}
 \frac{1}{\cos^2\theta} & (\text{式(\ref{tan_cos})で置き換え})\\
 =& \frac{ \frac{1}{\cos^2\theta} + \frac{\sin\theta}{\cos\theta} }{ \frac{1}{\cos^4\theta} }\times \frac{1}{\cos^2\theta}\\
 =& \frac{ \frac{1}{\cos^2\theta} + \frac{\sin\theta}{\cos\theta} }{ \frac{1}{\cos^2\theta} }
 &(\cos^2\theta\text{を分子分母に掛ける})\\
 =& 1 + \sin\theta\cos\theta & (\text{倍角の公式}\ \sin 2\theta)\\
 =& 1 + \frac{1}{2}\sin 2\theta
\end{align}

\begin{equation}
 \tan\theta = \frac{\sin\theta}{\cos\theta}, \quad \tan^2\theta +1 =\frac{1}{\cos^2\theta}
  \label{tan_cos}
\end{equation}

$\sin^2\theta + \cos^2\theta$ を $\cos 2\theta$として計算してました。
$\sin^2\theta + \cos^2\theta=1$で計算し直しています。

これにより
\begin{align}
 \int_{0}^{\frac{\pi}{2}}
 (1 +\sin\theta\cos\theta)
 \mathrm{d}\theta
 =&  \int_{0}^{\frac{\pi}{2}} \left(1 +\frac{1}{2}\sin2\theta\right) \mathrm{d}\theta\\
 =& \left[ \theta - \frac{1}{4}\cos 2\theta \right]_{\theta=0}^{\theta=\frac{\pi}{2}}\\
 =& \left( \frac{\pi}{2} - \frac{1}{4}\times(-1) \right) - \left( 0 - \frac{1}{4}\times(1) \right)\\
 =& \frac{\pi}{2}+\frac{1}{2}
\end{align}


\end{document}
