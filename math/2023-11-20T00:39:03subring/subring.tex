\documentclass[12pt,b5paper]{ltjsarticle}

%\usepackage[margin=15truemm, top=5truemm, bottom=5truemm]{geometry}
%\usepackage[margin=10truemm,left=15truemm]{geometry}
\usepackage[margin=10truemm]{geometry}

\usepackage{amsmath,amssymb}
%\pagestyle{headings}
\pagestyle{empty}

%\usepackage{listings,url}
%\renewcommand{\theenumi}{(\arabic{enumi})}

%\usepackage{graphicx}

%\usepackage{tikz}
%\usetikzlibrary {arrows.meta}
%\usepackage{wrapfig}
%\usepackage{bm}

% ルビを振る
%\usepackage{luatexja-ruby}	% required for `\ruby'

%% 核Ker 像Im Hom を定義
%\newcommand{\Img}{\mathop{\mathrm{Im}}\nolimits}
%\newcommand{\Ker}{\mathop{\mathrm{Ker}}\nolimits}
%\newcommand{\Hom}{\mathop{\mathrm{Hom}}\nolimits}

%\DeclareMathOperator{\Rot}{rot}
%\DeclareMathOperator{\Div}{div}
%\DeclareMathOperator{\Grad}{grad}
%\DeclareMathOperator{\arcsinh}{arcsinh}
%\DeclareMathOperator{\arccosh}{arccosh}
%\DeclareMathOperator{\arctanh}{arctanh}

%\usepackage{url}

%\usepackage{listings}
%
%\lstset{
%%プログラム言語(複数の言語に対応,C,C++も可)
%  language = Python,
%%  language = Lisp,
%%  language = C,
%  %背景色と透過度
%  %backgroundcolor={\color[gray]{.90}},
%  %枠外に行った時の自動改行
%  breaklines = true,
%  %自動改行後のインデント量(デフォルトでは20[pt])
%  breakindent = 10pt,
%  %標準の書体
%%  basicstyle = \ttfamily\scriptsize,
%  basicstyle = \ttfamily,
%  %コメントの書体
%%  commentstyle = {\itshape \color[cmyk]{1,0.4,1,0}},
%  %関数名等の色の設定
%  classoffset = 0,
%  %キーワード(int, ifなど)の書体
%%  keywordstyle = {\bfseries \color[cmyk]{0,1,0,0}},
%  %表示する文字の書体
%  %stringstyle = {\ttfamily \color[rgb]{0,0,1}},
%  %枠 "t"は上に線を記載, "T"は上に二重線を記載
%  %他オプション:leftline,topline,bottomline,lines,single,shadowbox
%  frame = TBrl,
%  %frameまでの間隔(行番号とプログラムの間)
%  framesep = 5pt,
%  %行番号の位置
%  numbers = left,
%  %行番号の間隔
%  stepnumber = 1,
%  %行番号の書体
%%  numberstyle = \tiny,
%  %タブの大きさ
%  tabsize = 4,
%  %キャプションの場所("tb"ならば上下両方に記載)
%  captionpos = t
%}

%\usepackage{cancel}
%\usepackage{bussproofs}
%\usepackage{proof}

\begin{document}

\hrulefill

$R$は単位元$e$を持つ環とする。

$S \subset R$ は $e\in S$である部分環、
$A$を$R$の部分集合とする。

$S$と$A$を含む$R$の部分環を$A_{\lambda}$とする。
つまり、$S \cup A \subset A_{\lambda} \subset R$である。

$\{ A_{\lambda} \}_{ \lambda \in \Lambda }$
を
部分環$A_{\lambda}$全てを集めたの集合族である。

\begin{equation}
 S[A] = \bigcap_{\lambda \in \Lambda} A_{\lambda}
\end{equation}
上記の様に$S[A]$を定める。
これにより$S[A]$は$R$の部分環となる。

$S[A]$は$S$上$A$で生成された部分環という。

特に、
$A=\{a_{1},\dots,a_{n}\}$である時、
$S[A] = S[a_{1},\dots,a_{n}]$とかく。



\hrulefill



$F$は体とする。

$K \subset F$ は
$F$の部分体、
$A$は$F$の部分集合とする。


$K$と$A$を含む$F$の部分体を$K_{\lambda}$とする。
つまり、$K \cup A \subset K_{\lambda} \subset F$である。

$\{ K_{\lambda} \}_{ \lambda \in \Lambda }$
を
部分体$K_{\lambda}$全てを集めたの集合族である。

\begin{equation}
 K(A) = \bigcap_{\lambda \in \Lambda} K_{\lambda}
\end{equation}
上記の様に$K(A)$を定める。
これにより$K(A)$は$F$の部分体となる。

$K(A)$は$K$上$A$で生成された部分体という。

特に、
$A=\{a_{1},\dots,a_{n}\}$である時、
$K(A) = K(a_{1},\dots,a_{n})$とかく。

\hrulefill




\begin{enumerate}
 \item
      $S[A]$は$S$と$A$を含むような最小の部分環である。

      \dotfill

      $\{ A_{\lambda} \}_{\lambda\in \Lambda}$は
      $S$と$A$を含むような部分環全体の集合族である。

      $A_{\alpha},A_{\beta} \in \{ A_{\lambda} \}_{\lambda \in \Lambda}$
      であれば次が満たされる。
      \begin{equation}
        A_{\alpha} \cap A_{\beta} \subset A_{\alpha}
         ,\quad
        A_{\alpha} \cap A_{\beta} \subset A_{\beta}
      \end{equation}

      $S$と$A$を含むような部分環$\tilde{A}$を取ってくる。

      この時、
      $\tilde{A} \in \{ A_{\lambda}\}_{\lambda \in \Lambda}$であり、
      $\bigcap_{\lambda \in \Lambda} A_{\lambda} \subset \tilde{A}$
      である。
      つまり、
      $S[A] \subset \tilde{A}$である。

      $\tilde{A}$は$S$と$A$を含むような部分環をどの様に選択しても
      必ず$S[A] \subset \tilde{A}$である。

      よって、
      $S[A]$は$S$と$A$を含む最小の部分環となる。


      \hrulefill

 \item
      $R$を単位元$e$を持つ可換環、
      $S$を$e$を含む$R$の部分環、
      $A=\{a_{1},\dots,a_{n}\}$を
      $R$の部分集合とする。

      \begin{equation}
       S[a_{1},\dots,a_{n}]
        =
        \left\{
         \sum_{有限和} c_{k_{1},\dots,k_{n}} a_{1}^{k_{1}} \dots a_{n}^{k_{n}}
         \;\middle|\;
         c_{k_{1},\dots,k_{n}} \in S ; k_{i}は 0 以上の整数
        \right\}
        \label{eq:saa}
      \end{equation}

      \dotfill


      等式
      \eqref{eq:saa}
      の右辺を$\tilde{S}$と書く。

      ${}^{\forall}a \in \tilde{S}$とする。

      $\tilde{S}$の定義より
      $a=\sum_{有限和} c_{k_{1},\dots,k_{n}} a_{1}^{k_{1}} \dots a_{n}^{k_{n}}$
      である。

      $c_{k_{1},\dots,k_{n}} \in S,\; A=\{a_{1}, \dots, a_{n}\}$であるので、
      $a$は$S$と$A$を含む$R$の部分環に含まれる。

      よって、$\tilde{S} \subset S[A]$である。

      また、
      $\tilde{S}$は$S$と$A$を含む為、
      $S$と$A$を含む最小の部分環$S[A]$を含む。
      つまり、$S[A] \subset \tilde{S}$である。

      以上により
      $S[A] = \tilde{S}$である。


      \hrulefill


 \item
      $K(A)$は$K$と$A$を含むような最小の部分体である。

      \dotfill

      $F$の部分体のうち、
      $K$と$A$を含むような部分体$K_{\lambda}$全体の集合を
      $\{K_{\lambda}\}_{\lambda \in \Lambda}$とする。

      ${}^{\forall}L \in \{K_{\lambda}\}_{\lambda \in \Lambda}$に対して
      $K(A) \subset L$を示せば、
      $K(A)$は最小の部分体であることといえる。

      $K$と$A$を含む部分環を$\tilde{K}$とする。

      この時、
      $\tilde{K}\in\{K_{\lambda}\}_{\lambda \in \Lambda}$であり、
      $\bigcap_{\lambda\in\Lambda}K_{\lambda} \subset \tilde{K}$である。
      つまり、
      $K(A) \subset \tilde{K}$である。

      $\tilde{K}$を$K$と$A$が含まれるようにどの様に取ってきても
      必ず$K(A) \subset \tilde{K}$であるので、
      $K(A)$は最小の部分体である。

      \hrulefill


 \item
      $F$を体とし、
      $K$を$F$の部分体とする。
      $A=\{a_{1},\dots,a_{n}\}$を$F$の部分集合とする。
      この時、次の式が成り立つ。
      \begin{equation}
       K(a_{1},\dots,a_{n})=
        \{ab^{-1} \mid a,b \in K[a_{1},\dots,a_{n}],\; b\ne 0 \}
        \label{eq:kaa}
      \end{equation}

      \dotfill

      等式\eqref{eq:kaa}の右辺を$\tilde{K}$とおく。

      $K(a_{1},\dots,a_{n}) \subset \tilde{K}$と
      $K(a_{1},\dots,a_{n}) \supset \tilde{K}$を示せれば
      $K(a_{1},\dots,a_{n}) = \tilde{K}$が言える。

      $K(a_{1},\dots,a_{n})$は$K,A$を含む最小の部分体である。
      $\tilde{K}$は
      $K \subset \tilde{K}$、$A \subset \tilde{K}$より
      上の定理から$K(a_{1},\dots,a_{n}) \subset \tilde{K}$である。

      $K[a_{1},\dots,a_{n}]$は体$K$に$a_{1},\dots,a_{n}$を付け加えた環である。


      この時、
      $a = \sum c_{k_{1},\dots,k_{n}} a_{1}^{k_{1}}\dots a_{n}^{k_{n}}$
      と表せる。

      $K[a_{1},\dots,a_{n}]$は$K$と$\{a_{1},\dots,a_{n}\}$を含む
      最小の部分環である。
      このため、
      $K[a_{1},\dots,a_{n}] \subset K(a_{1},\dots,a_{n})$
      である。

      $K[a_{1},\dots,a_{n}]$の
      逆元だけを集めた集合$K^{-1}[a_{1},\dots,a_{n}]$
      も
      $K^{-1}[a_{1},\dots,a_{n}] \subset K(a_{1},\dots,a_{n})$
      である。

      よって、
      $K[a_{1},\dots,a_{n}]$の元と
      $K^{-1}[a_{1},\dots,a_{n}]$の元をかけた集合$\tilde{K}$は
      $\tilde{K} \subset K(a_{1},\dots,a_{n})$
      となる。

      これにより次の等号が成り立つ。
      \begin{equation}
       K(a_{1},\dots,a_{n})
        = \tilde{K}
        = \{ab^{-1} \mid a,b \in K[a_{1},\dots,a_{n}],\; b\ne 0 \}
      \end{equation}

      \hrulefill



 \item
      有理数体$\mathbb{Q}$を
      複素数体$\mathbb{C}$の部分体と考え、
      純虚数$i=\sqrt{-1}$を取る。
      この時、次が成り立つ。
      \begin{equation}
       \mathbb{Q}(i)
        =\mathbb{Q}[i]
        = \{ a+bi \mid a,b\in \mathbb{Q} \}
      \end{equation}

      \dotfill

%      $\mathbb{Q}[i]$は
%      有理数体$\mathbb{Q}$に$i=\sqrt{-1}$を付け加えた環である。

%      $i^{2}=-1$なので、
%      $\mathbb{Q}[i] = \{ a+bi \mid a,b\in \mathbb{Q} \}$
%      である。

      $\mathbb{Q}[i]$は$\mathbb{Q}$と$i$を含む最小の環であり、
      $\mathbb{Q}(i)$は$\mathbb{Q}$と$i$を含む最小の体である。

      任意の$\mathbb{Q}[i]$の元は$\mathbb{Q}(i)$の元であるので、
      $\mathbb{Q}[i] \subset \mathbb{Q}$である。


      $\mathbb{Q}(i)$の元は
      $f,g\in \mathbb{Q}[i]$を用いて
      $fg^{-1}=f/g$とかける。
      
      $g\in \mathbb{Q}[i]$の共役複素数$\bar{g}\in\mathbb{Q}[i]$を用いて
      $g\bar{g}\in \mathbb{Q}$である。

      \begin{equation}
       fg^{-1} = \frac{f}{g} = \frac{f\bar{g}}{g\bar{g}} \in \mathbb{Q}[i]
      \end{equation}

      このため、$\mathbb{Q}(i) \subset \mathbb{Q}[i]$である。

      以上により
      $\mathbb{Q}(i) = \mathbb{Q}[i]$である。


      \hrulefill

\end{enumerate}



\hrulefill

\end{document}
