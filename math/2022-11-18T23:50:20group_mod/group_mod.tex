\documentclass[12pt,b5paper]{ltjsarticle}

%\usepackage[margin=15truemm, top=5truemm, bottom=5truemm]{geometry}
\usepackage[margin=10truemm]{geometry}

\usepackage{amsmath,amssymb}
%\pagestyle{headings}
\pagestyle{empty}

%\usepackage{listings,url}
%\renewcommand{\theenumi}{(\arabic{enumi})}

%\usepackage{graphicx}

%\usepackage{tikz}
%\usetikzlibrary {arrows.meta}
%\usepackage{wrapfig}	% required for `\wrapfigure' (yatex added)
%\usepackage{bm}	% required for `\bm' (yatex added)

% ルビを振る
%\usepackage{luatexja-ruby}	% required for `\ruby'

%% 核Ker 像Im Hom を定義
%\newcommand{\Img}{\mathop{\mathrm{Im}}\nolimits}
%\newcommand{\Ker}{\mathop{\mathrm{Ker}}\nolimits}
%\newcommand{\Hom}{\mathop{\mathrm{Hom}}\nolimits}

%\DeclareMathOperator{\Rot}{rot}
%\DeclareMathOperator{\Div}{div}
%\DeclareMathOperator{\Grad}{grad}
%\DeclareMathOperator{\arcsinh}{arcsinh}
%\DeclareMathOperator{\arccosh}{arccosh}
%\DeclareMathOperator{\arctanh}{arctanh}



%\usepackage{listings,url}
%
%\lstset{
%%プログラム言語(複数の言語に対応,C,C++も可)
%  language = Python,
%%  language = Lisp,
%%  language = C,
%  %背景色と透過度
%  %backgroundcolor={\color[gray]{.90}},
%  %枠外に行った時の自動改行
%  breaklines = true,
%  %自動改行後のインデント量(デフォルトでは20[pt])
%  breakindent = 10pt,
%  %標準の書体
%%  basicstyle = \ttfamily\scriptsize,
%  basicstyle = \ttfamily,
%  %コメントの書体
%%  commentstyle = {\itshape \color[cmyk]{1,0.4,1,0}},
%  %関数名等の色の設定
%  classoffset = 0,
%  %キーワード(int, ifなど)の書体
%%  keywordstyle = {\bfseries \color[cmyk]{0,1,0,0}},
%  %表示する文字の書体
%  %stringstyle = {\ttfamily \color[rgb]{0,0,1}},
%  %枠 "t"は上に線を記載, "T"は上に二重線を記載
%  %他オプション:leftline,topline,bottomline,lines,single,shadowbox
%  frame = TBrl,
%  %frameまでの間隔(行番号とプログラムの間)
%  framesep = 5pt,
%  %行番号の位置
%  numbers = left,
%  %行番号の間隔
%  stepnumber = 1,
%  %行番号の書体
%%  numberstyle = \tiny,
%  %タブの大きさ
%  tabsize = 4,
%  %キャプションの場所("tb"ならば上下両方に記載)
%  captionpos = t
%}



\begin{document}


\hrulefill
\textbf{問題}
\hrulefill


\begin{enumerate}
 \item
      $373$で割った余りが$109$でありかつ
      $511$で割った余りが$367$である正整数$n$
      であって$n\leq 1 000 000 = 10^{6}$
      を満たすものを1つ挙げよ。

      \dotfill

      $373$で割り切れて
      $511$でも割り切れる数を$m$とする。
      \begin{equation}
       m \equiv 0 \mod{373}, \quad
       m \equiv 0 \mod{511}
      \end{equation}
      $m=373\times 511$とおく。

      ユークリッドの互除法より
      \begin{equation}
       373 \times 137 + 511 \times (-100) =1
      \end{equation}
      である。
      これを移項し次の式が得られる。
      \begin{equation}
       373 \times 137 = 1 + 511 \times 100
      \end{equation}
      これは$373$で割り切れるが、
      $511$で割ると$1$余る。
      \begin{equation}
       373 \times 137 \equiv 0 \mod{373},\quad
        373 \times 137 %= 1 + 511 \times 100
        \equiv 1 \mod{511}
      \end{equation}



      $373$で割った余りが$109$である数$n$は次のように表される。
      \begin{equation}
       n = 373k + 109 \ (k\in\mathbb{Z})
        \label{num_org}
      \end{equation}

      これが次を満たすように$k$の値を求める。
      \begin{equation}
       n = 373k + 109 \equiv 367 \mod{511}
      \end{equation}

      $373 \times 137 \equiv 1 \mod{511}$であるので、
      \begin{equation}
       (373 \times 137)\times (367-109) + 109 \equiv 367 \mod{511}
      \end{equation}
      となる。
      つまり、$k=137\times (367-109) = 35346$となる。

      式(\ref{num_org})に$k$を当てはめて計算すると
      \begin{equation}
       n = 373 \times 35346 + 109 = 13 184 167
      \end{equation}
      となるが、$n>10^6$であるので
      $m=373\times 511$を繰り返し引き、この範囲に収まるようにする。
      \begin{equation}
       n = (373 \times 35346 + 109 ) - (69m)
        = 373 \times 87 + 109
        = 32560
      \end{equation}

      よって、$n=32560$である。


      \begin{equation}
       32560 = 511 \times 63 + 367
      \end{equation}


      \hrulefill

 \item
      $G$を群とする。
      任意の$x\in G$に対して、
      写像$C(x):G\rightarrow G$を
      $y\mapsto xyx^{-1}$と定義する。
      \begin{enumerate}
       \item
            任意の$x\in G$に対して、
            $C(x)$は$G$の自己同型写像であることを示せ。

            \dotfill

            $y,z\in G$に対して
            $C(x)(yz)=xyzx^{-1}=xyx^{-1}xzx^{-1}=C(x)(y)C(x)(z)$
            であるので$C(x)$は準同型写像である。

            $\mathrm{Ker}C(x)=\{e\}$であるので単射である。

            ${}^{\forall}z\in G$に対して
            $C(x)(x^{-1}zx)=xx^{-1}zxx^{-1}=z$であるので
            全射である。

            よって、$C(x)$は自己同型写像である。

            \hrulefill

       \item
            写像$C:G \to \mathrm{Aut} G$を
            $x\mapsto C(x)$と定義する。
            このとき、
            $C$は準同型写像であることを示せ。

            \dotfill

            $x_1,x_2\in G$に対して
            \begin{equation}
             C(x_1x_2)(y)=x_1x_2y(x_1x_2)^{-1}
              = x_1x_2yx_2^{-1}x_1^{-1}
              = C(x_1) \left( C(x_2)(y) \right)
            \end{equation}
            であるので
            $C(x_1x_2)=C(x_1)\circ C(x_2)$
            となり
            $C$は準同型写像である。

            \hrulefill

       \item
            $C$の核は
            $\{x\in G \mid {}^{\forall}y\in G,\ xy=yx\}$
            であることを示せ。

            \dotfill

            $\mathrm{Aut}G$の単位元は恒等写像$id_{G}$である。

            $C(x)(y)=xyx^{-1}=y$であるためには
            $xy=yx$であればよい。
            右から$x^{-1}$をかけることで$xyx^{-1}=y$となる。
            よって、$xy=yx$を満たせば$C(x)=id_{G}$となるので
            $\mathrm{Ker}C = \{x\in G \mid {}^{\forall}y\in G,\ xy=yx\}$
            である。


            \hrulefill

      \end{enumerate}

\end{enumerate}


\end{document}
