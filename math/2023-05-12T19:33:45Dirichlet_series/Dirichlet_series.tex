\documentclass[12pt,b5paper]{ltjsarticle}

%\usepackage[margin=15truemm, top=5truemm, bottom=5truemm]{geometry}
%\usepackage[margin=10truemm,left=15truemm]{geometry}
\usepackage[margin=10truemm]{geometry}

\usepackage{amsmath,amssymb}
%\pagestyle{headings}
\pagestyle{empty}

%\usepackage{listings,url}
%\renewcommand{\theenumi}{(\arabic{enumi})}

%\usepackage{graphicx}

%\usepackage{tikz}
%\usetikzlibrary {arrows.meta}
%\usepackage{wrapfig}
%\usepackage{bm}

% ルビを振る
%\usepackage{luatexja-ruby}	% required for `\ruby'

%% 核Ker 像Im Hom を定義
%\newcommand{\Img}{\mathop{\mathrm{Im}}\nolimits}
%\newcommand{\Ker}{\mathop{\mathrm{Ker}}\nolimits}
%\newcommand{\Hom}{\mathop{\mathrm{Hom}}\nolimits}

%\DeclareMathOperator{\Rot}{rot}
%\DeclareMathOperator{\Div}{div}
%\DeclareMathOperator{\Grad}{grad}
%\DeclareMathOperator{\arcsinh}{arcsinh}
%\DeclareMathOperator{\arccosh}{arccosh}
%\DeclareMathOperator{\arctanh}{arctanh}



%\usepackage{listings,url}
%
%\lstset{
%%プログラム言語(複数の言語に対応,C,C++も可)
%  language = Python,
%%  language = Lisp,
%%  language = C,
%  %背景色と透過度
%  %backgroundcolor={\color[gray]{.90}},
%  %枠外に行った時の自動改行
%  breaklines = true,
%  %自動改行後のインデント量(デフォルトでは20[pt])
%  breakindent = 10pt,
%  %標準の書体
%%  basicstyle = \ttfamily\scriptsize,
%  basicstyle = \ttfamily,
%  %コメントの書体
%%  commentstyle = {\itshape \color[cmyk]{1,0.4,1,0}},
%  %関数名等の色の設定
%  classoffset = 0,
%  %キーワード(int, ifなど)の書体
%%  keywordstyle = {\bfseries \color[cmyk]{0,1,0,0}},
%  %表示する文字の書体
%  %stringstyle = {\ttfamily \color[rgb]{0,0,1}},
%  %枠 "t"は上に線を記載, "T"は上に二重線を記載
%  %他オプション:leftline,topline,bottomline,lines,single,shadowbox
%  frame = TBrl,
%  %frameまでの間隔(行番号とプログラムの間)
%  framesep = 5pt,
%  %行番号の位置
%  numbers = left,
%  %行番号の間隔
%  stepnumber = 1,
%  %行番号の書体
%%  numberstyle = \tiny,
%  %タブの大きさ
%  tabsize = 4,
%  %キャプションの場所("tb"ならば上下両方に記載)
%  captionpos = t
%}



\begin{document}


\hrulefill

\textbf{ディリクレ級数 (Dirichlet series)}

複素数列$\{a_{n}\}_{n\in\mathbb{N}}$と
$s\in\mathbb{C}$に対して、
次で表される級数のことをディリクレ級数 (Dirichlet series)
という。
\begin{equation}
 \sum_{n=1}^{\infty}\frac{a_{n}}{n^{s}}
\end{equation}

\textbf{収束軸}

ディリクレ級数の$s$の実部$\mathrm{Re}(s)$に対し、
$\mathrm{Re}(s) > \sigma$の範囲で収束し、
$\mathrm{Re}(s) < \sigma$の範囲で発散する時、
$\sigma$を収束軸という。

ディリクレ級数が常に収束する時は収束軸は$-\infty$、
常に発散するときは$\infty$とする。


\textbf{収束軸の計算}

$s_{n}=\sum_{i=1}^{n}a_{i}$とする。
\begin{itemize}
 \item $s_{n}$が発散する場合
       \begin{equation}
        \limsup_{n\to \infty} \frac{\log{\lvert s_{n} \rvert}}{\log{n}}
       \end{equation}
 \item $s_{n}$が収束する場合
       \begin{equation}
        \limsup_{n\to \infty} \frac{\log{\lvert \sum_{i=n}^{\infty}a_{i} \rvert}}{\log{n}}
       \end{equation}
\end{itemize}



\dotfill

\textbf{Abel の級数変形法}

複素数列$\{a_{n}\}_{n\in\mathbb{N}}$は
その部分和$s_{n}=\sum_{k=1}^{n}a_{k}$のなす
数列$\{s_{n}\}_{n\in\mathbb{N}}$が有界であるとする。
すなわち、
${}^{\forall}N\in\mathbb{N}$に対して
$\lvert s_{N} \rvert = \left\lvert \sum_{n=1}^{N}a_{n} \right\rvert \leq M$
なる
$M\in\mathbb{R}$が存在する。
また、
実数列$\{\varepsilon_{n}\}_{n\in\mathbb{N}}$は
正項かつ単調減少
($\varepsilon_{1}\geq \varepsilon_{2}\geq \cdots \geq \varepsilon_{n}\geq \cdots \geq 0$)
であるとする。

このとき、
級数
$\displaystyle S= \sum_{n=1}^{\infty}a_{n}\varepsilon_{n}$
について
次が成り立つ。
\begin{enumerate}
 \item
      $\lim_{n\to\infty}\varepsilon_{n}=0$の時、
      級数
      $S= \sum_{n=1}^{\infty}a_{n}\varepsilon_{n}$
      は収束し、
      かつ$\lvert S \rvert \leq M\varepsilon_{1}$
      である。
 \item
      級数$\sum_{n=1}^{\infty}a_{n}$
      が収束する時、
      級数
      $S= \sum_{n=1}^{\infty}a_{n}\varepsilon_{n}$
      は収束し、
      かつ$\lvert S \rvert \leq M\varepsilon_{1}$
      である。
\end{enumerate}


\hrulefill


複素数$\omega\in\mathbb{C}$を
$\omega^{n}=1$となる
最小の自然数が$n=6$であるものとする。
この時、
$a_{n}=\omega^{n}$として定まる
Dirichlet級数
$\displaystyle \sum_{n=1}^{\infty}\frac{a_{n}}{n^{s}}
=\sum_{n=1}^{\infty}\frac{\omega^{n}}{n^{s}}$
の収束軸を求めよ。

\dotfill

$\omega = 1, \exp(\frac{\pi}{3}i), \exp(\frac{2\pi}{3}i), \exp(\pi i), \exp(\frac{4\pi}{3}i), \exp(\frac{5\pi}{3}i)$
は
$\omega^{6}=1$を満たす。

$\omega^{n}=1$となる最小の自然数が$6$であるので、
$1^1=1$、
$\exp(\pi i)^2=1$、
$\exp(\frac{2\pi}{3}i)^3=\exp(\frac{4\pi}{3}i)^3=1$は
$\omega$ではない。

つまり、
$\omega = \exp(\frac{\pi}{3}i), \exp(\frac{5\pi}{3}i)$
である。

$1+\omega^3=0$より
\begin{gather}
 \sum_{n=1}^{1}\omega^{n}=\omega,\
 \sum_{n=1}^{2}\omega^{n}=\omega+\omega^2,\
 \sum_{n=1}^{3}\omega^{n}=\omega+\omega^2+\omega^3=\omega+\omega^2-1\\
 \sum_{n=1}^{4}\omega^{n}=\omega^2+\omega^3=\omega^2-1,\
 \sum_{n=1}^{5}\omega^{n}=\omega^3=-1,\
 \sum_{n=1}^{6}\omega^{n}=0
\end{gather}
である。
$\omega^6=1$よりこの6種類が繰り返し現れる。

$s_{n}=\sum_{k=1}^{n}\omega^{k}$として
数列$\{s_{n}\}_{n\in\mathbb{N}}$を考えると、
上記の数列が繰り返し現れる数列になる。
\begin{equation}
 \{s_{n}\}_{n\in\mathbb{N}}=
  \{ \omega,\  \omega+\omega^2,\ \omega+\omega^2-1,\ \omega^2-1,\ -1,\ 0,\ \dots\}
\end{equation}
これより数列$\{ \lvert s_{n} \rvert\}_{n\in\mathbb{N}}$は次のようになる。
\begin{equation}
 \{ \lvert s_{n} \rvert\}_{n\in\mathbb{N}}=
  \{ 1, \sqrt{3}, 2, \sqrt{3}, 1, 0,\dots\}
\end{equation}

よって、
${}^{\forall}N\in\mathbb{N}$に対して
$\lvert s_{N} \rvert \leq 2$である。

また、
$s=\sigma + i t\ (\sigma,t\in\mathbb{R},\ i=\sqrt{-1})$とすれば、
\begin{gather}
 n^s
  =n^{\sigma+it}
  =\exp((\log{n})(\sigma+it))
  =\exp(\sigma\log{n})\exp(it\log{n})
  =n^{\sigma}\exp(i\log{n^t})\\
 \lvert n^{s} \rvert = n^{\sigma}
\end{gather}
である。

そこで、実数列$\{n^{-\sigma}\}_{n\in\mathbb{N}}$について考える。
${}^{\forall}n\in\mathbb{N}$に対して
$n^{-\sigma}>0$であり、
$\sigma>0$において単調減少な数列である。
極限を取ってみると次のようになる。
\begin{equation}
 \lim_{n\to\infty}n^{-\sigma}=
  \begin{cases}
   0 & (\sigma>0)\\
   1 & (\sigma=0)\\
   \infty & (\sigma < 0)
  \end{cases}
\end{equation}

Abelの級数変形法より
数列$S=\sum_{n=1}^{\infty}\omega^{n}n^{-\sigma}$
は$\sigma>0$において収束し、
$\lvert S \rvert \leq 2\cdot 1^{-\sigma}=2$である。

$\sigma<0$においては
$\lim_{n\to\infty}n^{-\sigma}=\infty$より
$S$は発散する。

つまり、
級数
$\sum_{n=1}^{\infty}\frac{\omega^n}{n^\sigma}=\sum_{n=1}^{\infty}\frac{\omega^n}{\lvert n^{s} \rvert}$
は
$\sigma>0$で収束、
$\sigma<0$で発散する。

よって、
$\sum_{n=1}^{\infty}\frac{\omega^n}{n^{s}}$
においても
$\sigma>0$で収束、
$\sigma<0$で発散する為、
収束軸は$\sigma=0$である。


%
%\dotfill
%
%
%$\lvert \omega^n \rvert =1$
%であり、
%$s=\sigma + i t$とすれば、
%$\lvert n^s \rvert = \lvert n^{\sigma+it} \rvert = n^{\sigma}$
%である。
%\begin{equation}
% \left\lvert \sum_{n=1}^{\infty}\frac{\omega^{n}}{n^{s}} \right\rvert
%  \leq
%  \sum_{n=1}^{\infty}\frac{\lvert \omega^{n} \rvert}{\lvert n^{s} \rvert}
%  =
%  \sum_{n=1}^{\infty}\frac{1}{n^{\sigma}}
%\end{equation}
%右辺は
%$\sigma=1$で発散し、
%$\sigma>1$で収束する。
%つまり、
%$\mathrm{Re}(s)>1$のとき
%$\sum_{n=1}^{\infty}\frac{\omega^{n}}{n^{s}}$
%は収束する。


\hrulefill

\end{document}
