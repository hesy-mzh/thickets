\documentclass[12pt,b5paper]{ltjsarticle}

%\usepackage[margin=15truemm, top=5truemm, bottom=5truemm]{geometry}
%\usepackage[margin=10truemm,left=15truemm]{geometry}
\usepackage[margin=10truemm]{geometry}

\usepackage{amsmath,amssymb}
%\pagestyle{headings}
\pagestyle{empty}

%\usepackage{listings,url}
%\renewcommand{\theenumi}{(\arabic{enumi})}

%\usepackage{graphicx}

%\usepackage{tikz}
%\usetikzlibrary {arrows.meta}
%\usepackage{wrapfig}	% required for `\wrapfigure' (yatex added)
%\usepackage{bm}	% required for `\bm' (yatex added)

% ルビを振る
%\usepackage{luatexja-ruby}	% required for `\ruby'

%% 核Ker 像Im Hom を定義
%\newcommand{\Img}{\mathop{\mathrm{Im}}\nolimits}
%\newcommand{\Ker}{\mathop{\mathrm{Ker}}\nolimits}
%\newcommand{\Hom}{\mathop{\mathrm{Hom}}\nolimits}

%\DeclareMathOperator{\Rot}{rot}
%\DeclareMathOperator{\Div}{div}
%\DeclareMathOperator{\Grad}{grad}
%\DeclareMathOperator{\arcsinh}{arcsinh}
%\DeclareMathOperator{\arccosh}{arccosh}
%\DeclareMathOperator{\arctanh}{arctanh}



%\usepackage{listings,url}
%
%\lstset{
%%プログラム言語(複数の言語に対応,C,C++も可)
%  language = Python,
%%  language = Lisp,
%%  language = C,
%  %背景色と透過度
%  %backgroundcolor={\color[gray]{.90}},
%  %枠外に行った時の自動改行
%  breaklines = true,
%  %自動改行後のインデント量(デフォルトでは20[pt])
%  breakindent = 10pt,
%  %標準の書体
%%  basicstyle = \ttfamily\scriptsize,
%  basicstyle = \ttfamily,
%  %コメントの書体
%%  commentstyle = {\itshape \color[cmyk]{1,0.4,1,0}},
%  %関数名等の色の設定
%  classoffset = 0,
%  %キーワード(int, ifなど)の書体
%%  keywordstyle = {\bfseries \color[cmyk]{0,1,0,0}},
%  %表示する文字の書体
%  %stringstyle = {\ttfamily \color[rgb]{0,0,1}},
%  %枠 "t"は上に線を記載, "T"は上に二重線を記載
%  %他オプション:leftline,topline,bottomline,lines,single,shadowbox
%  frame = TBrl,
%  %frameまでの間隔(行番号とプログラムの間)
%  framesep = 5pt,
%  %行番号の位置
%  numbers = left,
%  %行番号の間隔
%  stepnumber = 1,
%  %行番号の書体
%%  numberstyle = \tiny,
%  %タブの大きさ
%  tabsize = 4,
%  %キャプションの場所("tb"ならば上下両方に記載)
%  captionpos = t
%}



\begin{document}

\hrulefill

\textbf{微分演算子}

$\frac{\mathrm{d}}{\mathrm{d}x}f(x)$を
$Df(x)$や$D[f(x)]$などと書く時の$D$を微分演算子という。

早い話が$D = \frac{\mathrm{d}}{\mathrm{d}x}$である。

$Df(x)=g(x)$であるとき
$f(x)=\frac{1}{D}g(x)$である。
つまり、$\frac{1}{D}$は積分を表している。


\begin{align}
 \frac{1}{D-\alpha}f(x) =& e^{\alpha x}\int e^{-\alpha x}f(x)\mathrm{d}x\\
 \frac{1}{D-\alpha}e^{\alpha x} =& xe^{\alpha x}\\
 \frac{1}{D-\alpha}e^{\beta x} =& \frac{1}{\beta-\alpha}e^{\beta x}\\
 \frac{1}{\alpha-D}f(x) =& \frac{1}{\alpha}\left( 1+\frac{1}{\alpha}D+\frac{1}{\alpha^2}D^2+\cdots \right)\\
 =& \frac{1}{\alpha} \sum_{i=0}^{\infty} \left(\frac{1}{\alpha}D\right)^{i}f(x)\\
\end{align}

\hrulefill

$t$を独立変数とする関数$x=x(t)$についての
微分方程式について各問に答えよ。
\begin{equation}
 x^{\prime\prime}- x^{\prime}+7x=0
  \label{diff_eq01}
\end{equation}
\begin{enumerate}
 \item
      微分方程式(\ref{diff_eq01})を
      微分演算子$D$を用いて表し、
      特性方程式を求めよ。

      \dotfill

      微分演算子を
      $D=\frac{\mathrm{d}}{\mathrm{d}t}$
      として方程式を書き換える。
      \begin{align}
       x^{\prime\prime}- x^{\prime}+7x =& 0\\
       \frac{\mathrm{d}}{\mathrm{d}t}\frac{\mathrm{d}}{\mathrm{d}t}x
       -\frac{\mathrm{d}}{\mathrm{d}t}x
       +7x =& 0\\
       DDx-Dx+7x =& 0\\
       (D^2-D+7)x =& 0
      \end{align}

      よって、特性方程式は
      変数を$\lambda$とすると
      $\lambda^2-\lambda+7 = 0$
      である。

      \hrulefill

 \item
      微分方程式(\ref{diff_eq01})の
      基本解を求めよ。

      \dotfill

      特性方程式
      $\lambda^2-\lambda+7 = 0$
      の解は
      $\lambda=\frac{1}{2}\left( 1\pm 3\sqrt{3}i \right)$
      となる。

      基本解は$e^{\left(\frac{1}{2} \pm \frac{3\sqrt{3}}{2}i\right)t}$
      である。

      そこでオイラーの公式
      $e^{i\theta}=\cos{\theta}+i\sin{\theta}$
      を利用して一般解を変形する。
      \begin{align}
       x =& A_1e^{\left(\frac{1}{2} + \frac{3\sqrt{3}}{2}i\right)t}
        +A_2e^{\left(\frac{1}{2} - \frac{3\sqrt{3}}{2}i\right)t}\\
       =& A_1e^{\frac{1}{2}t}e^{\frac{3\sqrt{3}}{2}it}
       + A_2e^{\frac{1}{2}t}e^{-\frac{3\sqrt{3}}{2}it}\\
       =& A_1e^{\frac{1}{2}t}\left(\cos{\frac{3\sqrt{3}}{2}t}+i\sin{\frac{3\sqrt{3}}{2}t}\right)
       +  A_2e^{\frac{1}{2}t}\left(\cos{-\frac{3\sqrt{3}}{2}t}+i\sin{-\frac{3\sqrt{3}}{2}t}\right)\\
       =& A_1e^{\frac{1}{2}t}\left(\cos{\frac{3\sqrt{3}}{2}t}+i\sin{\frac{3\sqrt{3}}{2}t}\right)
       +  A_2e^{\frac{1}{2}t}\left(\cos{\frac{3\sqrt{3}}{2}t}-i\sin{\frac{3\sqrt{3}}{2}t}\right)\\
       =& (A_1+A_2)e^{\frac{1}{2}t}\cos{\frac{3\sqrt{3}}{2}t}
       +(A_1-A_2)ie^{\frac{1}{2}t}\sin{\frac{3\sqrt{3}}{2}t}
       \label{kai}
      \end{align}

      $A_1,A_2\in\mathbb{C}$を互いに共役として、
      $A_1=a+bi,A_2=a-bi \ (a,b\in\mathbb{R})$とおく。
      $A_1+A_2=2a,\ A_1-A_2=2bi$であるので、
      $C_1=2a,\ C_2=-2b$とすれば
      式(\ref{kai})は次のようになる。
      \begin{equation}
       x=
        C_1e^{\frac{1}{2}t}\cos{\frac{3\sqrt{3}}{2}t}
        +
        C_2e^{\frac{1}{2}t}\sin{\frac{3\sqrt{3}}{2}t}
      \end{equation}

      よって、基本解は次の2つになる。
      \begin{equation}
       e^{\frac{1}{2}t}\cos{\frac{3\sqrt{3}}{2}t}
        ,\quad
       e^{\frac{1}{2}t}\sin{\frac{3\sqrt{3}}{2}t}
      \end{equation}

      \hrulefill

\end{enumerate}


\hrulefill

$t$を独立変数とする関数$x=x(t)$についての
微分方程式について各問に答えよ。
\begin{equation}
 x^{\prime\prime}- 4x^{\prime}+7x=0
  \label{diff_eq02}
\end{equation}
\begin{enumerate}
 \item
      微分方程式(\ref{diff_eq02})を
      微分演算子$D$を用いて表し、
      特性方程式を求めよ。

      \dotfill

      微分演算子を用いると次の式が得られる。
      \begin{align}
       x^{\prime\prime}- 4x^{\prime}+7x =& 0\\
       DDx-4Dx+7x =& 0\\
       (D^2-4D+7)x =& 0
      \end{align}

      よって、特性方程式は
      変数を$\lambda$とすると
      $\lambda^2-4\lambda+7 = 0$
      である。

      \hrulefill

 \item
      微分方程式(\ref{diff_eq02})の
      基本解を求めよ。

      \dotfill

      特性方程式
      $\lambda^2-4\lambda+7 = 0$
      の解は
      $\lambda= 2\pm \sqrt{3}i$
      となる。

      基本解は
      $e^{\left(2 \pm \sqrt{3}i\right)t}$
      である。
      基本解が複素数となるので、
      変形を行うことにより
      基本解は次の2つになる。
      \begin{equation}
       e^{2t}\cos{\sqrt{3}t}
        ,\quad
       e^{2t}\sin{\sqrt{3}t}
      \end{equation}

      \hrulefill

\end{enumerate}

\hrulefill



\begin{equation}
 x^{\prime\prime}- 2x^{\prime}-3x=27t^2
  \label{diff_eq03}
\end{equation}

$t$を独立変数とする関数$x=x(t)$についての
微分方程式(\ref{diff_eq03})について、
一般解が次で与えられることを、
逆演算子を利用して確認せよ。

\begin{equation}
 x(t)=-9t^2+12t-14+C_1e^{3t}+C_2e^{-t}
  \qquad
  (C_1,C_2\text{:const})
\end{equation}

\dotfill

式(\ref{diff_eq03})を微分演算子を用いると次のようになる。
\begin{equation}
 (D^2-2D-3)x=27t^2
\end{equation}

これを逆演算子を用いて計算をする。
\begin{align}
 x =& \frac{1}{D^2-2D-3}27t^2\\
 =& \frac{1}{(D+1)(D-3)}27t^2\\
 =& \frac{1}{4}\left( \frac{1}{D-3}-\frac{1}{D+1} \right) 27t^2\\
 =& \frac{27}{4}\left( \frac{1}{D-3}t^2-\frac{1}{D+1}t^2 \right)
\end{align}

\begin{align}
 \frac{1}{D-3}t^2
  =& e^{3t} \int e^{-3t}t^2\mathrm{d}t\\
  =& e^{3t} \left( \frac{1}{-3}e^{-3t}t^2 - \int \frac{2}{-3}e^{-3t}t\mathrm{d}t \right)\\
  =& e^{3t} \left( \frac{1}{-3}e^{-3t}t^2 - \frac{2}{(-3)^2}e^{-3t}t + \int \frac{2}{(-3)^2}e^{-3t}\mathrm{d}t \right)\\
  =& e^{3t} \left( \frac{1}{-3}e^{-3t}t^2 - \frac{2}{(-3)^2}e^{-3t}t + \frac{2}{(-3)^3} e^{-3t} \right)\\
  =& -\frac{1}{3} t^2 - \frac{2}{9}t - \frac{2}{27}
\end{align}

\begin{align}
 \frac{1}{D+1}t^2
 =& e^{-t} \int e^{t} t^2 \mathrm{d}t\\
 =& e^{-t} \left( e^{t} t^2 - \int 2 e^{t} t \mathrm{d}t \right)\\
 =& e^{-t} \left( e^{t} t^2 - 2 e^{t} t + \int e^{t} \mathrm{d}t \right)\\
 =& e^{-t} \left( e^{t} t^2 - 2 e^{t} t + e^{t} \right)\\
 =& t^2 - 2 t + 1
\end{align}

\begin{align}
 x =& \frac{1}{D^2-2D-3}27t^2\\
 =& \frac{27}{4}\left( \frac{1}{D-3}t^2-\frac{1}{D+1}t^2 \right)\\
 =& \frac{27}{4}\left( -\frac{1}{3} t^2 - \frac{2}{9}t - \frac{2}{27} - t^2 + 2 t - 1 \right)\\
 =& \frac{27}{4}\left( -\frac{1}{3} t^2 - \frac{2}{9}t - \frac{2}{27} - t^2 + 2 t - 1 \right)\\
 =& -9 t^2 +12 t - \frac{29}{4}
 \end{align}



\hrulefill

\begin{equation}
 x^{\prime\prime}+x^{\prime}+x=7e^{2t}
  \label{diff_eq04}
\end{equation}

$t$を独立変数とする関数$x=x(t)$についての
微分方程式(\ref{diff_eq04})について、
一般解が次で与えられることを、
逆演算子を利用して確認せよ。

\begin{equation}
 x(t)=e^{2t}
  +C_1e^{-\frac{1}{2}t}\cos{\left(\frac{\sqrt{3}}{2}t\right)}
  +C_2e^{-\frac{1}{2}t}\sin{\left(\frac{\sqrt{3}}{2}t\right)}
  \qquad
  (C_1,C_2\text{:const})
\end{equation}


\dotfill




\hrulefill

\end{document}
