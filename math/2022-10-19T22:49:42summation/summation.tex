\documentclass[12pt,b5paper]{ltjsarticle}

%\usepackage[margin=15truemm, top=5truemm, bottom=5truemm]{geometry}
\usepackage[margin=10truemm]{geometry}

\usepackage{amsmath,amssymb}
%\pagestyle{headings}
\pagestyle{empty}

%\usepackage{listings,url}
%\renewcommand{\theenumi}{(\arabic{enumi})}

%\usepackage{graphicx}

%\usepackage{tikz}
%\usetikzlibrary {arrows.meta}
%\usepackage{wrapfig}	% required for `\wrapfigure' (yatex added)
%\usepackage{bm}	% required for `\bm' (yatex added)

% ルビを振る
%\usepackage{luatexja-ruby}	% required for `\ruby'

%% 核Ker 像Im Hom を定義
%\newcommand{\Img}{\mathop{\mathrm{Im}}\nolimits}
%\newcommand{\Ker}{\mathop{\mathrm{Ker}}\nolimits}
%\newcommand{\Hom}{\mathop{\mathrm{Hom}}\nolimits}

%\DeclareMathOperator{\Rot}{rot}
%\DeclareMathOperator{\Div}{div}
%\DeclareMathOperator{\Grad}{grad}
%\DeclareMathOperator{\arcsinh}{arcsinh}
%\DeclareMathOperator{\arccosh}{arccosh}
%\DeclareMathOperator{\arctanh}{arctanh}



\usepackage{listings,url}

\lstset{
%プログラム言語(複数の言語に対応,C,C++も可)
%  language = Python,
  language = Lisp,
  %背景色と透過度
  %backgroundcolor={\color[gray]{.90}},
  %枠外に行った時の自動改行
  breaklines = true,
  %自動改行後のインデント量(デフォルトでは20[pt])
  breakindent = 10pt,
  %標準の書体
%  basicstyle = \ttfamily\scriptsize,
  basicstyle = \ttfamily,
  %コメントの書体
%  commentstyle = {\itshape \color[cmyk]{1,0.4,1,0}},
  %関数名等の色の設定
  classoffset = 0,
  %キーワード(int, ifなど)の書体
%  keywordstyle = {\bfseries \color[cmyk]{0,1,0,0}},
  %表示する文字の書体
  %stringstyle = {\ttfamily \color[rgb]{0,0,1}},
  %枠 "t"は上に線を記載, "T"は上に二重線を記載
  %他オプション:leftline,topline,bottomline,lines,single,shadowbox
  frame = TBrl,
  %frameまでの間隔(行番号とプログラムの間)
  framesep = 5pt,
  %行番号の位置
  numbers = left,
  %行番号の間隔
  stepnumber = 1,
  %行番号の書体
%  numberstyle = \tiny,
  %タブの大きさ
  tabsize = 4,
  %キャプションの場所("tb"ならば上下両方に記載)
  captionpos = t
}



\begin{document}


%\hrulefill

%\hrulefill


\begin{gather}
 a_0=\frac{74}{320},\
 a_1=a_{-1}=\frac{67}{320},\
 a_2=a_{-2}=\frac{46}{320},\
 a_3=a_{-3}=\frac{21}{320}\\
 a_4=a_{-4}=\frac{3}{320},\
 a_5=a_{-5}=\frac{-5}{320},\
 a_6=a_{-6}=\frac{-6}{320},\
 a_7=a_{-7}=\frac{-3}{320}
\end{gather}

\begin{enumerate}
 \item $\displaystyle \sum_{k=-7}^7a_k$を求めよ。
 \item $\displaystyle \sum_{k=-7}^7 k^r a_k \ne 0$
       となる$r\in\mathbb{N}$のうち
       最小のものを求めよ。
\end{enumerate}

\dotfill

\begin{align}
 \sum_{k=-7}^7a_k
 = &
  \frac{-3}{320}
  + \frac{-6}{320}
  + \frac{-5}{320}
  + \frac{3}{320}
  + \frac{21}{320}
  + \frac{46}{320}
  + \frac{67}{320}\\
 &
  + \frac{74}{320}
  + \frac{67}{320}
  + \frac{46}{320}
  + \frac{21}{320}
  + \frac{3}{320}
  + \frac{-5}{320}
  + \frac{-6}{320}
  + \frac{-3}{320}\\
  =& 1
\end{align}

\dotfill

$r=1,r=2,r=3$の時$0$となり、
$r=4$の時$0$でなくなる。

\begin{equation}
 \sum_{k=-7}^7k^1a_k
  = \sum_{k=-7}^7k^2a_k
  = \sum_{k=-7}^7k^3a_k
  =0,\quad
 \sum_{k=-7}^7k^4a_k
 = \frac{-927}{10}
\end{equation}

%$r=1$の場合
%\begin{align}
% \sum_{k=-7}^7k^1a_k
% = &
%  ( -7 )^1 \cdot \frac{-3}{320}
%  + ( -6 )^1 \cdot \frac{-6}{320}
%  + ( -5 )^1 \cdot \frac{-5}{320}
%  + ( -4 )^1 \cdot \frac{3}{320}\\
% &
%  + ( -3 )^1 \cdot \frac{21}{320}
%  + ( -2 )^1 \cdot \frac{46}{320}
%  + ( -1 )^1 \cdot \frac{67}{320}
%  + ( 0 )^1 \cdot \frac{74}{320}\\
%&
%  + ( 1 )^1 \cdot \frac{67}{320}
%  + ( 2 )^1 \cdot \frac{46}{320}
%  + ( 3 )^1 \cdot \frac{21}{320}
%  + ( 4 )^1 \cdot \frac{3}{320}\\
%&
%  + ( 5 )^1 \cdot \frac{-5}{320}
%  + ( 6 )^1 \cdot \frac{-6}{320}
%  + ( 7 )^1 \cdot \frac{-3}{320}\\
%  =& 0
%\end{align}
%
%$r=2$の場合
%\begin{align}
% \sum_{k=-7}^7k^2a_k
% = &
%  ( -7 )^2 \cdot \frac{-3}{320}
%  + ( -6 )^2 \cdot \frac{-6}{320}
%  + ( -5 )^2 \cdot \frac{-5}{320}
%  + ( -4 )^2 \cdot \frac{3}{320}\\
% &
%  + ( -3 )^2 \cdot \frac{21}{320}
%  + ( -2 )^2 \cdot \frac{46}{320}
%  + ( -1 )^2 \cdot \frac{67}{320}
%  + ( 0 )^2 \cdot \frac{74}{320}\\
%&
%  + ( 1 )^2 \cdot \frac{67}{320}
%  + ( 2 )^2 \cdot \frac{46}{320}
%  + ( 3 )^2 \cdot \frac{21}{320}
%  + ( 4 )^2 \cdot \frac{3}{320}\\
%&
%  + ( 5 )^2 \cdot \frac{-5}{320}
%  + ( 6 )^2 \cdot \frac{-6}{320}
%  + ( 7 )^2 \cdot \frac{-3}{320}\\
%  =& 0
%\end{align}


Lispコード
\begin{lstlisting}
(defun sum (num)
  (+
   (* (expt 0 num) 74/320)
   (* (expt 1 num) 67/320)
   (* (expt -1 num) 67/320)
   (* (expt 2 num) 46/320)
   (* (expt -2 num) 46/320)
   (* (expt 3 num) 21/320)
   (* (expt -3 num) 21/320)
   (* (expt 4 num) 3/320)
   (* (expt -4 num) 3/320)
   (* (expt 5 num) -5/320)
   (* (expt -5 num) -5/320)
   (* (expt 6 num) -6/320)
   (* (expt -6 num) -6/320)
   (* (expt 7 num) -3/320)
   (* (expt -7 num) -3/320)
   ))

(sum 1)
0

(sum 2)
0

(sum 3)
0

(sum 4)
-927/10
\end{lstlisting}


\hrulefill


\end{document}
