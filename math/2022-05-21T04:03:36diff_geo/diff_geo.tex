\documentclass[12pt,b5paper]{ltjsarticle}

%\usepackage[margin=15truemm, top=5truemm, bottom=5truemm]{geometry}
\usepackage[margin=15truemm]{geometry}

\usepackage{amsmath,amssymb}
%\pagestyle{headings}
\pagestyle{empty}

%\usepackage{listings,url}
\renewcommand{\theenumi}{(\arabic{enumi})}

\usepackage{graphicx}

\usepackage{tikz}
\usetikzlibrary {arrows.meta}
\usepackage{wrapfig}	% required for `\wrapfigure' (yatex added)
\usepackage{bm}	% required for `\bm' (yatex added)
\usepackage{luatexja-ruby}	% required for `\ruby'
%% 像Im を定義
%\newcommand{\Img}{\mathop{\mathrm{Im}}\nolimits}

\begin{document}

\hrulefill

\begin{enumerate}
 \item
       $D \subset \mathbb{R}^n$を開集合とし、
       $\bm{f},\bm{g} : D\rightarrow \mathbb{R}^m$を
       $C^{k}$-級$(k\geq 1)$のベクトル値関数とする。
       関数$\langle \bm{f},\bm{g}\rangle$を次のように定義する。
       \begin{equation}
        \langle \bm{f},\bm{g}\rangle : D\rightarrow \mathbb{R}
         \qquad \bm{x} \mapsto
         %  \langle \bm{f},\bm{g}\rangle(\bm{x}) =
         \langle \bm{f}(\bm{x}),\bm{g}(\bm{x})\rangle
       \end{equation}
       $\langle \bm{f},\bm{g}\rangle$は$C^{k}$-級である。

       この時、次の2つを示せ。
       \begin{enumerate}
        \item
             \begin{equation}
              \frac{\partial}{\partial x_i}\langle \bm{f},\bm{g}\rangle
               = \left\langle \frac{\partial \bm{f}}{\partial x_i},\bm{g}\right\rangle
               + \left\langle \bm{f}, \frac{\partial \bm{g}}{\partial x_i}\right\rangle
               \quad (i=1,2,\cdots,n)
             \end{equation}
        \item
             \begin{equation}
              \lvert \bm{f} \rvert = c\; (\text{定数})
               \Leftrightarrow
               \left\langle \frac{\partial \bm{f}}{\partial x_i},\bm{f}\right\rangle
               \equiv 0
               \quad for \ {}^{\forall}i=1,2,\dots,n
             \end{equation}
       \end{enumerate}

\dotfill

 \item
       $D \subset \mathbb{R}^n$を開集合とし、
       $\bm{f},\bm{g} : D\rightarrow \mathbb{R}^3$を
       $C^{k}$-級$(k\geq 1)$のベクトル値関数とする。
       写像$\bm{f}\times\bm{g}$を次のように定義する。
       \begin{equation}
        \bm{f}\times\bm{g} : D\rightarrow \mathbb{R}^3
         \qquad \bm{x} \mapsto
         %  (\bm{f}\times\bm{g})(\bm{x}) =
         \bm{f}(\bm{x})\times\bm{g}(\bm{x}) \quad (\text{ベクトルの外積})
       \end{equation}
       $\bm{f}\times\bm{g}$は$C^{k}$-級のベクトル値関数である。

       この時、次の式が成り立つことを示せ。
       \begin{equation}
        \frac{\partial}{\partial x_i}(\bm{f}\times\bm{g})
         = \frac{\partial \bm{f}}{\partial x_i}\times\bm{g}
         + \bm{f}\times\frac{\partial \bm{g}}{\partial x_i}
         \qquad (i=1,2,\dots,n)
       \end{equation}
\end{enumerate}

\hrulefill

\begin{enumerate}
 \item
      $\bm{f},\bm{g} : D (\subset\mathbb{R}^n) \rightarrow \mathbb{R}^m$
      である為、
      $\bm{x}=(x_1,x_2,\dots,x_n)$について
      次のようにおく。
      \begin{equation}
       \bm{f}(\bm{x})=(f_1(\bm{x}),\dots,f_m(\bm{x})),
        \quad
      \bm{g}(\bm{x})=(g_1(\bm{x}),\dots,g_m(\bm{x}))
      \end{equation}

      ここから、内積を計算すると次の式になる。
      \begin{equation}
       \langle \bm{f},\bm{g}\rangle(\bm{x}) = \sum_{k=1}^{m}f_k(\bm{x})g_k(\bm{x})
      \end{equation}

      \begin{enumerate}
       \item
            $x_i$について偏微分する。
            \begin{align}
             \frac{\partial}{\partial x_i}\langle \bm{f},\bm{g}\rangle(\bm{x})
             =& \frac{\partial}{\partial x_i}\sum_{k=1}^{m}f_k(\bm{x})g_k(\bm{x})\\
             =& \sum_{k=1}^{m}\frac{\partial}{\partial x_i}f_k(\bm{x})g_k(\bm{x})\\
             =& \sum_{k=1}^{m}\left(\left(
             \frac{\partial}{\partial x_i}f_k(\bm{x})\right)g_k(\bm{x})
             + f_k(\bm{x})\left(\frac{\partial}{\partial x_i}g_k(\bm{x})
             \right)\right)\\
             =& \sum_{k=1}^{m}\left(
             \frac{\partial}{\partial x_i}f_k(\bm{x})\right)g_k(\bm{x})
             + \sum_{k=1}^{m}f_k(\bm{x})\left(\frac{\partial}{\partial x_i}g_k(\bm{x})
             \right)\\
             =& \left\langle \frac{\partial \bm{f}}{\partial x_i},\bm{g}\right\rangle(\bm{x})
             + \left\langle \bm{f},\frac{\partial \bm{g}}{\partial x_i}\right\rangle(\bm{x})
            \end{align}
       \item
            ($\Rightarrow$)
            $\lvert \bm{f} \rvert = c$とする。

            ${}^{\forall}\bm{x}\in D$に対して
            $\lvert \bm{f}(\bm{x}) \rvert = c$である。
            2乗して$x_i$で偏微分すると次のようになる。
            \begin{equation}
             \frac{\partial}{\partial x_i}\lvert \bm{f}(\bm{x}) \rvert^2 = 0
            \end{equation}
            左辺を変形すると次のようになる。
            \begin{align}
             \frac{\partial}{\partial x_i}\lvert \bm{f}(\bm{x}) \rvert^2
             =& \frac{\partial}{\partial x_i}\langle \bm{f}(\bm{x}), \bm{f}(\bm{x}) \rangle\\
             =& \left\langle \frac{\partial}{\partial x_i}\bm{f}(\bm{x}), \bm{f}(\bm{x}) \right\rangle
             + \left\langle \bm{f}(\bm{x}), \frac{\partial}{\partial x_i}\bm{f}(\bm{x}) \right\rangle\\
             =& 2\left\langle \frac{\partial}{\partial x_i}\bm{f}(\bm{x}), \bm{f}(\bm{x}) \right\rangle
            \end{align}
            これにより次の式が得られる。
            \begin{equation}
             \left\langle \frac{\partial}{\partial x_i}\bm{f}, \bm{f} \right\rangle(\bm{x}) =0
            \end{equation}
            ${}^{\forall}\bm{x}\in D$について言える為、恒等的に0である。
            \begin{equation}
             \left\langle \frac{\partial}{\partial x_i}\bm{f}, \bm{f} \right\rangle \equiv 0
            \end{equation}
            全ての$x_i \ (i=1,\dots,n)$について成立することから次が示せる。
            \begin{equation}
             \lvert \bm{f} \rvert = c
              \quad \Rightarrow \quad
               \left\langle \frac{\partial \bm{f}}{\partial x_i},\bm{f}\right\rangle
               \equiv 0
               \quad for \ {}^{\forall}i=1,2,\dots,n
            \end{equation}

            ($\Leftarrow$)
            ${}^{\forall}i=1,2,\dots,n$に対して
            $\left\langle \frac{\partial \bm{f}}{\partial x_i},\bm{f}\right\rangle \equiv 0$
            とする。

            \begin{equation}
             2\left\langle \frac{\partial \bm{f}}{\partial x_i},\bm{f}\right\rangle
             = \left\langle \frac{\partial \bm{f}}{\partial x_i},\bm{f}\right\rangle
             + \left\langle \bm{f}, \frac{\partial \bm{f}}{\partial x_i}\right\rangle
             = \frac{\partial}{\partial x_i}\left\langle \bm{f},\bm{f} \right\rangle
            \end{equation}
            であるので、${}^{\forall}\bm{x}\in D$に対して
            \begin{equation}
             \frac{\partial}{\partial x_i}\langle \bm{f},\bm{f}\rangle (\bm{x})=0
            \end{equation}

            これは、$x_i$の変化について$\langle \bm{f},\bm{f}\rangle (\bm{x})$は
            変化しないことを意味する。
            ${}^{\forall}i=1,2,\dots,n$について同じことが言えるため、
            $\langle \bm{f},\bm{f}\rangle (\bm{x})$は定数であることが分かる。
            \begin{equation}
             \langle \bm{f},\bm{f}\rangle = \lvert \bm{f} \rvert^2
            \end{equation}
            より
            $\lvert \bm{f} \rvert$が定数となる事がわかる。
      \end{enumerate}

\dotfill

 \item
      $\bm{f},\bm{g}:D\rightarrow \mathbb{R}^3$

      \begin{equation}
       \bm{f}(\bm{x})=(f_1(\bm{x}),f_2(\bm{x}),f_3(\bm{x})),
        \quad
        \bm{g}(\bm{x})=(g_1(\bm{x}),g_2(\bm{x}),g_3(\bm{x}))
      \end{equation}
      とする。
      この時、外積は次のようになる。
      \begin{align}
       (\bm{f}\times\bm{g})(\bm{x}) =&
       \bm{f}(\bm{x})\times\bm{g}(\bm{x})\\
       =& \left(
       \begin{vmatrix}f_2(\bm{x})&f_3(\bm{x})\\ g_2(\bm{x})&g_3(\bm{x})\end{vmatrix},
       \quad
       \begin{vmatrix}f_3(\bm{x})&f_1(\bm{x})\\ g_3(\bm{x})&g_1(\bm{x})\end{vmatrix},
       \quad
       \begin{vmatrix}f_1(\bm{x})&f_2(\bm{x})\\ g_1(\bm{x})&g_2(\bm{x})\end{vmatrix}
       \right)\label{compo}
      \end{align}

      成分の行列式を計算する。
      \begin{equation}
        \begin{vmatrix}f_2(\bm{x})&f_3(\bm{x})\\ g_2(\bm{x})&g_3(\bm{x})\end{vmatrix}
        = f_2(\bm{x})g_3(\bm{x}) - f_3(\bm{x})g_2(\bm{x})
      \end{equation}

      これを$x_i$で偏微分する。
      \begin{align}
       & \frac{\partial}{\partial x_i}
       \begin{vmatrix}f_2(\bm{x})&f_3(\bm{x})\\ g_2(\bm{x})&g_3(\bm{x})\end{vmatrix}\\
       =& \frac{\partial}{\partial x_i}
       \left(f_2(\bm{x})g_3(\bm{x}) - f_3(\bm{x})g_2(\bm{x})\right)\\
       =& \frac{\partial f_2(\bm{x})}{\partial x_i}g_3(\bm{x})
       + f_2(\bm{x})\frac{\partial g_3(\bm{x})}{\partial x_i}
       - \frac{\partial f_3(\bm{x})}{\partial x_i}g_2(\bm{x})
       - f_3(\bm{x})\frac{\partial g_2(\bm{x})}{\partial x_i}\\
       =& \begin{vmatrix}
          \frac{\partial}{\partial x_i}f_2(\bm{x})&\frac{\partial}{\partial x_i}f_3(\bm{x})\\
           g_2(\bm{x})&g_3(\bm{x})
          \end{vmatrix}
       +  \begin{vmatrix}
          f_2(\bm{x})&f_3(\bm{x})\\
          \frac{\partial}{\partial x_i}g_2(\bm{x})&\frac{\partial}{\partial x_i}g_3(\bm{x})
          \end{vmatrix}
      \end{align}

      これは式(\ref{compo})の他の成分でも同じように計算が出来る。
      この為、次の式が成り立つ。
      \begin{equation}
       \frac{\partial}{\partial x_i}
        (\bm{f}(\bm{x})\times\bm{g}(\bm{x}))
        =
        \frac{\partial \bm{f}(\bm{x})}{\partial x_i}\times\bm{g}(\bm{x})
        +
        \bm{f}(\bm{x})\times\frac{\partial \bm{g}(\bm{x})}{\partial x_i}
      \end{equation}


      \begin{equation}
       \frac{\partial}{\partial x_i}(\bm{f}\times\bm{g})
        = \frac{\partial \bm{f}}{\partial x_i}\times\bm{g}
        + \bm{f}\times\frac{\partial \bm{g}}{\partial x_i}
      \end{equation}

\end{enumerate}

\hrulefill

\end{document}
