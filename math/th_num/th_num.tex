\documentclass[12pt,b5paper]{ltjsarticle}

%\usepackage[margin=15truemm, top=5truemm, bottom=5truemm]{geometry}
\usepackage[margin=15truemm]{geometry}

\usepackage{amsmath,amssymb}
%\pagestyle{headings}
\pagestyle{empty}

%\usepackage{listings,url}
\renewcommand{\theenumi}{(\arabic{enumi})}

\usepackage{graphicx}

\usepackage{tikz}
\usetikzlibrary {arrows.meta}
\usepackage{wrapfig}	% required for `\wrapfigure' (yatex added)
\usepackage{bm}	% required for `\bm' (yatex added)
\usepackage{luatexja-ruby}	% required for `\ruby'
%% 像Im を定義
%\newcommand{\Img}{\mathop{\mathrm{Im}}\nolimits}

\begin{document}

\hrulefill

\textbf{命題}

$n\in\mathbb{Z}$が$n>4$の時、
$n^n+1$は合成数である。

\hrulefill


$n$が奇数の場合

$\mod 2$ のとき
$n \equiv 1$であるから
\begin{gather}
 n^n \equiv 1^n = 1 \mod 2\\
 n^n +1 \equiv 0 \mod 2
\end{gather}
となる。
これより$n^n+1$は2で割り切れる事がわかる。


$n$が偶数の場合

背理法で考える為、
$n^n +1$を素数と仮定する。

$n< n^n+1$より$n$と$n^n+1$は互いに素である。
フェルマーの小定理より次の式が成り立つ。
\begin{equation}
 n^{n^n+1-1} \equiv 1 \mod (n^n+1)
\end{equation}

この左辺を次のように変形する。
\begin{equation}
 n^{n^n} = n^{n\times n^{n-1}} = (n^n)^{n^{n-1}}
\end{equation}

$n$が偶数であるので、
$n^{n-1}$も偶数である。
この為、
\begin{align}
 n^{n^n} &= (n^n)^{n^{n-1}}\\
 &\equiv (-1)^{n^{n-1}} \mod (n^n+1)\\
 &= 1
\end{align}


\hrulefill


\hrulefill
フェルマーの小定理
\hrulefill

$p\in\mathbb{Z}$を素数とする。
$a\in\mathbb{Z}$が$p$と互いに素である時、
次が成り立つ。
\begin{equation}
 a^{p-1} \equiv 1 \ \mod p
\end{equation}

\hrulefill

\hrulefill
オイラーの定理
\hrulefill

$a$と$n$を互いに素な自然数とする。
この時次の式が成り立つ。
\begin{equation}
 a^{\phi(n)} \equiv 1 \mod n
\end{equation}
ただし、$\phi(n)$は$n$より小さい互いに素な自然数の個数。

\hrulefill

\hrulefill
ウィルソンの定理
\hrulefill

$p$を正の整数とする。
この時、
$p$が素数である事と
次の式は同値である。
\begin{equation}
(p-1)! \equiv -1 \mod p
\end{equation}


\hrulefill


\end{document}
