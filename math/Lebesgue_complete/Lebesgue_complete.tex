\documentclass[12pt,b5paper]{ltjsarticle}

%\usepackage[margin=15truemm, top=5truemm, bottom=5truemm]{geometry}
%\usepackage[margin=10truemm,left=15truemm]{geometry}
\usepackage[margin=10truemm]{geometry}

\usepackage{amsmath,amssymb}
%\pagestyle{headings}
\pagestyle{empty}

%\usepackage{listings,url}
%\renewcommand{\theenumi}{(\arabic{enumi})}

%\usepackage{graphicx}

%\usepackage{tikz}
%\usetikzlibrary {arrows.meta}
%\usepackage{wrapfig}
%\usepackage{bm}

% ルビを振る
\usepackage{luatexja-ruby}	% required for `\ruby'

%% 核Ker 像Im Hom を定義
%\newcommand{\Img}{\mathop{\mathrm{Im}}\nolimits}
%\newcommand{\Ker}{\mathop{\mathrm{Ker}}\nolimits}
%\newcommand{\Hom}{\mathop{\mathrm{Hom}}\nolimits}

%\DeclareMathOperator{\Rot}{rot}
%\DeclareMathOperator{\Div}{div}
%\DeclareMathOperator{\Grad}{grad}
%\DeclareMathOperator{\arcsinh}{arcsinh}
%\DeclareMathOperator{\arccosh}{arccosh}
%\DeclareMathOperator{\arctanh}{arctanh}



%\usepackage{listings,url}
%
%\lstset{
%%プログラム言語(複数の言語に対応,C,C++も可)
%  language = Python,
%%  language = Lisp,
%%  language = C,
%  %背景色と透過度
%  %backgroundcolor={\color[gray]{.90}},
%  %枠外に行った時の自動改行
%  breaklines = true,
%  %自動改行後のインデント量(デフォルトでは20[pt])
%  breakindent = 10pt,
%  %標準の書体
%%  basicstyle = \ttfamily\scriptsize,
%  basicstyle = \ttfamily,
%  %コメントの書体
%%  commentstyle = {\itshape \color[cmyk]{1,0.4,1,0}},
%  %関数名等の色の設定
%  classoffset = 0,
%  %キーワード(int, ifなど)の書体
%%  keywordstyle = {\bfseries \color[cmyk]{0,1,0,0}},
%  %表示する文字の書体
%  %stringstyle = {\ttfamily \color[rgb]{0,0,1}},
%  %枠 "t"は上に線を記載, "T"は上に二重線を記載
%  %他オプション:leftline,topline,bottomline,lines,single,shadowbox
%  frame = TBrl,
%  %frameまでの間隔(行番号とプログラムの間)
%  framesep = 5pt,
%  %行番号の位置
%  numbers = left,
%  %行番号の間隔
%  stepnumber = 1,
%  %行番号の書体
%%  numberstyle = \tiny,
%  %タブの大きさ
%  tabsize = 4,
%  %キャプションの場所("tb"ならば上下両方に記載)
%  captionpos = t
%}



\begin{document}

\hrulefill

\textbf{有限加法族}

集合$X$の部分集合族$\mathcal{F}$が
\textbf{有限加法族}である
とは次を満たすときをいう。
\begin{enumerate}
 \item $\emptyset \in \mathcal{F}$
 \item $A \in \mathcal{F} \Rightarrow X\backslash A \in\mathcal{F}$
 \item $A,B\in\mathcal{F}
       \Rightarrow A\cup B \in\mathcal{F}$
\end{enumerate}


\textbf{有限加法的測度}

集合$X$上の有限加法族$\mathcal{F}$について、
$m:\mathcal{F}\to [0,\infty]$が
$(X,\mathcal{F})$上の
\textbf{有限加法的測度}であるとは、
次の2つの条件を満たすときをいう。
\begin{enumerate}
 \item $m(\emptyset) =0$
 \item $A,B\in\mathcal{F}$が互いに素である時、
       $m(A\cup B) = m(A) + m(B)$
\end{enumerate}


\textbf{外測度}

$X$を集合とする。
$\Gamma : 2^{X}\to [0,\infty]$が
$X$上の\textbf{外測度}であるとは、
次の3つの条件を満たすときをいう。
\begin{enumerate}
 \item
      $\Gamma (\emptyset) = 0$
 \item
      $A,B \subset X$が
      $A\subset B$を満たす時、
      $\Gamma(A)\leq \Gamma(B)$
 \item
      $X$の任意の部分集合列$\{A_{n}\}_{n=1}^{\infty}$
      に対し、
      $\Gamma(\bigcup_{n=1}^{\infty}A_{n}) \leq \sum_{n=1}^{\infty}\Gamma(A_{n})$
\end{enumerate}


\textbf{$\Gamma$-可測}

$X$を集合とする。
$\Gamma : 2^{X}\to [0,\infty]$を
$X$上の外測度とする。

集合$E\subset X$が\textbf{$\Gamma$-可測}
(または \ruby{Carath\'eodory}{カラテオドリ}の意味で可測)
とは、
任意の$A\subset X$に対し次を満たすときをいう。
\begin{equation}
 \Gamma(A\cap E) + \Gamma(A\cap (X\backslash E)) = \Gamma(A)
\end{equation}

また、$\Gamma$-可測集合全体を$\mathcal{M}_{\Gamma}$と表す。



\textbf{命題}($X$上の外測度)

$X$を集合、
$\mathcal{F}$を$X$上の有限加法族、
$\mu$を$(X,\mathcal{F})$上の有限加法的測度
とする。
$\mu^{*}:2^{X}\to [0,\infty]$を次で定義する。
\begin{equation}
 \mu^{*}(A)
  = \inf\left\{
         \sum_{j=1}^{\infty}\mu(E_{j})
         \ \middle| \
         A \subset \bigcup_{j=1}^{\infty}E_{j}
         であり、
         E_{j}\in\mathcal{F}
         、j\in\mathbb{N}
        \right\}
\end{equation}

このとき、
$\mu^{*}$は$X$上の外側度である。


\textbf{可測関数}

$(X,\Sigma_{X}), \: (Y,\Sigma_{Y})$を可測空間、
つまり、
$X,Y$は集合で、
$\Sigma_{X},\Sigma_{Y}$は$\sigma$-加法族
とする。

関数$f: X\to Y$について
${}^{\forall}E\in \Sigma_{Y}$に対して
$f^{-1}(E)\in\Sigma_{X}$が成り立つとき、
関数$f: X\to Y$が可測であるという。
この集合$Y$が$\overline{\mathbb{R}}=[-\infty,\infty]$
の時、$\Sigma_{Y}$はボレル集合族として定義する。



\textbf{$\mu$-零集合}

$(X, \mathcal{M}, \mu)$を測度空間とする。
$A\subset X$が$\mu$-零集合であるとは,
$A\subset N$
かつ
$\mu(N) = 0$
を満たす
$ N\in \mathcal{M}$
が存在することをいう。


\textbf{完備}

$(X, \mathcal{M}, \mu)$を測度空間とする。
全ての$\mu$-零集合が
$\mathcal{M}$に属する時、
$(X, \mathcal{M}, \mu)$
あるいは$\mu$のことを
完備という。

. (X,M, µ)を測度空間とする. 全てのµ-零集合がMに属すとき, (X,M, µ),
あるいは µ のことを完備であるという.






\hrulefill

$(X,\mathcal{M},\mu)$を測度空間とし、
その完備化を
$(X,\overline{\mathcal{M}},\overline{\mu})$で表す。
また、$f:X\to \overline{\mathbb{R}}$とする。

\begin{enumerate}
 \item
      $g: X\to \overline{\mathbb{R}}$
      は
      $\mathcal{M}$-可測であるとする。
      $\{ x\in X \mid f(x)\ne g(x)\}$
      が$\mu$-零集合で
      あるならば、
      $f$は$\overline{\mathcal{M}}$-可測であることを示せ。


 \item
      $\{ f_{n} \}_{n=1}^{\infty}$は
      $X$上の$\overline{\mathbb{R}}$-値関数の列とし、
      任意の$n\in\mathbb{N}$に対し、
      $f_{n}$は$\mathcal{M}$-可則であるとする。
      $\{ x\in X \mid \lim_{n\to\infty}f_{n}(x)=f(x) \}$
      が$\mu$-零集合であるならば、
      $f$は
      $\overline{\mathcal{M}}$-可測になることを示せ。
\end{enumerate}



\dotfill






\hrulefill

\end{document}
