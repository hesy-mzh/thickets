\documentclass[12pt,b5paper]{ltjsarticle}

%\usepackage[margin=15truemm, top=5truemm, bottom=5truemm]{geometry}
\usepackage[margin=20truemm]{geometry}

\usepackage{amsmath,amssymb}
%\pagestyle{headings}
\pagestyle{empty}

%\usepackage{listings,url}
%\renewcommand{\theenumi}{(\arabic{enumi})}

\usepackage{graphicx}

\usepackage{tikz}
\usetikzlibrary {arrows.meta}
\usepackage{wrapfig}	% required for `\wrapfigure' (yatex added)
\usepackage{bm}	% required for `\bm' (yatex added)

% ルビを振る
%\usepackage{luatexja-ruby}	% required for `\ruby'

%% 核Ker 像Im Hom を定義
%\newcommand{\Img}{\mathop{\mathrm{Im}}\nolimits}
%\newcommand{\Ker}{\mathop{\mathrm{Ker}}\nolimits}
%\newcommand{\Hom}{\mathop{\mathrm{Hom}}\nolimits}

%\DeclareMathOperator{\Rot}{rot}
%\DeclareMathOperator{\Div}{div}
%\DeclareMathOperator{\Grad}{grad}
%\DeclareMathOperator{\arcsinh}{arcsinh}
%\DeclareMathOperator{\arccosh}{arccosh}
%\DeclareMathOperator{\arctanh}{arctanh}



\begin{document}

\hrulefill

$m,n$を自然数とする。

$m^2=2^n+1$を満たす$m,n$を求めよ。

\hrulefill

$m^2=2^n+1$を変形する。

\begin{gather}
 m^2 = 2^n +1\\
 (m+1)(m-1) = 2^n
\end{gather}


$2$は素数なので、
自然数$\alpha,\beta\ (\alpha>\beta)$を用いて
$m+1=2^\alpha, m-1=2^\beta$と分ける。
この時、$\alpha + \beta = n$である。

$(m+1)-(m-1) = 2$であり、
$(m+1)-(m-1) = 2^\alpha - 2^\beta$である。

これより、$2^\alpha - 2^\beta =2$である。
$2^\beta (2^{\alpha-\beta}-1)=2$となるが、
$2$は素数なので、
$(2^\beta,\ 2^{\alpha-\beta}-1)=(1,2)$
である場合と
$(2^\beta,\ 2^{\alpha-\beta}-1)=(2,1)$
の場合がありえる。
$2^{\alpha-\beta}-1=2$となる自然数$\alpha,\beta$は存在しない為、
前者はありえない。

\begin{equation}
 2^\beta=2, \quad 2^{\alpha-\beta}-1=1
\end{equation}
これを満たすのは$\alpha=2,\beta=1$であるので、
$(m,n)=(3,3)$となる。



\end{document}

