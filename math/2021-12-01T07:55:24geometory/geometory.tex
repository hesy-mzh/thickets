\documentclass[10pt,b5paper]{ltjsarticle}

\usepackage[margin=15truemm]{geometry}

\usepackage{amsmath}
\usepackage{amssymb}
\pagestyle{empty}

\usepackage{bm}	% required for `\bm' (yatex added)
\begin{document}


\fbox{\textbf{3}}
\begin{enumerate}
 \item $\mathbb{R}^2$の部分集合$A$が開集合であることの定義を述べよ。
 \item $A=\{\bm{x}\in\mathbb{R}^2 \mid d(\bm{x},\bm{0}) > 1\}$が
       $\mathbb{R}^2$の開集合となることを示せ。
 \item $\overline{A^\circ}\ne (\overline{A})^\circ$となる例をあげ、
       $\overline{A^\circ}$と$(\overline{A})^\circ$の包含関係を予想せよ。
\end{enumerate}

\dotfill

1)

任意の$\bm{a}\in A$に対し次を満たす近傍$U$が存在する。
\begin{equation}
 \bm{a}\in U \subset A
\end{equation}

$A$のすべての点には様々な近傍があります。
$A$が開集合であるためには、
各点において点ごとの近傍のうち
最低限一つは$A$に含まれる状態であればいいです。


2)

$A$の点$\bm{a}$に対し様々な近傍があり、
この近傍のうち一つ$U$が$A$に含まれる($U\subset A$)ことを示せばよい。
具体的に示すことは、次のようなことです。

$A$の任意の点は原点$\bm{0}$との距離が$1$より大きいので、
この間の距離で原点より離れるように近傍を取れればいい。
任意の点$\bm{a}\in A$に対し、
$\bm{a}$の近傍$U(\bm{a})$が存在し、
$U(\bm{a}) \subset A$となることを示す。

$\bm{a}=(a_1, a_2)$とおく。
$\delta = \frac{\sqrt{a_1^2 + a_2^2} -1}{2}$とおく。
この$\delta$を用いて$\bm{a}$の近傍$U(\bm{a},\delta)$を考える。

$U(\bm{a},\delta) \subset A$を示す。
$\bm{u}\in U(\bm{a},\delta)$について
次の計算により$d(\bm{u},\bm{0})>1$となり$\bm{u}\in A$である。
\begin{align}
 d(\bm{u},\bm{0}) \geq & d(\bm{a},\bm{0}) - d(\bm{u},\bm{a})\\
 > & \sqrt{a_1^2+a_2^2}-\frac{\sqrt{a_1^2 + a_2^2} -1}{2}\\
 = & \frac{\sqrt{a_1^2 + a_2^2} +1}{2}\\
 > & \frac{1+1}{2} =1
\end{align}

任意の$\bm{a}\in A$について $U(\bm{a},\delta) \subset A$であることがわかり、
$A$は近傍を含むため開集合である。


3)

$A$を$\mathbb{R}^2$の部分集合とする。
$\overline{A}$を$A$の閉包、
$A^\circ$を$A$の内部(又は 開核)とする。

$\overline{A}$は$A$を含む最小の閉集合で、
$A^\circ$は$A$の部分集合の内、最大の開集合です。
この様な関係です。
$A^\circ \subset A \subset \overline{A}$

集合$S_1$と$S_2$を
原点中心で半径$1$の円板の境界をつけない集合$S_1$と
境界をつけた集合$S_2$とします。
\begin{equation}
 S_1 = \{ \bm{s}\in \mathbb{R}^2 \mid d(\bm{s},\bm{0})<1 \}
  \quad
  S_2 = \{ \bm{s}\in \mathbb{R}^2 \mid d(\bm{s},\bm{0}) \leq 1 \}
\end{equation}

$S_1$は開集合、$S_2$は閉集合となり
それぞれの内部と閉包は次のような関係になります。
\begin{equation}
  S_1^\circ = S_1
  \quad
  \overline{S_1} =S_2
  \quad
  S_2^\circ = S_1
  \quad
  \overline{S_2}=S_2
\end{equation}

この為、$\overline{S_1^\circ}=S_2$であり、$(\overline{S_1})^\circ = S_1$
となります。
閉包は元の集合より大きくなり、内部は小さくなる為、
$\overline{A^\circ} \supset (\overline{A})^\circ$
となりそうです。


%\begin{align}
% \overline{S} =
% \{ \bm{s}\in \mathbb{R}^2 \mid d(\bm{s},\bm{0})\leq 1 \} \cup \{ (2,0) \in \mathbb{R}^2 \}
% \quad \Rightarrow & \quad
% (\overline{S})^\circ = \{ \bm{s}\in \mathbb{R}^2 \mid d(\bm{s},\bm{0})<1 \}\\
% S^\circ = \{ \bm{s}\in \mathbb{R}^2 \mid d(\bm{s},\bm{0})<1 \}
% \quad \Rightarrow & \quad
% \overline{S^\circ} = \{ \bm{s}\in \mathbb{R}^2 \mid d(\bm{s},\bm{0}) \leq 1 \}
%\end{align}



\newpage

%\hrulefill


\fbox{\textbf{4}}
 $\mathbb{R}^2$全体で定義される$\mathbb{R}^2$への連続写像(ベクトル値関数)$\bm{f}$について
\begin{enumerate}
 \item $\bm{f}$が点$\bm{a}\in\mathbb{R}^2$で連続であることの定義を述べよ。
 \item $\bm{f}(\mathbb{R}^2)$の任意の開集合$A$に対し、
       $\bm{f}^{-1}(A)$が$\mathbb{R}^2$の開集合となることを示せ。
 \item $\mathbb{R}^2$の任意の開集合$A$に対して、$\bm{f}(A)$が$\mathbb{R}^2$の開集合となるか?
       正しいなら証明を、正しくないなら反例を示せ。
\end{enumerate}

\dotfill

1)

$\bm{f}$が点$\bm{a}\in\mathbb{R}^2$で連続であるとは、
$\bm{f}(\bm{a})$の任意の近傍$U$に対し、
$\bm{f}^{-1}(U)$が$\bm{a}$の近傍であることです。
これは
写像$\bm{f}:\mathbb{R}^2 \rightarrow \mathbb{R}^2$によって
$\bm{a}$の行き先$\bm{f}(\bm{a})$に近傍$U$があり、
この近傍$U$の中に行き先がある点の集合($\subset \mathbb{R}^2$)
が$\bm{a}$の近傍であることを意味します。

解答に合わせると次のようにかけます。

%$\forall \epsilon >0$に対し次を満たす$\exists\delta >0$
任意の正の実数$\varepsilon >0$に対し次を満たす正の実数$\delta >0$が存在する。
\begin{equation}
 \bm{f}(U(\bm{a},\delta)) \subset U(\bm{f}(\bm{a}),\varepsilon)
\end{equation}

2)

$A$を開集合とする。
もし$\bm{f}^{-1}(A)=\emptyset$であれば開集合であるので、
$\bm{f}^{-1}(A)\ne \emptyset$とする。

$\bm{x}\in\bm{f}^{-1}(A)$とすれば、
$\bm{f}(\bm{x})\in A$であり$A$は近傍である。

$\bm{f}$は連続写像であるので、
$\bm{x}$の近傍$U(\bm{x})$で
$\bm{f}(U(\bm{x})) \subset A$を満たすものが存在する。

任意の点$\bm{x}\in\bm{f}^{-1}(A)$について
近傍が存在する為、
$\bm{f}^{-1}(A)$は開集合となる。

3)

次のような全て一点$\bm{0}$へ対応させる写像(定値写像)を考える。
\begin{equation}
 \bm{f}:\mathbb{R}^2 \rightarrow \mathbb{R}^2 ; \bm{x} \mapsto \bm{0}
\end{equation}

この$\bm{f}$は連続写像である。
これは任意の開集合$A\subset \mathbb{R}^2$に対し
次のように$\bm{f}^{-1}(A)$も開集合となる。
\begin{equation}
 \bm{f}^{-1}(A) =
 \begin{cases}
  \emptyset & \bm{0}\notin A\\
  \mathbb{R}^2 & \bm{0}\in A
 \end{cases}
\end{equation}

開集合$A$をうつした先は全て$\bm{0}$であるから
$\bm{f}(A)=\{\bm{0}\}$となる。
近傍を使い開集合を定めるのなら
1点のみの集合は開集合ではないので、
$\bm{f}(A)$は開集合ではない。
\end{document}
