\documentclass[12pt,b5paper]{ltjsarticle}

%\usepackage[margin=15truemm, top=5truemm, bottom=5truemm]{geometry}
\usepackage[margin=10truemm]{geometry}

\usepackage{amsmath,amssymb}
%\pagestyle{headings}
\pagestyle{empty}

%\usepackage{listings,url}
%\renewcommand{\theenumi}{(\arabic{enumi})}

%\usepackage{graphicx}

%\usepackage{tikz}
%\usetikzlibrary {arrows.meta}
%\usepackage{wrapfig}	% required for `\wrapfigure' (yatex added)
%\usepackage{bm}	% required for `\bm' (yatex added)

% ルビを振る
%\usepackage{luatexja-ruby}	% required for `\ruby'

%% 核Ker 像Im Hom を定義
%\newcommand{\Img}{\mathop{\mathrm{Im}}\nolimits}
%\newcommand{\Ker}{\mathop{\mathrm{Ker}}\nolimits}
%\newcommand{\Hom}{\mathop{\mathrm{Hom}}\nolimits}

%\DeclareMathOperator{\Rot}{rot}
%\DeclareMathOperator{\Div}{div}
%\DeclareMathOperator{\Grad}{grad}
%\DeclareMathOperator{\arcsinh}{arcsinh}
%\DeclareMathOperator{\arccosh}{arccosh}
%\DeclareMathOperator{\arctanh}{arctanh}



\begin{document}

\textbf{群}

$G,H$を群とする。
$f_1,f_2$を準同型写像とする。
\begin{equation}
 f_1:G\to H ,\qquad f_2:H\to G
\end{equation}

$g\in G$の時、
$f_1(g) \in H$である。

$h\in H$の時、
$f_2(h) \in G$である。


$f_1(g) \in H$であるので、
$(f_1(g))^{-1} \in H$であり、
$f_1(g)\cdot (f_1(g))^{-1} = e_H \in H$となる。


$f_2(h) \in G$であるので、
$(f_2(h))^{-1} \in G$であり、
$f_2(h)\cdot (f_2(h))^{-1} = e_G \in G$となる。

演算はそれぞれの群の中で行われる。
その為、単位元はそれぞれの群の単位元となる。

準同型写像であるので、
$G$の単位元$e_G$と$H$の単位元$e_H$について
次のような関係がある。
\begin{equation}
 f_1(e_G)=e_H,\quad f_2(e_H)=e_G
\end{equation}




\end{document}

