\documentclass[12pt,b5paper]{ltjsarticle}

%\usepackage[margin=15truemm, top=5truemm, bottom=5truemm]{geometry}
\usepackage[margin=15truemm]{geometry}

\usepackage{amsmath,amssymb}
%\pagestyle{headings}
\pagestyle{empty}

%\usepackage{listings,url}
\renewcommand{\theenumi}{(\arabic{enumi})}

\usepackage{graphicx}

\usepackage{tikz}
\usetikzlibrary {arrows.meta}
\usepackage{wrapfig}	% required for `\wrapfigure' (yatex added)
\usepackage{bm}	% required for `\bm' (yatex added)
\usepackage{luatexja-ruby}	% required for `\ruby'
%% 像Im を定義
%\newcommand{\Img}{\mathop{\mathrm{Im}}\nolimits}

\begin{document}

実$n$次元ベクトル空間$\mathbb{R}^n$において
$\bm{a}=(a_1,\dots,a_n), \bm{b}=(b_1,\dots,b_n)$
とする。

\textbf{ベクトルの大きさ(絶対値、2乗ノルム)}

\begin{equation}
 \lvert \bm{a} \rvert = \sqrt{ a_1^2 + \cdots a_n^2}
\end{equation}

\textbf{ベクトルの内積(inner product, scalar product)}

ベクトル同士で内積を定義できる。
内積はスカラーになる。

\begin{equation}
 \langle \bm{a}, \bm{b} \rangle = a_1b_1 + \cdots a_nb_n
\end{equation}

\textbf{ベクトルの外積(vector product, cross product)}

$3$次元ベクトル空間$\mathbb{R}^3$においては外積が定義できる。
外積はベクトルになる。

基底ベクトル$\bm{e_1}, \bm{e_2}, \bm{e_3}$を用いると、
$\bm{a}=a_1\bm{e_1} + a_2\bm{e_2} + a_3\bm{e_3}$
と書ける。

多くの場合、
基底ベクトルの成分表示は
$\bm{e_1}=(1,0,0), \bm{e_2}=(0,1,0), \bm{e_3}=(0,0,1)$である。

これにより外積を定義する。
\begin{equation}
 \bm{a}\times\bm{b} =
  \begin{vmatrix}
   \bm{e_1} & \bm{e_2} & \bm{e_3}\\
   a_1 & a_2 & a_3\\
   b_1 & b_2 & b_3
  \end{vmatrix}
\end{equation}

外積の成分表示は次のようになる。
\begin{equation}
 \bm{a}\times\bm{b} = (a_2b_3-a_3b_2, \ a_3b_1-a_1b_3, \ a_1b_2-a_2b_1)
\end{equation}
これを外積の定義とする場合もある。


\hrulefill

\textbf{命題}

%\ruby{Cauchy}{コーシー}–
\ruby{Schwarz}{シュワルツ}の不等式
$
%\begin{equation}
 \lvert \langle \bm{a}, \bm{b} \rangle \rvert
  \leq
 \lvert \bm{a} \rvert \lvert \bm{b} \rvert
%\end{equation}
\quad (\bm{a},\bm{b}\in\mathbb{R}^n)
$
の等号が成立する時は以下の場合のみである。
\begin{enumerate}
 \item $\bm{a}=\bm{0}$ 又は $\bm{b}=\bm{0}$
 \item $\bm{a}=\alpha\bm{b} \quad (\alpha\in\mathbb{R},\alpha\ne 0)$
\end{enumerate}

\dotfill

シュワルツの不等式と必要十分な条件(同値な条件)を示す必要がある。

まず、次の条件が成り立つ時、シュワルツの不等式の等号が成り立つことを示す。
\begin{enumerate}
 \item $\bm{a}=\bm{0}$ 又は $\bm{b}=\bm{0}$
 \item $\bm{a}=\alpha\bm{b} \quad (\alpha\in\mathbb{R},\alpha\ne 0)$
\end{enumerate}

$\bm{a}=\bm{0}$の時
$\langle \bm{0}, \bm{b} \rangle =0, \quad
\lvert \bm{0} \rvert \lvert \bm{b} \rvert =0$
より成立。

$\bm{a}=\alpha\bm{b} \quad (\bm{a}\ne\bm{0}, \ \bm{b}\ne \bm{0},\ \alpha\ne 0)$
の時、
左辺は
$\langle \bm{a}, \alpha\bm{a} \rangle = \alpha \langle \bm{a}, \bm{a} \rangle$
となり、
右辺は
$\lvert \bm{a} \rvert \lvert \alpha\bm{a} \rvert
= \lvert \alpha \rvert \lvert \bm{a} \rvert^2$となる。
$\lvert \bm{a} \rvert = \sqrt{\langle \bm{a}, \bm{a} \rangle}$
であるから次の式が成り立つ。
\begin{equation}
 \lvert \langle \bm{a}, \bm{b} \rangle \rvert
  =\lvert \alpha \langle \bm{a}, \bm{a} \rangle \rvert
  = \lvert \alpha \rvert \lvert \bm{a} \rvert^2
  = \lvert \bm{a} \rvert \lvert \bm{b} \rvert
\end{equation}


逆に
$\lvert \langle \bm{a}, \bm{b} \rangle \rvert
  = \lvert \bm{a} \rvert \lvert \bm{b} \rvert$
の場合を考える。

両辺を2乗し移項すると
$\langle \bm{a}, \bm{b} \rangle^2
- \lvert \bm{a} \rvert^2 \lvert \bm{b} \rvert^2 =0$
となる。
これを変形する。
\begin{align}
 0 &= \langle \bm{a}, \bm{b} \rangle^2
 - \lvert \bm{a} \rvert^2 \lvert \bm{b} \rvert^2\\
 &= \langle \bm{a}, \bm{b} \rangle \langle \bm{a}, \bm{b} \rangle
 - \langle \bm{a}, \bm{a} \rangle \langle \bm{b}, \bm{b} \rangle\\
 & = \langle \langle \bm{a}, \bm{b} \rangle \bm{a}, \bm{b} \rangle
 - \langle \langle \bm{a}, \bm{a} \rangle \bm{b}, \bm{b} \rangle\\
 & = \langle \langle \bm{a}, \bm{b} \rangle \bm{a}
 - \langle \bm{a}, \bm{a} \rangle \bm{b}, \bm{b} \rangle
\end{align}

この変形によりベクトル$\langle \bm{a}, \bm{b} \rangle \bm{a}
 - \langle \bm{a}, \bm{a} \rangle \bm{b}$
とベクトル$\bm{b}$
の内積が$0$であることがわかり、
ここから$\bm{b}=\bm{0}$が得られる。
同様の議論をベクトルを入れ替え
$\lvert \langle \bm{b}, \bm{a} \rangle \rvert
  = \lvert \bm{b} \rvert \lvert \bm{a} \rvert$
として行うと$\bm{a}=\bm{0}$が得られる。

また、
ベクトル$\langle \bm{a}, \bm{b} \rangle \bm{a}
 - \langle \bm{a}, \bm{a} \rangle \bm{b}$が$\bm{0}$になる場合を考えると、
\begin{gather}
 \langle \bm{a}, \bm{b} \rangle \bm{a}- \langle \bm{a}, \bm{a} \rangle \bm{b} = \bm{0}\\
 \langle \bm{a}, \bm{b} \rangle \bm{a} = \langle \bm{a}, \bm{a} \rangle \bm{b}\\
 \bm{a} = \frac{\langle \bm{a}, \bm{a} \rangle}{\langle \bm{a}, \bm{b} \rangle } \bm{b}
\end{gather}
である。
$\langle \bm{a}, \bm{a} \rangle$ や $\langle \bm{a}, \bm{b} \rangle$
はスカラー(実数)であるので、
$\bm{a}=\alpha\bm{b} \ (\alpha\in\mathbb{R})$
となることがわかる。

これより
一方が零ベクトルの場合 又は 一次従属の場合
がわかる。

\hrulefill

\textbf{命題}

$\bm{a},\bm{b},\bm{c}\in\mathbb{R}^3$
$\alpha,\beta\in\mathbb{R}$
\begin{enumerate}
 \item $\bm{a}\times \bm{b} = -\bm{b}\times \bm{a}$
 \item
       $\bm{a}\times (\alpha\bm{b}+\beta\bm{c})
      = \alpha(\bm{a}\times \bm{b})
      +\beta(\bm{a}\times \bm{c})$
\end{enumerate}

\dotfill

ベクトルの外積

$\bm{a}\times \bm{b} = -\bm{b}\times \bm{a}$

\begin{equation}
 \bm{a}\times\bm{b}
 = \begin{vmatrix}
    \bm{e_1} & \bm{e_2} & \bm{e_3}\\
    a_1 & a_2 & a_3\\
    b_1 & b_2 & b_3
   \end{vmatrix}
 = - \begin{vmatrix}
    \bm{e_1} & \bm{e_2} & \bm{e_3}\\
    b_1 & b_2 & b_3\\
    a_1 & a_2 & a_3
   \end{vmatrix}
 = - \bm{b}\times\bm{a}
\end{equation}

成分表示で計算した場合次のようになる。
\begin{gather}
 \bm{a}\times\bm{b} = (a_2b_3-a_3b_2, \ a_3b_1-a_1b_3, \ a_1b_2-a_2b_1)\\
 \bm{b}\times\bm{a} = (b_2a_3-b_3a_2, \ b_3a_1-b_1a_3, \ b_1a_2-b_2a_1)
\end{gather}
これにより
$\bm{a}\times \bm{b} = -\bm{b}\times \bm{a}$
である。

$\bm{a}\times (\alpha\bm{b}+\beta\bm{c})
 = \alpha(\bm{a}\times \bm{b})
 + \beta(\bm{a}\times \bm{c})$

\begin{align}
 \bm{a}\times (\alpha\bm{b}+\beta\bm{c})
 &= \begin{vmatrix}
    \bm{e_1} & \bm{e_2} & \bm{e_3}\\
    a_1 & a_2 & a_3\\
    \alpha b_1 + \beta c_1 & \alpha b_2 + \beta c_2 & \alpha b_3 + \beta c_3
   \end{vmatrix}\\
 &= \begin{vmatrix}
    \bm{e_1} & \bm{e_2} & \bm{e_3}\\
    a_1 & a_2 & a_3\\
    \alpha b_1 & \alpha b_2 & \alpha b_3
   \end{vmatrix}
 + \begin{vmatrix}
    \bm{e_1} & \bm{e_2} & \bm{e_3}\\
    a_1 & a_2 & a_3\\
    \beta c_1 & \beta c_2 & \beta c_3
   \end{vmatrix}\\
 &= \alpha
   \begin{vmatrix}
    \bm{e_1} & \bm{e_2} & \bm{e_3}\\
    a_1 & a_2 & a_3\\
    b_1 & b_2 & b_3
   \end{vmatrix}
 + \beta
   \begin{vmatrix}
    \bm{e_1} & \bm{e_2} & \bm{e_3}\\
    a_1 & a_2 & a_3\\
    c_1 & c_2 & c_3
   \end{vmatrix}\\
 &=\alpha(\bm{a}\times \bm{b}) + \beta(\bm{a}\times \bm{c})
\end{align}

成分表示で両辺を計算しても示すことが出来る。



\end{document}
