\documentclass[12pt,b5paper]{ltjsarticle}

%\usepackage[margin=15truemm, top=5truemm, bottom=5truemm]{geometry}
\usepackage[margin=15truemm]{geometry}

\usepackage{amsmath,amssymb}
%\pagestyle{headings}
\pagestyle{empty}

%\usepackage{listings,url}
\renewcommand{\theenumi}{(\arabic{enumi})}

\usepackage{graphicx}

\usepackage{tikz}
\usetikzlibrary {arrows.meta}
\usepackage{bm}	% required for `\bm' (yatex added)
\usepackage{luatexja-ruby}      % required for `\ruby'
%% 像Im を定義
%\newcommand{\Img}{\mathop{\mathrm{Im}}\nolimits}

\begin{document}

\hrulefill

\begin{enumerate}
 \item 行列式が$1$の$2$次直交行列を全て求めよ。
 \item ユニタリ行列の固有値の絶対値は$1$であることを証明せよ。
\end{enumerate}

\textbf{直交行列}

行列$A$について、
逆行列$A^{-1}$と転置行列${}^{t}\!\!A$が等しい時、
$A$を直交行列をいう。
${}^{t}\!\!AA=E$

\textbf{\ruby{unitary}{ユニタリ}行列}

行列$A$について、
逆行列$A^{-1}$と随伴行列$A^{*}(={}^{t}\!\bar{A})$が等しい時、
$A$をユニタリ行列をいう。
$A^{*}A=E$

\hrulefill

\begin{enumerate}
 \item 行列式が$1$の$2$次直交行列を全て求めよ。

       行列$A$を次のように置く。
       \begin{equation}
        A=
         \begin{pmatrix}
          a & b\\c& d
         \end{pmatrix}
       \end{equation}

       行列式が$1$なので、$\lvert A \rvert = ad-bc=1$である。

       逆行列$A^{-1}$と転置行列${}^{t}\!A$は次のようになる。
       \begin{equation}
        A^{-1}=\frac{1}{\lvert A \rvert}\begin{pmatrix}d & -c\\-b& a\end{pmatrix}
        = \begin{pmatrix}d & -c\\-b& a\end{pmatrix},
        \qquad
        {}^{t}\!A=\begin{pmatrix}a & c\\b& d\end{pmatrix}
       \end{equation}
       $A$は直交行列であるから$A^{-1}={}^{t}\!A$である。
       これより次の4つの式を得る。
       \begin{equation}
        d=a, \quad -c=c, \quad -b=b, \quad a=d
       \end{equation}
       つまり、
       $a=d,\ b=c=0$である。
       また、行列式が1なので、$ad=1$となり、$a=d=\pm 1$である。

       よって、行列式が1となる直交行列は次の2つである。
       \begin{equation}
        \begin{pmatrix}1 & 0\\0& 1\end{pmatrix},
        \quad
        \begin{pmatrix}-1 & 0\\0& -1\end{pmatrix}
       \end{equation}

       \dotfill

 \item ユニタリ行列の固有値の絶対値は$1$であることを証明せよ。

       行列$A$をユニタリ行列とし、
       $A$の固有値を$\lambda$、
       固有ベクトルを$\bm{x}$とする。
       固有値の定義より$A\bm{x}=\lambda\bm{x}$である。
       ベクトルの大きさを取ると次のようになる。
       \begin{align}
        \lvert A\bm{x}\rvert &= \lvert\lambda\bm{x}\rvert
        = \lvert\lambda\rvert\lvert\bm{x}\rvert
       \end{align}

       この左辺は2乗すると次のように変形できる。
       \begin{align}
        \lvert A\bm{x}\rvert^2 &= (A\bm{x})^{*}A\bm{x}\\
        &= \bm{x}^{*}A^{*}A\bm{x}\\
        &= \bm{x}^{*}E\bm{x} = \lvert \bm{x} \rvert^2
       \end{align}

       ここから
       $\lvert\lambda\rvert^2 \lvert\bm{x}\rvert^2 = \lvert \bm{x} \rvert^2$
       が得られる。
       これにより$\lvert \lambda \rvert =1$となる。
\end{enumerate}




\end{document}
