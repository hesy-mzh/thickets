\documentclass[12pt,b5paper]{ltjsarticle}

%\usepackage[margin=15truemm, top=5truemm, bottom=5truemm]{geometry}
\usepackage[margin=15truemm]{geometry}

\usepackage{amsmath,amssymb}
%\pagestyle{headings}
\pagestyle{empty}

%\usepackage{listings,url}
%\renewcommand{\theenumi}{(\arabic{enumi})}

\usepackage{graphicx}

\usepackage{tikz}
\usetikzlibrary {arrows.meta}
\usepackage{wrapfig}	% required for `\wrapfigure' (yatex added)
\usepackage{bm}	% required for `\bm' (yatex added)

% ルビを振る
%\usepackage{luatexja-ruby}	% required for `\ruby'

%% 核Ker 像Im Hom を定義
%\newcommand{\Img}{\mathop{\mathrm{Im}}\nolimits}
%\newcommand{\Ker}{\mathop{\mathrm{Ker}}\nolimits}
%\newcommand{\Hom}{\mathop{\mathrm{Hom}}\nolimits}
%\newcommand{\Rot}{\mathop{\mathrm{rot}}\nolimits}
%\newcommand{\Div}{\mathop{\mathrm{div}}\nolimits}
%\newcommand{\Grad}{\mathop{\mathrm{grad}}\nolimits}

\DeclareMathOperator{\Rot}{rot}
\DeclareMathOperator{\arcsinh}{arcsinh}

\begin{document}



\textbf{ベクトル解析}

\hrulefill

\begin{enumerate}
 \item
      $\mathbb{R}^2$内の開集合$U=\mathbb{R}^2\backslash \{0\}$上の
      $C^1$-級ベクトル場$\bm{f}:U\to\mathbb{R}^2$を次のように定める。
      \begin{equation}
       \bm{f}(x,y)={}^{t}\begin{pmatrix} \frac{-y}{\sqrt{x^2+y^2}} & \frac{x}{\sqrt{x^2+y^2}} \end{pmatrix}
      \end{equation}
       \begin{enumerate}
        \item
             $R>0$とし、
             $C^1$-級閉曲線$C_{R}:[0,2\pi]\to\mathbb{R}^2$を
             $C_{R}(t)=(R\cos t, R\sin t)$と定める。
             $\bm{f}$の$C_R$に沿った次の線積分を求めよ。
             \begin{equation}
              \int_{C_R}\bm{f}
             \end{equation}


        \item
             積分定理が使える領域$D_{R,r}\ (0<r<R)$を
             $D_{R,r}=\{(x,y)\in\mathbb{R}^2 \mid r^2\leq x^2+y^2\leq R^2\}$
             とする。
             この時次の値を求めよ。
             \begin{equation}
              \int_{D_{R,r}}\Rot\bm{f}\mathrm{d}x\mathrm{d}y
             \end{equation}

        \item
             次の命題(P)の真偽を評価し、その証明を与えよ。
             \begin{center}
              命題(P): ある$C^2$-級関数$g:U\to\mathbb{R}$が存在して
              $\bm{f}=\nabla g$となる
             \end{center}

       \end{enumerate}

      \dotfill

      \begin{enumerate}
       \item
            次のように$x,y$をおく。
            \begin{align}
             x =& R\cos t & y =& R\sin t\\
             \frac{\mathrm{d}x}{\mathrm{d}t} =& -R\sin t & \frac{\mathrm{d}y}{\mathrm{d}t} =& R\cos t
            \end{align}

            これを使うと$\bm{f}$は次のように$t$の関数となる。
            \begin{equation}
             \bm{f}(x,y)={}^{t}\begin{pmatrix} \frac{-y}{\sqrt{x^2+y^2}} & \frac{x}{\sqrt{x^2+y^2}} \end{pmatrix}
             ={}^{t}\begin{pmatrix} -\sin t & \cos t \end{pmatrix}
            \end{equation}

            その為、積分の値は次のように求められる。
            \begin{align}
             \int_{C_R}\bm{f} =& \int_{0}^{2\pi}
             \begin{pmatrix} -\sin t \\ \cos t \end{pmatrix}\cdot
             \begin{pmatrix} \frac{\mathrm{d}x}{\mathrm{d}t} \\ \frac{\mathrm{d}y}{\mathrm{d}t} \end{pmatrix}
             \mathrm{d}t\\
             =& \int_{0}^{2\pi} (R\sin^2 t + R\cos^2 t) \mathrm{d}t
             = [Rt]_{t=0}^{t=2\pi} =2\pi R
            \end{align}

       \item
            ストークスの定理より次の式が成り立つ。
            \begin{equation}
             \int_{D_{R,r}}\Rot\bm{f}\mathrm{d}x\mathrm{d}y
              = \int_{\partial D_{R,r}}\bm{f}\mathrm{d}s
            \end{equation}

            $\partial D_{R,r}$は2つの円からなる。
            半径$R$の円を$S_R$、$r$の円を$S_r$とすると2つの積分に分かれる。
            \begin{equation}
             \int_{\partial D_{R,r}}\bm{f}\mathrm{d}s
              = \int_{S_{R}}\bm{f}\mathrm{d}s
              - \int_{S_{r}}\bm{f}\mathrm{d}s
            \end{equation}

            先程の問と同じようにして次の積分が求まる。
            \begin{equation}
             \int_{S_{R}}\bm{f}\mathrm{d}s = 2\pi R
             ,\qquad
             \int_{S_{r}}\bm{f}\mathrm{d}s = 2\pi r
            \end{equation}

            これにより積分値は次のように求まる。
            \begin{equation}
             \int_{D_{R,r}}\Rot\bm{f}\mathrm{d}x\mathrm{d}y
              = 2\pi R - 2\pi r
            \end{equation}

       \item
            次の2つの関数の偏微分がある。
            \begin{align}
             \frac{\partial}{\partial x}
              y\arcsinh\left( \frac{x}{\sqrt{y^2}} \right)
              =& y \cdot \frac{1}{\sqrt{1+\frac{x^2}{y^2}}} \cdot \frac{1}{\sqrt{y^2}}
              = \frac{y}{\sqrt{y^2+x^2}}\\
             \frac{\partial}{\partial y}
              x\arcsinh\left( \frac{y}{\sqrt{x^2}} \right)
              =& x \cdot \frac{1}{\sqrt{1+\frac{y^2}{x^2}}} \cdot \frac{1}{\sqrt{x^2}}
              = \frac{x}{\sqrt{x^2+y^2}}
            \end{align}

            これより関数$g$が次の2つを満たすことはない。
            \begin{equation}
             \frac{\partial g}{\partial x} = \frac{-y}{\sqrt{y^2+x^2}}
              ,\qquad
             \frac{\partial g}{\partial y} = \frac{x}{\sqrt{y^2+x^2}}
            \end{equation}

            よって、$\bm{f}=\nabla g$となる関数$g$は存在しない。


      \end{enumerate}


      \hrulefill

      \newpage

 \item
      $f:\mathbb{R}^2\to\mathbb{R}$を$C^3$-級関数とし、
      $C^2$-級の関数$g_i :\mathbb{R}^2\to\mathbb{R}\ (i=1,2)$を
      
\end{enumerate}


\end{document}
