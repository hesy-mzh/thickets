\documentclass[12pt,b5paper]{ltjsarticle}

%\usepackage[margin=15truemm, top=5truemm, bottom=5truemm]{geometry}
\usepackage[margin=10truemm]{geometry}

\usepackage{amsmath,amssymb}
%\pagestyle{headings}
\pagestyle{empty}

%\usepackage{listings,url}
%\renewcommand{\theenumi}{(\arabic{enumi})}

%\usepackage{graphicx}

%\usepackage{tikz}
%\usetikzlibrary {arrows.meta}
%\usepackage{wrapfig}	% required for `\wrapfigure' (yatex added)
%\usepackage{bm}	% required for `\bm' (yatex added)

% ルビを振る
%\usepackage{luatexja-ruby}	% required for `\ruby'

%% 核Ker 像Im Hom を定義
%\newcommand{\Img}{\mathop{\mathrm{Im}}\nolimits}
%\newcommand{\Ker}{\mathop{\mathrm{Ker}}\nolimits}
%\newcommand{\Hom}{\mathop{\mathrm{Hom}}\nolimits}

%\DeclareMathOperator{\Rot}{rot}
%\DeclareMathOperator{\Div}{div}
%\DeclareMathOperator{\Grad}{grad}
%\DeclareMathOperator{\arcsinh}{arcsinh}
%\DeclareMathOperator{\arccosh}{arccosh}
%\DeclareMathOperator{\arctanh}{arctanh}



%\usepackage{listings,url}
%
%\lstset{
%%プログラム言語(複数の言語に対応,C,C++も可)
%  language = Python,
%%  language = Lisp,
%%  language = C,
%  %背景色と透過度
%  %backgroundcolor={\color[gray]{.90}},
%  %枠外に行った時の自動改行
%  breaklines = true,
%  %自動改行後のインデント量(デフォルトでは20[pt])
%  breakindent = 10pt,
%  %標準の書体
%%  basicstyle = \ttfamily\scriptsize,
%  basicstyle = \ttfamily,
%  %コメントの書体
%%  commentstyle = {\itshape \color[cmyk]{1,0.4,1,0}},
%  %関数名等の色の設定
%  classoffset = 0,
%  %キーワード(int, ifなど)の書体
%%  keywordstyle = {\bfseries \color[cmyk]{0,1,0,0}},
%  %表示する文字の書体
%  %stringstyle = {\ttfamily \color[rgb]{0,0,1}},
%  %枠 "t"は上に線を記載, "T"は上に二重線を記載
%  %他オプション:leftline,topline,bottomline,lines,single,shadowbox
%  frame = TBrl,
%  %frameまでの間隔(行番号とプログラムの間)
%  framesep = 5pt,
%  %行番号の位置
%  numbers = left,
%  %行番号の間隔
%  stepnumber = 1,
%  %行番号の書体
%%  numberstyle = \tiny,
%  %タブの大きさ
%  tabsize = 4,
%  %キャプションの場所("tb"ならば上下両方に記載)
%  captionpos = t
%}



\begin{document}


\hrulefill
\textbf{定義}
\hrulefill

\textbf{ゾルゲンフライ直線}

実数$\mathbb{R}$の開集合$U$を$p,q\in\mathbb{R} \ ( p<q )$を用いて
次のように定義する。
\begin{equation}
 U = (p,q]
  \label{openset}
\end{equation}

$U$は左半開区間である。

左半開区間全体の集合を開集合の基底として
$\mathbb{R}$に位相を導入する。
この位相空間$(\mathbb{R},\{U\})$を
ゾルゲンフライ直線 (Sorgenfrey line)
という。

式(\ref{openset})の開集合を
右半開区間$[p,q)$とする定義もある。



\textbf{第二可算公理}

開集合族に
高々可算な基底が存在する
とき、
第二可算公理を満たすという。



\hrulefill
\textbf{問題}
\hrulefill


\begin{enumerate}
 \item
      \
      [距離空間上の閉集合]

      距離空間$(X,d)$の部分集合$F$が閉集合であるための必要十分条件は
      $X$の点$x$に収束する$F$からなる任意の点列において、
      $x\in F$になることであることを示せ。

      \dotfill

      \begin{equation}
       F:\text{閉集合}
        \Rightarrow
        \lim_{n\to \infty}x_n \in F
      \end{equation}

      $F$を閉集合、$n\in\mathbb{N}$に対し$x_n\in F$とする。
      この時、点列$\{x_n\}$の極限が$x$であるとは、
      次を満たす事をいう。
      \begin{center}
       任意の$\varepsilon>0$に対し、
       次を満たすような
       $N_0\in\mathbb{N}$が
       存在する。
       \begin{equation}
        N>N_0 \Rightarrow d(x,x_N) <\varepsilon
         \label{lim}
       \end{equation}
      \end{center}


      $F$は閉集合であるので補集合$X\backslash F$は開集合である。
      開集合であれば、任意の点$p\in X\backslash F$に対して
      $\varepsilon$近傍$U_{\varepsilon}$が存在し、
      $U_{\varepsilon}\subset X\backslash F$となる。

      $\{x_n\}$の極限$x$は
      任意の$\varepsilon>0$に対し、
      $d(x,x_N)<\varepsilon$
      となる$x_N\in F$が存在するので、
      $x\not\in X\backslash F$である。
      つまり、$x\in F$となる。


      \begin{equation}
       F:\text{閉集合}
        \Leftarrow
        \lim_{n\to \infty}x_n \in F
      \end{equation}

      $n\in\mathbb{N}$に対し$x_n\in F$であり、
      $\displaystyle x=\lim_{n\to\infty}x_n\in F$とする。

      式(\ref{lim})より
      極限$x$の$\varepsilon$近傍$U_{\varepsilon}$
      には$x_{N}\in U_{\varepsilon}$となる
      $x_{N}\in F$が存在する。
      これは任意の$\varepsilon$に対して成り立つ為、
      $U_{\varepsilon}\cap F\ne\emptyset$となり
      $x\in F$が触点であることとなる。

      任意の点列$\{x_n\}$の極限が$F$に含まれるのであれば
      $F$の閉包$\bar{F}$が$F$と一致する。
      つまり、$F$は閉集合である。


      \hrulefill

 \item
      \
      [可算公理]

      ゾルゲンフライ直線$\mathbb{R}_{l}$は
      第二可算公理を満たさないことを示せ。

      \dotfill

      第二可算公理を満たすと仮定する。

      開基
      $\mathfrak{O}$
      が可算であるとする。

      $\mathbb{R}_{l}$の開集合$(x-1,x]$に対して
      これに含まれる開集合
      $O_{x}\in\mathfrak{O}$
      を$x$が含まれるように一つ選択する。
      つまり、$x\in O_{x} \subset (x-1,x]$である。

      この開集合$O_{x}$を
      $\mathbb{R}_{l}$
      の点ごとに選び、
      集合族$\mathfrak{A}$
      を作る。
      \begin{equation}
       \mathfrak{A}=
        \{ O_{x}\in\mathfrak{O} \mid
        x\in\mathbb{R},\ x\in O_{x} \subset (x-1,x] \}
      \end{equation}

      $\mathfrak{A} \subset \mathfrak{O}$
      であるので、
      それぞれの濃度は
      $\lvert \mathfrak{A} \rvert \leq \lvert \mathfrak{O} \rvert$
      である。

      $x,y\in\mathbb{R}$が$x<y$とした時、
      $x\in O_{x} \subset (x-1,x]$である為、
      $y\not\in O_{x}$となる。
      つまり、$x<y$であれば$O_{x}\ne O_{y}$である。

      これにより次の写像が単射であることがわかる。
      \begin{equation}
       f:\mathbb{R} \to \mathfrak{A}
        \quad
        f(x) = O_{x}
      \end{equation}

      これにより集合の濃度は
      $\lvert \mathbb{R} \rvert \leq \lvert \mathfrak{A} \rvert$
      である。
      つまり、$\mathfrak{O}$が可算であることに矛盾する。

      よって、第二可算公理を
      $\mathbb{R}_{l}$は
      満たさないことがわかる。

      \hrulefill

\end{enumerate}


\end{document}
