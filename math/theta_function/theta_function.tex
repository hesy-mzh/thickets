\documentclass[12pt,b5paper]{ltjsarticle}

%\usepackage[margin=15truemm, top=5truemm, bottom=5truemm]{geometry}
%\usepackage[margin=10truemm,left=15truemm]{geometry}
\usepackage[margin=10truemm]{geometry}

\usepackage{amsmath,amssymb}
%\pagestyle{headings}
\pagestyle{empty}

%\usepackage{listings,url}
%\renewcommand{\theenumi}{(\arabic{enumi})}

%\usepackage{graphicx}

%\usepackage{tikz}
%\usetikzlibrary {arrows.meta}
%\usepackage{wrapfig}
%\usepackage{bm}

% ルビを振る
%\usepackage{luatexja-ruby}	% required for `\ruby'

%% 核Ker 像Im Hom を定義
%\newcommand{\Img}{\mathop{\mathrm{Im}}\nolimits}
%\newcommand{\Ker}{\mathop{\mathrm{Ker}}\nolimits}
%\newcommand{\Hom}{\mathop{\mathrm{Hom}}\nolimits}

%\DeclareMathOperator{\Rot}{rot}
%\DeclareMathOperator{\Div}{div}
%\DeclareMathOperator{\Grad}{grad}
%\DeclareMathOperator{\arcsinh}{arcsinh}
%\DeclareMathOperator{\arccosh}{arccosh}
%\DeclareMathOperator{\arctanh}{arctanh}

%\usepackage{url}

%\usepackage{listings}
%
%\lstset{
%%プログラム言語(複数の言語に対応,C,C++も可)
%  language = Python,
%%  language = Lisp,
%%  language = C,
%  %背景色と透過度
%  %backgroundcolor={\color[gray]{.90}},
%  %枠外に行った時の自動改行
%  breaklines = true,
%  %自動改行後のインデント量(デフォルトでは20[pt])
%  breakindent = 10pt,
%  %標準の書体
%%  basicstyle = \ttfamily\scriptsize,
%  basicstyle = \ttfamily,
%  %コメントの書体
%%  commentstyle = {\itshape \color[cmyk]{1,0.4,1,0}},
%  %関数名等の色の設定
%  classoffset = 0,
%  %キーワード(int, ifなど)の書体
%%  keywordstyle = {\bfseries \color[cmyk]{0,1,0,0}},
%  %表示する文字の書体
%  %stringstyle = {\ttfamily \color[rgb]{0,0,1}},
%  %枠 "t"は上に線を記載, "T"は上に二重線を記載
%  %他オプション:leftline,topline,bottomline,lines,single,shadowbox
%  frame = TBrl,
%  %frameまでの間隔(行番号とプログラムの間)
%  framesep = 5pt,
%  %行番号の位置
%  numbers = left,
%  %行番号の間隔
%  stepnumber = 1,
%  %行番号の書体
%%  numberstyle = \tiny,
%  %タブの大きさ
%  tabsize = 4,
%  %キャプションの場所("tb"ならば上下両方に記載)
%  captionpos = t
%}

%\usepackage{cancel}
%\usepackage{bussproofs}
%\usepackage{proof}

\begin{document}


\hrulefill

\begin{enumerate}
 \item 加法定理
      \begin{equation}
       \begin{split}
        f(x-u)f(x+u)f(y-v) & f(y+v)
        -f(x-v)f(x+v)f(y-u)f(y+u)\\
        =& f(x-y)f(x+y)f(u-v)f(u+v)
       \end{split}
       \label{eq:add}
      \end{equation}
      $f(x)$が次の式である時、
      上の加法定理\eqref{eq:add}が成立することを示せ。
      \begin{enumerate}
       \item $f(x)=x$
       \item $f(x)=\sin{x}$
       \item $f(x)=\theta_{1}(x \mid \tau)$
      \end{enumerate}

      \hrulefill

      \begin{enumerate}
       \item $f(x)=x$

             \dotfill

             \vspace{-30pt}
             \begin{align}
              (左辺) =&\; (x-u)(x+u)(y-v)(y+v)
              - (x-v)(x+v)(y-u)(y+u)\\
              =&\; \left(x^{2}-u^{2}\right) \left(y^{2}-v^{2}\right)
              - \left(x^{2}-v^{2}\right) \left(y^{2}-u^{2}\right)\\
              =&\; - x^{2}v^{2} - y^{2}u^{2} + x^{2}u^{2} + y^{2}v^{2}\\
              =&\; x^{2}(u^{2}-v^{2}) - y^{2}(u^{2}-v^{2})\\
              =&\; (x^{2}-y^{2})(u^{2}-v^{2})\\
              =&\; (x-y)(x+y)(u-v)(u+v) = (右辺)
             \end{align}

             \hrulefill

       \item $f(x)=\sin{x}$

             \dotfill

             $\sin{\alpha}$と$\sin{\beta}$の積の公式
             \begin{equation}
              \sin{\alpha}\sin{\beta}
               =\frac{1}{2}\left(\cos{(\alpha-\beta)}
                            -\cos{(\alpha+\beta)}\right)
             \end{equation}

             \vspace{-30pt}
             \begin{gather}
              \sin{(x-u)}\sin{(x+u)}
              =\frac{1}{2}\left( \cos{2u} -\cos{2x} \right)\\
              \sin{(y-v)}\sin{(y+v)}
              =\frac{1}{2}\left( \cos{2v} -\cos{2y} \right)\\
              \sin{(x-v)}\sin{(x+v)}
              =\frac{1}{2}\left( \cos{2v} -\cos{2x} \right)\\
              \sin{(y-u)}\sin{(y+u)}
              =\frac{1}{2}\left( \cos{2u} -\cos{2y} \right)\\
              \sin{(x-y)}\sin{(x+y)}
              =\frac{1}{2}\left( \cos{2y} -\cos{2x} \right)\\
              \sin{(u-v)}\sin{(u+v)}
              =\frac{1}{2}\left( \cos{2v} -\cos{2u} \right)
             \end{gather}



             \begin{align}
              (左辺) =&\; \frac{1}{4}(\cos{2u} -\cos{2x})(\cos{2v} -\cos{2y})\\
              & \qquad - \frac{1}{4}(\cos{2v} -\cos{2x})(\cos{2u} -\cos{2y})\\
              =&\; \frac{1}{4} (-\cos{2x}\cos{2v}-\cos{2y}\cos{2u}
              +\cos{2x}\cos{2u}+\cos{2y}\cos{2v})\\
              =&\; \frac{1}{4} (\cos{2x}-\cos{2y})(\cos{2u}-\cos{2v})\\
              =&\; \sin{(x-y)}\sin{(x+y)}\sin{(u-v)}\sin{(u+v)}
              =(右辺)
             \end{align}

             \hrulefill

       \item $f(x)=\theta_{1}(x \mid \tau)$

             \dotfill

             \hrulefill

      \end{enumerate}


 \item
      テータ関数
       \begin{gather}
        \vartheta \begin{bmatrix} a \\ b \end{bmatrix} (u \mid \tau)
        = \sum_{n\in\mathbb{Z}} \mathbf{e}
%        \overset{\mathrm{def}}{=} \sum_{n\in\mathbb{Z}} \mathbf{e}
         \left[ (n+a)^{2}\tau + 2(n+a)(u+b) \right]\\
         \left(
        a,b\in\mathbb{R},\
        u\in\mathbb{C},\;
        \tau\in\{z\in\mathbb{C}\mid \mathrm{Im}z \geq 0\},\
        \mathbf{e}[x] \overset{\mathrm{def}}{=} %\exp{\sqrt{-\pi}x} \ (
        \exp{\sqrt{-1}\pi x}% ?)
        \right)
       \end{gather}
       の変換性(準周期性)は
       \begin{align}
        \vartheta \begin{bmatrix} a \\ b \end{bmatrix} (u+1 \mid \tau)
        =&\; \mathbf{e} [2a]
        \vartheta \begin{bmatrix} a \\ b \end{bmatrix} (u \mid \tau)\\
        \vartheta \begin{bmatrix} a \\ b \end{bmatrix} (u+\tau \mid \tau)
        =&\; \mathbf{e} [-\tau -2(u+b)]
        \vartheta \begin{bmatrix} a \\ b \end{bmatrix} (u \mid \tau)
       \end{align}
       であり、
       $\displaystyle \vartheta \begin{bmatrix} a \\ b \end{bmatrix} (u \mid \tau) =0$ の
       $\bmod{(\mathbb{Z} \oplus \tau\mathbb{Z})}$での
       zero点の個数は 1個で、
       それは
       $\displaystyle \frac{1}{2} - \frac{a}{2} + \left( \frac{1}{2}-\frac{b}{2} \right)\tau \mod{(\mathbb{Z} \oplus \tau\mathbb{Z})}$であった。

       テータ関数の記号は次のものを用いる。
       \begin{align}
        \theta_{1}(u\mid\tau) =&\; -\vartheta \begin{bmatrix} 1/2 \\ 1/2 \end{bmatrix} (u+\tau \mid \tau)
        , &
        \theta_{2}(u\mid\tau) =&\; \vartheta \begin{bmatrix} 1/2 \\ 0 \end{bmatrix} (u+\tau \mid \tau)\\
        \theta_{3}(u\mid\tau) =&\; \vartheta \begin{bmatrix} 0 \\ 0 \end{bmatrix} (u+\tau \mid \tau)
        , &
        \theta_{4}(u\mid\tau) =&\; \vartheta \begin{bmatrix} 0 \\ 1/2 \end{bmatrix} (u+\tau \mid \tau)
       \end{align}


       \begin{enumerate}
        \item テータ関数の定義を用いて次を表わせ。

              \dotfill

              \begin{equation}
               \alpha\in\mathbb{R}に対して
                \quad
                \vartheta \begin{bmatrix} a \\ b \end{bmatrix} (u+\alpha \mid \tau)
                = \vartheta \begin{bmatrix} a \\ b+\alpha \end{bmatrix} (u \mid \tau)
              \end{equation}

              \dotfill
             \begin{enumerate}
              \item $\displaystyle \theta_{1} \left(u+\frac{1}{2} \middle| \tau \right)$
                    を
                    $\displaystyle \theta_{2} \left(u \mid \tau \right)$
                    を用いて表わせ。

                    \dotfill

                    \begin{align}
                     \theta_{1} \left(u+\frac{1}{2} \middle| \tau \right)
                      =&\; -\vartheta \begin{bmatrix} 1/2 \\ 1/2 \end{bmatrix} (u+\frac{1}{2}+\tau \mid \tau)\\
                      =&\; -\vartheta \begin{bmatrix} 1/2 \\ 0 \end{bmatrix} (u+1+\tau \mid \tau)\\
                      =&\; - \mathbf{e}[1] \vartheta \begin{bmatrix} 1/2 \\ 0 \end{bmatrix} (u+\tau \mid \tau)\\
                     =&\; -\mathbf{e}[1]\theta_{2}(u\mid\tau)
                    \end{align}

                    \hrulefill


              \item $\displaystyle \theta_{2} \left(u+\frac{1}{2} \middle| \tau \right)$
                    を
                    $\displaystyle \theta_{1} \left(u \middle| \tau \right)$
                    を用いて表わせ。

                    \dotfill

                    \begin{align}
                     \theta_{2} \left(u+\frac{1}{2} \middle| \tau \right)
                      =&\; \vartheta \begin{bmatrix} 1/2 \\ 0 \end{bmatrix} (u+\frac{1}{2}+\tau \mid \tau)\\
                      =&\; \vartheta \begin{bmatrix} 1/2 \\ 1/2 \end{bmatrix} (u+\tau \mid \tau)\\
                     =&\; -\theta_{1}(u\mid\tau)
                    \end{align}

                    \hrulefill


              \item $\displaystyle \theta_{3} \left(u+\frac{1}{2} \middle| \tau \right)$
                    を
                    $\displaystyle \theta_{4} \left(u \middle| \tau \right)$
                    を用いて表わせ。

                    \dotfill

                    \begin{align}
                       \theta_{3}(u+\frac{1}{2}\mid\tau)
                         =&\; \vartheta \begin{bmatrix} 0 \\ 0 \end{bmatrix} (u+\frac{1}{2}+\tau \mid \tau)\\
                         =&\; \vartheta \begin{bmatrix} 0 \\ 1/2 \end{bmatrix} (u+\tau \mid \tau)\\
                       =&\; \theta_{4}(u\mid\tau)
                    \end{align}

                    \hrulefill

              \item $\displaystyle \theta_{4} \left(u+\frac{1}{2} \middle| \tau \right)$
                    を
                    $\displaystyle \theta_{3} \left(u \middle| \tau \right)$
                    を用いて表わせ。

                    \dotfill

                    \begin{align}
                       \theta_{4}(u+\frac{1}{2}\mid\tau)
                         =&\; \vartheta \begin{bmatrix} 0 \\ 1/2 \end{bmatrix} (u+\frac{1}{2}+\tau \mid \tau)\\
                         =&\; \vartheta \begin{bmatrix} 0 \\ 0 \end{bmatrix} (u+1+\tau \mid \tau)\\
                       =&\; \mathbf{e}[0] \vartheta \begin{bmatrix} 0 \\ 0 \end{bmatrix} (u+\tau \mid \tau)\\
                       =&\; \theta_{3}(u\mid\tau)
                    \end{align}

                    \hrulefill
             \end{enumerate}

        \item テータ関数の定義を用いて次を表わせ。

              \dotfill

              \begin{align}
                \vartheta \begin{bmatrix} a \\ b \end{bmatrix} (u \mid \tau)
                =& \sum_{n\in\mathbb{Z}} \mathbf{e}
                 \left[ (n+a)^{2}\tau + 2(n+a)(u+b) \right]\\
                =& \sum_{n\in\mathbb{Z}} \mathbf{e}
                 \left[ n^{2}\tau +2n(u+a\tau+b)+ a^{2}\tau + 2a(u+b) \right]\\
                =& \mathbf{e}[a^{2}\tau + 2a(u+b) ] \sum_{n\in\mathbb{Z}}
                  \mathbf{e} [ n^{2}\tau +2n(u+a\tau+b)]\\
                =& \mathbf{e}[a^{2}\tau + 2a(u+b) ]
                \vartheta \begin{bmatrix} 0 \\ b \end{bmatrix} (u+a\tau \mid \tau)
              \end{align}

              \dotfill
             \begin{enumerate}
              \item $\displaystyle \theta_{1} \left(u+\frac{\tau}{2} \middle| \tau \right)$
                    を
                    $\displaystyle \theta_{4} \left(u \middle| \tau \right)$
                    を用いて表わせ。

                    \dotfill

                    \begin{align}
                     & \theta_{1}(u + \frac{\tau}{2} \mid \tau)\\
                        =&\; -\vartheta \begin{bmatrix} 1/2 \\ 1/2 \end{bmatrix} (u+\frac{3}{2}\tau \mid \tau)\\
                        =&\; -\mathbf{e}[\frac{\tau}{4}+u+\frac{3}{2}\tau+\frac{1}{2}]
                          \vartheta \begin{bmatrix} 0 \\ 1/2 \end{bmatrix} (u+\frac{1}{2}\tau+\frac{3}{2}\tau \mid \tau)\\
                        =&\; -\mathbf{e}[\frac{\tau}{4}+u+\frac{3}{2}\tau+\frac{1}{2}]
                     \mathbf{e}[-\tau-2(u+\tau+\frac{1}{2})]
                     \vartheta \begin{bmatrix} 0 \\ 1/2 \end{bmatrix} (u+\tau \mid \tau)\\
                        =&\; -\mathbf{e}[-\frac{5}{4}\tau-u-\frac{1}{2}]\theta_{4}(u\mid\tau)
                    \end{align}

                    \hrulefill

              \item $\displaystyle \theta_{2} \left(u+\frac{\tau}{2} \middle| \tau \right)$
                    を
                    $\displaystyle \theta_{3} \left(u \middle| \tau \right)$
                    を用いて表わせ。

                    \dotfill

                    \begin{align}
                     \theta_{2}(u+\frac{\tau}{2} \mid \tau)
                      =&\; \vartheta \begin{bmatrix} 1/2 \\ 0 \end{bmatrix} (u+\frac{3}{2}\tau \mid \tau)\\
                      =&\;
                     \mathbf{e}[\frac{\tau}{4} + u + \frac{3}{2}\tau]
                        \vartheta \begin{bmatrix} 0 \\ 0 \end{bmatrix} (u+2\tau \mid \tau)\\
                      =&\;
                     \mathbf{e}[\frac{\tau}{4} + u + \frac{3}{2}\tau]
                        \mathbf{e}[-\tau -2(u+\tau)]
                               \vartheta \begin{bmatrix} 0 \\ 0 \end{bmatrix} (u+\tau \mid \tau)\\
                      =&\;
                     \mathbf{e}[-\frac{5}{4}\tau-u] \theta_{3}(u\mid \tau)
                    \end{align}

                    \hrulefill

              \item $\displaystyle \theta_{3} \left(u+\frac{\tau}{2} \middle| \tau \right)$
                    を
                    $\displaystyle \theta_{2} \left(u \middle| \tau \right)$
                    を用いて表わせ。

                    \dotfill

                    \begin{align}
                     \theta_{3} \left(u+\frac{\tau}{2} \middle| \tau \right)
                     =&\;
                     \vartheta \begin{bmatrix} 0 \\ 0 \end{bmatrix} (u+\frac{\tau}{2}+\tau \mid \tau)\\
                     =&\; \mathbf{e}[-\frac{\tau}{4}-u-\tau]
                     \vartheta \begin{bmatrix} 1/2 \\ 0 \end{bmatrix} (u+\tau \mid \tau)\\
                     =&\; \mathbf{e}[-\frac{5}{4}\tau-u]
                     \theta_{2} \left(u \middle| \tau \right)
                    \end{align}

                    \hrulefill

              \item $\displaystyle \theta_{4} \left(u+\frac{\tau}{2} \middle| \tau \right)$
                    を
                    $\displaystyle \theta_{1} \left(u \middle| \tau \right)$
                    を用いて表わせ。

                    \dotfill

                    \begin{align}
                     \theta_{4} \left(u+\frac{\tau}{2} \middle| \tau \right)
                     =&\;
                     \vartheta \begin{bmatrix} 0 \\ 1/2 \end{bmatrix} (u+\frac{\tau}{2}+\tau \mid \tau)\\
                     =&\; \mathbf{e}[-\frac{\tau}{4} -(u+\tau+\frac{1}{2})]
                     \vartheta \begin{bmatrix} 1/2 \\ 1/2 \end{bmatrix} (u+\tau \mid \tau)\\
                     =&\; \mathbf{e}[-\frac{5}{4}\tau -u-\frac{1}{2}]
                     \theta_{1} \left(u \middle| \tau \right)
                    \end{align}

                    \hrulefill
             \end{enumerate}

        \item
             \begin{enumerate}
              \item $\displaystyle \theta_{1} \left(u+\frac{1}{2} \middle| 2\tau \right)$
                    を
                    $\displaystyle \theta_{2} \left(u \middle| 2\tau \right)$
                    を用いて表わせ。

                    \dotfill

                    \begin{align}
                     \theta_{1} \left(u+\frac{1}{2} \middle| 2\tau \right)
                     =&\;
                     -\vartheta \begin{bmatrix} 1/2 \\ 1/2 \end{bmatrix} (u+\frac{1}{2}+2\tau \mid 2\tau)\\
                     =&\;
                     -\vartheta \begin{bmatrix} 1/2 \\ 0 \end{bmatrix} (u+1+2\tau \mid 2\tau)\\
                     =&\; -\mathbf{e}[1]
                     \vartheta \begin{bmatrix} 1/2 \\ 0 \end{bmatrix} (u+2\tau \mid 2\tau)\\
                     =&\; -\mathbf{e}[1] \theta_{2} \left(u \middle| 2\tau \right)
                    \end{align}

                    \hrulefill

              \item $\displaystyle \theta_{2} \left(u+\frac{1}{2} \middle| 2\tau \right)$
                    を
                    $\displaystyle \theta_{1} \left(u \middle| 2\tau \right)$
                    を用いて表わせ。

                    \dotfill

                    \begin{align}
                     \theta_{2} \left(u+\frac{1}{2} \middle| 2\tau \right)
                     =&\;
                     \vartheta \begin{bmatrix} 1/2 \\ 0 \end{bmatrix} (u+\frac{1}{2}+2\tau \mid 2\tau)\\
                     =&\;
                     \vartheta \begin{bmatrix} 1/2 \\ 1/2 \end{bmatrix} (u+2\tau \mid 2\tau)\\
                     =&\;
                     -\theta_{1} \left(u \middle| 2\tau \right)
                    \end{align}

                    \hrulefill

              \item $\displaystyle \theta_{3} \left(u+\frac{1}{2} \middle| 2\tau \right)$
                    を
                    $\displaystyle \theta_{4} \left(u \middle| 2\tau \right)$
                    を用いて表わせ。

                    \dotfill

                    \begin{align}
                     \theta_{3} \left(u+\frac{1}{2} \middle| 2\tau \right)
                     =&\;
                     \vartheta \begin{bmatrix} 0 \\ 0 \end{bmatrix} (u+\frac{1}{2}+2\tau \mid 2\tau)\\
                     =&\;
                     \vartheta \begin{bmatrix} 0 \\ 1/2 \end{bmatrix} (u+2\tau \mid 2\tau)\\
                     =&\; \theta_{4} \left(u \middle| 2\tau \right)
                    \end{align}

                    \hrulefill

              \item $\displaystyle \theta_{4} \left(u+\frac{1}{2} \middle| 2\tau \right)$
                    を
                    $\displaystyle \theta_{3} \left(u \middle| 2\tau \right)$
                    を用いて表わせ。

                    \dotfill

                    \begin{align}
                     \theta_{4} \left(u+\frac{1}{2} \middle| 2\tau \right)
                     =&\;
                     \vartheta \begin{bmatrix} 0 \\ 1/2 \end{bmatrix} (u+\frac{1}{2}+2\tau \mid 2\tau)\\
                     =&\;
                     \vartheta \begin{bmatrix} 0 \\ 0 \end{bmatrix} (u+1+2\tau \mid 2\tau)\\
                     =&\; \mathbf{e}[0]
                     \vartheta \begin{bmatrix} 0 \\ 0 \end{bmatrix} (u+2\tau \mid 2\tau)\\
                     =&\;
                     \theta_{3} \left(u \middle| 2\tau \right)
                    \end{align}

                    \hrulefill
             \end{enumerate}

        \item
             \begin{enumerate}
              \item $\displaystyle \theta_{1} \left(u+\tau \middle| 2\tau \right)$
                    を
                    $\displaystyle \theta_{4} \left(u \middle| 2\tau \right)$
                    を用いて表わせ。

                    \dotfill

                    \begin{align}
                     & \theta_{1} \left(u+\tau \middle| 2\tau \right)\\
                     =&\;
                     -\vartheta \begin{bmatrix} 1/2 \\ 1/2 \end{bmatrix} (u+\tau+2\tau \mid 2\tau)\\
                     =&\; -\mathbf{e}[\frac{2\tau}{4}+u+3\tau +\frac{1}{2}]
                     \vartheta \begin{bmatrix} 0 \\ 1/2 \end{bmatrix} (u+2\tau+2\tau \mid 2\tau)\\
                     =&\; -\mathbf{e}[\frac{7}{2}\tau+u +\frac{1}{2}]
                     \mathbf{e}[-2\tau-2(u+2\tau+\frac{1}{2})]
                     \vartheta \begin{bmatrix} 0 \\ 1/2 \end{bmatrix} (u+2\tau \mid 2\tau)\\
                     =&\; -\mathbf{e}[-\frac{5}{2}\tau-u-\frac{1}{2}]
                     \theta_{4} \left(u \middle| 2\tau \right)
                    \end{align}

                    \hrulefill

              \item $\displaystyle \theta_{2} \left(u+\tau \middle| 2\tau \right)$
                    を
                    $\displaystyle \theta_{3} \left(u \middle| 2\tau \right)$
                    を用いて表わせ。

                    \dotfill

                    \begin{align}
                     \theta_{2} \left(u+\tau \middle| 2\tau \right)
                     =&\;
                     \vartheta \begin{bmatrix} 1/2 \\ 0 \end{bmatrix} (u+\tau+2\tau \mid 2\tau)\\
                     =&\;
                     \mathbf{e}[\frac{2\tau}{4}+u+3\tau]
                     \vartheta \begin{bmatrix} 0 \\ 0 \end{bmatrix} (u+2\tau+2\tau \mid 2\tau)\\
                     =&\;
                     \mathbf{e}[\frac{7}{2}\tau+u]
                     \mathbf{e}[-2\tau -2(u+2\tau)]
                     \vartheta \begin{bmatrix} 0 \\ 0 \end{bmatrix} (u+2\tau \mid 2\tau)\\
                     =&\;
                     \mathbf{e}[-\frac{5}{2}\tau-u]
                     \theta_{3} \left(u \mid 2\tau \right)
                    \end{align}

                    \hrulefill

              \item $\displaystyle \theta_{3} \left(u+\tau \middle| 2\tau \right)$
                    を
                    $\displaystyle \theta_{2} \left(u \middle| 2\tau \right)$
                    を用いて表わせ。

                    \dotfill

                    \begin{align}
                     \theta_{3} \left(u+\tau \middle| 2\tau \right)
                     =&\;
                     \vartheta \begin{bmatrix} 0 \\ 0 \end{bmatrix} (u+\tau+2\tau \mid 2\tau)\\
                     =&\;
                     \mathbf{e}[-\frac{2\tau}{4} -u-2\tau]
                     \vartheta \begin{bmatrix} 1/2 \\ 0 \end{bmatrix} (u+2\tau \mid 2\tau)\\
                     =&\;
                     \mathbf{e}[-\frac{5}{2}\tau-u]
                     \theta_{2} \left(u \mid 2\tau \right)
                    \end{align}

                    \hrulefill

              \item $\displaystyle \theta_{4} \left(u+\tau \middle| 2\tau \right)$
                    を
                    $\displaystyle \theta_{1} \left(u \middle| 2\tau \right)$
                    を用いて表わせ。

                    \dotfill

                    \begin{align}
                     \theta_{4} \left(u+\tau \mid 2\tau \right)
                     =&\;
                     \vartheta \begin{bmatrix} 0 \\ 1/2 \end{bmatrix} (u+\tau+2\tau \mid 2\tau)\\
                     =&\;
                     \mathbf{e}[-\frac{2\tau}{4} -u-2\tau -\frac{1}{2}]
                     \vartheta \begin{bmatrix} 1/2 \\ 1/2 \end{bmatrix} (u+2\tau \mid 2\tau)\\
                     =&\; -\mathbf{e}[-\frac{5}{2}\tau -u -\frac{1}{2}]
                     \theta_{1} \left(u \mid 2\tau \right)
                    \end{align}

                    \hrulefill

             \end{enumerate}


        \item
             $\theta_{1}(u\mid 2\tau),\;\theta_{2}(u\mid 2\tau)
             ,\;\theta_{3}(u\mid 2\tau),\;\theta_{4}(u\mid 2\tau)$
             の$u \mapsto u+1$としたときの
             変換性(準周期性)を求めよ。

             \dotfill

             \begin{align}
              \theta_{1} (u + \frac{1}{2} \mid 2\tau)
              =&\; -\mathbf{e}[1] \theta_{2} \left(u \mid 2\tau \right)\\
              \theta_{2} (u+\frac{1}{2} \mid 2\tau )
              =&\; -\theta_{1} \left(u \mid 2\tau \right)\\
              \theta_{3} (u+\frac{1}{2} \mid 2\tau )
              =&\; \theta_{4} (u \mid 2\tau)\\
              \theta_{4} (u+\frac{1}{2} \mid 2\tau )
              =&\; \theta_{3} (u \mid 2\tau )
             \end{align}

             \begin{align}
              \theta_{1} (u +1 \mid 2\tau)
              =&\;
              -\mathbf{e}[1] \theta_{2} (u+\frac{1}{2} \mid 2\tau )
              =
              \mathbf{e}[1] \theta_{1} \left(u \mid 2\tau \right)\\
%
              \theta_{2} (u +1 \mid 2\tau)
              =&\;
              -\theta_{1} (u+\frac{1}{2} \mid 2\tau )
              =
              \mathbf{e}[1] \theta_{2} \left(u \mid 2\tau \right)\\
%
              \theta_{3} (u +1 \mid 2\tau)
              =&\; \theta_{4} (u+\frac{1}{2} \mid 2\tau)
              = \theta_{3} (u \mid 2\tau)\\
%
              \theta_{4} (u +1 \mid 2\tau)
              =&\; \theta_{4} (u+\frac{1}{2} \mid 2\tau)
              = \theta_{3} (u \mid 2\tau)
             \end{align}




             \hrulefill

        \item
             $\theta_{1}(u\mid 2\tau),\;\theta_{2}(u\mid 2\tau)
             ,\;\theta_{3}(u\mid 2\tau),\;\theta_{4}(u\mid 2\tau)$
             の$u \mapsto u+2\tau$としたときの
             変換性(準周期性)を求めよ。

             \dotfill

             \begin{align}
              \theta_{1} (u + \tau \mid 2\tau)
              =&\; -\mathbf{e}[-\frac{5}{2}\tau-u-\frac{1}{2}]
              \theta_{4} (u \mid 2\tau )\\
              \theta_{2} (u+\tau \mid 2\tau )
              =&\; \mathbf{e}[-\frac{5}{2}\tau-u]
              \theta_{3} (u \mid 2\tau )\\
              \theta_{3} (u+\tau \mid 2\tau )
              =&\; \mathbf{e}[-\frac{5}{2}\tau-u]
              \theta_{2} (u \mid 2\tau)\\
              \theta_{4} (u+\tau \mid 2\tau )
              =&\; -\mathbf{e}[-\frac{5}{2}\tau-u-\frac{1}{2}]
              \theta_{1} (u \mid 2\tau )
             \end{align}

             \begin{align}
              \theta_{1} (u + 2\tau \mid 2\tau)
              =& -\mathbf{e}[-\frac{5}{2}\tau-u-\frac{1}{2}]
              \theta_{4} (u+\tau \mid 2\tau )\\
              =& \mathbf{e}[-\frac{5}{2}\tau-u-\frac{1}{2}]
              \mathbf{e}[-\frac{5}{2}\tau-u-\frac{1}{2}]
              \theta_{1} (u \mid 2\tau )\\
              =& \mathbf{e}[-5\tau-2u-1]\theta_{1} (u \mid 2\tau )\\
              %
              \theta_{2} (u + 2\tau \mid 2\tau)
              =& \mathbf{e}[-\frac{5}{2}\tau-u]
              \theta_{3} (u+\tau \mid 2\tau )\\
              =& \mathbf{e}[-\frac{5}{2}\tau-u]
              \mathbf{e}[-\frac{5}{2}\tau-u]
              \theta_{2} (u \mid 2\tau )\\
              =& \mathbf{e}[-5\tau-2u]
              \theta_{2} (u \mid 2\tau )\\
              %
              \theta_{3} (u + 2\tau \mid 2\tau)
              =& \mathbf{e}[-\frac{5}{2}\tau-u]
              \theta_{2} (u+\tau \mid 2\tau )\\
              =& \mathbf{e}[-\frac{5}{2}\tau-u]
              \mathbf{e}[-\frac{5}{2}\tau-u]
              \theta_{3} (u \mid 2\tau )\\
              =& \mathbf{e}[-5\tau-2u]
              \theta_{3} (u \mid 2\tau )\\
              %
              \theta_{4} (u + 2\tau \mid 2\tau)
              =& -\mathbf{e}[-\frac{5}{2}\tau-u-\frac{1}{2}]
              \theta_{1} (u+\tau \mid 2\tau )\\
              =& \mathbf{e}[-\frac{5}{2}\tau-u-\frac{1}{2}]
              \mathbf{e}[-\frac{5}{2}\tau-u-\frac{1}{2}]
              \theta_{4} (u \mid 2\tau )\\
              =& \mathbf{e}[-5\tau-2u-1]\theta_{4} (u \mid 2\tau)
             \end{align}




             \hrulefill

       \end{enumerate}

\end{enumerate}

\hrulefill

\end{document}
