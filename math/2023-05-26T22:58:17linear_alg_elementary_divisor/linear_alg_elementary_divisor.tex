\documentclass[12pt,b5paper]{ltjsarticle}

%\usepackage[margin=15truemm, top=5truemm, bottom=5truemm]{geometry}
%\usepackage[margin=10truemm,left=15truemm]{geometry}
\usepackage[margin=10truemm]{geometry}

\usepackage{amsmath,amssymb}
%\pagestyle{headings}
\pagestyle{empty}

%\usepackage{listings,url}
%\renewcommand{\theenumi}{(\arabic{enumi})}

%\usepackage{graphicx}

%\usepackage{tikz}
%\usetikzlibrary {arrows.meta}
%\usepackage{wrapfig}
%\usepackage{bm}

% ルビを振る
%\usepackage{luatexja-ruby}	% required for `\ruby'

%% 核Ker 像Im Hom を定義
%\newcommand{\Img}{\mathop{\mathrm{Im}}\nolimits}
%\newcommand{\Ker}{\mathop{\mathrm{Ker}}\nolimits}
%\newcommand{\Hom}{\mathop{\mathrm{Hom}}\nolimits}

%\DeclareMathOperator{\Rot}{rot}
%\DeclareMathOperator{\Div}{div}
%\DeclareMathOperator{\Grad}{grad}
%\DeclareMathOperator{\arcsinh}{arcsinh}
%\DeclareMathOperator{\arccosh}{arccosh}
%\DeclareMathOperator{\arctanh}{arctanh}



\usepackage{listings,url}

\lstset{
%プログラム言語(複数の言語に対応,C,C++も可)
  language = Python,
%  language = Lisp,
%  language = C,
  %背景色と透過度
  %backgroundcolor={\color[gray]{.90}},
  %枠外に行った時の自動改行
  breaklines = true,
  %自動改行後のインデント量(デフォルトでは20[pt])
  breakindent = 10pt,
  %標準の書体
%  basicstyle = \ttfamily\scriptsize,
  basicstyle = \ttfamily,
  %コメントの書体
%  commentstyle = {\itshape \color[cmyk]{1,0.4,1,0}},
  %関数名等の色の設定
  classoffset = 0,
  %キーワード(int, ifなど)の書体
%  keywordstyle = {\bfseries \color[cmyk]{0,1,0,0}},
  %表示する文字の書体
  %stringstyle = {\ttfamily \color[rgb]{0,0,1}},
  %枠 "t"は上に線を記載, "T"は上に二重線を記載
  %他オプション:leftline,topline,bottomline,lines,single,shadowbox
  frame = TBrl,
  %frameまでの間隔(行番号とプログラムの間)
  framesep = 5pt,
  %行番号の位置
  numbers = left,
  %行番号の間隔
  stepnumber = 1,
  %行番号の書体
%  numberstyle = \tiny,
  %タブの大きさ
  tabsize = 4,
  %キャプションの場所("tb"ならば上下両方に記載)
  captionpos = t
}



\begin{document}

\hrulefill

\textbf{単因子}

行列に対し、
行と列に対する変形を繰り返し行い
対角成分以外を$0$とする。
このとき、対角成分を因子の順に並べたものを
単因子という。

例えば、$(2,6,18)$など。


\hrulefill

\begin{enumerate}
 \item
      $R=\mathbb{Z}$として、
      次の行列の単因子を求めよ。
      \begin{equation}
       A=
        \begin{pmatrix}
         2 & 6 & 4 \\
         3 & 7 & 5 \\
         -4 & 2 & 8
        \end{pmatrix}
      \end{equation}

      \dotfill

      行における変形は左から、
      列における変形は右からかけることにより行える。

      次の行列$P_{1},\dots,P_{5}$を
      $A$にかけることにより
      行列の変形をおこなう。
      \begin{align}
       A P_{1} =
        \begin{pmatrix}
         2 & 0 & 0 \\
         3 & -2 & -1 \\
         -4 & 14 & 16
        \end{pmatrix},
        &\qquad
        P_{1} =
        \begin{pmatrix}
         1 & -3 & -2 \\
         0 & 1 & 0 \\
         0 & 0 & 1
        \end{pmatrix}
       \\
       A P_{1} P_{2} =
        \begin{pmatrix}
         2 & 0 & 0 \\
         0 & 0 & -1 \\
         44 & -18 & 16
        \end{pmatrix},
        &\qquad
        P_{2} =
        \begin{pmatrix}
         1 & 0 & 0 \\
         0 & 1 & 0 \\
         3 & -2 & 1
        \end{pmatrix}
       \\
       P_{3} A P_{1} P_{2}=
        \begin{pmatrix}
         2 & 0 & 0 \\
         0 & 0 & -1 \\
         0 & -18 & 0
        \end{pmatrix},
        &\qquad
        P_{3} =
        \begin{pmatrix}
         1 & 0 & 0 \\
         0 & 1 & 0 \\
         -22 & 16 & 1
        \end{pmatrix}
       \\
       P_{3} A P_{1} P_{2} P_{4}=
        \begin{pmatrix}
         0 & 2 & 0 \\
         -1 & 0 & 0 \\
         0 & 0 & -18
        \end{pmatrix},
        &\qquad
        P_{4} =
        \begin{pmatrix}
         0 & 1 & 0 \\
         0 & 0 & 1 \\
         1 & 0 & 0
        \end{pmatrix}
       \\
       P_{5} P_{3} A P_{1} P_{2} P_{4}=
        \begin{pmatrix}
         -1 & 0 & 0 \\
         0 & 2 & 0 \\
         0 & 0 & -18
        \end{pmatrix},
        &\qquad
        P_{5} =
        \begin{pmatrix}
         0 & 1 & 0 \\
         1 & 0 & 0 \\
         0 & 0 & 1
        \end{pmatrix}
      \end{align}

      よって、単因子は
      $(-1,2,-18)$である。

      \hrulefill

 \item
      $R=\mathbb{C}[x]$として、
      次の行列の単因子を求めよ。
      \begin{equation}
       A=
        \begin{pmatrix}
         x-2 & -1 & 0 \\
         -1 & x-2 & 0 \\
         -1 & 1 & x-3
        \end{pmatrix}
      \end{equation}

      \dotfill


      
      \begin{align}
       P_{1} A=
        \begin{pmatrix}
         0 & (x-1)(x-3) & 0 \\
         -1 & x-2 & 0 \\
         0 & -x+3 & x-3
        \end{pmatrix},
        &\qquad
        P_{1} =
        \begin{pmatrix}
         1 & x-2 & 0 \\
         0 & 1 & 0 \\
         0 & -1 & 1
        \end{pmatrix}
       \\
       P_{1} A P_{2}=
        \begin{pmatrix}
         0 & (x-1)(x-3) & 0 \\
         -1 & 0 & 0 \\
         0 & 0 & x-3
        \end{pmatrix},
        &\qquad
        P_{2} =
        \begin{pmatrix}
         1 & x-2 & 0 \\
         0 & 1 & 0 \\
         0 & 1 & 1
        \end{pmatrix}
       \\
       P_{1} A P_{2} P_{3}=
        \begin{pmatrix}
         0 & 0 & (x-1)(x-3) \\
         -1 & 0 & 0 \\
         0 & x-3 & 0
        \end{pmatrix},
        &\qquad
        P_{3} =
        \begin{pmatrix}
         1 & 0 & 0 \\
         0 & 0 & 1 \\
         0 & 1 & 0
        \end{pmatrix}
       \\
       P_{4} P_{1} A P_{2} P_{3}=
        \begin{pmatrix}
         -1 & 0 & 0 \\
         0 & x-3 & 0 \\
         0 & 0 & (x-1)(x-3)
        \end{pmatrix},
        &\qquad
        P_{4} =
        \begin{pmatrix}
         0 & 1 & 0 \\
         0 & 0 & 1 \\
         1 & 0 & 0
        \end{pmatrix}
      \end{align}

      これにより
      単因子は
      $(-1,x-3,(x-1)(x-3))$
      である。


      \dotfill

      Sagemath にて確認用計算。
\begin{lstlisting}
A=matrix(CC['x'],[[x-2,-1,0],[-1,x-2,0],[-1,1,x-3]])
print(A,"\n")

P1=matrix(CC['x'],[[1,x-2,0],[0,1,0],[0,-1,1]])
print(P1*A,"\n")

P2=matrix(CC['x'],[[1,x-2,0],[0,1,0],[0,1,1]])
print(P1*A*P2,"\n")

P3=matrix(CC['x'],[[1,0,0],[0,0,1],[0,1,0]])
print(P1*A*P2*P3,"\n")

P4=matrix(CC['x'],[[0,1,0],[0,0,1],[1,0,0]])
print(P4*P1*A*P2*P3,"\n")
\end{lstlisting}




\end{enumerate}

\hrulefill



\end{document}
