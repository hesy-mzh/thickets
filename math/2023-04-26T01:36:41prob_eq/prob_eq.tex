\documentclass[12pt,b5paper]{ltjsarticle}

%\usepackage[margin=15truemm, top=5truemm, bottom=5truemm]{geometry}
%\usepackage[margin=10truemm,left=15truemm]{geometry}
\usepackage[margin=10truemm]{geometry}

\usepackage{amsmath,amssymb}
%\pagestyle{headings}
\pagestyle{empty}

%\usepackage{listings,url}
%\renewcommand{\theenumi}{(\arabic{enumi})}

%\usepackage{graphicx}

%\usepackage{tikz}
%\usetikzlibrary {arrows.meta}
%\usepackage{wrapfig}
%\usepackage{bm}

% ルビを振る
%\usepackage{luatexja-ruby}	% required for `\ruby'

%% 核Ker 像Im Hom を定義
%\newcommand{\Img}{\mathop{\mathrm{Im}}\nolimits}
%\newcommand{\Ker}{\mathop{\mathrm{Ker}}\nolimits}
%\newcommand{\Hom}{\mathop{\mathrm{Hom}}\nolimits}

%\DeclareMathOperator{\Rot}{rot}
%\DeclareMathOperator{\Div}{div}
%\DeclareMathOperator{\Grad}{grad}
%\DeclareMathOperator{\arcsinh}{arcsinh}
%\DeclareMathOperator{\arccosh}{arccosh}
%\DeclareMathOperator{\arctanh}{arctanh}



%\usepackage{listings,url}
%
%\lstset{
%%プログラム言語(複数の言語に対応,C,C++も可)
%  language = Python,
%%  language = Lisp,
%%  language = C,
%  %背景色と透過度
%  %backgroundcolor={\color[gray]{.90}},
%  %枠外に行った時の自動改行
%  breaklines = true,
%  %自動改行後のインデント量(デフォルトでは20[pt])
%  breakindent = 10pt,
%  %標準の書体
%%  basicstyle = \ttfamily\scriptsize,
%  basicstyle = \ttfamily,
%  %コメントの書体
%%  commentstyle = {\itshape \color[cmyk]{1,0.4,1,0}},
%  %関数名等の色の設定
%  classoffset = 0,
%  %キーワード(int, ifなど)の書体
%%  keywordstyle = {\bfseries \color[cmyk]{0,1,0,0}},
%  %表示する文字の書体
%  %stringstyle = {\ttfamily \color[rgb]{0,0,1}},
%  %枠 "t"は上に線を記載, "T"は上に二重線を記載
%  %他オプション:leftline,topline,bottomline,lines,single,shadowbox
%  frame = TBrl,
%  %frameまでの間隔(行番号とプログラムの間)
%  framesep = 5pt,
%  %行番号の位置
%  numbers = left,
%  %行番号の間隔
%  stepnumber = 1,
%  %行番号の書体
%%  numberstyle = \tiny,
%  %タブの大きさ
%  tabsize = 4,
%  %キャプションの場所("tb"ならば上下両方に記載)
%  captionpos = t
%}



\begin{document}

\hrulefill

$p\ne q$である時次の連立方程式を解け。
\begin{equation}
 \begin{cases}
  \alpha + \beta =0 \\
  \alpha + \beta \left(\frac{q}{p}\right)^{N} + \frac{N}{q-p} =0
 \end{cases}
\end{equation}

\dotfill

1つ目の式より$\beta=-\alpha$であるから、
これを2つ目に代入する。
\begin{align}
 \alpha - \alpha \left(\frac{q}{p}\right)^{N} + \frac{N}{q-p} =& 0\\
% \alpha \left( 1 - \left(\frac{q}{p}\right)^{N} \right) + \frac{N}{q-p} =& 0\\
 \alpha \left( 1 - \left(\frac{q}{p}\right)^{N} \right) =& - \frac{N}{q-p}\\
 \alpha =& - \frac{N}{q-p} \frac{p^{N}}{p^{N}-q^{N}}
\end{align}

\begin{equation}
 \alpha = - \frac{Np^{N}}{(q-p)(p^{N}-q^{N})}
  ,\quad
  \beta = \frac{Np^{N}}{(q-p)(p^{N}-q^{N})}
\end{equation}

\hrulefill

%$N\in\mathbb{N},\ N\geq 2$とする。
%この時、次の漸化式を解け。
%
%\begin{equation}
% \begin{cases}
%  V_{n} = (1/2)V_{n+1} + (1/2)V_{n-1} + 1\\
%  V_{0} = 0,\ \ V_{N}=0
% \end{cases}
%\end{equation}
%
%\dotfill
%
%\begin{align}
% V_{n} =& (1/2)V_{n+1} + (1/2)V_{n-1} + 1\\
% (1/2)V_{n+1} - (1/2)V_{n} =& (1/2)V_{n} - (1/2)V_{n-1} - 1
%\end{align}
%
% $a_{n} =  (1/2)(V_{n} - V_{n-1})$とおけば、
% $a_{n+1} = a_{n} - 1$
% であるので、
% $a_{n} = a_{1} -(n-1)$
%
%
%\hrulefill

\end{document}
