\documentclass[12pt,b5paper]{ltjsarticle}

%\usepackage[margin=15truemm, top=5truemm, bottom=5truemm]{geometry}
\usepackage[margin=15truemm]{geometry}

\usepackage{amsmath,amssymb}
\pagestyle{empty}

\usepackage{listings,url}

\begin{document}

\begin{enumerate}
 \item[問3]
以下の1次元複体$X, Y$が位相同型であるかどうかを判定せよ。
\begin{enumerate}
 \item $X=\{ x \in\mathbb{R}^1 \mid -2<x<5\},
       y=\{y\in\mathbb{R}^1 \mid y>0\}$

       \dotfill

       写像$f$を次のように定義します。
       \begin{equation}
        f : X \rightarrow Y \qquad x \mapsto -\frac{x+2}{x-5}
       \end{equation}
       この$f$は連続写像であり全単射です。
       \begin{equation}
        \lim_{x\rightarrow -2}f(x)=0 \qquad \lim_{x\rightarrow 5}f(x)=\infty
       \end{equation}
       この時、逆写像が存在し次の式で表せます。
       \begin{equation}
        f^{-1} : Y \rightarrow X \qquad y \mapsto \frac{5y-2}{y+1}
       \end{equation}
       この写像$f^{-1}$は
       \begin{equation}
        \lim_{y\rightarrow 0}f^{-1}(y)=-2 \qquad \lim_{y\rightarrow \infty}f(y)=5
       \end{equation}
       であり、全単射な連続写像です。
       写像$f$は全単射な連続写像でその逆写像$f^{-1}$も連続写像であるので、
       位相同型写像となり、$X$と$Y$は位相同型であることがわかります。

       \hrulefill
 \item $X=\bigcirc\llap{\small ---}, Y=\bigcirc\llap{\large +}$

       \dotfill
       
\end{enumerate}
\end{enumerate}

\end{document}
