\documentclass[12pt,b5paper]{ltjsarticle}

%\usepackage[margin=15truemm, top=5truemm, bottom=5truemm]{geometry}
%\usepackage[margin=10truemm,left=15truemm]{geometry}
\usepackage[margin=10truemm]{geometry}

\usepackage{amsmath,amssymb}
%\pagestyle{headings}
\pagestyle{empty}

%\usepackage{listings,url}
%\renewcommand{\theenumi}{(\arabic{enumi})}

%\usepackage{graphicx}

%\usepackage{tikz}
%\usetikzlibrary {arrows.meta}
%\usepackage{wrapfig}	% required for `\wrapfigure' (yatex added)
%\usepackage{bm}	% required for `\bm' (yatex added)

% ルビを振る
%\usepackage{luatexja-ruby}	% required for `\ruby'

%% 核Ker 像Im Hom を定義
%\newcommand{\Img}{\mathop{\mathrm{Im}}\nolimits}
%\newcommand{\Ker}{\mathop{\mathrm{Ker}}\nolimits}
%\newcommand{\Hom}{\mathop{\mathrm{Hom}}\nolimits}

%\DeclareMathOperator{\Rot}{rot}
%\DeclareMathOperator{\Div}{div}
%\DeclareMathOperator{\Grad}{grad}
%\DeclareMathOperator{\arcsinh}{arcsinh}
%\DeclareMathOperator{\arccosh}{arccosh}
%\DeclareMathOperator{\arctanh}{arctanh}



%\usepackage{listings,url}
%
%\lstset{
%%プログラム言語(複数の言語に対応,C,C++も可)
%  language = Python,
%%  language = Lisp,
%%  language = C,
%  %背景色と透過度
%  %backgroundcolor={\color[gray]{.90}},
%  %枠外に行った時の自動改行
%  breaklines = true,
%  %自動改行後のインデント量(デフォルトでは20[pt])
%  breakindent = 10pt,
%  %標準の書体
%%  basicstyle = \ttfamily\scriptsize,
%  basicstyle = \ttfamily,
%  %コメントの書体
%%  commentstyle = {\itshape \color[cmyk]{1,0.4,1,0}},
%  %関数名等の色の設定
%  classoffset = 0,
%  %キーワード(int, ifなど)の書体
%%  keywordstyle = {\bfseries \color[cmyk]{0,1,0,0}},
%  %表示する文字の書体
%  %stringstyle = {\ttfamily \color[rgb]{0,0,1}},
%  %枠 "t"は上に線を記載, "T"は上に二重線を記載
%  %他オプション:leftline,topline,bottomline,lines,single,shadowbox
%  frame = TBrl,
%  %frameまでの間隔(行番号とプログラムの間)
%  framesep = 5pt,
%  %行番号の位置
%  numbers = left,
%  %行番号の間隔
%  stepnumber = 1,
%  %行番号の書体
%%  numberstyle = \tiny,
%  %タブの大きさ
%  tabsize = 4,
%  %キャプションの場所("tb"ならば上下両方に記載)
%  captionpos = t
%}



\begin{document}

\begin{enumerate}
 \item
      $a|bc$を満たす零でない整数 $a,b,c$に対して
      $\frac{a}{GCD(a,b)}|c$であることを示せ。

      \dotfill

      $g=GCD(a,b)$とする。
      これにより$a=a^\prime g, \ b=b^\prime g$となる
      $a^\prime,b^\prime \in \mathbb{Z}$が存在する。

      $a|bc$より$a^\prime | b^\prime c$である。
      $g$は$a$と$b$の最大公約数であるので、
      $a^\prime,b^\prime$は互いに素である。
      よって、$a^\prime | b^\prime c$から、
      $a^\prime | c$であることになる。

      $a=a^\prime g =a^\prime GCD(a,b)$より
      $a^\prime = \frac{a}{GCD(a,b)}$が得られる。

      $a^\prime | c$であるので、
      $\frac{a}{GCD(a,b)}|c$であることがわかる。

      \hrulefill

 \item
      \hrulefill

      環$R$に対して、部分環$I\subset R$が次を満たす時、
      $I$をイデアルという。
      \begin{equation}
       r\in R,\ x\in I \Rightarrow rx\in I
      \end{equation}

      積が非可換であるときは$rx\in I$や$xr\in I$と
      異なるものになるので、
      右側イデアル(左イデアル)、左側イデアル(右イデアル)と呼ばれる。

      $rx\in I$の様に右側にイデアルの元がある時に右側イデアル、
      左側に$R$の元がある場合に左イデアルということが多いが、
      テキストなどによって異なる場合もあるので注意が必要。
      この場合、右側イデアルと左イデアルは同じものを指すが、
      右側イデアルや右イデアルは (right ideal) と書き、
      左側イデアルや左イデアルは (left ideal) と
      書いてあったりする。

      \hrulefill

      \begin{enumerate}
       \item
            $R=\mathbb{Z}$とする。
            $(100,160,240)=(a)$を満たす$a\in\mathbb{Z}$を一つ挙げよ。

            \dotfill

            $-3,2\in R$より
            $20=-3\cdot 100 + 2\cdot160 \in (100,160,240)$である。
            よって、$(20)\subset (100,160,240)$である。

            $5,8,12\in R$より
            $100=5\cdot 20\in (20),\ 160=8\cdot20\in (20),\ 240=12\cdot20 \in (20)$である。
            これにより$(100,160,240)\subset (20)$である。

            よって、$(100,160,240)=(20)$

            \hrulefill

       \item
            $R=\mathbb{Q}[x]$とする。
            $(x^2,x^3+3x+1)=(ax^3+bx^2+cx+1)$を
            満たす$a,b,c\in\mathbb{Z}$を一組挙げよ。

            \dotfill

            $f=x^2,\ g=x^3+3x+1$とおく。
            \begin{gather}
             (g-xf)-3(x(g-xf)-3f) = 1\\
             ( (x^3+3x+1) - x (x^2) )
             -3( x( (x^3+3x+1)-x(x^2) )-3(x^2)) = 1
            \end{gather}

            $(1-3x)g +(3x^2-x+9)f = 1$であるので、
            $(1)\subset (x^2,x^3+3x+1)$である。

            $(1)=R$であるので、
            $R\subset (x^2,x^3+3x+1)$である。

            よって、
            $(1)=(ax^3+bx^2+cx+1)$であるので、
            $(a,b,c)=(0,0,0)$である。

            \hrulefill

       \item
            $R=\mathbb{R}[x]$とする。
            $(x^3-1,x^5-1)=(ax^5+bx^4+cx^3+dx^2+ex+1)$を
            満たす$a,b,c,d,e\in\mathbb{Z}$を一組挙げよ。

            \dotfill

            $f=x^3-1,g=x^5-1$とする。
            \begin{gather}
             -f+x(-x^2f+g) = -x+1
            -(x^3-1)+x( -x^2(x^3-1) +  (x^5-1) ) = -x+1
            \end{gather}

            $(-1-x^3)f+xg = -x+1$であるので、
            $-x+1\in (x^3-1,x^5-1)$であり、
            $(-x+1) \subset (x^3-1,x^5-1)$である。

            $x^3-1,x^5-1$はそれぞれ次のような変形ができる。
            \begin{equation}
             x^3-1 = -(x^2+x+1)(-x+1),
              \quad
             x^5-1 = -(x^4+x^3+x^2+x+1)(-x+1)
            \end{equation}

            つまり、$x^3-1\in (-x+1)$であり、
            $x^5-1\in (-x+1)$である。
            その為、$(x^3-1,x^5-1)\subset (-x+1)$である。

            これらより
            $(x^3-1,x^5-1)=(-x+1)$である事がわかる。
            よって、
            $(a,b,c,d,e)=(0,0,0,0,-1)$である。

            \hrulefill

       \item
            $R=\mathbb{Z}[x]$とする。
            $(x^3-1,x^4+1)=(ax^2+b,cx+d)$を
            満たす$a,b,c,d\in\mathbb{Z}$を一組挙げよ。

            \dotfill

            \begin{gather}
             -x(x^3-1) + (x^4+1) = x+1\\
             -(x^3-1) + x^2( -x(x^3-1) + (x^4+1) ) = x^2+1
            \end{gather}
            これにより
            $x+1\in (x^3-1,x^4+1)$、
            $x^2+1\in (x^3-1,x^4+1)$であり、
            $(x+1,x^2+1)\subset (x^3-1,x^4+1)$である。

            \begin{gather}
             (x+1)(x^2+1) -(x+1) -(x^2+1) = x^3-1\\
             (x^2+1)^2 -(x^2+1) -x*(x+1) +(x+1) = x^4+1
            \end{gather}
            これにより
            $x^3-1 \in (x+1,x^2+1)$、
            $x^4+1 \in (x+1,x^2+1)$であり、
            $(x^3-1,x^4+1) \subset (x+1,x^2+1)$である。

            よって、
            $(x^3-1,x^4+1) = (x+1,x^2+1)$であるので、
            $(a,b,c,d)=(1,1,1,1)$である。
            
            \hrulefill

      \end{enumerate}
\end{enumerate}


\end{document}
