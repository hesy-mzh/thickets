\documentclass[12pt,b5paper]{ltjsarticle}

%\usepackage[margin=15truemm, top=5truemm, bottom=5truemm]{geometry}
%\usepackage[margin=10truemm,left=15truemm]{geometry}
\usepackage[margin=10truemm]{geometry}

\usepackage{amsmath,amssymb}
%\pagestyle{headings}
\pagestyle{empty}

%\usepackage{listings,url}
%\renewcommand{\theenumi}{(\arabic{enumi})}

%\usepackage{graphicx}

%\usepackage{tikz}
%\usetikzlibrary {arrows.meta}
%\usepackage{wrapfig}
%\usepackage{bm}

% ルビを振る
%\usepackage{luatexja-ruby}	% required for `\ruby'

%% 核Ker 像Im Hom を定義
%\newcommand{\Img}{\mathop{\mathrm{Im}}\nolimits}
%\newcommand{\Ker}{\mathop{\mathrm{Ker}}\nolimits}
%\newcommand{\Hom}{\mathop{\mathrm{Hom}}\nolimits}

%\DeclareMathOperator{\Rot}{rot}
%\DeclareMathOperator{\Div}{div}
%\DeclareMathOperator{\Grad}{grad}
%\DeclareMathOperator{\arcsinh}{arcsinh}
%\DeclareMathOperator{\arccosh}{arccosh}
%\DeclareMathOperator{\arctanh}{arctanh}



%\usepackage{listings,url}
%
%\lstset{
%%プログラム言語(複数の言語に対応,C,C++も可)
%  language = Python,
%%  language = Lisp,
%%  language = C,
%  %背景色と透過度
%  %backgroundcolor={\color[gray]{.90}},
%  %枠外に行った時の自動改行
%  breaklines = true,
%  %自動改行後のインデント量(デフォルトでは20[pt])
%  breakindent = 10pt,
%  %標準の書体
%%  basicstyle = \ttfamily\scriptsize,
%  basicstyle = \ttfamily,
%  %コメントの書体
%%  commentstyle = {\itshape \color[cmyk]{1,0.4,1,0}},
%  %関数名等の色の設定
%  classoffset = 0,
%  %キーワード(int, ifなど)の書体
%%  keywordstyle = {\bfseries \color[cmyk]{0,1,0,0}},
%  %表示する文字の書体
%  %stringstyle = {\ttfamily \color[rgb]{0,0,1}},
%  %枠 "t"は上に線を記載, "T"は上に二重線を記載
%  %他オプション:leftline,topline,bottomline,lines,single,shadowbox
%  frame = TBrl,
%  %frameまでの間隔(行番号とプログラムの間)
%  framesep = 5pt,
%  %行番号の位置
%  numbers = left,
%  %行番号の間隔
%  stepnumber = 1,
%  %行番号の書体
%%  numberstyle = \tiny,
%  %タブの大きさ
%  tabsize = 4,
%  %キャプションの場所("tb"ならば上下両方に記載)
%  captionpos = t
%}



\begin{document}

\hrulefill

次の極限値を求めよ。
\begin{enumerate}
 \item $\displaystyle \lim_{x\to 1+0} 3^{\frac{2}{1-x}}$

       \dotfill

       $\displaystyle \lim_{x\to 1+0} {\frac{2}{1-x}} = -\infty$であるので
       \begin{equation}
        \lim_{x\to 1+0} 3^{\frac{2}{1-x}}
         =
         \lim_{ \frac{2}{1-x}\to -\infty} 3^{\frac{2}{1-x}}
         =
         \lim_{ X\to -\infty} 3^{X}
         =
         0
       \end{equation}

       \hrulefill

 \item $\displaystyle \lim_{x\to +0} \left( \frac{1}{5} \right) ^{\frac{1}{x}}$

       \dotfill

       $\displaystyle \lim_{x\to +0} {\frac{1}{x}} = +\infty$であるので
       \begin{equation}
        \lim_{x\to +0} \left( \frac{1}{5} \right) ^{\frac{1}{x}}
         =
         \lim_{\frac{1}{x}\to +\infty} \left( \frac{1}{5} \right) ^{\frac{1}{x}}
         =
         \lim_{X \to +\infty} \left( \frac{1}{5} \right) ^{X}
         =
         0
       \end{equation}

       \hrulefill

 \item $\displaystyle \lim_{x\to -0} \frac{1}{1+2^{\frac{1}{x}}}$

       \dotfill

       $\displaystyle \lim_{x\to -0} {2^{\frac{1}{x}}} = +0$であるので
       \begin{equation}
        \lim_{x\to -0} \frac{1}{1+2^{\frac{1}{x}}}
         =
         \lim_{2^{\frac{1}{x}}\to +0} \frac{1}{1+2^{\frac{1}{x}}}
         =
        \lim_{X\to +0} \frac{1}{1+X}
        = 1
       \end{equation}

       \hrulefill

 \item $\displaystyle \lim_{x\to \frac{\pi}{2}-0} \frac{1}{1+\tan{x}}$

       \dotfill

       $\displaystyle \lim_{x\to \frac{\pi}{2}-0} {\tan{x}} = +\infty$であるので
       \begin{equation}
        \lim_{x\to \frac{\pi}{2}-0} \frac{1}{1+\tan{x}}
         =
         \lim_{\tan{x}\to +\infty} \frac{1}{1+\tan{x}}
         =
         \lim_{X\to +\infty} \frac{1}{1+X}
         =
         0
       \end{equation}

       \hrulefill

 \item $\displaystyle \lim_{x\to +0} \frac{\log_{2}{x}}{\log_{2}{x}+3}$

       \dotfill

       $\displaystyle \lim_{x\to +0} {\log_{2}{x}} = -\infty$であるので、
       \begin{align}
        \lim_{x\to +0} \frac{\log_{2}{x}}{\log_{2}{x}+3}
         &=
         \lim_{\log_{2}{x}\to -\infty} \frac{\log_{2}{x}}{\log_{2}{x}+3}\\
         &=
         \lim_{X\to -\infty} \frac{X}{X+3}
         =
         \lim_{X\to -\infty} \frac{1}{1+ \frac{3}{X}}
        = 1
       \end{align}

       \hrulefill

 \item $\displaystyle \lim_{x\to 3+0} \left( \log_{6}{(x^{2}-9)} -\log_{6}{(x-3)} \right)$

       \dotfill

       \begin{align}
        \lim_{x\to 3+0} \left( \log_{6}{(x^{2}-9)} -\log_{6}{(x-3)} \right)
         &=
         \lim_{x\to 3+0} \log_{6}{\frac{x^{2}-9}{x-3}}\\
         &=
         \lim_{x\to 3+0} \log_{6}{(x+3)}
        = 1
       \end{align}

       \hrulefill

 \item $\displaystyle \lim_{x\to \infty} \left( \log_{3}{x^{2}}-\log_{2}(x^{2}+5x+2) \right)$

       \dotfill

       $\log_{2}(x^{2}+5x+2) = \log_{2}{x^{2}(1+\frac{5}{x}+\frac{2}{x^{2}})} = \log_{2}{x^{2}} + \log_{2}{(1+\frac{5}{x}+\frac{2}{x^{2}})}$
       である。
       この為、問の式は次のように変形できる。
       \begin{align}
        & \lim_{x\to \infty} \left( \log_{3}{x^{2}}-\log_{2}(x^{2}+5x+2) \right)\\
         = &
         \lim_{x\to \infty} \left( \log_{3}{x^{2}}-\log_{2}{x^{2}} - \log_{2}{\left(1+\frac{5}{x}+\frac{2}{x^{2}}\right)} \right)\\
        = &
        \lim_{x\to \infty} \left( \log_{3}{x^{2}}-\log_{2}{x^{2}} \right) - \lim_{x\to\infty} \log_{2}{\left(1+\frac{5}{x}+\frac{2}{x^{2}}\right)}
       \end{align}

       次のように、後半部分は$0$である。
       \begin{equation}
        \lim_{x\to\infty} \log_{2}{\left(1+\frac{5}{x}+\frac{2}{x^{2}}\right)}
         =
         \log_{2}{1}= 0
         \label{eq:log_2}
       \end{equation}

       そこで、前半部分を計算する。
       \begin{align}
        \log_{3}{x^{2}}-\log_{2}{x^{2}}
        & = \frac{\log_{6}{x^{2}}}{\log_{6}{3}} - \frac{\log_{6}{x^{2}}}{\log_{6}{2}}
        = \frac{2\log_{6}{x}}{\log_{6}{3}} - \frac{2\log_{6}{x}}{\log_{6}{2}}\\
        &= \left( \frac{2}{\log_{6}{3}} - \frac{2}{\log_{6}{2}} \right)\log_{6}{x}
       \end{align}

       $0< \log_{6}{2} < \log_{6}{3} <1$より
       \begin{equation}
        \frac{2}{\log_{6}{3}} - \frac{2}{\log_{6}{2}}
         = \frac{2(\log_{6}{2}-\log_{6}{3})}{\log_{6}{3}\log_{6}{2}} < 0
       \end{equation} 

       また、
       $\displaystyle \lim_{x\to \infty} \log_{6}{x} = \infty$であるので、
       \begin{align}
        \lim_{x\to \infty} \left( \log_{3}{x^{2}}-\log_{2}{x^{2}} \right)
         &=
         \lim_{x\to \infty} \left( \frac{2}{\log_{6}{3}} - \frac{2}{\log_{6}{2}} \right)\log_{6}{x}\\
         &=
         \left( \frac{2}{\log_{6}{3}} - \frac{2}{\log_{6}{2}} \right) \lim_{x\to \infty} \log_{6}{x}
        = -\infty
        \label{eq:log_1}
       \end{align}

       \eqref{eq:log_2}と\eqref{eq:log_1}より
       \begin{equation}
        \lim_{x\to \infty} \left( \log_{3}{x^{2}}-\log_{2}(x^{2}+5x+2) \right)
         = -\infty + 0
         = -\infty
       \end{equation}

       \hrulefill

\end{enumerate}

\hrulefill





\end{document}
