\documentclass[12pt,b5paper]{ltjsarticle}

%\usepackage[margin=15truemm, top=5truemm, bottom=5truemm]{geometry}
\usepackage[margin=15truemm]{geometry}

\usepackage{amsmath,amssymb}
%\pagestyle{headings}
\pagestyle{empty}

%\usepackage{listings,url}
%\renewcommand{\theenumi}{(\arabic{enumi})}

\usepackage{graphicx}

\usepackage{tikz}
\usetikzlibrary {arrows.meta}
\usepackage{wrapfig}	% required for `\wrapfigure' (yatex added)
\usepackage{bm}	% required for `\bm' (yatex added)
\usepackage{luatexja-ruby}	% required for `\ruby'
%% 像Im を定義
%\newcommand{\Img}{\mathop{\mathrm{Im}}\nolimits}

\begin{document}



\begin{enumerate}
 \item 
       \begin{enumerate}
        \item
             $D$は $y^2=x$と$y=x-2$が囲まれた領域
             \begin{equation}
              \int_{D}xy \,\mathrm{d}x\,\mathrm{d}y
             \end{equation}
        \item
             $D= \{ (x,y)\in\mathbb{R}^2 \mid 0\leq x \leq \frac{1}{2}, \, x\leq y\leq 1-x \}$
             \begin{equation}
              \int_{D}(x+y) \,\mathrm{d}x\,\mathrm{d}y
             \end{equation}
        \item
             $D= \{ (x,y)\in\mathbb{R}^2 \mid x^2 + y^2 \leq 1, \, 0\leq y\leq x \}$
             \begin{equation}
              \int_{D}xy^3 \,\mathrm{d}x\,\mathrm{d}y
             \end{equation}
       \end{enumerate}

       \dotfill

       \begin{enumerate}
        \item
             $D=\{ (x,y)\in\mathbb{R}^2 \mid -1\leq y \leq 2, \, y^2\leq x\leq y+2 \}$である。
             \begin{align}
              \int_{D}xy \,\mathrm{d}x\,\mathrm{d}y
              =& \int_{-1}^{2} \int_{y^2}^{y+2}xy \,\mathrm{d}x\,\mathrm{d}y\\
              =& \int_{-1}^{2} \left[ \frac{1}{2}x^2y \right]_{x=y^2}^{x=y+2}\,\mathrm{d}y\\
              =&  \frac{1}{2} \int_{-1}^{2} \left( (y+2)^2y-y^5 \right)\,\mathrm{d}y\\
              =&  \frac{1}{2} \int_{-1}^{2} \left( y^3 + 4y^2 + 4y-y^5 \right)\,\mathrm{d}y\\
              =&  \frac{1}{2} \left[ \frac{1}{4}y^4 + \frac{4}{3}y^3 + 2y^2-\frac{1}{6}y^6 \right]_{y=-1}^{y=2} \\
%              =&  \frac{1}{2} \left( \frac{1}{4}(2^4-1) + \frac{4}{3}(2^3+1) + 2(2^2-1)-\frac{1}{6}(2^6-1) \right)\\
              =& \frac{45}{8}
             \end{align}


%
% sagemathのコード
%
%var("x,y")
%
%print( integral( x*y, x) )
%f(x) = integral( x*y, x, y^2, y+2)
%print( f(x) )
%print( integral( f(x), y) )
%
%integral( integral( x*y, x, y^2, y+2), y , -1, 2)
%



        \item
             \begin{align}
              \int_{D}(x+y) \,\mathrm{d}x\,\mathrm{d}y
              =& \int_{0}^{\frac{1}{2}} \int_{x}^{1-x}(x+y) \,\mathrm{d}y\,\mathrm{d}x\\
              =& \int_{0}^{\frac{1}{2}} \left[ xy + \frac{1}{2}y^2 \right]_{y=x}^{y=1-x} \,\mathrm{d}x\\
              =& \int_{0}^{\frac{1}{2}} \left( x(1-x-x) + \frac{1}{2}((1-x)^2-x^2) \right) \,\mathrm{d}x\\
              =&  \frac{1}{2} \int_{0}^{\frac{1}{2}} \left( -4x^2 + 1 \right) \,\mathrm{d}x\\
              =&  \frac{1}{2} \left[ -\frac{4}{3}x^3 +x \right]_{x=0}^{x=\frac{1}{2}}\\
              =&  \frac{1}{2} \left( -4\left(\frac{1}{2}\right)^3 + \frac{1}{2} \right)\\
              =& \frac{1}{6}
             \end{align}

% sagemath のコード
%
%var("x,y")
%
%print( integral( x+y, y) )
%f(x) = integral( x+y, y, x, 1-x)
%print( f(x) )
%print( integral( f(x), x) )
%
%integral( integral( x+y, y, x, 1-x), x , 0, 1/2)


        \item
             領域$D$を極座標で表記すると次のようになる。
             \begin{equation}
              D=\left\{ (r,\theta)\in\mathbb{R}^2 \mid 0\leq r \leq 1,\ 0\leq \theta \leq \frac{\pi}{4} \right\}
             \end{equation}
             変数変換は次のように行う。
             \begin{equation}
              x=r\cos\theta,\ y=r\sin\theta
             \end{equation}
             これによりヤコビアンは次のようになる。
             \begin{equation}
              \begin{vmatrix}
               \frac{\partial x}{\partial r} & \frac{\partial x}{\partial \theta}\\
               \frac{\partial y}{\partial r} & \frac{\partial y}{\partial \theta}
              \end{vmatrix}
              =
              \begin{vmatrix}
               \cos\theta & -r\sin\theta\\
               \sin\theta & r\cos\theta\\
              \end{vmatrix}
              =r
             \end{equation}

             \begin{align}
              \int_{D}xy^3 \,\mathrm{d}x\,\mathrm{d}y
              =& \int_{0}^{1} \!\!\!\int_{0}^{\frac{\pi}{4}}
              r\cos\theta(r\sin\theta)^3\cdot r\mathrm{d}\theta\mathrm{d}r\\
              &= \int_{0}^{1} r^5\!\!\! \int_{0}^{\frac{\pi}{4}}
                \cos\theta \sin^3\theta\,\mathrm{d}\theta\,\mathrm{d}r \label{tr_sin}\\
              =& \int_{0}^{1} r^5\!\!\! \int_{0}^{\frac{\pi}{4}}
              \left(\frac{1}{4}\sin 2\theta - \frac{1}{8}\sin 4\theta\right)
                \,\mathrm{d}\theta\,\mathrm{d}r\\
              =& \int_{0}^{1} r^5
                \left[ -\frac{1}{8}\cos 2\theta + \frac{1}{32}\cos 4\theta \right]_{\theta=0}^{\theta=\frac{\pi}{4}}
                \,\mathrm{d}r\\
              =& \int_{0}^{1} \frac{r^5}{16}\,\mathrm{d}r\\
              =& \left[ \frac{r^6}{96} \right]_{r=0}^{r=1} = \frac{1}{96}
             \end{align}

%             \begin{equation}
%              \sin 3\theta = 3\sin\theta -4\sin^3\theta
%               \quad \Leftrightarrow \quad
%               \sin^3\theta = \frac{1}{4}( 3\sin\theta - \sin 3\theta )
%             \end{equation}

             式(\ref{tr_sin})は次のように変形をした。
             \begin{align}
              \cos\theta \sin^3\theta
               =& \cos\theta \sin\theta \cdot \sin^2\theta
               = \frac{1}{2}\sin 2\theta \cdot \frac{1}{2}(1-\cos 2\theta)\\
               =& \frac{1}{4}\sin 2\theta - \frac{1}{4}\sin 2\theta\cos 2\theta
               = \frac{1}{4}\sin 2\theta - \frac{1}{8}\sin 4\theta
             \end{align}
             この変形に次の式を用いた。
             \begin{align}
              \sin 2\theta = 2\sin\theta\cos\theta
              \quad \Leftrightarrow & \quad
              \sin\theta\cos\theta = \frac{1}{2}\sin 2\theta\\
              \cos 2\theta = 1-2\sin^2\theta
              \quad \Leftrightarrow & \quad
              \sin^2\theta = \frac{1}{2}(1-\cos 2\theta)
             \end{align}


       \end{enumerate}

       \hrulefill
 \item
      \begin{align}
       C_1:& \left[0,\pi\right]\to \mathbb{R}^2 ,  \quad C_1(t)=(\cos{t},\ \sin{t})\\
       C_2:& \left[\sqrt{\pi},\sqrt{2\pi}\right]\to \mathbb{R}^2 ,  \quad C_2(t)=(\cos{t^2},\ \sin{t^2})\\
       D=& \{ (x,y)\in\mathbb{R}^2 \mid x^2 + y^2 <4 \}\\
       \bm{f}:& D\to \mathbb{R}^2 ,  \quad \bm{f}(x,y)={}^{t}\begin{pmatrix}1 & 0\end{pmatrix}
      \end{align}
      \begin{equation}
       \int_{(C_1,\ C_2)} \bm{f} %\,\mathrm{d}x\,\mathrm{d}y
      \end{equation}

      \dotfill

      写像$C_1,\,C_2$の像は全て\bm{f}の定義域に含まれる。
      つまり、$C_1 \subset D,\ C_2 \subset D$。

      $C_1$上の位置ベクトルを$\bm{c}_1=(\cos t, \sin t)$とする。
      \begin{equation}
       \mathrm{d}\bm{c}_1 =
       \frac{\mathrm{d}\bm{c}_1}{\mathrm{d}t}\mathrm{d}t
        = (-\sin t, \cos t) \mathrm{d}t
      \end{equation}

      \begin{align}
       \int_{C_1}\bm{f}\mathrm{d}\bm{c}_1
       =& \int_{0}^{\pi}(1,0)\cdot (-\sin t, \cos t) \mathrm{d}t\\
       =& \int_{0}^{\pi}(-\sin t) \mathrm{d}t\\
       =& \left[\cos t \right]_{0}^{\pi}
       = -2
      \end{align}

      $C_2$上の位置ベクトルを$\bm{c}_2=(\cos t^2, \sin t^2)$とする。
      \begin{equation}
       \mathrm{d}\bm{c}_2 =
       \frac{\mathrm{d}\bm{c}_2}{\mathrm{d}t}\mathrm{d}t
        = (-2t\sin t^2, 2t\cos t^2) \mathrm{d}t
      \end{equation}

      \begin{align}
       \int_{C_2}\bm{f}\mathrm{d}\bm{c}_2
       =& \int_{\sqrt{\pi}}^{\sqrt{2\pi}}(1,0)\cdot (-2t\sin t^2, 2t\cos t^2) \mathrm{d}t\\
       =& \int_{\sqrt{\pi}}^{\sqrt{2\pi}} (-2t\sin t^2) \mathrm{d}t\\
       =& \left[ \cos t^2\right]_{\sqrt{\pi}}^{\sqrt{2\pi}} = 2
      \end{align}

      $C_1,\, C_2$上の積分を合わせると次のようになる。
      \begin{equation}
       \int_{C_1}\bm{f}\mathrm{d}\bm{c}_1
        + \int_{C_2}\bm{f}\mathrm{d}\bm{c}_2
        =0
      \end{equation}



      \dotfill

      問題は次の式でした。
      \begin{equation}
       \int_{(C_1,\ C_2)} \bm{f} %\,\mathrm{d}x\,\mathrm{d}y
      \end{equation}

      この積分の$(C_1,C_2)$は2つの曲線全体にわたって積分するという意味で捉えました。
      


\end{enumerate}




\end{document}
