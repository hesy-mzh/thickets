\documentclass[12pt,b5paper]{ltjsarticle}

%\usepackage[margin=15truemm, top=5truemm, bottom=5truemm]{geometry}
%\usepackage[margin=10truemm,left=15truemm]{geometry}
\usepackage[margin=10truemm]{geometry}

\usepackage{amsmath,amssymb}
%\pagestyle{headings}
\pagestyle{empty}

%\usepackage{listings,url}
%\renewcommand{\theenumi}{(\arabic{enumi})}

%\usepackage{graphicx}

%\usepackage{tikz}
%\usetikzlibrary {arrows.meta}
%\usepackage{wrapfig}
%\usepackage{bm}

% ルビを振る
%\usepackage{luatexja-ruby}	% required for `\ruby'

%% 核Ker 像Im Hom を定義
%\newcommand{\Img}{\mathop{\mathrm{Im}}\nolimits}
%\newcommand{\Ker}{\mathop{\mathrm{Ker}}\nolimits}
%\newcommand{\Hom}{\mathop{\mathrm{Hom}}\nolimits}

%\DeclareMathOperator{\Rot}{rot}
%\DeclareMathOperator{\Div}{div}
%\DeclareMathOperator{\Grad}{grad}
%\DeclareMathOperator{\arcsinh}{arcsinh}
%\DeclareMathOperator{\arccosh}{arccosh}
%\DeclareMathOperator{\arctanh}{arctanh}



%\usepackage{listings,url}
%
%\lstset{
%%プログラム言語(複数の言語に対応,C,C++も可)
%  language = Python,
%%  language = Lisp,
%%  language = C,
%  %背景色と透過度
%  %backgroundcolor={\color[gray]{.90}},
%  %枠外に行った時の自動改行
%  breaklines = true,
%  %自動改行後のインデント量(デフォルトでは20[pt])
%  breakindent = 10pt,
%  %標準の書体
%%  basicstyle = \ttfamily\scriptsize,
%  basicstyle = \ttfamily,
%  %コメントの書体
%%  commentstyle = {\itshape \color[cmyk]{1,0.4,1,0}},
%  %関数名等の色の設定
%  classoffset = 0,
%  %キーワード(int, ifなど)の書体
%%  keywordstyle = {\bfseries \color[cmyk]{0,1,0,0}},
%  %表示する文字の書体
%  %stringstyle = {\ttfamily \color[rgb]{0,0,1}},
%  %枠 "t"は上に線を記載, "T"は上に二重線を記載
%  %他オプション:leftline,topline,bottomline,lines,single,shadowbox
%  frame = TBrl,
%  %frameまでの間隔(行番号とプログラムの間)
%  framesep = 5pt,
%  %行番号の位置
%  numbers = left,
%  %行番号の間隔
%  stepnumber = 1,
%  %行番号の書体
%%  numberstyle = \tiny,
%  %タブの大きさ
%  tabsize = 4,
%  %キャプションの場所("tb"ならば上下両方に記載)
%  captionpos = t
%}



\begin{document}

\hrulefill

すべての自然数$n\in\mathbb{N}$について、
以下のことが成立することを
それぞれ数学的帰納法で示せ。
\begin{enumerate}
 \item $\displaystyle \begin{pmatrix}\alpha & 0 \\ 0 & \beta \end{pmatrix}^{\!\!n} = \begin{pmatrix}\alpha^{n} & 0 \\ 0 & \beta^{n} \end{pmatrix}$

       \hrulefill

       $n=1$の場合を考える。
       この時、左辺と右辺は一致する。
       \begin{equation}
        \begin{pmatrix}\alpha & 0 \\ 0 & \beta \end{pmatrix}^{\!\!1} = \begin{pmatrix}\alpha^{1} & 0 \\ 0 & \beta^{1} \end{pmatrix}
       \end{equation}

       \dotfill

       $n=k$の時、式が成り立っていると仮定し、$n=k+1$の場合を考える。
       $n=k$では等号が成立するので、次の式を利用する。
       \begin{equation}
        \begin{pmatrix}\alpha & 0 \\ 0 & \beta \end{pmatrix}^{\!\!k} = \begin{pmatrix}\alpha^{k} & 0 \\ 0 & \beta^{k} \end{pmatrix}
       \end{equation}

       $n=k+1$の場合の左辺を計算する。
       \begin{align}
        \begin{pmatrix}\alpha & 0 \\ 0 & \beta \end{pmatrix}^{\!\!k+1}
           &= \begin{pmatrix}\alpha & 0 \\ 0 & \beta \end{pmatrix}^{k} \begin{pmatrix}\alpha & 0 \\ 0 & \beta \end{pmatrix}
           = \begin{pmatrix}\alpha^{k} & 0 \\ 0 & \beta^{k} \end{pmatrix} \begin{pmatrix}\alpha & 0 \\ 0 & \beta \end{pmatrix}\\
           &= \begin{pmatrix} \alpha^{k}\alpha+0\cdot0 & \alpha^{k}\cdot0+0\cdot\beta \\ 0\cdot\alpha+\beta^{k}\cdot0 & 0\cdot0+\beta^{k}\beta \end{pmatrix}
           = \begin{pmatrix}\alpha^{k+1} & 0 \\ 0 & \beta^{k+1} \end{pmatrix}
       \end{align}
       
       よって、$n=k$が成り立つなら$n=k+1$も成り立つことがわかる。

       \dotfill

       以上により、
       $\displaystyle \begin{pmatrix}\alpha & 0 \\ 0 & \beta \end{pmatrix}^{\!\!n} = \begin{pmatrix}\alpha^{n} & 0 \\ 0 & \beta^{n} \end{pmatrix}$
       は成立する。

       \hrulefill

 \item $\displaystyle \begin{pmatrix} \lambda & 1 \\ 0 & \lambda \end{pmatrix}^{\!\!n} = \begin{pmatrix} \lambda^{n} & n\lambda^{n-1} \\ 0 & \lambda^{n} \end{pmatrix}$

       \hrulefill

       $n=1$の場合を考える。
       この時、左辺と右辺は一致する。($\lambda\ne 0$の時)
       \begin{equation}
        \begin{pmatrix} \lambda & 1 \\ 0 & \lambda \end{pmatrix}^{\!\!1} = \begin{pmatrix} \lambda^{1} & 1\cdot\lambda^{1-1} \\ 0 & \lambda^{1} \end{pmatrix}
       \end{equation}

       \dotfill

       $n=2$の場合を考える。
       この時、左辺と右辺は一致する。
       \begin{equation}
        \begin{pmatrix} \lambda & 1 \\ 0 & \lambda \end{pmatrix}^{\!\!2}
          = \begin{pmatrix} \lambda & 1 \\ 0 & \lambda \end{pmatrix}\begin{pmatrix} \lambda & 1 \\ 0 & \lambda \end{pmatrix}
          = \begin{pmatrix} \lambda\lambda + 1\cdot0 & \lambda\cdot1+1\cdot\lambda \\ 0\cdot\lambda+\lambda\cdot0 & 0\cdot1+\lambda\cdot\lambda \end{pmatrix}
          = \begin{pmatrix} \lambda^{2} & 2\cdot\lambda^{2-1} \\ 0 & \lambda^{2} \end{pmatrix}
       \end{equation}

       \dotfill

       $n=k$の時、式が成り立っていると仮定し、$n=k+1$の場合を考える。($k>1$)
       $n=k$では等号が成立するので、次の式を利用する。
       \begin{equation}
        \begin{pmatrix} \lambda & 1 \\ 0 & \lambda \end{pmatrix}^{\!\!k} = \begin{pmatrix} \lambda^{k} & k\lambda^{k-1} \\ 0 & \lambda^{k} \end{pmatrix}
       \end{equation}

       $n=k+1$の場合の左辺を計算する。
       \begin{align}
        \begin{pmatrix} \lambda & 1 \\ 0 & \lambda \end{pmatrix}^{\!\!k+1}
           &= \begin{pmatrix} \lambda & 1 \\ 0 & \lambda \end{pmatrix}^{\!\!k} \begin{pmatrix} \lambda & 1 \\ 0 & \lambda \end{pmatrix}
           = \begin{pmatrix} \lambda^{k} & k\lambda^{k-1} \\ 0 & \lambda^{k} \end{pmatrix} \begin{pmatrix} \lambda & 1 \\ 0 & \lambda \end{pmatrix}\\
           &= \begin{pmatrix} \lambda^{k}\cdot\lambda+k\lambda^{k-1}\cdot0 & \lambda^{k}\cdot1+k\lambda^{k-1}\cdot\lambda \\ 0\cdot\lambda+\lambda^{k}\cdot0 & 0\cdot1+\lambda^{k}\cdot\lambda \end{pmatrix}\\
           &= \begin{pmatrix} \lambda^{k+1} & (k+1)\lambda^{k} \\ 0 & \lambda^{k+1} \end{pmatrix}
       \end{align}

       よって、$n=k$が成り立つなら$n=k+1$も成り立つことがわかる。

       \dotfill

       以上により、次が示せた。
       
       \begin{itemize}
        \item
             $\lambda\ne 0$の時、$1$以上の全ての整数で成り立つ。
        \item
             $\lambda= 0$の時、$2$以上の全ての整数で成り立つ。
       \end{itemize}

       \hrulefill

\end{enumerate}

\hrulefill

\end{document}
