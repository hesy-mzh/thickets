\documentclass[12pt,b5paper]{ltjsarticle}

%\usepackage[margin=15truemm, top=5truemm, bottom=5truemm]{geometry}
\usepackage[margin=10truemm]{geometry}

\usepackage{amsmath,amssymb}
%\pagestyle{headings}
\pagestyle{empty}

%\usepackage{listings,url}
%\renewcommand{\theenumi}{(\arabic{enumi})}

%\usepackage{graphicx}

%\usepackage{tikz}
%\usetikzlibrary {arrows.meta}
%\usepackage{wrapfig}	% required for `\wrapfigure' (yatex added)
%\usepackage{bm}	% required for `\bm' (yatex added)

% ルビを振る
%\usepackage{luatexja-ruby}	% required for `\ruby'

%% 核Ker 像Im Hom を定義
%\newcommand{\Img}{\mathop{\mathrm{Im}}\nolimits}
%\newcommand{\Ker}{\mathop{\mathrm{Ker}}\nolimits}
%\newcommand{\Hom}{\mathop{\mathrm{Hom}}\nolimits}

%\DeclareMathOperator{\Rot}{rot}
%\DeclareMathOperator{\Div}{div}
%\DeclareMathOperator{\Grad}{grad}
%\DeclareMathOperator{\arcsinh}{arcsinh}
%\DeclareMathOperator{\arccosh}{arccosh}
%\DeclareMathOperator{\arctanh}{arctanh}



\begin{document}

\textbf{計算}

\begin{equation}
 \sqrt{57^3 - 38^3}
\end{equation}

\hrulefill

$57 = 60-3$より
\begin{equation}
 57^3 = (60 -3 )^3 =( 3(20-1) )^3 = 3^3\times 19^3
\end{equation}

$38 = 20 -2$より
\begin{equation}
 38^3 = (20 - 2)^3 =(2(20-1))^3 = 2^3 \times 19^3
\end{equation}

これらを用いて$57^3 - 38^3$を計算する。
\begin{equation}
 57^3 - 38^3 = 3^3\times 19^3 - 2^3 \times 19^3 = 19^3(3^3 - 2^3) = 19^4
\end{equation}

よって
\begin{equation}
 \sqrt{57^3 - 38^3} = \sqrt{19^4} = 19^2
\end{equation}


$19=20-1$より
\begin{equation}
 19^2 = (20-1)^2 = 20^2 -2\times 20 \times 1 + 1 = 400 - 40 +1 = 361
\end{equation}

\dotfill

\begin{equation}
 \sqrt{57^3 - 38^3} = 361
\end{equation}


\end{document}

