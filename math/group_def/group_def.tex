\documentclass[12pt,b5paper]{ltjsarticle}

%\usepackage[margin=15truemm, top=5truemm, bottom=5truemm]{geometry}
\usepackage[margin=10truemm]{geometry}

\usepackage{amsmath,amssymb}
%\pagestyle{headings}
\pagestyle{empty}

%\usepackage{listings,url}
%\renewcommand{\theenumi}{(\arabic{enumi})}

%\usepackage{graphicx}

%\usepackage{tikz}
%\usetikzlibrary {arrows.meta}
%\usepackage{wrapfig}	% required for `\wrapfigure' (yatex added)
%\usepackage{bm}	% required for `\bm' (yatex added)

% ルビを振る
%\usepackage{luatexja-ruby}	% required for `\ruby'

%% 核Ker 像Im Hom を定義
%\newcommand{\Img}{\mathop{\mathrm{Im}}\nolimits}
%\newcommand{\Ker}{\mathop{\mathrm{Ker}}\nolimits}
%\newcommand{\Hom}{\mathop{\mathrm{Hom}}\nolimits}

%\DeclareMathOperator{\Rot}{rot}
%\DeclareMathOperator{\Div}{div}
%\DeclareMathOperator{\Grad}{grad}
%\DeclareMathOperator{\arcsinh}{arcsinh}
%\DeclareMathOperator{\arccosh}{arccosh}
%\DeclareMathOperator{\arctanh}{arctanh}



\begin{document}

\hrulefill
\textbf{群}
\hrulefill

定義

集合$G$がある演算$\blacktriangle$で閉じているとする。

\begin{enumerate}
 \item
      結合法則を満たす。
      \begin{equation}
       ({}^{\forall} a,b,c \in G) \qquad
        (a \blacktriangle b) \blacktriangle c = a \blacktriangle (b \blacktriangle c)
      \end{equation}

 \item
      単位元$e$が存在する。
      \begin{equation}
       ({}^{\forall} a \in G) \qquad
        a \blacktriangle e = e \blacktriangle a = a
      \end{equation}

 \item
      逆元が存在する。
      \begin{equation}
       ({}^{\forall} a \in G) \qquad
        {}^\exists a^{-1} \in G \quad
        s.t. \quad a \blacktriangle a^{-1} = a^{-1} \blacktriangle a = e
      \end{equation}
\end{enumerate}


\hrulefill
\textbf{問題}
\hrulefill



集合$\{0,2,3\}$上の演算$\blacktriangle$を次のように定義する。
\begin{align}
 0\blacktriangle0 =& 0 & 0\blacktriangle2 =& 2 & 0\blacktriangle3 =& 3\\
 2\blacktriangle0 =& 2 & 2\blacktriangle2 =& 2 & 2\blacktriangle3 =& 0\\
 3\blacktriangle0 =& 3 & 3\blacktriangle2 =& 0 & 3\blacktriangle3 =& 3
\end{align}

次の表は上の演算をまとめたものである。
\begin{center}
 \begin{tabular}{|c||c|c|c|}
  \hline
  左 $\backslash$ 右 & 0 & 2 & 3 \\
  \hline \hline
  0 & 0 & 2 & 3 \\ \hline
  2 & 2 & 2 & 0 \\ \hline
  3 & 3 & 0 & 3 \\ \hline
 \end{tabular}
\end{center}
この時、集合$\{0,2,3\}$と演算$\blacktriangle$について
次の性質が成立するか判定せよ。
\begin{enumerate}
 \item 単位元が存在する。
 \item 任意の元の逆元が存在する。
 \item 結合率を満たす。
\end{enumerate}

\dotfill

\begin{enumerate}
 \item
      単位元を$e$とおくと、
      次の3つの式を満たす。
      \begin{align}
       0 \blacktriangle e =& e \blacktriangle 0 = 0 \label{unit_zero}\\
       2 \blacktriangle e =& e \blacktriangle 2 = 2 \label{unit_two}\\
       3 \blacktriangle e =& e \blacktriangle 3 = 3 \label{unit_three}
      \end{align}

      式(\ref{unit_zero})を満たすのは$0\in \{0,2,3\}$である。
      式(\ref{unit_two})を満たすのは$0,2\in \{0,2,3\}$である。
      式(\ref{unit_three})を満たすのは$0,3\in \{0,2,3\}$である。

      以上により単位元は$0\in\{0,2,3\}$である。

 \item
      上の内容から$0$が単位元である。
      $a$の逆元を$a^{-1}$とすると
      次の3つの式を満たす。
      \begin{align}
       0 \blacktriangle 0^{-1} =& 0^{-1} \blacktriangle 0 = 0 \label{inverse_zero}\\
       2 \blacktriangle 2^{-1} =& 2^{-1} \blacktriangle 2 = 0 \label{inverse_two}\\
       3 \blacktriangle 3^{-1} =& 3^{-1} \blacktriangle 3 = 0 \label{inverse_three}
      \end{align}

      式(\ref{inverse_zero})を満たす$0^{-1}$は$0^{-1}=0$である。
      式(\ref{inverse_two})を満たす$2^{-1}$は$2^{-1}=3$である。
      式(\ref{inverse_three})を満たす$3^{-1}$は$3^{-1}=2$である。

      以上により$\{0,2,3\}$の任意の元に対し逆元は存在する。

 \item
      結合律を確認する。
%      演算$\blacktriangle$が可換であることを仮定しないので、
      次の6つの式が成立するか確認する。
      \begin{align}
       0 \blacktriangle ( 2 \blacktriangle 3 ) = (0 \blacktriangle  2) \blacktriangle 3 \label{asso023}\\
       0 \blacktriangle ( 3 \blacktriangle 2 ) = (0 \blacktriangle  3) \blacktriangle 2 \label{asso032}\\
       2 \blacktriangle ( 3 \blacktriangle 0 ) = (2 \blacktriangle  3) \blacktriangle 0 \label{asso230}\\
       2 \blacktriangle ( 0 \blacktriangle 3 ) = (2 \blacktriangle  0) \blacktriangle 3 \label{asso203}\\
       3 \blacktriangle ( 0 \blacktriangle 2 ) = (3 \blacktriangle  0) \blacktriangle 2 \label{asso302}\\
       3 \blacktriangle ( 2 \blacktriangle 0 ) = (3 \blacktriangle  2) \blacktriangle 0 \label{asso320}
      \end{align}


      式(\ref{asso023})について
      \begin{equation}
       \text{左辺} \quad
        0 \blacktriangle ( 2 \blacktriangle 3 )
        = 0 \blacktriangle 0 = 0 \qquad
       \text{右辺} \quad
        (0 \blacktriangle  2) \blacktriangle 3
        = 2 \blacktriangle 3 = 0
      \end{equation}

      式(\ref{asso032})について
      \begin{equation}
       \text{左辺} \quad
        0 \blacktriangle ( 3 \blacktriangle 2 )
        = 0 \blacktriangle 0 = 0 \qquad
       \text{右辺} \quad
        (0 \blacktriangle  3) \blacktriangle 2
        = 3 \blacktriangle 2 = 0
      \end{equation}

      式(\ref{asso230})について
      \begin{equation}
       \text{左辺} \quad
        2 \blacktriangle ( 3 \blacktriangle 0 )
        = 2 \blacktriangle 3 = 0 \qquad
       \text{右辺} \quad
        (2 \blacktriangle  3) \blacktriangle 0
        = 0 \blacktriangle 0 = 0
      \end{equation}

      式(\ref{asso203})について
      \begin{equation}
       \text{左辺} \quad
        2 \blacktriangle ( 0 \blacktriangle 3 )
        = 2 \blacktriangle 3 = 0 \qquad
       \text{右辺} \quad
        (2 \blacktriangle  0) \blacktriangle 3
        = 2 \blacktriangle 3 = 0
      \end{equation}

      式(\ref{asso302})について
      \begin{equation}
       \text{左辺} \quad
        3 \blacktriangle ( 0 \blacktriangle 2 )
        = 3 \blacktriangle 2 = 0 \qquad
       \text{右辺} \quad
        (3 \blacktriangle  0) \blacktriangle 2
        = 3 \blacktriangle 2 = 0
      \end{equation}

      式(\ref{asso320})について
      \begin{equation}
       \text{左辺} \quad
        3 \blacktriangle ( 2 \blacktriangle 0 )
        = 3 \blacktriangle 2 = 0 \qquad
       \text{右辺} \quad
        (3 \blacktriangle  2) \blacktriangle 0
        = 0 \blacktriangle 0 = 0
      \end{equation}


      以上により結合律を満たす。


\end{enumerate}


\end{document}

