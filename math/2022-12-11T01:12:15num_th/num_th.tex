\documentclass[12pt,b5paper]{ltjsarticle}

%\usepackage[margin=15truemm, top=5truemm, bottom=5truemm]{geometry}
%\usepackage[margin=10truemm,left=15truemm]{geometry}
\usepackage[margin=10truemm]{geometry}

\usepackage{amsmath,amssymb}
%\pagestyle{headings}
\pagestyle{empty}

%\usepackage{listings,url}
%\renewcommand{\theenumi}{(\arabic{enumi})}

%\usepackage{graphicx}

%\usepackage{tikz}
%\usetikzlibrary {arrows.meta}
%\usepackage{wrapfig}	% required for `\wrapfigure' (yatex added)
%\usepackage{bm}	% required for `\bm' (yatex added)

% ルビを振る
%\usepackage{luatexja-ruby}	% required for `\ruby'

%% 核Ker 像Im Hom を定義
%\newcommand{\Img}{\mathop{\mathrm{Im}}\nolimits}
%\newcommand{\Ker}{\mathop{\mathrm{Ker}}\nolimits}
%\newcommand{\Hom}{\mathop{\mathrm{Hom}}\nolimits}

%\DeclareMathOperator{\Rot}{rot}
%\DeclareMathOperator{\Div}{div}
%\DeclareMathOperator{\Grad}{grad}
%\DeclareMathOperator{\arcsinh}{arcsinh}
%\DeclareMathOperator{\arccosh}{arccosh}
%\DeclareMathOperator{\arctanh}{arctanh}



%\usepackage{listings,url}
%
%\lstset{
%%プログラム言語(複数の言語に対応,C,C++も可)
%  language = Python,
%%  language = Lisp,
%%  language = C,
%  %背景色と透過度
%  %backgroundcolor={\color[gray]{.90}},
%  %枠外に行った時の自動改行
%  breaklines = true,
%  %自動改行後のインデント量(デフォルトでは20[pt])
%  breakindent = 10pt,
%  %標準の書体
%%  basicstyle = \ttfamily\scriptsize,
%  basicstyle = \ttfamily,
%  %コメントの書体
%%  commentstyle = {\itshape \color[cmyk]{1,0.4,1,0}},
%  %関数名等の色の設定
%  classoffset = 0,
%  %キーワード(int, ifなど)の書体
%%  keywordstyle = {\bfseries \color[cmyk]{0,1,0,0}},
%  %表示する文字の書体
%  %stringstyle = {\ttfamily \color[rgb]{0,0,1}},
%  %枠 "t"は上に線を記載, "T"は上に二重線を記載
%  %他オプション:leftline,topline,bottomline,lines,single,shadowbox
%  frame = TBrl,
%  %frameまでの間隔(行番号とプログラムの間)
%  framesep = 5pt,
%  %行番号の位置
%  numbers = left,
%  %行番号の間隔
%  stepnumber = 1,
%  %行番号の書体
%%  numberstyle = \tiny,
%  %タブの大きさ
%  tabsize = 4,
%  %キャプションの場所("tb"ならば上下両方に記載)
%  captionpos = t
%}



\begin{document}


\begin{enumerate}
 \item
      次の$a,b$の最大公約数$d$、および $sa+tb=d$を満たす
      整数$s,t$を求めよ。
      \begin{enumerate}
       \item
            $a=210,b=54$

            \dotfill

            ユークリッドの互除法を用いる。
            \begin{align}
             210 =& 54 \times 3 + 48\label{e1-1}\\
             54 =& 48 \times 1 + 6\label{e1-2}\\
             48 =& 6 \times 8
            \end{align}

            これにより最大公約数は$6$である。

            式(\ref{e1-1})と(\ref{e1-2})を変形すると
            $210 - 54\times 3 = 48$、
            $54 - 48\times 1 =6$となるので、
            $48$に代入をする。
            \begin{align}
             54 - 48\times 1 =6\\
             54 - (210 - 54\times 3)\times 1 =6\\
             54 - 210 + 54\times 3 =6\\
             54 \times 4 - 210 =6\\
             4b-a=6
            \end{align}

            よって、$(s,t)=(-1,4),d=6$である。

            $-a+4b=6$であるが、最大公約数$6$で両辺を割る。
            \begin{equation}
             -1\times 35 +4\times 9 =1
            \end{equation}

            $sa+tb=6$を満たす$s,t$であれば
            同様に最大公約数で割って次の式が得られる。
            \begin{equation}
             s\times 35 + t\times 9 =1
            \end{equation}

            2つの式の差を考えれば次の式が得られる。
            \begin{align}
             35(s+1)+9(t-4)=0\\
             35(s+1)=-9(t-4)\label{e1-3}
            \end{align}

            $a$と$b$を最大公約数で割って$35$と$9$を得たので
            この2つの数は互いに素である。
            その為、等号が成り立っているので
            $s+1$は9の倍数となる。

            $s+1=9k (k\in\mathbb{Z})$とすると
            式(\ref{e1-3})を変形し$t$が得られる。
            \begin{align}
             35\times 9k =& -9(t-4)\\
             t=& -35k+4
            \end{align}

            よって、$s,t$は次のようになる。
            \begin{equation}
             (s,t) = ( 9k-1,\ -35k+4 ) \qquad k\in\mathbb{Z}
            \end{equation}


            \hrulefill

       \item
            $a=210,b=55$

            \dotfill

            上の問いと同じように
            ユークリッドの互除法
            を利用する。

            \begin{align}
             210 =& 55\times 4 -10\\
             55 =& (-10)\times (-5) +5\\
             -10 =& 5\times (-2)
            \end{align}
            これにより最大公約数は5である。

            $-10$に代入する。
            \begin{align}
             55 - (-10)\times (-5) = 5\\
             55 - (210-55\times 4)\times (-5) = 5\\
             55 - 210 \times (-5) +55\times (-20) = 5\\
             210\times 5 + 55\times (-19) =5\\
             5a-19b=5
            \end{align}

            よって、$s,t$は次のようになる。
            $(s,t)=(5,-19),d=5$


            $5a-19b=5$と$sa+tb=5$を
            最大公約数5 で割ると次の式が得られる。
            \begin{align}
             5\times 42 -19\times 11 =&1\\
             s\times 42 +t\times 11 =&1
            \end{align}

            差を取って式を移項すると次の式が得られる。
            \begin{equation}
             42(s-5) = -11(t+19)\label{e2-3}
            \end{equation}

            $42$と$11$は互いに素であるので、
            $s-5$が$11$の倍数である。
            そこで$k\in\mathbb{Z}$を用いて
            $s=11k+5$とする。

            これを式(\ref{e2-3})に代入すると
            $t= -42k -19$となる。

            よって求めるべき数は次のようになる。
            \begin{equation}
             (s,t) = (11k+5,\ -42k-19) \qquad k\in\mathbb{Z}
            \end{equation}



            \hrulefill

       \item
            $a=210,b=56$

            \dotfill

            \begin{align}
             210 =& 56 \times 4 -14\\
             56 =& (-14)\times (-4)
            \end{align}
            最大公約数は$14$である。

            \begin{align}
             210 - 56 \times 4 = -14\\
             -210 + 56 \times 4 = 14\\
             -a+4b=14
            \end{align}

            これにより$s,t$は次のように求まる。
            $(s,t)=(-1,4),d=14$


            $-a+4b=14,\ sa+tb=14$を最大公約数で割る。
            \begin{align}
             -1\times 15 +4\times 4 =& 1\\
             s\times 15 +t\times 4 =& 1\\
            \end{align}

            差を取って移項する。
            \begin{equation}
             15(s+1) = -4(t-4)
            \end{equation}

            $15$と$4$は互いに素なので
            $k\in\mathbb{Z}$を用いて
            $s=4k-1$とする。
            これを代入すると
            $t=-15k+4$となる。

            \begin{equation}
             (s,t)=(4k-1,\ -15k+4) \qquad k\in\mathbb{Z}
            \end{equation}
            



            \hrulefill
      \end{enumerate}
 \item
      $x^{100}-3x^{99}+x$を$x^2-2x-3$で割った余りを答えよ。

      \dotfill

      $x^2-2x-3$は2次式であるので
      余りは1次式となる。
      余りを$r(x)=r_1x+r_2$とおく。
      商を$q(x)$とすると次のような関係がある。

      \begin{align}
       x^{100}-3x^{99}+x
       =& (x^2-2x-3)q(x)+r(x)\\
       =& (x+1)(x-3)q(x)+r(x)
      \end{align}

      この為、
      $x^{100}-3x^{99}+x$に
      $x=-1$を代入した値と$r(-1)$が等しくなり、
      $x=3$を代入した値と$r(3)$が等しくなる。

      \begin{align}
       (-1)^{100}-3(-1)^{99}+(-1) =& 3\\
       3^{100}-3(3)^{99}+3 =& 3
      \end{align}

      そこで$x=-1,x=3$の場合の式を作って連立方程式を解く。
      \begin{equation}
       \begin{cases}
        -r_1+r_2 = 3\\
        3r_1+r_2 = 3
       \end{cases}
      \end{equation}

      $(r_1,r_2)=(0,3)$となるので、
      余りは$r(x)=3$となる。


      
      \hrulefill

\end{enumerate}

\end{document}
