\documentclass[12pt,b5paper]{ltjsarticle}

%\usepackage[margin=15truemm, top=5truemm, bottom=5truemm]{geometry}
%\usepackage[margin=10truemm,left=15truemm]{geometry}
\usepackage[margin=10truemm]{geometry}

\usepackage{amsmath,amssymb}
%\pagestyle{headings}
\pagestyle{empty}

%\usepackage{listings,url}
%\renewcommand{\theenumi}{(\arabic{enumi})}

\usepackage{graphicx}

%\usepackage{tikz}
%\usetikzlibrary {arrows.meta}
%\usepackage{wrapfig}
%\usepackage{bm}

% ルビを振る
\usepackage{luatexja-ruby}

%% 核Ker 像Im Hom を定義
%\newcommand{\Img}{\mathop{\mathrm{Im}}\nolimits}
%\newcommand{\Ker}{\mathop{\mathrm{Ker}}\nolimits}
%\newcommand{\Hom}{\mathop{\mathrm{Hom}}\nolimits}

%\DeclareMathOperator{\Rot}{rot}
%\DeclareMathOperator{\Div}{div}
%\DeclareMathOperator{\Grad}{grad}
%\DeclareMathOperator{\arcsinh}{arcsinh}
%\DeclareMathOperator{\arccosh}{arccosh}
%\DeclareMathOperator{\arctanh}{arctanh}



%\usepackage{listings,url}
%
%\lstset{
%%プログラム言語(複数の言語に対応,C,C++も可)
%  language = Python,
%%  language = Lisp,
%%  language = C,
%  %背景色と透過度
%  %backgroundcolor={\color[gray]{.90}},
%  %枠外に行った時の自動改行
%  breaklines = true,
%  %自動改行後のインデント量(デフォルトでは20[pt])
%  breakindent = 10pt,
%  %標準の書体
%%  basicstyle = \ttfamily\scriptsize,
%  basicstyle = \ttfamily,
%  %コメントの書体
%%  commentstyle = {\itshape \color[cmyk]{1,0.4,1,0}},
%  %関数名等の色の設定
%  classoffset = 0,
%  %キーワード(int, ifなど)の書体
%%  keywordstyle = {\bfseries \color[cmyk]{0,1,0,0}},
%  %表示する文字の書体
%  %stringstyle = {\ttfamily \color[rgb]{0,0,1}},
%  %枠 "t"は上に線を記載, "T"は上に二重線を記載
%  %他オプション:leftline,topline,bottomline,lines,single,shadowbox
%  frame = TBrl,
%  %frameまでの間隔(行番号とプログラムの間)
%  framesep = 5pt,
%  %行番号の位置
%  numbers = left,
%  %行番号の間隔
%  stepnumber = 1,
%  %行番号の書体
%%  numberstyle = \tiny,
%  %タブの大きさ
%  tabsize = 4,
%  %キャプションの場所("tb"ならば上下両方に記載)
%  captionpos = t
%}



\begin{document}

\hrulefill

次の広義2重積分を計算せよ。

\begin{equation}
 \iint_{D}\frac{2y}{x^2+y^2}\mathrm{d}x\mathrm{d}y,
  \quad
  D=\{(x,y)\in\mathbb{R} \mid 0\leq y \leq x \leq 1 , x > 0\}
\end{equation}

\dotfill

$\displaystyle f(x)=\frac{2y}{x^2+y^2}$とし、
$D_{n}=\{ (x,y) \mid \frac{1}{n} \leq x \leq 1,\ 0\leq y \leq x \}$
とする。
$\{D_n\}$は$D$の近似列である。

$D$において
$x>0,y\geq0$であるので
$f(x)\geq0$である。

\begin{center}
 \fbox{
 \begin{minipage}[c]{400pt}
  $D$は領域の英語 Domain の頭文字から、
  $n$は自然数 natural number または 数 number の頭文字からとっているので
  暗にそれを示唆している

  特に$n$を自然数として数列を構成したい為、
  $D_n$は$\frac{1}{n}$を用いて定義している
 \end{minipage}
 }

 \begin{equation}
  \frac{\mathrm{d}}{\mathrm{d}y} \log(x^2+y^2)
   = \frac{2y}{x^2+y^2}
   = f(x)
 \end{equation}
 このように$y$の偏微分を考えると$f(x)$が得られる。
 この為、次の積分が得られる。
 \begin{equation}
  \int f(x) \mathrm{d}y = \log(x^2+y^2) +C
   \qquad (C:\text{積分定数})
 \end{equation}
\end{center}

$y$は$0 \leq y \leq x$であるから
定積分は次のようになる。
\begin{equation}
 \int_{0}^{x}f(x)\mathrm{d}y
  = \left[ \log(x^2+y^2) \right]_{y=0}^{y=x}
  = \log(x^2+x^2) - \log(x^2+0^2)
  = \log 2
\end{equation}

領域$D_{n}$上での積分値を$I_{n}$として
数列$\{ I_{n} \}$を作る。
この数列の極限が広義積分の値となる。
\begin{align}
 \lim_{n\to\infty}I_{n}
  =& \lim_{n\to\infty}
  \iint_{D_{n}} \frac{2y}{x^2+y^2} \mathrm{d}x\mathrm{d}y
  = \lim_{n\to\infty}
  \int _{\frac{1}{n}}^{1}\int_{0}^{x} \frac{2y}{x^2+y^2} \mathrm{d}y\mathrm{d}x\\
 =& \lim_{n\to\infty}
  \int _{\frac{1}{n}}^{1} \log2 \mathrm{d}x
 = \lim_{n\to\infty}
 \left( 1-\frac{1}{n} \right)\log2
 = \log 2
\end{align}

よって広義積分は次のように求められる。
\begin{equation}
 \iint_{D} \frac{2y}{x^2+y^2} \mathrm{d}x\mathrm{d}y
  =\log 2
\end{equation}


\hrulefill

\end{document}
