\documentclass[12pt,b5paper]{ltjsarticle}

%\usepackage[margin=15truemm, top=5truemm, bottom=5truemm]{geometry}
%\usepackage[margin=10truemm,left=15truemm]{geometry}
\usepackage[margin=10truemm]{geometry}

\usepackage{amsmath,amssymb}
%\pagestyle{headings}
\pagestyle{empty}

%\usepackage{listings,url}
%\renewcommand{\theenumi}{(\arabic{enumi})}

%\usepackage{graphicx}

%\usepackage{tikz}
%\usetikzlibrary {arrows.meta}
%\usepackage{wrapfig}	% required for `\wrapfigure' (yatex added)
%\usepackage{bm}	% required for `\bm' (yatex added)

% ルビを振る
%\usepackage{luatexja-ruby}	% required for `\ruby'

%% 核Ker 像Im Hom を定義
%\newcommand{\Img}{\mathop{\mathrm{Im}}\nolimits}
%\newcommand{\Ker}{\mathop{\mathrm{Ker}}\nolimits}
%\newcommand{\Hom}{\mathop{\mathrm{Hom}}\nolimits}

%\DeclareMathOperator{\Rot}{rot}
%\DeclareMathOperator{\Div}{div}
%\DeclareMathOperator{\Grad}{grad}
%\DeclareMathOperator{\arcsinh}{arcsinh}
%\DeclareMathOperator{\arccosh}{arccosh}
%\DeclareMathOperator{\arctanh}{arctanh}



%\usepackage{listings,url}
%
%\lstset{
%%プログラム言語(複数の言語に対応,C,C++も可)
%  language = Python,
%%  language = Lisp,
%%  language = C,
%  %背景色と透過度
%  %backgroundcolor={\color[gray]{.90}},
%  %枠外に行った時の自動改行
%  breaklines = true,
%  %自動改行後のインデント量(デフォルトでは20[pt])
%  breakindent = 10pt,
%  %標準の書体
%%  basicstyle = \ttfamily\scriptsize,
%  basicstyle = \ttfamily,
%  %コメントの書体
%%  commentstyle = {\itshape \color[cmyk]{1,0.4,1,0}},
%  %関数名等の色の設定
%  classoffset = 0,
%  %キーワード(int, ifなど)の書体
%%  keywordstyle = {\bfseries \color[cmyk]{0,1,0,0}},
%  %表示する文字の書体
%  %stringstyle = {\ttfamily \color[rgb]{0,0,1}},
%  %枠 "t"は上に線を記載, "T"は上に二重線を記載
%  %他オプション:leftline,topline,bottomline,lines,single,shadowbox
%  frame = TBrl,
%  %frameまでの間隔(行番号とプログラムの間)
%  framesep = 5pt,
%  %行番号の位置
%  numbers = left,
%  %行番号の間隔
%  stepnumber = 1,
%  %行番号の書体
%%  numberstyle = \tiny,
%  %タブの大きさ
%  tabsize = 4,
%  %キャプションの場所("tb"ならば上下両方に記載)
%  captionpos = t
%}



\begin{document}


\begin{enumerate}
 \item
      次の広義積分が絶対収束することを示せ。
      \begin{enumerate}
       \item
            $\displaystyle
            \int_{0}^{\infty} \frac{\sqrt{x}}{x^2-2x+2}\mathrm{d}x$


            \dotfill

            $x\geq0$の時
            $\sqrt{x}>0,x^2-2x+2>0$である。



            $a>0$とすると問の積分は次の様になる。
            \begin{equation}
             \int_{0}^{\infty} \frac{\sqrt{x}}{x^2-2x+2}\mathrm{d}x
              =
              \lim_{a\to\infty}\int_{0}^{a}
              \frac{\sqrt{x}}{x^2-2x+2}\mathrm{d}x
            \end{equation}

            





            \hrulefill


       \item
            $\displaystyle
            \int_{-\infty}^{\infty}
            \frac{x^ne^{-itx}}{\cosh{x}}\mathrm{d}x$
            \qquad
            $(n\in\mathbb{Z}_{\geq 0},\ t\in\mathbb{R})$

            \dotfill






            \hrulefill

      \end{enumerate}

 \item
      \begin{enumerate}
       \item
            $a\in\mathrm{R}$に対して
            広義積分
            $\displaystyle
            \int_{-\infty}^{\infty}e^{2iax-x^2}\mathrm{d}x$
            は絶対収束することを示せ。

            \dotfill




            \hrulefill


       \item
            $a>0$と仮定する。
            また、$R>0$とする。
            $\mathbb{C}$上の正則関数
            $f(z)=e^{2iaz-z^2}$を
            $-R,R,R+ia,-R+ia$を頂点とする長方形の境界に沿って
            積分してから$R\to\infty$とすることで
            $\displaystyle
            \int_{-\infty}^{\infty}e^{2iax-x^2}\mathrm{d}x$
            を求めよ。

            (ヒント:虚軸に並行な線積分は$R\to\infty$のとき
            0に収束することを示す)

            \dotfill






            \hrulefill

      \end{enumerate}
\end{enumerate}



\end{document}
