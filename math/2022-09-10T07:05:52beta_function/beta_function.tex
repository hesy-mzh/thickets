\documentclass[12pt,b5paper]{ltjsarticle}

%\usepackage[margin=15truemm, top=5truemm, bottom=5truemm]{geometry}
\usepackage[margin=10truemm]{geometry}

\usepackage{amsmath,amssymb}
%\pagestyle{headings}
\pagestyle{empty}

%\usepackage{listings,url}
%\renewcommand{\theenumi}{(\arabic{enumi})}

%\usepackage{graphicx}

%\usepackage{tikz}
%\usetikzlibrary {arrows.meta}
%\usepackage{wrapfig}	% required for `\wrapfigure' (yatex added)
%\usepackage{bm}	% required for `\bm' (yatex added)

% ルビを振る
%\usepackage{luatexja-ruby}	% required for `\ruby'

%% 核Ker 像Im Hom を定義
%\newcommand{\Img}{\mathop{\mathrm{Im}}\nolimits}
%\newcommand{\Ker}{\mathop{\mathrm{Ker}}\nolimits}
%\newcommand{\Hom}{\mathop{\mathrm{Hom}}\nolimits}

%\DeclareMathOperator{\Rot}{rot}
%\DeclareMathOperator{\Div}{div}
%\DeclareMathOperator{\Grad}{grad}
%\DeclareMathOperator{\arcsinh}{arcsinh}
%\DeclareMathOperator{\arccosh}{arccosh}
%\DeclareMathOperator{\arctanh}{arctanh}



\begin{document}


\textbf{ベータ関数}
\qquad $x>0,\, y>0$
\begin{equation}
 B(x,y) = \int_0^1 t^{x-1}(1-t)^{y-1}\textrm{d}t
\end{equation}

\textbf{多次元ベータ関数(ディリクレ積分)}
\qquad $a>0,\, b>0,\, c>0,\, d>0$
\begin{align}
 B(a,b,c,d)
  =& \int\!\!\!\int\!\!\!\int_{K}
  s^{a-1} t^{b-1} u^{c-1} (1-(s+t+u))^{d-1}
  \,\textrm{d}s\,\textrm{d}t\,\textrm{d}u \label{d1}\\
 K =& \{
   (s,t,u)\in\mathbb{R}^3 \mid s>0,\, t>0,\, u>0,\, s+t+u<1
 \}
\end{align}

\textbf{ガンマ関数}
\qquad $x>0$
\begin{equation}
 \Gamma(x)=\int_0^{\infty} t^{x-1}e^{-x}\textrm{d}x
  ,\qquad
  \Gamma(1)=1
  ,\quad
  \Gamma(x+1) = x\Gamma(x)
  \label{gamma}
\end{equation}

\textbf{ベータ関数とガンマ関数の関係}
\begin{align}
 B(x,y) =& \frac{\Gamma(x)\Gamma(y)}{\Gamma(x+y)}\\
 B(a,b,c,d) =& \frac{\Gamma(a)\Gamma(b)\Gamma(c)\Gamma(d)}{\Gamma(a+b+c+d)} \label{beta_gamma}
\end{align}

\hrulefill


\begin{equation}
 f_S(s)=\begin{cases} \frac{1}{3} & (0<s<3) \\ 0 & (\text{otherwise}) \end{cases}
 ,\quad
 f_T(t)=\begin{cases} \frac{1}{2} & (0<t<2) \\ 0 & (\text{otherwise}) \end{cases}
 ,\quad
 f_U(u)=\begin{cases} \frac{1}{5} & (0<u<5) \\ 0 & (\text{otherwise}) \end{cases}
\end{equation}


上の式を使って定数関数を置き換える。

\begin{align}
 & \int\!\!\!\int\!\!\!\int_{\substack{0<s<3 \\ 0<t<2 \\ 0<u<5 \\ s+t+u<1}}
  f_S(s) \cdot f_T(t) \cdot f_U(u) \textrm{d}s\,  \textrm{d}t\,  \textrm{d}u \label{prob}\\
 =& \int\!\!\!\int\!\!\!\int_{\substack{0<s<3 \\ 0<t<2 \\ 0<u<5 \\ s+t+u<1}}
  \frac{1}{3} \cdot \frac{1}{2} \cdot \frac{1}{5} \textrm{d}s\,  \textrm{d}t\,  \textrm{d}u
 = \frac{1}{3} \cdot \frac{1}{2} \cdot \frac{1}{5}
 \int\!\!\!\int\!\!\!\int_{\substack{0<s<3 \\ 0<t<2 \\ 0<u<5 \\ s+t+u<1}}
   \textrm{d}s\,  \textrm{d}t\,  \textrm{d}u\\
\end{align}

積分範囲 $0<s<3,\ 0<t<2,\ 0<u<5,\ s+t+u<1$については
$s+t+u<1$があるので、$s,t,u$は$1$より大きくなることはない。
したがって次のようになる。
\begin{equation}
 \int\!\!\!\int\!\!\!\int_{\substack{0<s<3 \\ 0<t<2 \\ 0<u<5 \\ s+t+u<1}}
  \textrm{d}s\,  \textrm{d}t\,  \textrm{d}u
 =\int\!\!\!\int\!\!\!\int_{\substack{0<s<1 \\ 0<t<1 \\ 0<u<1 \\ s+t+u<1}}
  \textrm{d}s\,  \textrm{d}t\,  \textrm{d}u
  \label{d2}
\end{equation}

範囲$0<s ,\, 0<t ,\, 0<u ,\, s+t+u<1$について
次の式が成り立つ。
\begin{equation}
 1 = s^0 t^0 u^0 (1-s-t-u)^0
\end{equation}

これを式(\ref{d2})に当てはめる。
\begin{equation}
 \int\!\!\!\int\!\!\!\int_{\substack{0<s<1 \\ 0<t<1 \\ 0<u<1 \\ s+t+u<1}}
  \textrm{d}s\,  \textrm{d}t\,  \textrm{d}u
  =\int\!\!\!\int\!\!\!\int_{\substack{0<s<1 \\ 0<t<1 \\ 0<u<1 \\ s+t+u<1}}
  s^0 t^0 u^0 (1-s-t-u)^0 \textrm{d}s\,  \textrm{d}t\,  \textrm{d}u
\end{equation}

これは式(\ref{d1})の$(a,b,c,d)=(1,1,1,1)$の時の式である。
よって、式(\ref{prob})は次のようになる。
\begin{equation}
 \int\!\!\!\int\!\!\!\int_{\substack{0<s<3 \\ 0<t<2 \\ 0<u<5 \\ s+t+u<1}}
  f_S(s) \cdot f_T(t) \cdot f_U(u) \textrm{d}s\,  \textrm{d}t\,  \textrm{d}u
 = \frac{1}{3}\cdot\frac{1}{2}\cdot\frac{1}{5} B(1,1,1,1)
 = \frac{1}{180}
\end{equation}


ただし、$B(1,1,1,1)$は
式(\ref{gamma})と式(\ref{beta_gamma})より
次のように求まる。

\begin{equation}
 B(1,1,1,1) = \frac{\Gamma(1)\Gamma(1)\Gamma(1)\Gamma(1)}{\Gamma(1+1+1+1)}
 = \frac{1 \cdot 1 \cdot 1 \cdot 1}{3 \cdot 2 \cdot 1 \cdot \Gamma(1)}
 = \frac{1}{6}
\end{equation}






\end{document}

