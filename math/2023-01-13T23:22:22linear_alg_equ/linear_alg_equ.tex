\documentclass[12pt,b5paper]{ltjsarticle}

%\usepackage[margin=15truemm, top=5truemm, bottom=5truemm]{geometry}
%\usepackage[margin=10truemm,left=15truemm]{geometry}
\usepackage[margin=10truemm]{geometry}

\usepackage{amsmath,amssymb}
%\pagestyle{headings}
\pagestyle{empty}

%\usepackage{listings,url}
%\renewcommand{\theenumi}{(\arabic{enumi})}

\usepackage{graphicx}

%\usepackage{tikz}
%\usetikzlibrary {arrows.meta}
%\usepackage{wrapfig}
\usepackage{bm}

% ルビを振る
%\usepackage{luatexja-ruby}	% required for `\ruby'

%% 核Ker 像Im Hom を定義
%\newcommand{\Img}{\mathop{\mathrm{Im}}\nolimits}
%\newcommand{\Ker}{\mathop{\mathrm{Ker}}\nolimits}
%\newcommand{\Hom}{\mathop{\mathrm{Hom}}\nolimits}

%\DeclareMathOperator{\Rot}{rot}
%\DeclareMathOperator{\Div}{div}
%\DeclareMathOperator{\Grad}{grad}
%\DeclareMathOperator{\arcsinh}{arcsinh}
%\DeclareMathOperator{\arccosh}{arccosh}
%\DeclareMathOperator{\arctanh}{arctanh}



%\usepackage{listings,url}
%
%\lstset{
%%プログラム言語(複数の言語に対応,C,C++も可)
%  language = Python,
%%  language = Lisp,
%%  language = C,
%  %背景色と透過度
%  %backgroundcolor={\color[gray]{.90}},
%  %枠外に行った時の自動改行
%  breaklines = true,
%  %自動改行後のインデント量(デフォルトでは20[pt])
%  breakindent = 10pt,
%  %標準の書体
%%  basicstyle = \ttfamily\scriptsize,
%  basicstyle = \ttfamily,
%  %コメントの書体
%%  commentstyle = {\itshape \color[cmyk]{1,0.4,1,0}},
%  %関数名等の色の設定
%  classoffset = 0,
%  %キーワード(int, ifなど)の書体
%%  keywordstyle = {\bfseries \color[cmyk]{0,1,0,0}},
%  %表示する文字の書体
%  %stringstyle = {\ttfamily \color[rgb]{0,0,1}},
%  %枠 "t"は上に線を記載, "T"は上に二重線を記載
%  %他オプション:leftline,topline,bottomline,lines,single,shadowbox
%  frame = TBrl,
%  %frameまでの間隔(行番号とプログラムの間)
%  framesep = 5pt,
%  %行番号の位置
%  numbers = left,
%  %行番号の間隔
%  stepnumber = 1,
%  %行番号の書体
%%  numberstyle = \tiny,
%  %タブの大きさ
%  tabsize = 4,
%  %キャプションの場所("tb"ならば上下両方に記載)
%  captionpos = t
%}



\begin{document}

%\hrulefill
\textbf{方程式の解空間}
\begin{equation}
 \begin{pmatrix}
  1 & 1 & 1 & 1 & 1 \\
  1 & -1 & 1 & 0 & 2 \\
  2 & 1 & 2 & -1 & 5
 \end{pmatrix}
 \bm{x}=\bm{0}
 \quad
 \text{ ただし }
 \bm{x}=
 \begin{pmatrix}
  x_1 \\ x_2 \\ x_3 \\ x_4 \\ x_5
 \end{pmatrix}
\end{equation}
上記方程式の解を調べよ。

\dotfill

左辺の行列に行に関する変形を行うと次のような行列が得られる。

\begin{equation}
 \begin{pmatrix}
  1 & 0 & 1 & 0 & 2\\
  0 & 1 & 0 & 0 & 0\\
  0 & 0 & 0 & 1 & -1
 \end{pmatrix}
\end{equation}

つまり、
最初の方程式の解は次の方程式の解でもある。
\begin{equation}
 \begin{pmatrix}
  1 & 0 & 1 & 0 & 2\\
  0 & 1 & 0 & 0 & 0\\
  0 & 0 & 0 & 1 & -1
 \end{pmatrix}
 \begin{pmatrix}
  x_1 \\ x_2 \\ x_3 \\ x_4 \\ x_5
 \end{pmatrix}
 =
 \begin{pmatrix}
  0 \\ 0 \\ 0 \\ 0 \\ 0
 \end{pmatrix}
\end{equation}


上記行列の積から次の三つの式が得られる。
\begin{equation}
 \begin{cases}
  x_1+x_3+2x_5=0\\
  x_2=0\\
  x_4-x_5=0
 \end{cases}
 \to
 \begin{cases}
  x_1=-x_3-2x_5\\
  x_2=0\\
  x_4=x_5
 \end{cases}
\end{equation}

3つ目の式から1つ目の式の$x_5$を$x_4$に置き換えると
ベクトル$\bm{x}$は次のようになる。
\begin{equation}
 \bm{x}=
  \begin{pmatrix}
     x_1 \\ x_2 \\ x_3 \\ x_4 \\ x_5
  \end{pmatrix}
  =
  \begin{pmatrix}
     -x_3-2x_4 \\ 0 \\ x_3 \\ x_4 \\ x_4
  \end{pmatrix}
  =
  \begin{pmatrix}
     -x_3 \\ 0 \\ x_3 \\ 0 \\ 0
  \end{pmatrix}
  +
  \begin{pmatrix}
     -2x_4 \\ 0 \\ 0 \\ x_4 \\ x_4
  \end{pmatrix}
  =
  x_3
  \begin{pmatrix}
     -1 \\ 0 \\ 1 \\ 0 \\ 0
  \end{pmatrix}
  +
  x_4
  \begin{pmatrix}
     -2 \\ 0 \\ 0 \\ 1 \\ 1
  \end{pmatrix}
\end{equation}

つまり、$x_3,x_4$が何か値を決めればほかの3つの成分の値も決まることになり
2次元の解空間であることがわかる。

$x_3,x_4$はどんな値をとってもよいので、
この2つを$c_1,c_2$と置きなおすと
$\bm{x}は次のようになる。$
\begin{equation}
 \bm{x}=
   c_1
  \begin{pmatrix}
     -1 \\ 0 \\ 1 \\ 0 \\ 0
  \end{pmatrix}
  +
  c_2
  \begin{pmatrix}
     -2 \\ 0 \\ 0 \\ 1 \\ 1
  \end{pmatrix}
\end{equation}

%\hrulefill

\end{document}
