\documentclass[12pt,b5paper]{ltjsarticle}

%\usepackage[margin=15truemm, top=5truemm, bottom=5truemm]{geometry}
%\usepackage[margin=10truemm,left=15truemm]{geometry}
\usepackage[margin=10truemm]{geometry}

\usepackage{amsmath,amssymb}
%\pagestyle{headings}
\pagestyle{empty}

%\usepackage{listings,url}
%\renewcommand{\theenumi}{(\arabic{enumi})}

%\usepackage{graphicx}

%\usepackage{tikz}
%\usetikzlibrary {arrows.meta}
%\usepackage{wrapfig}
%\usepackage{bm}

% ルビを振る
\usepackage{luatexja-ruby}	% required for `\ruby'

%% 核Ker 像Im Hom を定義
%\newcommand{\Img}{\mathop{\mathrm{Im}}\nolimits}
%\newcommand{\Ker}{\mathop{\mathrm{Ker}}\nolimits}
%\newcommand{\Hom}{\mathop{\mathrm{Hom}}\nolimits}

%\DeclareMathOperator{\Rot}{rot}
%\DeclareMathOperator{\Div}{div}
%\DeclareMathOperator{\Grad}{grad}
%\DeclareMathOperator{\arcsinh}{arcsinh}
%\DeclareMathOperator{\arccosh}{arccosh}
%\DeclareMathOperator{\arctanh}{arctanh}

\usepackage{url}

%\usepackage{listings}
%
%\lstset{
%%プログラム言語(複数の言語に対応,C,C++も可)
%  language = Python,
%%  language = Lisp,
%%  language = C,
%  %背景色と透過度
%  %backgroundcolor={\color[gray]{.90}},
%  %枠外に行った時の自動改行
%  breaklines = true,
%  %自動改行後のインデント量(デフォルトでは20[pt])
%  breakindent = 10pt,
%  %標準の書体
%%  basicstyle = \ttfamily\scriptsize,
%  basicstyle = \ttfamily,
%  %コメントの書体
%%  commentstyle = {\itshape \color[cmyk]{1,0.4,1,0}},
%  %関数名等の色の設定
%  classoffset = 0,
%  %キーワード(int, ifなど)の書体
%%  keywordstyle = {\bfseries \color[cmyk]{0,1,0,0}},
%  %表示する文字の書体
%  %stringstyle = {\ttfamily \color[rgb]{0,0,1}},
%  %枠 "t"は上に線を記載, "T"は上に二重線を記載
%  %他オプション:leftline,topline,bottomline,lines,single,shadowbox
%  frame = TBrl,
%  %frameまでの間隔(行番号とプログラムの間)
%  framesep = 5pt,
%  %行番号の位置
%  numbers = left,
%  %行番号の間隔
%  stepnumber = 1,
%  %行番号の書体
%%  numberstyle = \tiny,
%  %タブの大きさ
%  tabsize = 4,
%  %キャプションの場所("tb"ならば上下両方に記載)
%  captionpos = t
%}


\usepackage{bussproofs}


\begin{document}

\hrulefill

NK

自然演繹
--Natural deduction

\begin{enumerate}
  \item

      $A$を命題論理の論理式とする。
      $A\to A$の
      HK
      における証明図を書け。
      また、$(\neg A) \vee A$
      の
      NK
      における証明図を書け。

      \dotfill


       \begin{prooftree}
        %ここに証明図を書く
        \AxiomC{$(\Phi \land \Psi)$}
        \UnaryInfC{$\Phi$}
       \end{prooftree}

       \begin{prooftree}
        %ここに証明図を書く
        \AxiomC{$\Phi$}
        \AxiomC{$\Psi$}
        \BinaryInfC{$\Phi \land \Psi$}
       \end{prooftree}


       \begin{prooftree}
        %ここに証明図を書く
        \AxiomC{$\Phi$}
        \AxiomC{$\Psi$}
        \BinaryInfC{$\Phi \land \Psi$}
        \AxiomC{$X \to \Upsilon$}
        \BinaryInfC{$(\Phi \land \Psi) \land (X \to \Upsilon)$}
       \end{prooftree}

       \begin{prooftree}
        %ここに証明図を書く
        \AxiomC{$\Gamma$}
        \AxiomC{$[\Phi]_1$}
        \BinaryInfC{$\Psi$}
        \RightLabel{{\scriptsize 1}}
        \UnaryInfC{$\Phi \to \Psi$}
       \end{prooftree}



       \begin{prooftree}
        \AxiomC{A}
        \AxiomC{B}
        \AxiomC{C}
        \BinaryInfC{D}
        \RightLabel{\scriptsize(1)}
        \BinaryInfC{E}
        \DisplayProof
        \qquad
        \AxiomC{A}
       \end{prooftree}


      \hrulefill

  \item

      $\varphi(x)$を任意の$L$論理式とする。
      $\vdash_{\mathrm{NK}} \forall x \varphi(x) \leftrightarrow \neg \exists x \neg \varphi(x)$
       を示す
      NK
      の証明図をかけ。
      また、
       $\vdash_{\mathrm{NK}} \exists x \varphi(x) \leftrightarrow \neg \forall x \neg \varphi(x)$
      を示せ。


      \dotfill




      \hrulefill

\end{enumerate}

\hrulefill

\end{document}
