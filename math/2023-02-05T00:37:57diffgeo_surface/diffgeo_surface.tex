\documentclass[12pt,b5paper]{ltjsarticle}

%\usepackage[margin=15truemm, top=5truemm, bottom=5truemm]{geometry}
%\usepackage[margin=10truemm,left=15truemm]{geometry}
\usepackage[margin=10truemm]{geometry}

\usepackage{amsmath,amssymb}
%\pagestyle{headings}
\pagestyle{empty}

%\usepackage{listings,url}
%\renewcommand{\theenumi}{(\arabic{enumi})}

\usepackage{graphicx}

%\usepackage{tikz}
%\usetikzlibrary {arrows.meta}
%\usepackage{wrapfig}	% required for `\wrapfigure' (yatex added)
%\usepackage{bm}	% required for `\bm' (yatex added)

% ルビを振る
%\usepackage{luatexja-ruby}	% required for `\ruby'

%% 核Ker 像Im Hom を定義
%\newcommand{\Img}{\mathop{\mathrm{Im}}\nolimits}
%\newcommand{\Ker}{\mathop{\mathrm{Ker}}\nolimits}
%\newcommand{\Hom}{\mathop{\mathrm{Hom}}\nolimits}

%\DeclareMathOperator{\Rot}{rot}
%\DeclareMathOperator{\Div}{div}
%\DeclareMathOperator{\Grad}{grad}
%\DeclareMathOperator{\arcsinh}{arcsinh}
%\DeclareMathOperator{\arccosh}{arccosh}
%\DeclareMathOperator{\arctanh}{arctanh}



%\usepackage{listings,url}
%
%\lstset{
%%プログラム言語(複数の言語に対応,C,C++も可)
%  language = Python,
%%  language = Lisp,
%%  language = C,
%  %背景色と透過度
%  %backgroundcolor={\color[gray]{.90}},
%  %枠外に行った時の自動改行
%  breaklines = true,
%  %自動改行後のインデント量(デフォルトでは20[pt])
%  breakindent = 10pt,
%  %標準の書体
%%  basicstyle = \ttfamily\scriptsize,
%  basicstyle = \ttfamily,
%  %コメントの書体
%%  commentstyle = {\itshape \color[cmyk]{1,0.4,1,0}},
%  %関数名等の色の設定
%  classoffset = 0,
%  %キーワード(int, ifなど)の書体
%%  keywordstyle = {\bfseries \color[cmyk]{0,1,0,0}},
%  %表示する文字の書体
%  %stringstyle = {\ttfamily \color[rgb]{0,0,1}},
%  %枠 "t"は上に線を記載, "T"は上に二重線を記載
%  %他オプション:leftline,topline,bottomline,lines,single,shadowbox
%  frame = TBrl,
%  %frameまでの間隔(行番号とプログラムの間)
%  framesep = 5pt,
%  %行番号の位置
%  numbers = left,
%  %行番号の間隔
%  stepnumber = 1,
%  %行番号の書体
%%  numberstyle = \tiny,
%  %タブの大きさ
%  tabsize = 4,
%  %キャプションの場所("tb"ならば上下両方に記載)
%  captionpos = t
%}



\begin{document}

曲面の助変数表示を$p(u,v)$とし、
接ベクトルを
$p_{u}(u,v),p_{v}(u,v)$
とする。
$p_u$は$u$での偏微分、$p_v$は$v$での偏微分を表す。


\textbf{単位法線ベクトル}

二つの接ベクトルに直交する単位ベクトル$\nu$を単位法線ベクトルという。
\begin{equation}
 \nu(u,v)=
  \frac{p_{u}(u,v) \times p_{v}(u,v)}
  {\lvert p_{u}(u,v) \times p_{v}(u,v) \rvert}
\end{equation}

\textbf{正則曲面}

二つのベクトル$p_{u}(u,v),p_{v}(u,v)$
が一時独立であるとき
$p(u,v)$が表す曲面を正則曲面という。


\textbf{第一基本量}

接ベクトルの内積で表される以下の2変数関数を
第一基本量という。
\begin{align}
 E =& p_{u}(u,v) \cdot p_{u}(u,v) = \lvert p_{u}(u,v) \rvert^2
  \\
 F =& p_{u}(u,v) \cdot p_{v}(u,v)
  \\
 G =& p_{v}(u,v) \cdot p_{v}(u,v) = \lvert p_{v}(u,v) \rvert^2
\end{align}


\textbf{第一基本形式}

第一基本量の形式的な和を第一基本形式という。
\begin{equation}
 ds^2 = dp \cdot dp = E du^2 + 2F dudv + G dv^2
\end{equation}

行列で表すと次のようになる。
\begin{equation}
 ds^2 =
  \begin{pmatrix}
   du & dv
  \end{pmatrix}
  \begin{pmatrix}
   E & F \\
   F & G
  \end{pmatrix}
  \begin{pmatrix}
   du \\ dv
  \end{pmatrix}
\end{equation}


\textbf{第二基本量}

次のように内積で定義された2変数関数を第二基本量という。
\begin{align}
 L =& - p_{u}(u,v) \cdot \nu_{u}(u,v) = p_{uu}(u,v) \cdot \nu(u,v)
  \\
 M =& - p_{u}(u,v) \cdot \nu_{v}(u,v) = - p_{v}(u,v) \cdot \nu_{u}(u,v) = p_{uv}(u,v) \cdot \nu(u,v)
  \\
 N =&  - p_{v}(u,v) \cdot \nu_{v}(u,v) = p_{vv}(u,v) \cdot \nu(u,v)
\end{align}


\textbf{第二基本形式}

内積$-dp \cdot d\nu$を第二基本形式という。
次のように第二基本量を用いて表せる。
\begin{align}
 \mathrm{II} =& -dp \cdot d\nu\\
 =&
 - (p_{u}(u,v)du + p_{v}(u,v)dv) \cdot (\nu_{u}(u,v)du + \nu_{v}(u,v)dv)
 \\
 =&
 L du^2 + 2M dudv + N dv^2
\end{align}

行列で表すと次のようになる。
\begin{equation}
 \mathrm{II}=
   \begin{pmatrix}
   du & dv
  \end{pmatrix}
  \begin{pmatrix}
   L & M \\
   M & N
  \end{pmatrix}
  \begin{pmatrix}
   du \\ dv
  \end{pmatrix}
\end{equation}



\textbf{Gauss 曲率}

ガウス曲率$K$は
第一基本量と第二基本量により
次のように定義される。
\begin{equation}
 K= \frac{LN-M^2}{EG-F^2}
\end{equation}

これは行列を使い次のようにも表現できる。
\begin{equation}
 K=\det{A}
  ,\quad
 A=
  \begin{pmatrix}
   E & F \\
   F & G
  \end{pmatrix}^{-1}
  \begin{pmatrix}
   L & M \\
   M & N
  \end{pmatrix}
\end{equation}


\hrulefill

\begin{enumerate}
 \item
      第一基本量、第二基本量が
      \begin{equation}
       E=G=1,\
        F=0,\
        L=1,\
        M=0,\
        N=-1
      \end{equation}
      となる正則曲面が存在するかどうか
      理由をつけて答えよ。

      \dotfill

      $L,M,N$は単位法線ベクトルとの内積で定義される。

      正則曲面でないなら単位法線ベクトルは0となるので、
      $(L,M,N)=(0,0,0)$となる。

      $(L,M,N)\ne(0,0,0)$となるので、
      正則曲面は存在する。


      \hrulefill

 \item
      $\lambda$は正値関数とする。
      第一基本形式が
      $\lambda(du^2+dv^2)$で与えられる正則曲面の
      ガウス曲率$K$を$\varphi = \log{\lambda}$を
      用いて表せ。

      \dotfill

      ガウス曲率$K$を第一基本量を用いて表すと次のようになる。
      \begin{equation}
        \begin{split}
         K =&
         \frac{E  (E_{v}G_{v} - 2F_{u}G_{v} + G_{u}^{2})}
             {4(E G - F^{2})^{2}}\\
         & + \frac{F (E_{u}G_{v} - E_{v}G_{u} - 2E_{v}F_{v} - 2F_{u}G_{u} + 4F_{u}F_{v})}
             {4(E G - F^{2})^{2}}\\
         & + \frac{G (E_{u}G_{u} - 2E_{u}F_{v} + E_{v}^{2})}
             {4(EG - F^{2})^{2}}
         - \frac{E_{vv} - 2F_{uv} + G_{uu}}{2(EG-F^2)}
        \end{split}
      \end{equation}

      第一基本形式が
      $\lambda(du^2+dv^2)$
      であるので、
      第一基本量は
      $E=G=\lambda,F=0$
      である。
      これを上の式に当てはめる。

      \begin{align}
       K =&
         \frac{\lambda  (\lambda_{v}\lambda_{v} + \lambda_{u}^{2})}
             {4(\lambda \lambda )^{2}}
          + \frac{\lambda (\lambda_{u}\lambda_{u} + \lambda_{v}^{2})}
             {4(\lambda \lambda )^{2}}
         - \frac{\lambda_{vv} + \lambda_{uu}}{2(\lambda \lambda)}\\
       =&
          \frac{\lambda_{u}^2 + \lambda_{v}^{2}}{2\lambda^{3}}
         - \frac{\lambda_{vv} + \lambda_{uu}}{2\lambda^2}
      \end{align}


      $\varphi = \log{\lambda}$より
      $u$と$v$の偏微分を計算する。
      \begin{equation}
       \varphi_{u} = \frac{\lambda_{u}}{\lambda}
        ,\
       \varphi_{v} = \frac{\lambda_{v}}{\lambda}
        ,\
       \varphi_{uu} = \frac{\lambda_{uu}\lambda -\lambda_{u}^2}{\lambda^2}
        ,\
       \varphi_{vv} = \frac{\lambda_{vv}\lambda -\lambda_{v}^2}{\lambda^2}
      \end{equation}

      $\varphi_{uu}$と
      $\varphi_{vv}$の和からガウス曲率の式が得られる。
      \begin{align}
       \varphi_{uu} + \varphi_{vv}
         =& \frac{\lambda_{uu}+ \lambda_{vv}}{\lambda}
         - \frac{\lambda_{u}^2 + \lambda_{v}^2}{\lambda^2}\\
       -\frac{\varphi_{uu} + \varphi_{vv}}{2\lambda}
         =& \frac{\lambda_{u}^2 + \lambda_{v}^2}{2\lambda^3}
           - \frac{\lambda_{uu}+ \lambda_{vv}}{2\lambda^2}
      \end{align}

      $\varphi = \log{\lambda}$は
      $\lambda = e^{\varphi}$であるので、
      曲率は次のように表せる。
      \begin{equation}
       K = -\frac{\varphi_{uu} + \varphi_{vv}}{2e^{\varphi}}
      \end{equation}


      \hrulefill

\end{enumerate}



\hrulefill

\end{document}
