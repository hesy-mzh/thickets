\documentclass[12pt,b5paper]{ltjsarticle}

%\usepackage[margin=15truemm, top=5truemm, bottom=5truemm]{geometry}
\usepackage[margin=10truemm]{geometry}

\usepackage{amsmath,amssymb}
%\pagestyle{headings}
\pagestyle{empty}

%\usepackage{listings,url}
%\renewcommand{\theenumi}{(\arabic{enumi})}

%\usepackage{graphicx}

%\usepackage{tikz}
%\usetikzlibrary {arrows.meta}
%\usepackage{wrapfig}	% required for `\wrapfigure' (yatex added)
%\usepackage{bm}	% required for `\bm' (yatex added)

% ルビを振る
%\usepackage{luatexja-ruby}	% required for `\ruby'

%% 核Ker 像Im Hom を定義
%\newcommand{\Img}{\mathop{\mathrm{Im}}\nolimits}
%\newcommand{\Ker}{\mathop{\mathrm{Ker}}\nolimits}
%\newcommand{\Hom}{\mathop{\mathrm{Hom}}\nolimits}

%\DeclareMathOperator{\Rot}{rot}
%\DeclareMathOperator{\Div}{div}
%\DeclareMathOperator{\Grad}{grad}
%\DeclareMathOperator{\arcsinh}{arcsinh}
%\DeclareMathOperator{\arccosh}{arccosh}
%\DeclareMathOperator{\arctanh}{arctanh}



\begin{document}

\hrulefill
\textbf{定義}
\hrulefill

\textbf{曲率}

パラメータ表示された曲線$\gamma(s)$と
接ベクトル$\gamma^{\prime}(s)$、
法線ベクトル$n(s)$が次のように表される。
\begin{equation}
 \gamma(s) = (x(s), y(s))
  ,\qquad
 \gamma^{\prime}(s) = (x^{\prime}(s), y^{\prime}(s))
  ,\qquad
 n(s) = (- y^{\prime}(s), x^{\prime}(s))
\end{equation}

$\gamma^{\prime}(s)$が
単位接ベクトル($\lvert \gamma^{\prime}(s)\rvert =1$)
であれば、
$n(s)$は単位法線ベクトル($\lvert n(s) \rvert =1$)である。

この時、次を満たす$\kappa(s)$を曲率関数という。
\begin{equation}
 \gamma^{\prime\prime}(s)=\kappa(s)n(s)
\end{equation}

\textbf{曲率の求め方}

\begin{enumerate}
 \item 単位接ベクトルを求める
 \item 単位接ベクトルを微分する
 \item このベクトルが単位法線ベクトルのスカラー倍になっている
 \item このスカラーが曲率である
\end{enumerate}


パラメータ表示された曲線$\gamma(t)=(\gamma_{x}(t),\gamma_{y}(t))$
に対して微分を求める。
\begin{equation}
 \gamma^{\prime}(t)=(\gamma_{x}^{\prime} (t),\gamma_{y}^{\prime} (t))
  ,\quad
 \gamma^{\prime\prime}(t)=(\gamma_{x}^{\prime\prime} (t),\gamma_{y}^{\prime\prime} (t))
\end{equation}

これを列に並べた行列式
\begin{equation}
 \det(\gamma^{\prime}(t), \gamma^{\prime\prime}(t))
  =
  \begin{vmatrix}
   \gamma_{x}^{\prime} (t) & \gamma_{x}^{\prime\prime} (t)\\
   \gamma_{y}^{\prime} (t) & \gamma_{y}^{\prime\prime} (t))
  \end{vmatrix}
\end{equation}
を用いて曲率$\kappa(t)$は次のように求めることが出来る。
\begin{equation}
 \kappa(t) = \frac{\det(\gamma^{\prime}(t), \gamma^{\prime\prime}(t))}{\lvert \gamma^{\prime}(t) \rvert^3}
\end{equation}





\hrulefill
\textbf{問題}
\hrulefill

\begin{enumerate}
 \item
      次の平面曲線の曲率関数$\kappa (t)$を求めよ。
      \begin{enumerate}
       \item
            $f:\mathbb{R} \to \mathbb{R}^2, f(t)=(a\cos t, b\sin t), \quad a,b>0$

       \item
            $f:\mathbb{R} \to \mathbb{R}^2, f(t)=(t, \cosh t)$

       \item
            $f:(0,\infty) \to \mathbb{R}^2, f(t)=(\frac{\cos t}{t}, \frac{\sin t}{t})$
      \end{enumerate}

\dotfill

\begin{enumerate}
 \item
      $f(t)=(a\cos t, b\sin t)$より、
      $f^{\prime}(t), f^{\prime\prime}(t), \lvert f^{\prime}(t) \rvert$を求める。
      \begin{gather}
       f^{\prime}(t) = (-a\sin t, b\cos t),\quad
       f^{\prime\prime}(t) = (-a\cos t, -b\sin t)\\
       \lvert f^{\prime}(t) \rvert = (a^2\sin^2 t+b^2\cos^2 t)^{\frac{1}{2}}
      \end{gather}

      これを用いて$\det(f^{\prime},f^{\prime\prime})$を求める。
      \begin{equation}
       \det(f^{\prime},f^{\prime\prime})=
        \begin{vmatrix}
         -a\sin t & -a\cos t\\
         b\cos t & -b\sin t
        \end{vmatrix}
        = ab\sin^2 t + ab\cos^2 t = ab
      \end{equation}

      よって、曲率$\kappa(t)$は次のようになる。
      \begin{equation}
       \kappa(t) = \frac{\det(f^{\prime},f^{\prime\prime})}{\lvert f^{\prime} \rvert^3}
        = \frac{ab}{(a^2\sin^2 t+b^2\cos^2 t)^{\frac{3}{2}}}
      \end{equation}


 \item
      $f(t)=(t, \cosh t)$より
      $f^{\prime}(t), f^{\prime\prime}(t), \lvert f^{\prime}(t) \rvert$を求める。
      \begin{gather}
       f^{\prime}(t) = (1,\sinh t),\
       f^{\prime\prime}(t) = (0,\cosh t),\
       \lvert f^{\prime}(t) \rvert = \frac{1}{2}(e^{2t}+e^{-2t}+2)^{\frac{1}{2}}
      \end{gather}

      これを用いて$\det(f^{\prime},f^{\prime\prime})$を求める。
      \begin{equation}
       \det(f^{\prime},f^{\prime\prime})=
        \begin{vmatrix}
         1 & 0\\
         \sinh t & \cosh t
        \end{vmatrix}
        = \cosh t
      \end{equation}

      よって、曲率$\kappa(t)$は次のようになる。
      \begin{equation}
       \kappa(t) = \frac{\det(f^{\prime},f^{\prime\prime})}{\lvert f^{\prime} \rvert^3}
        = \frac{8\cosh t}{(e^{2t}+e^{-2t}+2)^{\frac{3}{2}}}
      \end{equation}



 \item
      $f(t)=(\frac{\cos t}{t}, \frac{\sin t}{t})$より
      $f^{\prime}(t), f^{\prime\prime}(t), \lvert f^{\prime}(t) \rvert$を求める。

      \begin{gather}
       f^{\prime}(t) = \left( -\frac{\sin t}{t}-\frac{\cos t}{t^2},\frac{\cos t}{t}-\frac{\sin t}{t^2} \right)\\
       f^{\prime\prime}(t) = \left(
       -\frac{\cos t}{t}+\frac{2\sin t}{t^2}+\frac{2\cos t}{t^3},
       -\frac{\sin t}{t}-\frac{2\cos t}{t^2}-\frac{2\sin t}{t^3}
       \right)\\
       \lvert f^{\prime}(t) \rvert = \left( \frac{1}{t^2}+\frac{1}{t^4} \right)^{\frac{1}{2}}
       =\frac{1}{t^2}(t^2+1)^{\frac{1}{2}}
      \end{gather}

      これを用いて$\det(f^{\prime},f^{\prime\prime})$を求める。
      \begin{align}
       \det(f^{\prime},f^{\prime\prime})=&
        \begin{vmatrix}
         -\frac{\sin t}{t}-\frac{\cos t}{t^2} & -\frac{\cos t}{t}+\frac{2\sin t}{t^2}+\frac{2\cos t}{t^3}\\
         \frac{\cos t}{t}-\frac{\sin t}{t^2} & -\frac{\sin t}{t}-\frac{2\cos t}{t^2}-\frac{2\sin t}{t^3}\\
        \end{vmatrix}\\
        =& \frac{1}{t^5}(t^3+4t\sin^2 t + 4\sin t\cos t)
      \end{align}

      よって、曲率$\kappa(t)$は次のようになる。
      \begin{equation}
       \kappa(t) = \frac{\det(f^{\prime},f^{\prime\prime})}{\lvert f^{\prime} \rvert^3}
        = \frac{ t(t^3+4t\sin^2 t + 4\sin t\cos t) }{( t^2+1 )^{\frac{3}{2}}}
      \end{equation}





\end{enumerate}

\dotfill

\begin{enumerate}
 \item
      $f^{\prime}(t)=(-a\sin t, b\cos t)$であるので、
      $\lvert f^{\prime}(t) \rvert = \sqrt{a^2\sin^2 t+b^2\cos^2 t}$
      である。
      単位接ベクトルと単位法線ベクトルは
      \begin{align}
       \frac{1}{\lvert f^{\prime}(t) \rvert}f^{\prime}(t)
        =& \frac{1}{\sqrt{a^2\sin^2 t+b^2\cos^2 t}} (-a\sin t, b\cos t)\\
       n(t) =& \frac{1}{\sqrt{a^2\sin^2 t+b^2\cos^2 t}} (-b\cos t , -a\sin t)
      \end{align}
      である。
      \begin{align}
       & \left( \frac{1}{\lvert f^{\prime}(t) \rvert}f^{\prime}(t) \right)^{\prime}\\
        =& \frac{1}{\sqrt{a^2\sin^2 t+b^2\cos^2 t}}f^{\prime\prime}(t)
       - \frac{\sin t \cos t (a^2-b^2)}{(a^2\sin^2 t+b^2\cos^2 t)^{\frac{3}{2}}}f^{\prime}(t)\\
        =& \frac{(a^2\sin^2 t+b^2\cos^2 t)f^{\prime\prime}(t)
       - \sin t \cos t (a^2-b^2)f^{\prime}(t)}{(a^2\sin^2 t+b^2\cos^2 t)^{\frac{3}{2}}}\label{quo01}
      \end{align}
      
      $f^{\prime\prime}(t)=(-a\cos t, -b\sin t)$であるので、
      式(\ref{quo01})の分子の各成分を計算すると
      \begin{gather}
       (a^2\sin^2 t+b^2\cos^2 t)(-a\cos t)
       - \sin t \cos t (a^2-b^2)(-a\sin t)
       = -ab^2\cos t\\
       (a^2\sin^2 t+b^2\cos^2 t)(-b\sin t)
       - \sin t \cos t (a^2-b^2)(b\cos t)
       = -a^2b\sin t
      \end{gather}

      \begin{align}
       \left( \frac{1}{\lvert f^{\prime}(t) \rvert}f^{\prime}(t) \right)^{\prime}
        =& \frac{ab}{(a^2\sin^2 t+b^2\cos^2 t)^{\frac{3}{2}}}(-b\cos t,-a\sin t)
%      (\text{不明?一致しない?})  =& \frac{ab}{(a^2\sin^2 t+b^2\cos^2 t)^{\frac{3}{2}}}n(t)
      \end{align}

%      曲率関数$\kappa(t)$は次のようになる。
%      \begin{equation}
%       \kappa(t) = \frac{ab}{(a^2\sin^2 t+b^2\cos^2 t)^{\frac{3}{2}}}
%      \end{equation}


 \item
      $f(t)=(t, \cosh t)$より
      $f^{\prime}(t) = (1,\sinh t),\ f^{\prime\prime}(t) = (0,\cosh t)$
      である。
      また、
      $\lvert f^{\prime}(t) \rvert = \frac{1}{2}\sqrt{e^{2t}+e^{-2t}+2}$
      であるため単位接ベクトルは次のようになる。
      \begin{equation}
       \frac{1}{\lvert f^{\prime}(t) \rvert}f^{\prime}(t)=
         \frac{2}{\sqrt{e^{2t}+e^{-2t}+2}}(1,\sinh t)
      \end{equation}

      \begin{align}
        \left( \frac{1}{\lvert f^{\prime}(t) \rvert}f^{\prime}(t) \right)^{\prime}
       =& \frac{2}{\sqrt{e^{2t}+e^{-2t}+2}} f^{\prime\prime}(t)
       - \frac{2e^{2t}-2e^{-2t}}{(e^{2t}+e^{-2t}+2)^{\frac{3}{2}}} f^{\prime}(t)\\
       =& \frac{2(e^{2t}+e^{-2t}+2)f^{\prime\prime}(t)-(2e^{2t}-2e^{-2t})f^{\prime}(t)}{(e^{2t}+e^{-2t}+2)^{\frac{3}{2}}}
      \end{align}

      \begin{gather}
       2(e^{2t}+e^{-2t}+2)(0)-(2e^{2t}-2e^{-2t})(1)\\
       = -2(e^{t}+e^{-t})(e^{t}-e^{-t})=-8\sinh t \cosh t\\
       2(e^{2t}+e^{-2t}+2)(\cosh t)-(2e^{2t}-2e^{-2t})(\sinh t)\\
       = (e^{2t}+e^{-2t}+2)(e^{t}+e^{-t})-(e^{2t}-e^{-2t})(e^{t}-e^{-t}) = 4(e^{t}+e^{-t})
      \end{gather}

      \begin{align}
       \left( \frac{1}{\lvert f^{\prime}(t) \rvert}f^{\prime}(t) \right)^{\prime}=&
        \frac{4(e^{t}+e^{-t})}{(e^{2t}+e^{-2t}+2)^{\frac{3}{2}}}(-\sinh t,1)\\
       =&
        \frac{8\cosh t}{(e^{2t}+e^{-2t}+2)^{\frac{3}{2}}}(-\sinh t,1)
      \end{align}



 \item
      $f(t)=(\frac{\cos t}{t}, \frac{\sin t}{t})$より
      $f^{\prime}(t), f^{\prime\prime}(t)$を求める。
      \begin{align}
       f^{\prime}(t)=&
       \left(
         -\frac{\sin t}{t}-\frac{\cos t}{t^2},
         \frac{\cos t}{t}-\frac{\sin t}{t^2}
       \right)\\
       f^{\prime\prime}(t)=&
       \left(
       -\frac{\cos t}{t}+\frac{2\sin t}{t^2}+\frac{2\cos t}{t^3},
       -\frac{\sin t}{t}-\frac{2\cos t}{t^2}-\frac{2\sin t}{t^3}
       \right)
      \end{align}
\end{enumerate}




\hrulefill


 \item
      正の値を取る$C^2$-級関数$f(x)$について、
      極座標表示された曲線
      \begin{equation}
       \theta \mapsto (x(\theta), y(\theta))
        = (f(\theta)\cos\theta, f(\theta)\sin\theta )
      \end{equation}
      の点$(x(\theta), y(\theta))$における曲線$\kappa(\theta)$は
      \begin{equation}
       \kappa =
        \frac{f^2+2(f^{\prime})^2-ff^{\prime\prime}}
        {\{f^2+(f^{\prime})^2\}^{\frac{3}{2}}}
      \end{equation}
      で与えられることを示せ。

\dotfill

      $x(\theta),y(\theta)$の2階の導関数を求める。
      \begin{align}
       x^{\prime}(\theta) =& f^{\prime}(\theta)\cos\theta-f(\theta)\sin\theta\\
       y^{\prime}(\theta) =& f^{\prime}(\theta)\sin\theta+f(\theta)\cos\theta\\
       x^{\prime\prime}(\theta) =& f^{\prime\prime}(\theta)\cos\theta - 2f^{\prime}(\theta)\sin\theta - f(\theta)\cos\theta\\
       y^{\prime\prime}(\theta) =& f^{\prime\prime}(\theta)\sin\theta + 2f^{\prime}(\theta)\cos\theta - f(\theta)\sin\theta
      \end{align}

      次の行列を計算する。
      \begin{equation}
       \begin{vmatrix}
        x^{\prime}(\theta) & x^{\prime\prime}(\theta)\\
        y^{\prime}(\theta) & y^{\prime\prime}(\theta)
       \end{vmatrix}
       =x^{\prime}(\theta)y^{\prime\prime}(\theta) - x^{\prime\prime}(\theta)y^{\prime}(\theta)
      \end{equation}

      \begin{align}
       x^{\prime}(\theta)y^{\prime\prime}(\theta)
        =& (f^{\prime}(\theta)\cos\theta-f(\theta)\sin\theta)(f^{\prime\prime}(\theta)\sin\theta + 2f^{\prime}(\theta)\cos\theta - f(\theta)\sin\theta)\\
      =& f^{\prime}(\theta)\cos\theta(f^{\prime\prime}(\theta)\sin\theta + 2f^{\prime}(\theta)\cos\theta - f(\theta)\sin\theta)\\
      & -f(\theta)\sin\theta(f^{\prime\prime}(\theta)\sin\theta + 2f^{\prime}(\theta)\cos\theta - f(\theta)\sin\theta)\\
       =&
       f^{\prime}(\theta) f^{\prime\prime}(\theta)\sin\theta\cos\theta + 2(f^{\prime}(\theta))^2\cos^2\theta - f^{\prime}(\theta) f(\theta)\sin\theta\cos\theta\\
      & -f(\theta) f^{\prime\prime}(\theta)\sin^2\theta - 2f(\theta)f^{\prime}(\theta)\sin\theta \cos\theta + (f(\theta))^2\sin^2\theta\\
%
       x^{\prime\prime}(\theta)y^{\prime}(\theta)
        =& (f^{\prime}(\theta)\sin\theta+f(\theta)\cos\theta)(f^{\prime\prime}(\theta)\cos\theta - 2f^{\prime}(\theta)\sin\theta - f(\theta)\cos\theta)\\
       =& f^{\prime}(\theta)\sin\theta(f^{\prime\prime}(\theta)\cos\theta - 2f^{\prime}(\theta)\sin\theta - f(\theta)\cos\theta)\\
       & + f(\theta)\cos\theta(f^{\prime\prime}(\theta)\cos\theta - 2f^{\prime}(\theta)\sin\theta - f(\theta)\cos\theta)\\
       =& f^{\prime}(\theta) f^{\prime\prime}(\theta)\sin\theta\cos\theta - 2(f^{\prime}(\theta))^2\sin^2\theta - f(\theta)f^{\prime}(\theta)\sin\theta\cos\theta\\
       & + f(\theta)f^{\prime\prime}(\theta)\cos^2\theta - 2f^{\prime}(\theta)f(\theta)\sin\theta\cos\theta - (f(\theta))^2\cos^2\theta
      \end{align}

      \begin{equation}
       \begin{vmatrix}
        x^{\prime}(\theta) & x^{\prime\prime}(\theta)\\
        y^{\prime}(\theta) & y^{\prime\prime}(\theta)
       \end{vmatrix}
       = 2(f^{\prime}(\theta))^2 - f(\theta)f^{\prime\prime}(\theta) + (f(\theta))^2
      \end{equation}

      ベクトルの大きさ$\lvert (x^{\prime}(\theta),y^{\prime}(\theta)) \rvert$を求める。
      \begin{align}
       (x^{\prime}(\theta))^2+(y^{\prime}(\theta))^2
        =& (f^{\prime}(\theta)\cos\theta-f(\theta)\sin\theta)^2+(f^{\prime}(\theta)\sin\theta+f(\theta)\cos\theta)^2\\
        =& (f^{\prime}(\theta))^2 + (f(\theta))^2\\
       \lvert (x^{\prime}(\theta),y^{\prime}(\theta)) \rvert
       =& ((f^{\prime}(\theta))^2 + (f(\theta))^2)^{\frac{1}{2}}
      \end{align}

      これらを合わせて曲率関数$\kappa(\theta)$は次のように求まる。
      \begin{equation}
       \kappa(\theta) = \frac{1}{\lvert (x^{\prime}(\theta),y^{\prime}(\theta)) \rvert^3}
        \begin{vmatrix}
        x^{\prime}(\theta) & x^{\prime\prime}(\theta)\\
        y^{\prime}(\theta) & y^{\prime\prime}(\theta)
        \end{vmatrix}
        =\frac{2(f^{\prime}(\theta))^2 - f(\theta)f^{\prime\prime}(\theta) + (f(\theta))^2}{((f^{\prime}(\theta))^2 + (f(\theta))^2)^{\frac{3}{2}}}
      \end{equation}

\hrulefill


 \item
      2変数の$C^2$-級関数$f(x,y)$は次を満たすとする。
      \begin{itemize}
       \item
            $Z_{f} = \{ (x,y)\in\mathbb{R}^2 \mid f(x,y)=0 \} \ne \emptyset$
       \item
            ${}^{\forall}(a,b)\in Z_{f}$に対して
            $(f_{x}(a,b),f_{y}(a,b))\ne (0,0)$
      \end{itemize}

      この時、陰関数定理より、
      $Z_f$は$Z_f$の各点の近傍で曲線片として表せる。
      よって、$(a,b)\in Z_f$に対して、
      $(a,b)$における$Z_f$の曲率$\kappa(a,b)$を定義することが出来る。
      (ただし、曲線の向き付けによって曲率の符号が逆になる。)

      曲率$\kappa(a,b)$は次の式で表されることを示せ。
      \begin{equation}
       \kappa(a,b)=
        \pm \frac{f_y^2f_{xx}-2f_xf_yf_{xy}+f_x^2f_{yy}}{(f_x^2+f_y^2)^{\frac{3}{2}}}(a,b)
      \end{equation}

\dotfill

%\textbf{陰関数定理}

%\dotfill

%\begin{align}
% Z_{f_x} =& \{ (a,b) \in Z_{f} \mid f_{x}(a,b) \ne 0 \}\\
% Z_{f_y} =& \{ (a,b) \in Z_{f} \mid f_{y}(a,b) \ne 0 \}\\
% Z_{f} =& Z_{f_x} \cup Z_{f_y}
%\end{align}



      $(a,b)\in Z_{f}$における曲線$f(x,y)=0$の
      法ベクトルは$(f_{x}(a,b),f_{y}(a,b))$であるので、
      単位法ベクトルは次の式で表される。
      \begin{equation}
       n(a,b)
%        = \left(
%           \frac{f_{x}}{(f_{x}^2+f_{y}^2)^\frac{1}{2}}(a,b),
%           \frac{f_{y}}{(f_{x}^2+f_{y}^2)^\frac{1}{2}}(a,b)
%        \right)
        = \frac{1}{(f_{x}^2+f_{y}^2)^\frac{1}{2}}(f_{x},f_{y})(a,b)
        \label{unitn}
      \end{equation}

      つまり、
      $n=\frac{1}{(f_{x}^2+f_{y}^2)^\frac{1}{2}}(f_{x},f_{y})$
      に対して、単位接ベクトルは次のようなベクトルである。
      \begin{equation}
       \frac{1}{(f_{x}^2+f_{y}^2)^\frac{1}{2}}(-f_{y},f_{x})
        \quad \text{or} \quad
       \frac{1}{(f_{x}^2+f_{y}^2)^\frac{1}{2}}(f_{y},-f_{x})
      \end{equation}

      そこで、
      単位接ベクトルの一つを次のように$T$とおく。
      \begin{equation}
       T=\frac{1}{(f_{x}^2+f_{y}^2)^\frac{1}{2}}(-f_{y},f_{x})
      \end{equation}
      

      弧長パラメータ$s$を用いて
      $x=\gamma_{x}(s),y=\gamma_{y}(s)$
      とすると
      曲線$f(x,y)=0$は
      $f(\gamma_{x}(s),\gamma_{y}(s))=0$
      と表せる。

      接ベクトルは
      $\gamma^{\prime}(s)=(\gamma_{x}^{\prime}(s),\gamma_{y}^{\prime}(s))$
      であるので、
      正負の違いをまとめて書けば次のようになる。
      \begin{equation}
       (\gamma_{x}^{\prime},\gamma_{y}^{\prime})
        =
        \pm\frac{1}{(f_{x}^2+f_{y}^2)^\frac{1}{2}}(-f_{y},f_{x})
      \end{equation}

      単位接ベクトル$T$に合わせ、
      次のように置いて考える。
      \begin{equation}
       \gamma_{x}^{\prime}
       =\frac{-f_{y}}{(f_{x}^2+f_{y}^2)^\frac{1}{2}}
       ,\qquad
       \gamma_{y}^{\prime}
       =\frac{f_{x}}{(f_{x}^2+f_{y}^2)^\frac{1}{2}}
       \label{diffxandy}
      \end{equation}

      曲率は法ベクトル$\gamma^{\prime\prime}(s)$が
      単位ベクトルの何倍であるかを求めることで得られる。
      つまり、単位法ベクトル[式(\ref{unitn})]との内積が曲率となる。
      \begin{equation}
       \gamma^{\prime\prime}
       = \frac{\mathrm{d}\gamma^{\prime}}{\mathrm{d}s}
       = \frac{\mathrm{d}}{\mathrm{d}s}
       \left(\gamma_{x}^{\prime},\gamma_{y}^{\prime}\right)
       ,\qquad
       \kappa = \langle \gamma^{\prime\prime}, n \rangle
      \end{equation}

      ベクトル$\gamma^{\prime\prime}$の各成分ごとに
      合成関数の微分の式を利用すると次の式が得られる。
      \begin{equation}
       \frac{\mathrm{d}}{\mathrm{d}s}\gamma_{x}^{\prime}
       = \frac{\partial \gamma_{x}^{\prime}}{\partial x}
       \frac{\mathrm{d}x}{\mathrm{d}s}
       + \frac{\partial \gamma_{x}^{\prime}}{\partial y}
       \frac{\mathrm{d}y}{\mathrm{d}s}
%       =& \frac{\mathrm{d}}{\mathrm{d}s}
%       \frac{-f_{y}}{(f_{x}^2+f_{y}^2)^\frac{1}{2}}\\
       ,\qquad
       \frac{\mathrm{d}}{\mathrm{d}s}\gamma_{y}^{\prime}
       = \frac{\partial \gamma_{y}^{\prime}}{\partial x}
       \frac{\mathrm{d}x}{\mathrm{d}s}
       + \frac{\partial \gamma_{y}^{\prime}}{\partial y}
       \frac{\mathrm{d}y}{\mathrm{d}s}
      \end{equation}

      これにより$\gamma^{\prime\prime}$は
      $
      \frac{\partial \gamma_{x}^{\prime}}{\partial x},
      \frac{\partial \gamma_{x}^{\prime}}{\partial y},
      \frac{\partial \gamma_{y}^{\prime}}{\partial x},
      \frac{\partial \gamma_{y}^{\prime}}{\partial y}
      $
      と
      $\frac{\mathrm{d}x}{\mathrm{d}s},\frac{\mathrm{d}y}{\mathrm{d}s}$
      を計算することで得られる。

      $x=\gamma_{x}(s),y=\gamma_{y}(s)$であるので、
      式(\ref{diffxandy})より
      $\frac{\mathrm{d}x}{\mathrm{d}s},\frac{\mathrm{d}y}{\mathrm{d}s}$
      は次のようになる。
      \begin{align}
       \frac{\mathrm{d}x}{\mathrm{d}s}
       = \frac{\mathrm{d}\gamma_{x}}{\mathrm{d}s}
       = \gamma_{x}^{\prime}
       = \frac{-f_{y}}{(f_{x}^2+f_{y}^2)^\frac{1}{2}}\\
       \frac{\mathrm{d}y}{\mathrm{d}s}
       = \frac{\mathrm{d}\gamma_{y}}{\mathrm{d}s}
       = \gamma_{y}^{\prime}
       = \frac{f_{x}}{(f_{x}^2+f_{y}^2)^\frac{1}{2}}
      \end{align}

      
      $
      \frac{\partial \gamma_{x}^{\prime}}{\partial x},
      \frac{\partial \gamma_{x}^{\prime}}{\partial y},
      \frac{\partial \gamma_{y}^{\prime}}{\partial x},
      \frac{\partial \gamma_{y}^{\prime}}{\partial y}
      $
      もそれぞれ求める。
      \begin{align}
       \frac{\partial \gamma_{x}^{\prime}}{\partial x}
       =& \frac{\partial}{\partial x}
       \frac{-f_{y}}{(f_{x}^2+f_{y}^2)^\frac{1}{2}}
       = \frac{-f_{xy}}{(f_{x}^2+f_{y}^2)^\frac{1}{2}}
       + \frac{f_{y}(f_{x}f_{xx}+f_{y}f_{xy})}{(f_{x}^2+f_{y}^2)^\frac{3}{2}}\\
       =& \frac{-f_{x}^2f_{xy} + f_{y}f_{x}f_{xx}}{(f_{x}^2+f_{y}^2)^\frac{3}{2}}\\
%
       \frac{\partial \gamma_{x}^{\prime}}{\partial y}
       =& \frac{\partial}{\partial y}
       \frac{-f_{y}}{(f_{x}^2+f_{y}^2)^\frac{1}{2}}
       = \frac{-f_{yy}}{(f_{x}^2+f_{y}^2)^\frac{1}{2}}
       + \frac{f_{y}(f_{x}f_{xy}+f_{y}f_{yy})}{(f_{x}^2+f_{y}^2)^\frac{3}{2}}\\
       =& \frac{-f_{x}^2f_{yy} + f_{y}f_{x}f_{xy}}{(f_{x}^2+f_{y}^2)^\frac{3}{2}}\\
%
      \frac{\partial \gamma_{y}^{\prime}}{\partial x}
       =& \frac{\partial}{\partial x}
       \frac{f_{x}}{(f_{x}^2+f_{y}^2)^\frac{1}{2}}
       = \frac{f_{xx}}{(f_{x}^2+f_{y}^2)^\frac{1}{2}}
       + \frac{-f_{x}(f_{x}f_{xx}+f_{y}f_{xy})}{(f_{x}^2+f_{y}^2)^\frac{3}{2}}\\
       =& \frac{f_{y}^2 f_{xx}-f_{x}f_{y}f_{xy}}{(f_{x}^2+f_{y}^2)^\frac{3}{2}}\\
%
      \frac{\partial \gamma_{y}^{\prime}}{\partial y}
       =& \frac{\partial}{\partial y}
       \frac{f_{x}}{(f_{x}^2+f_{y}^2)^\frac{1}{2}}
       = \frac{f_{xy}}{(f_{x}^2+f_{y}^2)^\frac{1}{2}}
       + \frac{-f_{x}(f_{x}f_{xy}+f_{y}f_{yy})}{(f_{x}^2+f_{y}^2)^\frac{3}{2}}\\
       =& \frac{f_{y}^2 f_{xy}-f_{x}f_{y}f_{yy}}{(f_{x}^2+f_{y}^2)^\frac{3}{2}}
      \end{align}


      $\gamma^{\prime\prime}$を複数のベクトルに分ける。
      \begin{align}
       \gamma^{\prime\prime}
       =& \left(
       \frac{\partial \gamma_{x}^{\prime}}{\partial x}
       \frac{\mathrm{d}x}{\mathrm{d}s}
       + \frac{\partial \gamma_{x}^{\prime}}{\partial y}
       \frac{\mathrm{d}y}{\mathrm{d}s}
       ,
       \frac{\partial \gamma_{y}^{\prime}}{\partial x}
       \frac{\mathrm{d}x}{\mathrm{d}s}
       + \frac{\partial \gamma_{y}^{\prime}}{\partial y}
       \frac{\mathrm{d}y}{\mathrm{d}s}
       \right)\\
       =&
       \left(
       \frac{\partial \gamma_{x}^{\prime}}{\partial x}
       \frac{\mathrm{d}x}{\mathrm{d}s}
       ,
       \frac{\partial \gamma_{y}^{\prime}}{\partial x}
       \frac{\mathrm{d}x}{\mathrm{d}s}
       \right)
       +
       \left(
       \frac{\partial \gamma_{x}^{\prime}}{\partial y}
       \frac{\mathrm{d}y}{\mathrm{d}s}
       ,
       \frac{\partial \gamma_{y}^{\prime}}{\partial y}
       \frac{\mathrm{d}y}{\mathrm{d}s}
       \right)\\
       =& \frac{\partial (\gamma_{x}^{\prime},\gamma_{y}^{\prime})}{\partial x}
       \frac{\mathrm{d}x}{\mathrm{d}s}
       +
       \frac{\partial (\gamma_{x}^{\prime},\gamma_{y}^{\prime})}{\partial y}
       \frac{\mathrm{d}y}{\mathrm{d}s}\\
      \end{align}

      この為、曲率を次のような2つの内積に分けて求める。
      \begin{equation}
       \kappa =
       \langle \gamma^{\prime\prime},n \rangle
       = \frac{\mathrm{d}x}{\mathrm{d}s}
       \left\langle
       \frac{\partial (\gamma_{x}^{\prime},\gamma_{y}^{\prime})}{\partial x},n \right\rangle
       + \frac{\mathrm{d}y}{\mathrm{d}s}
       \left\langle \frac{\partial (\gamma_{x}^{\prime},\gamma_{y}^{\prime})}{\partial y},n \right\rangle
      \end{equation}


%      $n=\frac{1}{(f_{x}^2+f_{y}^2)^\frac{1}{2}}(f_{x},f_{y})$

      2つの内積は次のような式となる。
      \begin{align}
       &
       \left\langle
       \frac{\partial (\gamma_{x}^{\prime},\gamma_{y}^{\prime})}{\partial x},n \right\rangle\\
       =& \frac{-f_{x}^2f_{xy} + f_{y}f_{x}f_{xx}}{(f_{x}^2+f_{y}^2)^\frac{3}{2}}
       \frac{f_{x}}{(f_{x}^2+f_{y}^2)^\frac{1}{2}}
       +\frac{f_{y}^2 f_{xx}-f_{x}f_{y}f_{xy}}{(f_{x}^2+f_{y}^2)^\frac{3}{2}}
       \frac{f_{y}}{(f_{x}^2+f_{y}^2)^\frac{1}{2}}\\
       =& \frac{f_{y} f_{xx}-f_{x}f_{xy}}{f_{x}^2+f_{y}^2}\\
%
       &
       \left\langle
       \frac{\partial (\gamma_{x}^{\prime},\gamma_{y}^{\prime})}{\partial y},n \right\rangle\\
       =& \frac{-f_{x}^2f_{yy} + f_{y}f_{x}f_{xy}}{(f_{x}^2+f_{y}^2)^\frac{3}{2}}
       \frac{f_{x}}{(f_{x}^2+f_{y}^2)^\frac{1}{2}}
       +\frac{f_{y}^2 f_{xy}-f_{x}f_{y}f_{yy}}{(f_{x}^2+f_{y}^2)^\frac{3}{2}}
       \frac{f_{y}}{(f_{x}^2+f_{y}^2)^\frac{1}{2}}\\
       =& \frac{f_{y} f_{xy}-f_{x}f_{yy}}{f_{x}^2+f_{y}^2}
      \end{align}

      以上により曲率$\kappa$は次のようになる。
      \begin{align}
       \kappa =& \langle \gamma^{\prime\prime}, n\rangle\\
       =&
       \frac{f_{y} f_{xx}-f_{x}f_{xy}}{f_{x}^2+f_{y}^2}
       \frac{-f_{y}}{(f_{x}^2+f_{y}^2)^\frac{1}{2}}
       +
       \frac{f_{y} f_{xy}-f_{x}f_{yy}}{f_{x}^2+f_{y}^2}
       \frac{f_{x}}{(f_{x}^2+f_{y}^2)^\frac{1}{2}}\\
       =&
%       \kappa(a,b)=
%       \pm
       \frac{-f_y^2f_{xx}+2f_xf_yf_{xy}-f_x^2f_{yy}}{(f_x^2+f_y^2)^{\frac{3}{2}}}
       %(a,b)
      \end{align}

      式(\ref{diffxandy})にて接ベクトルを一方に固定したが、
      逆向きのベクトルでも同様に計算できる。
      この場合、$\gamma^{\prime\prime}$が$-1$倍されるので、
      曲率$\kappa$の符号も逆になる。

      よって、曲率は次のような式で求められる。
      \begin{equation}
%       \kappa(a,b)=
       \kappa=
       \pm
       \frac{f_y^2f_{xx}-2f_xf_yf_{xy}+f_x^2f_{yy}}{(f_x^2+f_y^2)^{\frac{3}{2}}}
       %(a,b)
      \end{equation}

      \dotfill

      \begin{equation}
       \frac{f_y^2f_{xx}-2f_xf_yf_{xy}+f_x^2f_{yy}}{(f_x^2+f_y^2)^{\frac{3}{2}}}
        =
        \frac{1}{(f_x^2+f_y^2)^{\frac{3}{2}}}
        \begin{vmatrix}
         f_{xx} & f_{xy} & f_{x}\\
         f_{xy} & f_{yy} & f_{y}\\
         f_{x} & f_{y} & 0
        \end{vmatrix}
      \end{equation}
\end{enumerate}

\hrulefill


\end{document}
