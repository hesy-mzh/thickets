\documentclass[12pt,b5paper]{ltjsarticle}

%\usepackage[margin=15truemm, top=5truemm, bottom=5truemm]{geometry}
%\usepackage[margin=10truemm,left=15truemm]{geometry}
\usepackage[margin=10truemm]{geometry}

\usepackage{amsmath,amssymb}
%\pagestyle{headings}
\pagestyle{empty}

%\usepackage{listings,url}
\renewcommand{\theenumi}{(\arabic{enumi})}

%\usepackage{graphicx}

%\usepackage{tikz}
%\usetikzlibrary {arrows.meta}
%\usepackage{wrapfig}
\usepackage{bm}

% ルビを振る
%\usepackage{luatexja-ruby}	% required for `\ruby'

%% 核Ker 像Im Hom を定義
%\newcommand{\Img}{\mathop{\mathrm{Im}}\nolimits}
%\newcommand{\Ker}{\mathop{\mathrm{Ker}}\nolimits}
%\newcommand{\Hom}{\mathop{\mathrm{Hom}}\nolimits}

%\DeclareMathOperator{\Rot}{rot}
%\DeclareMathOperator{\Div}{div}
%\DeclareMathOperator{\Grad}{grad}
%\DeclareMathOperator{\arcsinh}{arcsinh}
%\DeclareMathOperator{\arccosh}{arccosh}
%\DeclareMathOperator{\arctanh}{arctanh}

\usepackage{url}

%\usepackage{listings}
%
%\lstset{
%%プログラム言語(複数の言語に対応,C,C++も可)
%  language = Python,
%%  language = Lisp,
%%  language = C,
%  %背景色と透過度
%  %backgroundcolor={\color[gray]{.90}},
%  %枠外に行った時の自動改行
%  breaklines = true,
%  %自動改行後のインデント量(デフォルトでは20[pt])
%  breakindent = 10pt,
%  %標準の書体
%%  basicstyle = \ttfamily\scriptsize,
%  basicstyle = \ttfamily,
%  %コメントの書体
%%  commentstyle = {\itshape \color[cmyk]{1,0.4,1,0}},
%  %関数名等の色の設定
%  classoffset = 0,
%  %キーワード(int, ifなど)の書体
%%  keywordstyle = {\bfseries \color[cmyk]{0,1,0,0}},
%  %表示する文字の書体
%  %stringstyle = {\ttfamily \color[rgb]{0,0,1}},
%  %枠 "t"は上に線を記載, "T"は上に二重線を記載
%  %他オプション:leftline,topline,bottomline,lines,single,shadowbox
%  frame = TBrl,
%  %frameまでの間隔(行番号とプログラムの間)
%  framesep = 5pt,
%  %行番号の位置
%  numbers = left,
%  %行番号の間隔
%  stepnumber = 1,
%  %行番号の書体
%%  numberstyle = \tiny,
%  %タブの大きさ
%  tabsize = 4,
%  %キャプションの場所("tb"ならば上下両方に記載)
%  captionpos = t
%}

%\usepackage{cancel}
%\usepackage{bussproofs}
%\usepackage{proof}

\begin{document}

\hrulefill

\begin{description}
 \item [問題 I]
       次の$3 \times 3$の実対称行列$A$を考える。
             \begin{equation}
              A=
               \begin{pmatrix}
                -1 & -1 & 1 \\
                -1 & 1 & -1 \\
                1 & -1 & -1
               \end{pmatrix}
             \end{equation}
       また、
       線形写像$f:\mathbb{R}^{3}\to\mathbb{R}^{3}$ を $f(\bm{x})=A\bm{x}$
       と定義する。
       ここで、
       $\bm{x}\in\mathbb{R}^{3}$は3次元列ベクトルである。

       \begin{enumerate}
        \item
             $f$の合成写像を次のように与える。
             \begin{equation}
              g(\bm{x}) = f\circ f (\bm{x}) = f\left( f (\bm{x})\right)
             \end{equation}
             この合成写像は$g(\bm{x})=B\bm{x}$と表すことができる。
             行列$B$を行列$A$を用いて表せ。

        \item \label{enu:eigenvec}
             以下の$\bm{v}_{1},\bm{v}_{2},\bm{v}_{3}$は
              行列$A$の固有ベクトルである。
             \begin{equation}
              \bm{v}_{1}= \begin{pmatrix} -1 \\ 0 \\ 1 \end{pmatrix},
              \bm{v}_{2}= \begin{pmatrix} 1 \\ 1 \\ 1 \end{pmatrix},
              \bm{v}_{3}= \begin{pmatrix} 1 \\ -2 \\ 1 \end{pmatrix}
             \end{equation}
             各ベクトルに対応する固有値をそれぞれ答えよ。

        \item \label{enu:inv}
             行列$P$を\ref{enu:eigenvec}の
             $\bm{v}_{1},\bm{v}_{2},\bm{v}_{3}$を用いて
             次のように定義する。
             \begin{equation}
              P = (\bm{v}_{1},\bm{v}_{2},\bm{v}_{3})
               =
               \begin{pmatrix}
                -1 & 1 & 1 \\
                0 & 1 & -2 \\
                1 & 1 & 1
               \end{pmatrix}
             \end{equation}
             この行列の逆行列$P^{-1}$は、
             ある行列$X$を用いて
             $P^{-1}=X \;{}^{t}\!P$と表される。
             行列$X$を求めよ。
             ただし、${}^{t}\!P$は$P$の転置行列である。



        \item
             実数パラメータ$a,b,c$を用いて、
             ベクトル$\bm{x}$が
             $\bm{x}=a\bm{v}_{1}+b\bm{v}_{2}+c\bm{v}_{3}$と
             表されるとき、
             次を満たす行列$Y$を求めよ。
             \begin{equation}
              A\bm{x} = PY\begin{pmatrix} a \\ b \\ c \end{pmatrix}
             \end{equation}
             ただし、
             $\bm{v}_{1},\bm{v}_{2},\bm{v}_{3}$および
             $P$は
             \ref{enu:eigenvec}-\ref{enu:inv}で用いた
             ベクトルおよび行列である。


        \item
             $f$を5回合成した写像$h(\bm{x})=f\circ f\circ f\circ f\circ f(\bm{x})=f(f(f(f(f(\bm{x})))))$は、
             \ref{enu:inv}の行列$P$を用いて
             $h(\bm{x})=(PZ\;{}^{t}\!P)\bm{x}$と表すことができる。
             行列$Z$を求めよ。


       \end{enumerate}

       \dotfill

       \begin{enumerate}
        \item
             \begin{equation}
              g(\bm{x})=f(f(\bm{x}))=f(A\bm{x})=AA\bm{x}
             \end{equation}
             よって、$B=A^{2}$である。


        \item
             行列の固有値$\lambda$、固有ベクトル$\bm{v}$は
             $A\bm{v}=\lambda\bm{v}$を満たす。

             固有ベクトル$\bm{v}_{i} \ (i=1,2,3)$がわかっているのであれば、
             対応する固有値$\lambda_{i}$は
             $A\bm{v}_{i}=\lambda_{i}\bm{v}_{i}$を満たす。

             それぞれを計算すると
             $\lambda_{1}=-2,\ \lambda_{2}=-1,\ \lambda_{3}=2$
             である。


        \item
             実対称行列の固有値は実数となり、
             固有値が異なるときの固有ベクトルは直交する。

             $i\ne j$であれば、
             内積$\bm{v}_{i}\cdot\bm{v}_{j}=0$
             である。
             \begin{equation}
              {}^{t}\!P P =
              \begin{pmatrix}
               \lvert \bm{v}_{1} \rvert^{2} & 0 & 0 \\
               0 & \lvert \bm{v}_{2} \rvert^{2} & 0 \\
               0 & 0 & \lvert \bm{v}_{3} \rvert^{2}
              \end{pmatrix}
              =
              \begin{pmatrix}
               2 & 0 & 0 \\
               0 & 3 & 0 \\
               0 & 0 & 6
              \end{pmatrix}
             \end{equation}

             ここに、逆行列をかけると次の式が得られる。
             \begin{equation}
              \begin{pmatrix}
               2 & 0 & 0 \\
               0 & 3 & 0 \\
               0 & 0 & 6
              \end{pmatrix}^{-1}
              {}^{t}\!P
              = P^{-1}
             \end{equation}

             よって、$P^{-1}=X\;{}^{t}P$を満たす行列$X$は以下のようになる。
             \begin{equation}
              X=
              \begin{pmatrix}
               2 & 0 & 0 \\
               0 & 3 & 0 \\
               0 & 0 & 6
              \end{pmatrix}^{-1}
              =
              \begin{pmatrix}
               \frac{1}{2} & 0 & 0 \\
               0 & \frac{1}{3} & 0 \\
               0 & 0 & \frac{1}{6}
              \end{pmatrix}
             \end{equation}


        \item
             $\bm{x}=a\bm{v}_{1}+b\bm{v}_{2}+c\bm{v}_{3}$より
             \begin{align}
              A\bm{x}
               &=A(a\bm{v}_{1}+b\bm{v}_{2}+c\bm{v}_{3})
               =aA\bm{v}_{1}+bA\bm{v}_{2}+cA\bm{v}_{3}\\
               &=a\lambda_{1}\bm{v}_{1}+b\lambda_{2}\bm{v}_{2}+c\lambda_{3}\bm{v}_{3}
              =
              \begin{pmatrix}
               \lambda_{1}\bm{v}_{1} & \lambda_{2}\bm{v}_{2} & \lambda_{3}\bm{v}_{3}
              \end{pmatrix}
              \begin{pmatrix}
               a \\ b \\ c
              \end{pmatrix}\\
              &=
              \begin{pmatrix}
               \bm{v}_{1} & \bm{v}_{2} & \bm{v}_{3}
              \end{pmatrix}
              \begin{pmatrix}
               \lambda_{1} & 0 & 0 \\
               0 & \lambda_{2} & 0 \\
               0 & 0 & \lambda_{3}
              \end{pmatrix}
              \begin{pmatrix}
               a \\ b \\ c
              \end{pmatrix}
              =
              P
              \begin{pmatrix}
               \lambda_{1} & 0 & 0 \\
               0 & \lambda_{2} & 0 \\
               0 & 0 & \lambda_{3}
              \end{pmatrix}
              \begin{pmatrix}
               a \\ b \\ c
              \end{pmatrix}
             \end{align}

             よって、
             $Y=
             \begin{pmatrix}
               -2 & 0 & 0 \\
               0 & -1 & 0 \\
               0 & 0 & 2
             \end{pmatrix}$
             である。


        \item
             $h(\bm{x})=A^{5}\bm{x}$である。

             行列$P$を用いて行列$A$は次のように対角化できる。
             \begin{equation}
              P^{-1}AP =
               \begin{pmatrix}
                -2 & 0 & 0 \\
                0 & -1 & 0 \\
                0 & 0 & 2
               \end{pmatrix}
             \end{equation}

             これを変形すると次の式が得られる。
             \begin{equation}
              A= P
               \begin{pmatrix}
                -2 & 0 & 0 \\
                0 & -1 & 0 \\
                0 & 0 & 2
               \end{pmatrix}
               P^{-1}
             \end{equation}

             これと、\ref{enu:inv}を用いて$A^{5}$を計算する。
             \begin{align}
              A^{5}
              &=P
               \begin{pmatrix}
                -2 & 0 & 0 \\
                0 & -1 & 0 \\
                0 & 0 & 2
               \end{pmatrix}
               P^{-1}
               \cdots
               P
               \begin{pmatrix}
                -2 & 0 & 0 \\
                0 & -1 & 0 \\
                0 & 0 & 2
               \end{pmatrix}
               P^{-1}\\
               &=P
               \begin{pmatrix}
                -2 & 0 & 0 \\
                0 & -1 & 0 \\
                0 & 0 & 2
               \end{pmatrix}^{5}
               P^{-1}
              =
              P
              \begin{pmatrix}
                -2 & 0 & 0 \\
                0 & -1 & 0 \\
                0 & 0 & 2
              \end{pmatrix}^{5}
              \begin{pmatrix}
               \frac{1}{2} & 0 & 0 \\
               0 & \frac{1}{3} & 0 \\
               0 & 0 & \frac{1}{6}
              \end{pmatrix}
              {}^{t}\!P\\
              &=
              P
              \begin{pmatrix}
                -16 & 0 & 0 \\
                0 & -\frac{1}{3} & 0 \\
                0 & 0 & \frac{16}{3}
              \end{pmatrix}
              {}^{t}\!P
             \end{align}
       \end{enumerate}

        よって、
        $Z=\begin{pmatrix}
                -16 & 0 & 0 \\
                0 & -\frac{1}{3} & 0 \\
                0 & 0 & \frac{16}{3}
              \end{pmatrix}$
             である。

       \hrulefill

 \item[問題2]

       \begin{enumerate}
%        \item
%             実数関数$f(x)=\exp{(x^{2}/2)} \ (x\in\mathbb{R})$とする。
%             $f^{(n)}(x) \ (n=1,2,\dots)$は
%             $f(x)$の$n$階導関数である。
%             ただし、
%             $f^{(0)}(x)=f(x)$とする。
%             以下の問いに答えよ。
%             \begin{enumerate}
%              \item
%                   次式を示せ。
%                   \begin{equation}
%                    f^{(1)}(x)=xf(x)
%                     \qquad
%                    f^{(2)}(x)=xf^{(1)}(x) + f(x)
%                   \end{equation}
%              \item
%
%              \item
%
%              \item
%             \end{enumerate}

        \setcounter{enumi}{1}
        \item
             実数関数$g(x)$は以下のように定義される。
             \begin{equation}
              g(x)=
               \begin{cases}
                \frac{1}{\sqrt{2\pi}x}\exp{\left(-\frac{1}{2}(\ln{x})^{2}\right)} & (x>0)\\
                0 & (x\leq 0)
               \end{cases}
             \end{equation}
             以下の問に答えよ。
             ただし、
             $\int_{-\infty}^{\infty} \exp{(-x^{2}/2)}dx = \sqrt{2\pi}$を
             証明なしで使ってもよい。
             \begin{enumerate}
              \item
                   $\int_{-\infty}^{\infty} g(x) dx$を求めよ。

              \item
                   $\int_{-\infty}^{\infty} xg(x) dx$を求めよ。

              \item
                   $\int_{-\infty}^{\infty} x^{n}g(x) dx$を求めよ。
                   ただし、$n$は自然数である。
             \end{enumerate}



             \dotfill

             \begin{enumerate}
              \item
                   $\int_{-\infty}^{\infty} \exp{(-x^{2}/2)}dx = \sqrt{2\pi}$を
                   利用するため、
                   $t=\ln{x}$と置く。
                   これにより
                   $dt/dx=1/x$である。

                   そこで、
                   $t= \ln{x}$と置いて置換積分を考える。
                   \begin{align}
                    \int_{-\infty}^{\infty} g(x) dx
                    &=
                    \int_{0}^{\infty} \frac{1}{\sqrt{2\pi}x}\exp{\left(-\frac{1}{2}(\ln{x})^{2}\right)} dx\\
                    &=
                    \int_{-\infty}^{\infty} \frac{1}{\sqrt{2\pi}}\exp{\left(-\frac{t^{2}}{2}\right)} dt\\
                    &= \frac{1}{\sqrt{2\pi}} \int_{-\infty}^{\infty}\exp{\left(-\frac{t^{2}}{2}\right)} dt
                    = \frac{1}{\sqrt{2\pi}} \times \sqrt{2\pi}
                    =1
                   \end{align}


              \item
                   $t=\ln{x}$と置くと
                   $x=\exp{(t)}$である。
                   そこで、$xg(x)$を変形する。
                   \begin{align}
                    xg(x)
                    &= x \frac{1}{\sqrt{2\pi}x}\exp{\left(-\frac{1}{2}(\ln{x})^{2}\right)}
                    = \frac{1}{\sqrt{2\pi}x}\exp{(t)}\exp{\left(-\frac{1}{2}t^{2}\right)}\\
                    &= \frac{1}{\sqrt{2\pi}x}\exp{\left(-\frac{1}{2}(t-1)^{2}+\frac{1}{2}\right)}
                    = \frac{\exp{(\frac{1}{2})}}{\sqrt{2\pi}x}\exp{\left(-\frac{1}{2}(t-1)^{2}\right)}
                   \end{align}

                   ここで、$s=t-1$と置くと
                   $\int_{-\infty}^{\infty} \exp{(-x^{2}/2)}dx$の形の式
                   が得られる。

                   そこで、
                   $s=\ln{x}-1$として置換積分を行う。
                   このとき、$ds/dx=1/x$である。
                   \begin{align}
                    \int_{-\infty}^{\infty} xg(x) dx
                    &=
                    \int_{0}^{\infty} x \frac{1}{\sqrt{2\pi}x}\exp{\left(-\frac{1}{2}(\ln{x})^{2}\right)} dx\\
                    &= \frac{\exp{(\frac{1}{2})}}{\sqrt{2\pi}} \int_{-\infty}^{\infty} \exp{\left(-\frac{1}{2}s^{2}\right)} ds
                    = \exp{\left(\frac{1}{2}\right)}
                   \end{align}


              \item
                   上記の問いを踏まえ、
                   $u=\ln{x} -n$と置く。
                   このとき、$x=\exp{(u+n)}$であり、$du/dx=1/x$である。
                   \begin{align}
                    \int_{-\infty}^{\infty} x^{n}g(x) dx
                    &= \int_{0}^{\infty} x^{n} \frac{1}{\sqrt{2\pi}}\exp{\left(-\frac{1}{2}(\ln{x})^{2}\right)} \frac{dx}{x}\\
                    &= \int_{-\infty}^{\infty} \exp{n(u+n)} \frac{1}{\sqrt{2\pi}}\exp{\left(-\frac{1}{2}(u+n)^{2}\right)} du\\
                    &= \frac{1}{\sqrt{2\pi}} \int_{-\infty}^{\infty} \exp{\left(-\frac{u^{2}}{2}+ \frac{n^{2}}{2}\right)}du\\
                    &= \frac{\exp{(\frac{n^{2}}{2})}}{\sqrt{2\pi}} \int_{-\infty}^{\infty} \exp{\left(-\frac{u^{2}}{2}\right)}du\\
                    &= \exp{\left(\frac{n^{2}}{2}\right)}
                   \end{align}



             \end{enumerate}

       \end{enumerate}

       \hrulefill

\end{description}

\hrulefill

\end{document}
