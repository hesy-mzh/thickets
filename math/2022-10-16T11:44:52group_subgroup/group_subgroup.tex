\documentclass[12pt,b5paper]{ltjsarticle}

%\usepackage[margin=15truemm, top=5truemm, bottom=5truemm]{geometry}
\usepackage[margin=10truemm]{geometry}

\usepackage{amsmath,amssymb}
%\pagestyle{headings}
\pagestyle{empty}

%\usepackage{listings,url}
%\renewcommand{\theenumi}{(\arabic{enumi})}

%\usepackage{graphicx}

%\usepackage{tikz}
%\usetikzlibrary {arrows.meta}
%\usepackage{wrapfig}	% required for `\wrapfigure' (yatex added)
%\usepackage{bm}	% required for `\bm' (yatex added)

% ルビを振る
%\usepackage{luatexja-ruby}	% required for `\ruby'

%% 核Ker 像Im Hom を定義
%\newcommand{\Img}{\mathop{\mathrm{Im}}\nolimits}
%\newcommand{\Ker}{\mathop{\mathrm{Ker}}\nolimits}
%\newcommand{\Hom}{\mathop{\mathrm{Hom}}\nolimits}

%\DeclareMathOperator{\Rot}{rot}
%\DeclareMathOperator{\Div}{div}
%\DeclareMathOperator{\Grad}{grad}
%\DeclareMathOperator{\arcsinh}{arcsinh}
%\DeclareMathOperator{\arccosh}{arccosh}
%\DeclareMathOperator{\arctanh}{arctanh}



\begin{document}

\hrulefill
\textbf{定義}
\hrulefill

\textbf{群}

集合$G$について
演算$\star$が定義されていて、
この演算で閉じているとする。
つまり、
${}^{\forall}a,b\in G \Rightarrow a\star b\in G$
である。

次の3条件を満たす時、$G$を群(group)という。
\begin{enumerate}
 \item \textbf{結合律}

       ${}^{\forall}a,b,c\in G$に対して
       \begin{equation}
        (a\star b) \star c = a\star (b \star c)
       \end{equation}

 \item \textbf{単位元}

       元$e\in G$が存在し、
       ${}^{\forall}a\in G$に対して
       \begin{equation}
        a\star e = e \star a = a
       \end{equation}

 \item \textbf{逆元}

       ${}^{\forall}a\in G$に対して、
       次を満たす元$b\in G$が存在
       \begin{equation}
        a\star b = b \star a = e
       \end{equation}
\end{enumerate}

単位元は$e$で表し、
$a$の逆元を$a^{-1}$で表す事が多いが、
演算が加法であるときは単位元を$0$
逆元を$-a$と書く事が多い。

\textbf{部分群}

群$G$の部分集合$H$が群である時、
$H$を$G$の部分群という。


\textbf{互いに素}


$a,b\in\mathbb{Z}$が互いに素であるとは、
$a$と$b$を素因数と単元($1$と$-1$)の積で表した時に
共通に現れる数が$1$と$-1$だけである事をいう。

例えば、$2,3$であれば、
\begin{equation}
 2 = 1\times 2
  \quad
 3 = 1\times 3
\end{equation}
であるので互いに素である。
これは$2=1\times 1\times 2 = (-1)\times (-1) \times 2$などの
等様々な積で表せるが、
共通に現れるのは$1,-1$のみである。

これにより$0$と$1$は互いに素であるが
$0$と$2$は互いに素でない。



\textbf{対称群}

8次対称群$S_{8}$は
集合$\{1,2,\dots,8\}$から集合$\{1,2,\dots,8\}$への
全単射全体の集合であり、
写像の合成により群を成す。

$S_{8}$の元$\sigma$は全単射であり、次のような表記で表す。
\begin{equation}
 \begin{pmatrix}
  1&2&3&4&5&6&7&8\\
  1&2&5&4&8&6&7&3\\
 \end{pmatrix}
 =(3,5,8)
\end{equation}

恒等写像を$(1)$で表し、
3を7、7を3に写しそれ以外は同じ数に写す写像を互換といい$(3,7)$と表す。
\begin{equation}
 (1)=
 \begin{pmatrix}
  1&2&3&4&5&6&7&8\\
  1&2&3&4&5&6&7&8\\
 \end{pmatrix}
 ,\quad
 (3,7)=
 \begin{pmatrix}
  1&2&3&4&5&6&7&8\\
  1&2&7&4&5&6&3&8\\
 \end{pmatrix}
\end{equation}

対称群$S_{8}$の元$\sigma$は全て互換の積で表される。
この互換の個数を$n$とするとき、
$(-1)^n$を$\sigma$の符号(signature)といい、
$\mathrm{sgn}(\sigma)$と書く。

%\textbf{
符号の定義は色々あり、
定義に従って確認するのであれば状況に応じて判断する必要がある。
%}

\textbf{符号の定義}

\begin{equation}
 \sigma =
  \begin{pmatrix}
   1&2&3&4&5&6&7&8\\
   \sigma_1 & \sigma_2 & \sigma_3 & \sigma_4 & \sigma_5 & \sigma_6 & \sigma_7 & \sigma_8\\
  \end{pmatrix}
\end{equation}
上記のような$\sigma$に対して転倒数$N(\sigma)$を次のように定める。
\begin{equation}
 N(\sigma) = \lvert \{(i,j) \mid 1\leq i<j \leq 8 , \sigma_i>\sigma_j \} \rvert
\end{equation}
この時 $\sigma$の符号$\mathrm{sgn}(\sigma)$を次のように定義する。
\begin{equation}
 \mathrm{sgn}(\sigma)=(-1)^{N(\sigma)}
\end{equation}



\hrulefill
\textbf{問題}
\hrulefill

\begin{enumerate}
 \item
      次の$(\mathbb{R},+)$の部分集合は
      部分群であるかどうか調べよ。
      \begin{enumerate}
       \item
            $\{ x \in \mathbb{Z} \mid x \geq 0\}$

       \item
            $\{ \frac{a}{b} \in \mathbb{Q} \mid b\in 2\mathbb{Z}, aとbは互いに素\}$

       \item
            $\{ \frac{a}{b} \in \mathbb{Q} \mid bは奇数, aとbは互いに素\}$

       \item
            $\{ \frac{a}{b} \in \mathbb{Q} \mid b\in\{1,2\}, aとbは互いに素\}$

       \item
            $\{ \frac{a}{b} \in \mathbb{Q} \mid b\in\{1,2,3\}, aとbは互いに素\}$
      \end{enumerate}


\dotfill

      部分群であることを示すためには
      演算で閉じていることと
      単位元、逆元の存在を示せばいい。
      結合律については、
      部分集合が閉じていれば
      もとの群の結合律がそのまま成立する。

      $(\mathbb{R},+)$の単位元は$0$であり、
      $x\in \mathbb{R}$の逆元は$-x\in\mathbb{R}$である。

      \begin{enumerate}
       \item
            部分集合$\{ x \in \mathbb{Z} \mid x \geq 0 \}$に
            $1$は含まれていても$-1$は含まれない。
            つまり、$1$の逆元は存在しない。
            この為、この部分集合は部分群ではない。

       \item
            部分集合
            $\{ \frac{a}{b} \in \mathbb{Q} \mid b\in 2\mathbb{Z}, aとbは互いに素\}$
            に$0$が含まれない。
            これは、分母が偶数で分子は偶数ではないので
            分子は$0$でない。
            この為、単位元を含まないので部分群ではない。

       \item
            $\{ \frac{a}{b} \in \mathbb{Q} \mid bは奇数, aとbは互いに素\}$

            $\frac{0}{1}=0$が単位元であり、
            $\frac{a}{b}$に対し$\frac{-a}{b}$が逆元となる。

            $\frac{a}{b},\frac{c}{d}$に対し
            $\frac{a}{b}+\frac{c}{d} = \frac{ad+bc}{bd}$である。
            $b$と$d$は奇数であるので$bd$も奇数となる。
            $\mathbb{R}$上で$\frac{ad+bc}{bd}$は既約な分数表記に出来る。
            この為、演算$+$で閉じていることが分かる。

            以上により部分群であることが分かる。


       \item
            $\{ \frac{a}{b} \in \mathbb{Q} \mid b\in\{1,2\}, aとbは互いに素\}$

            すべての整数は$1$と互いに素であるので単位元$0$を含む。
            また、$\frac{a}{b}$に対し$\frac{-a}{b}$を含む為
            逆元も含む。

            $\frac{a}{b}+\frac{c}{d}=\frac{ad+bc}{bd}$である。
            分母は$\{1,2\}$の元であるので
            $bd$も$\{1,2\}$の元である。
            $\frac{ad+bc}{bd}$は$\mathbb{R}$の元であるので
            既約な分数が一つ存在する。

            この為、演算$+$で閉じている事がわかる。

            以上により部分群となることが分かる。


       \item
            $\{ \frac{a}{b} \in \mathbb{Q} \mid b\in\{1,2,3\}, aとbは互いに素\}$

            $\frac{1}{2},\frac{1}{3}$は部分集合に含まれるが
            和$\frac{1}{2}+\frac{1}{3}=\frac{5}{6}$は含まれない為、
            演算$+$で閉じていない。
            つまり、部分群ではない。
      \end{enumerate}

\hrulefill

 \item
      $8$次対照群$S_{8}$の元$\sigma$を
      以下のように与える。
      それぞれの符号を求めよ。
      \begin{enumerate}
       \item
            $(6,4)$

       \item
            $(4,5)(1,7)(2,9)$

       \item
            $(5,8,2,7,3)$

       \item
            $
            \begin{pmatrix}
             1 & 2 & 3 & 4 & 5 & 6 & 7 & 8\\
             \downarrow & \downarrow & \downarrow & \downarrow & \downarrow & \downarrow & \downarrow & \downarrow \\
             6 & 7 & 2 & 3 & 4 & 1 & 8 & 5
            \end{pmatrix}
            $
      \end{enumerate}

\dotfill

      \begin{enumerate}
       \item
            $(6,4)$

            互換1つだけであるので
            $\mathrm{sgn}(6,4)=(-1)^1=-1$
            である。

       \item
            $(4,5)(1,7)(2,9)$

            互換3つの積であるので
            $\mathrm{sgn}( (4,5)(1,7)(2,9) )=(-1)^3=-1$
            である。

       \item
            $(5,8,2,7,3)$

            互換の積に表すと次のようになる。
            \begin{equation}
             (5,8,2,7,3)=
             \begin{pmatrix}
              1 & 2 & 3 & 4 & 5 & 6 & 7 & 8\\
              1 & 7 & 5 & 4 & 8 & 6 & 3 & 2\\
             \end{pmatrix}
             =
             (2,8)(2,3)(3,5)(2,7)
            \end{equation}

            これは次のような順に上から下に向かい
            2つを入れ替える作業を行うと
            $(5,8,2,7,3)$が構成できる。
            \begin{equation}
             \begin{matrix}
              & 1 & 2 & 3 & 4 & 5 & 6 & 7 & 8\\
              (2,7) & \downarrow & \downarrow & \downarrow & \downarrow & \downarrow & \downarrow & \downarrow & \downarrow \\
              & 1 & 7 & 3 & 4 & 5 & 6 & 2 & 8\\
              (3,5) & \downarrow & \downarrow & \downarrow & \downarrow & \downarrow & \downarrow & \downarrow & \downarrow \\
              & 1 & 7 & 5 & 4 & 3 & 6 & 2 & 8\\
              (2,3) & \downarrow & \downarrow & \downarrow & \downarrow & \downarrow & \downarrow & \downarrow & \downarrow \\
              & 1 & 7 & 5 & 4 & 2 & 6 & 3 & 8\\
              (2,8) & \downarrow & \downarrow & \downarrow & \downarrow & \downarrow & \downarrow & \downarrow & \downarrow \\
              & 1 & 7 & 5 & 4 & 8 & 6 & 3 & 2\\
             \end{matrix}
            \end{equation}

            これにより$\mathrm{sgn}(5,8,2,7,3)=(-1)^4=1$である。

       \item
            $
            \begin{pmatrix}
             1 & 2 & 3 & 4 & 5 & 6 & 7 & 8\\
             \downarrow & \downarrow & \downarrow & \downarrow & \downarrow & \downarrow & \downarrow & \downarrow \\
             6 & 7 & 2 & 3 & 4 & 1 & 8 & 5
            \end{pmatrix}
            $

            これは$(1,6)$と$(2,7,8,5,4,3)$に分解できる。
            $(2,7,8,5,4,3)$を互換の積で表す。
            \begin{equation}
             (2,7,8,5,4,3) = (5,8)(4,5)(3,4)(2,3)(2,7)
            \end{equation}

            順に上から下に向かい
            2つを入れ替える作業を行うと
            $(2,7,8,5,4,3)$が構成できる。
            \begin{equation}
             \begin{matrix}
              & 1 & 2 & 3 & 4 & 5 & 6 & 7 & 8\\
              (2,7) & \downarrow & \downarrow & \downarrow & \downarrow & \downarrow & \downarrow & \downarrow & \downarrow \\
              & 1 & 7 & 3 & 4 & 5 & 6 & 2 & 8\\
              (2,3) & \downarrow & \downarrow & \downarrow & \downarrow & \downarrow & \downarrow & \downarrow & \downarrow \\
              & 1 & 7 & 2 & 4 & 5 & 6 & 3 & 8\\
              (3,4) & \downarrow & \downarrow & \downarrow & \downarrow & \downarrow & \downarrow & \downarrow & \downarrow \\
              & 1 & 7 & 2 & 3 & 5 & 6 & 4 & 8\\
              (4,5) & \downarrow & \downarrow & \downarrow & \downarrow & \downarrow & \downarrow & \downarrow & \downarrow \\
              & 1 & 7 & 2 & 3 & 4 & 6 & 5 & 8\\
              (5,8) & \downarrow & \downarrow & \downarrow & \downarrow & \downarrow & \downarrow & \downarrow & \downarrow \\
              & 1 & 7 & 2 & 3 & 4 & 6 & 8 & 5\\
             \end{matrix}
            \end{equation}

            以上により$(1,6)(5,8)(4,5)(3,4)(2,3)(2,7)$と
            互換の積で表される。
            よって、符号は$(-1)^6=1$となる。
      \end{enumerate}
\end{enumerate}






\hrulefill


\end{document}

