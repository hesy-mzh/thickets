\documentclass[12pt,b5paper]{ltjsarticle}

%\usepackage[margin=15truemm, top=5truemm, bottom=5truemm]{geometry}
\usepackage[margin=10truemm]{geometry}

\usepackage{amsmath,amssymb}
%\pagestyle{headings}
\pagestyle{empty}

%\usepackage{listings,url}
%\renewcommand{\theenumi}{(\arabic{enumi})}

%\usepackage{graphicx}

%\usepackage{tikz}
%\usetikzlibrary {arrows.meta}
%\usepackage{wrapfig}	% required for `\wrapfigure' (yatex added)
\usepackage{bm}	% required for `\bm' (yatex added)

% ルビを振る
%\usepackage{luatexja-ruby}	% required for `\ruby'

%% 核Ker 像Im Hom を定義
%\newcommand{\Img}{\mathop{\mathrm{Im}}\nolimits}
%\newcommand{\Ker}{\mathop{\mathrm{Ker}}\nolimits}
%\newcommand{\Hom}{\mathop{\mathrm{Hom}}\nolimits}

%\DeclareMathOperator{\Rot}{rot}
%\DeclareMathOperator{\Div}{div}
%\DeclareMathOperator{\Grad}{grad}
%\DeclareMathOperator{\arcsinh}{arcsinh}
%\DeclareMathOperator{\arccosh}{arccosh}
%\DeclareMathOperator{\arctanh}{arctanh}



\usepackage{listings,url}

\lstset{
%プログラム言語(複数の言語に対応,C,C++も可)
  language = Python,
%  language = Lisp,
%  language = C,
  %背景色と透過度
  %backgroundcolor={\color[gray]{.90}},
  %枠外に行った時の自動改行
  breaklines = true,
  %自動改行後のインデント量(デフォルトでは20[pt])
  breakindent = 10pt,
  %標準の書体
%  basicstyle = \ttfamily\scriptsize,
  basicstyle = \ttfamily,
  %コメントの書体
%  commentstyle = {\itshape \color[cmyk]{1,0.4,1,0}},
  %関数名等の色の設定
  classoffset = 0,
  %キーワード(int, ifなど)の書体
%  keywordstyle = {\bfseries \color[cmyk]{0,1,0,0}},
  %表示する文字の書体
  %stringstyle = {\ttfamily \color[rgb]{0,0,1}},
  %枠 "t"は上に線を記載, "T"は上に二重線を記載
  %他オプション:leftline,topline,bottomline,lines,single,shadowbox
  frame = TBrl,
  %frameまでの間隔(行番号とプログラムの間)
  framesep = 5pt,
  %行番号の位置
  numbers = left,
  %行番号の間隔
  stepnumber = 1,
  %行番号の書体
%  numberstyle = \tiny,
  %タブの大きさ
  tabsize = 4,
  %キャプションの場所("tb"ならば上下両方に記載)
  captionpos = t
}



\begin{document}

\hrulefill
\textbf{定義}
\hrulefill

\textbf{確率過程}

確率変数$X_t$が順序付けされてまとめられたものを
\textbf{時系列データ}といい、
$X = \{ X_t \mid t\in\mathbb{Z} \}$
のように表す。


時系列$X$の確率変数$X_i,X_k$の
共分散$\textrm{Cov}(X_i,X_k)$を
$\lvert i-k \rvert$次の\textbf{自己共分散}という。
この$\lvert i-k \rvert$のことをラグと呼ぶ。

${}^\forall X_i \in X$に対して
$E[X_i] = \mu$となる値$\mu$が1つ存在し、
${}^\forall X_i,X_{i+k} \in X$に対して
$Cov(X_i,X_{i+k})=\sigma_k$となる値$\sigma_k$が
ラグ$k$についてのみ依存して定まる時、
時系列$X$は弱定常性を持つという。


\textbf{ホワイトノイズ}

ホワイトノイズとは、
弱定常性を持つ時系列$X$で
次の性質をもつものをいう。
\begin{equation}
 {}^{\forall}X_i,X_k\in X \ (i\ne k)\text{ に対し }
  \quad
  E[X_i]=0
  ,\
  %Cov(X_i,X_i)=
  V[X_i]=\sigma^2
  ,\
  Cov(X_i,X_k)=0
\end{equation}


\dotfill
\textbf{性質}
\dotfill
\begin{align}
 E[aX+b] =& aE[X]+b\\
 E[X_1+X_2] =& E[X_1]+E[X_2]\\
 V[X] =& E[(X-E[X])^2]\\
 V[X] =& E[X^2] - (E[X])^2\\
 Cov(X_1,X_2) =& E[(X_1-E[X_1])(X_2-E[X_2])]\\
 V[X] =& Cov(X,X)\\
 Cov(X_1,X_2) =& E[X_1X_2]- E[X_1]E[X_2]\\
 V[X_1+X_2] =& V[X_1]+V[X_2]+2Cov(X_1,X_2)\\
 Cov(X_1+X_2,X_3) =& Cov(X_1,X_3)+ Cov(X_2,X_3)
\end{align}

\hrulefill
\textbf{問題}
\hrulefill


\begin{enumerate}
 \item
      $X=\{X_{t};t\in\mathbb{Z}\}$
      を実数値弱定常時系列で、
      平均関数が$0$であるものとし、
      $X$の自己共分散関数を$\gamma$とする。
      \begin{enumerate}
       \item
            ${}^{\forall}h\in\mathbb{Z}$
            に対して、
            $\gamma(-h)=\gamma(h)$が成り立つことを示せ。

\dotfill

            自己共分散関数$\gamma$は
            $\gamma(h)=Cov(X_{i+h},X_{i})$
            であるとすると
            $\gamma(-h)=Cov(X_{i+h},X_{i})$
            となる。

            共分散の定義から次の性質が得られる。
            \begin{align}
             Cov(X_{i+h},X_{i})
              =& E[(X_{i+h}-E[X_{i+h}])(X_i-E[X_i])]\\
              =& E[(X_{i}-E[X_{i}])(X_{i+h}-E[X_{i+h}])]
              = Cov(X_{i},X_{i+h})
            \end{align}

            よって、
            $\gamma(-h)=\gamma(h)$
            である。

\hrulefill
       \item
            $n\in\mathbb{N}$とするとき、
            次の$n$次正方行列$\Gamma_{n}$の
            固有値が全て$0$以上であることを示せ。
            \begin{equation}
             \Gamma_{n}=
              \begin{pmatrix}
               \gamma(0) & \gamma(1) & \gamma(2) & \cdots & \gamma(n-1)\\
               \gamma(1) & \gamma(0) & \gamma(1) & \cdots & \gamma(n-2)\\
               \vdots & \vdots & \vdots & \ddots & \vdots \\
               \gamma(n-1) & \gamma(n-2) & \gamma(n-3) & \cdots & \gamma(0)\\
              \end{pmatrix}
            \end{equation}
\dotfill


            時系列$X$より連続した
            $n$個の確率変数$X_1,\cdots,X_n$を取り出し
            分散共分散行列$\Sigma$を作る。
            \begin{equation}
             \Sigma =
             \begin{pmatrix}
              Cov(X_1,X_1) & Cov(X_1,X_2) & \cdots & Cov(X_1,X_n)\\
              Cov(X_2,X_1) & Cov(X_2,X_2) & \cdots & Cov(X_2,X_n)\\
              \vdots & \vdots & \ddots & \vdots\\
              Cov(X_n,X_1) & Cov(X_n,X_2) & \cdots & Cov(X_n,X_n)\\
             \end{pmatrix}
            \end{equation}

            取り出した確率変数は連続していれば全て同じ行列となり
            $\Gamma_n = \Sigma$である。

            分散共分散行列は半正定値行列であるので
            固有値は全て0以上である。

\dotfill

            共分散は偏差の積の平均である。
            \begin{align}
             Cov(X_i,X_k)=&E[(X_i-E[X_i])(X_k-E[X_k])]\\
             =&\frac{1}{N}\sum_{j=1}^{N}
             (X_i^{(j)}-E[X_i])(X_k^{(j)}-E[X_k])
            \end{align}
            (ただし、$X_i$は$N$個のデータから成る確率変数で、
            $E[X_i]=\frac{1}{N}\sum_{j=1}^{N}X_I^{(j)}$とする)

            この為、共分散$Cov(X_i,X_k)$はベクトルの内積を用いて
            次のように書ける。
            \begin{align}
             & Cov(X_i,X_k)\\
             =&
              \frac{1}{N}
              \begin{pmatrix}
               X_i^{(1)}-E[X_i] & X_i^{(2)}-E[X_i] &
               \cdots & X_i^{(N)}-E[X_i]
              \end{pmatrix}
              \begin{pmatrix}
               X_k^{(1)}-E[X_k] \\ X_k^{(2)}-E[X_k] \\
               \vdots \\ X_k^{(N)}-E[X_k]
              \end{pmatrix}
            \end{align}

            そこで$n\times N$行列$C$を次のように作る。
            \begin{equation}
             C=
              \begin{pmatrix}
               X_1^{(1)}-E[X_1] & \cdots & X_1^{(N)}-E[X_1]\\
               \vdots & \ddots & \vdots\\
               X_n^{(1)}-E[X_n] & \cdots & X_n^{(N)}-E[X_n]
              \end{pmatrix}
            \end{equation}

            これにより行列$C$とその転置行列${}^{t}\!C$
            を用いて
            分散共分散行列を次のように書ける。
            \begin{equation}
             \Sigma = \frac{1}{N}C\, {}^{t}\!C
            \end{equation}


            任意のベクトル$\bm{x}$を用いて
            ${}^{t}\!\bm{x}\Sigma \bm{x}$を計算する。
            \begin{equation}
             {}^{t}\!\bm{x}\Sigma \bm{x}
              = \frac{1}{N} {}^{t}\!\bm{x} C\, {}^{t}\!C \bm{x}
              = \frac{1}{N} {}^{t}({}^{t}\!C\bm{x}) {}^{t}\!C \bm{x}
            \end{equation}

            ここで$\bm{y}={}^{t}\!C\bm{x}$とすると上の式は
            次のようになる。
            \begin{equation}
             {}^{t}\!\bm{x}\Sigma \bm{x}
              = \frac{1}{N} {}^{t}\!\bm{y} \bm{y}
              = \frac{1}{N} \sum_{j=1}^{N}y_j^2 \geq 0
            \end{equation}

            よって、分散共分散行列$\Sigma$は
            半正定値行列である。

\dotfill

            \textbf{半正定値}

            対称行列$A$が次を満たす時、
            半正定値という。
            \begin{equation}
             \text{任意のベクトル }\bm{x}
              \text{ に対して }
              {}^{t}\!\bm{x}A\bm{x} \geq 0
            \end{equation}

            \textbf{半正定値の性質}

            次は同値な条件である。
            \begin{itemize}
             \item $A$は半正定値
             \item $A$の固有値は全て非負
            \end{itemize}

\hrulefill

      \end{enumerate}

 \item
      実数値確率変数$X,Y$は次を満たすものとする。
      \begin{equation}
       E[X]=E[Y]=0,\ E[X^2]<\infty,\ E[Y^2]<\infty
      \end{equation}
      この時、次の不等式が成り立つことを示せ。
      \begin{equation}
       (Cov(X,Y))^2 \leq V[X]V[Y]
      \end{equation}

\dotfill

      確率変数$XY$の分散は次の式で求められる。
      \begin{equation}
       V[XY]=E[(XY)^2]-(E[XY])^2
      \end{equation}

      分散は定義から常に0以上の値を取るので、上の式から次が得られる。
      \begin{equation}
       E[(XY)^2] \geq (E[XY])^2
      \end{equation}

      条件($E[X]=E[Y]=0$)より
      $Cov(X,Y),V[X],V[Y]$は次のように変形できる。
      \begin{align}
       Cov(X,Y) =& E[XY]- E[X]E[Y]=E[XY]\\
        V[X] =& E[X^2]- (E[X])^2=E[X^2]\\
        V[Y] =& E[Y^2]- (E[Y])^2=E[Y^2]
      \end{align}

      \begin{equation}
       E[X^2Y^2] \geq (Cov(X,Y))^2
      \end{equation}

      $X,Y$の相関係数を$\rho$とすると
      $-1\leq \rho \leq 1$であり、
      $0 \leq \rho^2 \leq 1$である。

      $\rho$は次の式で求められる。
      \begin{equation}
       \rho = \frac{Cov(X,Y)}{\sqrt{V[X]}\sqrt{V[Y]}}
      \end{equation}


      よって、
      \begin{equation}
       \rho^2=\frac{(Cov(X,Y))^2}{V[X]V[Y]}\leq 1
      \end{equation}
      であり、分母を払うことで次が得られる。
      \begin{equation}
       (Cov(X,Y))^2 \leq V[X]V[Y]
      \end{equation}


\hrulefill

 \item
      $\sigma >0$を定数とし、
      弱定常時系列$X=\{X_{t};t\in\mathbb{Z}\}$
      はホワイトノイズ$\mathrm{WN}(0,\sigma^2)$
      であるとする。
      時系列$S=\{S_{t};t\in\mathbb{N}\cup\{0\}\}$を
      次のように定義する。
      \begin{equation}
       S_{t}=
        \begin{cases}
         0, & (t=0)\\
         \sum_{j=1}^{t}X_{j}, & (t\in\mathbb{N})
        \end{cases}
      \end{equation}
      \dotfill

      ホワイトノイズであるということなので、
      各$X_{i}\in X$に対して
      $E[X_{i}]=0,\ V[X_{i}]=\sigma^2$
      であり、
      $X_{i},X_{k}\in X\ (i\ne k)$に対して
      $Cov(X_{i},X_{k})=0$である。

      \hrulefill
      \begin{enumerate}
       \item
            各$t\in\mathbb{N}$に対して、
            $S_{t}$の平均を求めよ。

\dotfill

            $t=0$の時、$E[S_0]=E[0]=0$である。

            $t>0$の場合を考える。

            次のように平均を計算できる。
            \begin{equation}
             E[S_{t}]
              =E\left[\sum_{j=1}^{t}X_{j}\right]
              =\sum_{j=1}^{t} E[X_{j}]
              =\sum_{j=1}^{t} 0
              =0
            \end{equation}

            よって、
            ${}^{\forall}S_{t}\in S$に対して
            $E[S_{t}]=0$である。

\hrulefill

       \item
            各$t\in\mathbb{N}$に対して、
            $S_{t}$の分散を求めよ。

\dotfill

            $t=0$の時、$V[S_0]=V[0]=0$である。

            $t>0$の場合を考える。

            $S_{t}$の分散$V[S_{t}]$を計算する。
            \begin{equation}
             V[S_{t}]
              =V[S_{t-1}+X_{t}]
              =V[S_{t-1}]+V[X_{t}]+2Cov(S_{t-1},X_{t})
              \label{V01}
            \end{equation}

            ここで、$Cov(S_{t-1},X_{t})$は
            \begin{equation}
             Cov(S_{t-1},X_{t})
              =Cov(X_{1},X_{t})+Cov(X_{2},X_{t})
              + \cdots +Cov(X_{t-1},X_{t})
            \end{equation}
            と分解できるが、ホワイトノイズであるので
            共分散は全て0となる。

            これより、
            式(\ref{V01})は
            $V[S_{t}]=V[S_{t-1}]+V[X_{t}]$
            となる。
            これを繰り返し行うと次の式が得られる。
            \begin{equation}
             V[S_{t}]=\sum_{j=1}^{t}V[X_{j}]
            \end{equation}

            ホワイトノイズであるので
            全ての分散は$V[X_{j}]=\sigma^2$である。
            これにより分散は
            $V[S_{t}]=t\sigma^2$
            となる。

\hrulefill

       \item
            各$t,k\in\mathbb{N}$に対して、
            $S_{t+k}$と$S_{t}$の共分散を求めよ。

\dotfill

            $k=0$の時、
            $Cov(S_{t},S_{t})=V[S_{t}]=t\sigma^2$である。

            $k>0$について考える。

            $k=1$の場合次のような変形が行える。
            \begin{equation}
             Cov(S_{t+1},S_{t})
              =Cov(S_{t}+X_{t+1},S_{t})
              =Cov(S_{t},S_{t}) + Cov(X_{t+1},S_{t})
            \end{equation}

            $Cov(X_{t+1},S_{t})$は
            \begin{equation}
             Cov(X_{t+1},S_{t})
              =Cov(X_{t+1},X_{1})+\cdots +Cov(X_{t+1},X_{t})
              =0
            \end{equation}
            となる。
            その為、次のように共分散が求まる。
            \begin{equation}
             Cov(S_{t+1},S_{t})
              =Cov(S_{t},S_{t})
              =t\sigma^2
            \end{equation}

            $k>1$の場合、
            次のように$S_{t+k}$を$S_{t+k-1}$にすることができる。
            \begin{align}
             Cov(S_{t+k},S_{t})
              =& Cov(S_{t+k-1}+X_{t+k},S_{t})\\
              =& Cov(S_{t+k-1},S_{t}) + Cov(X_{t+k},S_{t})\\
              =& Cov(S_{t+k-1},S_{t})
            \end{align}


            よって、
            $S_{t+k}$と$S_{t}$の共分散は
            $Cov(S_{t+k},S_{t})=t\sigma^2$
            となる。

\hrulefill
       \item
            時系列 $\{\nabla S_{t} ; t\in\mathbb{N}\cup\{0\} \}$
            が弱定常性を持つかどうか説明せよ。

\dotfill


\hrulefill
      \end{enumerate}

 \item
      $p$を$0$以上の整数とし、
      時系列$X=\{X_{t};t\in\mathbb{Z}\}$の
      トレンド成分が次のように与えられているとする。
      \begin{equation}
       m(t)=\sum_{r=0}^{p}b_{r}t^{r}
      \end{equation}

      各$a_{-2},\dots,a_{2}$を次のようにする。
      \begin{equation}
       a_{0}=\frac{1}{4},\
       a_{-1}=a_{1}=\frac{1}{4},\
       a_{-2}=a_{2}=\frac{1}{8}
      \end{equation}
      
      この時、
      次の式が成り立つような$p$を求めよ。
      \begin{equation}
       \sum_{k=-2}^{2}a_{k}m(t+k)=m(t)
      \end{equation}
\dotfill

      $\sum$記号を展開し分数をまとめると次のような形になる。
      \begin{equation}
       \sum_{k=-2}^{2}a_{k}m(t+k)
        =\frac{1}{8}( m(t-2)+2m(t-1)+2m(t)+2m(t+1)+m(t+2) )
      \end{equation}

      これが$m(t)$と等しいため、
      次のように分母を払い移行し式を整理する。
      \begin{align}
       m(t) =& \frac{1}{8}( m(t-2)+2m(t-1)+2m(t)+2m(t+1)+m(t+2) )\\
       0 =& m(t-2)+2m(t-1)-6m(t)+2m(t+1)+m(t+2) \\
        =& \sum_{r=0}^{p}b_{r}(t-2)^{r}+2\sum_{r=0}^{p}b_{r}(t-1)^{r}\\
       & -6\sum_{r=0}^{p}b_{r}t^{r}
       +2\sum_{r=0}^{p}b_{r}(t+1)^{r}
       +\sum_{r=0}^{p}b_{r}(t+2)^{r} \\
       =& \sum_{r=0}^{p}b_{r}( (t-2)^r +2(t-1)^r -6t^r +2(t+1)^r +(t+2)^r )
      \end{align}

      $b_{r}$が定まっていないためこれの係数
      $( (t-2)^r +2(t-1)^r -6t^r +2(t+1)^r +(t+2)^r )$
      が$0$となるような$r$を求める。


      \begin{align}
       (t-2)^0 +2(t-1)^0 -6t^0 +2(t+1)^0 +(t+2)^0
       =& 0\\
       (t-2)^1 +2(t-1)^1 -6t^1 +2(t+1)^1 +(t+2)^1
       =& 0\\
       (t-2)^2 +2(t-1)^2 -6t^2 +2(t+1)^2 +(t+2)^2
       =& 12\\
       (t-2)^3 +2(t-1)^3 -6t^3 +2(t+1)^3 +(t+2)^3
       =& 36t\\
       (t-2)^4 +2(t-1)^4 -6t^4 +2(t+1)^4 +(t+2)^4
       =& 72t^2 + 36
      \end{align}

      これにより
      $p=0,1$の時
      $\sum_{k=-2}^{2}a_{k}m(t+k)-m(t) =0$
      であり、
      $p=2$のとき、
      $\sum_{k=-2}^{2}a_{k}m(t+k)-m(t) =12b_{2}$
      である。

      $p>2$の時
      $\sum_{k=-2}^{2}a_{k}m(t+k)-m(t)$は
      $t$の多項式となり
      $0$にならない。

      よって、
      $\sum_{k=-2}^{2}a_{k}m(t+k)=m(t)$
      を満たすのは$p=0,\ p=1$の時となる。

\hrulefill



\end{enumerate}

sagemath 用コード

\url{https://sagecell.sagemath.org/}
\begin{lstlisting}
x = var('x')  # 変数x の定義
t = var('t')  # 変数t の定義

# 変数x,t の多項式を定義
pol(x,t) = (t-2)^x + 2*(t-1)^x - 6*t^x + 2*(t+1)^x + (t+2)^x

# 変数x を0から増やしながら式を展開し出力
for i in range(7):
    print("r=", i, " ",expand(pol(i,t)))
\end{lstlisting}


\end{document}
