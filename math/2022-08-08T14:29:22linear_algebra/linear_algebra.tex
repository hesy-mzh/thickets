\documentclass[12pt,b5paper]{ltjsarticle}

%\usepackage[margin=15truemm, top=5truemm, bottom=5truemm]{geometry}
\usepackage[margin=10truemm]{geometry}

\usepackage{amsmath,amssymb}
%\pagestyle{headings}
\pagestyle{empty}

%\usepackage{listings,url}
%\renewcommand{\theenumi}{(\arabic{enumi})}

\usepackage{graphicx}

\usepackage{tikz}
\usetikzlibrary {arrows.meta}
\usepackage{wrapfig}	% required for `\wrapfigure' (yatex added)
\usepackage{bm}	% required for `\bm' (yatex added)

% ルビを振る
%\usepackage{luatexja-ruby}	% required for `\ruby'

%% 核Ker 像Im Hom を定義
%\newcommand{\Img}{\mathop{\mathrm{Im}}\nolimits}
%\newcommand{\Ker}{\mathop{\mathrm{Ker}}\nolimits}
%\newcommand{\Hom}{\mathop{\mathrm{Hom}}\nolimits}

%\DeclareMathOperator{\Rot}{rot}
%\DeclareMathOperator{\Div}{div}
%\DeclareMathOperator{\Grad}{grad}
%\DeclareMathOperator{\arcsinh}{arcsinh}
%\DeclareMathOperator{\arccosh}{arccosh}
%\DeclareMathOperator{\arctanh}{arctanh}



\begin{document}

\hrulefill
\textbf{線形代数}
\hrulefill

\begin{enumerate}
 \item
      $x,y\in\mathbb{Z}$が$x^2 + y^2 \leq 25$を満たすとする。
      この時、次の3つのベクトルが一次独立となる$x,y$の組$(x,y)$の個数を求めよ。
      ただし、$(x,y)$と$(y,x)$は別の組として数えるものとする。
     \begin{equation}
      \begin{pmatrix}x\\y\\y-4\end{pmatrix}
      ,\qquad
      \begin{pmatrix}y\\x-3\\-4\end{pmatrix}
      ,\qquad
      \begin{pmatrix}x+1\\y\\y-5\end{pmatrix}
     \end{equation}

      \dotfill

      $x^2 + y^2 \leq 25$を満たす整数の組$(x,y)$は次の26個である。
      \begin{center}
       \begin{tabular}{c||cccccc}
        $x\ \backslash\ y$ & 0 & 1 & 2 & 3 & 4 & 5 \\
        \hline\hline
        0 & $(0,0)$ & $(0,1)$ & $(0,2)$ & $(0,3)$ & $(0,4)$ & $(0,5)$ \\
        1 & $(1,0)$ & $(1,1)$ & $(1,2)$ & $(1,3)$ & $(1,4)$ & \\
        2 & $(2,0)$ & $(2,1)$ & $(2,2)$ & $(2,3)$ & $(2,4)$ & \\
        3 & $(3,0)$ & $(3,1)$ & $(3,2)$ & $(3,3)$ & $(3,4)$ & \\
        4 & $(4,0)$ & $(4,1)$ & $(4,2)$ & $(4,3)$ & & \\
        5 & $(5,0)$ & & & & &
       \end{tabular}
      \end{center}

      \dotfill

      3つのベクトル$v_1,v_2,v_3$を次のように定める。
      \begin{equation}
       v_1 = \begin{pmatrix}x\\y\\y-4\end{pmatrix}
      ,\
      v_2 = \begin{pmatrix}y\\x-3\\-4\end{pmatrix} - 4v_3
      = \begin{pmatrix}y-4\\x-3\\0\end{pmatrix}
      ,\
      v_3 = \begin{pmatrix}x+1\\y\\y-5\end{pmatrix} - v_1%\begin{pmatrix}x\\y\\y-4\end{pmatrix}
      = \begin{pmatrix}1\\0\\-1\end{pmatrix}
      \end{equation}

%      \begin{equation}
%       \begin{pmatrix}
%        x & y-4 & 1\\
%        y & x-3 & 0\\
%        y-4 & 0 & -1
%       \end{pmatrix}
%       \begin{pmatrix} a_1 \\ a_2 \\ a_3 \end{pmatrix}
%       =
%        \begin{pmatrix} 0 \\ 0 \\ 0 \end{pmatrix}
%      \end{equation}

      この3つのベクトルを並べた行列$A$を次のようにおく。
      \begin{equation}
       A=(v_3, v_1, v_2)=
       \begin{pmatrix}
        1 & x & y-4 \\
        0 & y & x-3 \\
        -1 & y-4 & 0
       \end{pmatrix}
      \end{equation}

      3つのベクトルが一次従属である場合は
      $\det A = 0$であるので、
      この時の$(x,y)$を求める。
      \begin{equation}
       \det A = y(y-4) - (x+y-4)(x-3) = 0
      \end{equation}
      移項をすると
      $y(y-4)=(x+y-4)(x-3)$である。

      \begin{itemize}
       \item
            $y=0$の時、$0\leq x \leq 5$の範囲で$0=(x-4)(x-3)$を満たす整数は$x=3,4$である。
       \item
            $y=1$の時、$0\leq x \leq 4$の範囲で$-3=(x-3)^2$を満たす整数はない。
       \item
            $y=2$の時、$0\leq x \leq 4$の範囲で$-4=(x-2)(x-3)$を満たす整数はない。
       \item
            $y=3$の時、$0\leq x \leq 4$の範囲で$-3=(x-1)(x-3)$を満たす整数はない。
       \item
            $y=4$の時、$0\leq x \leq 3$の範囲で$0=x(x-3)$を満たす整数は$x=0,3$である。
       \item
            $y=5$の時、$0\leq x \leq 0$の範囲で$5=(x+1)(x-3)$を満たす整数はない。
      \end{itemize}

      以上より次の組が一次従属となる。
      \begin{equation}
       (x,y)=(3,0),\ (4,0),\ (0,4),\ (3,4)
      \end{equation}

      これにより一次独立となる$(x,y)$は$22$個であることがわかる。





      \hrulefill

 \item
      $n\in\mathbb{N}$が$n\geq2$とする。
      $i,j\in\mathbb{N}$はそれぞれ$1\leq i,j \leq n$で$i\ne j$とし、
      $c$は定数とする。
      $n$次単位行列$E_n$の$i$行の$c$倍を$j$行に足すことで
      得られる行列を$P$とし、
      $E_n$の$i$列の$c$倍を$j$列に足すことで
      得られる行列を$Q$とする。

      この時、$\det P$と$\det Q$を求めよ。

      \dotfill

      $n$次正方行列である成分のみ$c$でそれ以外が$0$の行列を$R(c)$と書くと、
      $P,Q$は次のように表せる。
      \begin{equation}
       E_n=
        \begin{pmatrix}
         1 & 0 & \cdots & 0\\
         0 & 1 & \ddots & \vdots\\
         \vdots & \ddots & \ddots& 0\\
         0 & \cdots & 0 & 1
        \end{pmatrix}
        ,\qquad
        P=E_n + R(c)
        ,\quad
        Q=E_n + R(c)
      \end{equation}

      $\det(R(c))=0$より行列式は次のようになる。
      \begin{align}
       \det P = \det(E_n + R(c)) = \det(E_n) + \det(R(c)) =1\\
       \det Q = \det(E_n + R(c)) = \det(E_n) + \det(R(c)) =1
      \end{align}

      \hrulefill

 \item
      3つの$n$次正方行列$A,B,C$に対し、
      次の行列式の等式を証明せよ。
      \begin{equation}
       \begin{vmatrix}
        A+B+C & -A+B+2C & B+2C\\
        A+B & B+C & B+C\\
        A & -A+C & C
       \end{vmatrix}
       =
       (-1)^n \det(A) \det(A-C) \det(B+C)
      \end{equation}

      \dotfill

      1段目は$n$行あるのでこれを$-1$倍した為、行列式に$(-1)^n$をかける。(\ref{1st})

      1段目に2段目と3段目を加える。(\ref{2nd})

      零行列を含んでいる為、2つの行列式の積に分ける。(\ref{3rd})

      後ろの行列式の列方向の1段を2段目から引く。(\ref{4th})

      各行列式の積になる。
      \begin{align}
       & \begin{vmatrix}
        A+B+C & -A+B+2C & B+2C\\
        A+B & B+C & B+C\\
        A & -A+C & C
       \end{vmatrix}\\
       =&
       (-1)^n
       \begin{vmatrix}
        -(A+B+C) & -(-A+B+2C) & -(B+2C)\\
        A+B & B+C & B+C\\
        A & -A+C & C
       \end{vmatrix}\label{1st}\\
       =&
       (-1)^n
       \begin{vmatrix}
        A-C & 0 & 0\\
        A+B & B+C & B+C\\
        A & -A+C & C
       \end{vmatrix}\label{2nd}\\
       =&
       (-1)^n\det(A-C)
       \begin{vmatrix}
        B+C & B+C\\
        -A+C & C
       \end{vmatrix}\label{3rd}\\
       =&
       (-1)^n\det(A-C)
       \begin{vmatrix}
        B+C & 0\\
        -A+C & A
       \end{vmatrix}\label{4th}\\
       =&
       (-1)^n\det(A-C)\det(B+C)\det(A)
      \end{align}



      \hrulefill

 \item
      次の正方行列の固有値を求めよ。
      対角化可能であれば対角行列とその正則行列を求めよ。
      \begin{equation}
       A=
       \begin{pmatrix}
        2 & 1 & 1\\
        0 & 3 & 1\\
        4 & -4 & 3
       \end{pmatrix}
       ,\qquad
       B=
       \begin{pmatrix}
        1+\sqrt{3} & -1 & 1\\
        2 & \sqrt{3} & 1\\
        3-\sqrt{3} & 1 & 3
       \end{pmatrix}
      \end{equation}

      \dotfill

      行列$A$の固有値を$\lambda$とすると固有方程式は次のようになる。
      \begin{equation}
       \det(A-\lambda E) =0
      \end{equation}
      この式を計算すると$(\lambda -2)(\lambda -3)^2=0$となるので、
      固有値は$\lambda = 2,3$の2つである。

      それぞれの固有値に対応した固有ベクトルを求める。
      固有ベクトル$\bm{x}$は$A\bm{x}=\lambda\bm{x}$を満たすベクトルであるので、
      $(A-\lambda E)\bm{x}=0$を満たすベクトルとして求める。

      $\lambda = 2$の場合、$\bm{x}= k_1 \begin{pmatrix} -5 \\ -4 \\ 4 \end{pmatrix}$である。($k_1$は定数)

      $\lambda = 3$の場合、$\bm{x}= k_2 \begin{pmatrix} 1 \\ 1 \\ 0 \end{pmatrix}$である。($k_2$は定数)

      3次正方行列$A$に対して固有ベクトルが2つしか無い為、$A$は対角化は出来ない。

      同様に$B$について考える。
      \begin{equation}
       \det(B-\lambda E) = -(\lambda -4)(\lambda^2 -2\sqrt{3}\lambda +4) =0
      \end{equation}
      これを解くと$\lambda = 4, \sqrt{3}\pm i$である。

      それぞれの固有値$\lambda$に対し固有ベクトル$\bm{x}$を求める。

      $\lambda = 4$の場合、$\bm{x}= k_1 \begin{pmatrix} 15+2\sqrt{3} \\ 32+9\sqrt{3} \\ 71 \end{pmatrix}$である。($k_1$は定数)

      $\lambda = \sqrt{3}+i$の場合、$\bm{x}= k_2 \begin{pmatrix} 1 \\ -i \\ -1 \end{pmatrix}$である。($k_2$は定数)

      $\lambda = \sqrt{3}-i$の場合、$\bm{x}= k_3 \begin{pmatrix} 1 \\ i \\ -1 \end{pmatrix}$である。($k_3$は定数)

      固有ベクトルが3つ得られたので、
      これを用いて正則行列$P$を作る。
      $P$を利用することで
      $B$は対角化可能である。
      \begin{equation}
       P=
        \begin{pmatrix}
         15+2\sqrt{3} & 1 & 1 \\
         32+9\sqrt{3} & -i & i \\
         71 & -1 & -1
        \end{pmatrix}
      ,\qquad
      P^{-1}BP =
        \begin{pmatrix}
         4 & 0 & 0 \\
         0 & \sqrt{3}+i & 0 \\
         0 & 0 & \sqrt{3}-i
        \end{pmatrix}
      \end{equation}


\end{enumerate}







\end{document}

