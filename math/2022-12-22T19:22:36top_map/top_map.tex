\documentclass[12pt,b5paper]{ltjsarticle}

%\usepackage[margin=15truemm, top=5truemm, bottom=5truemm]{geometry}
%\usepackage[margin=10truemm,left=15truemm]{geometry}
\usepackage[margin=10truemm]{geometry}

\usepackage{amsmath,amssymb}
%\pagestyle{headings}
\pagestyle{empty}

%\usepackage{listings,url}
%\renewcommand{\theenumi}{(\arabic{enumi})}

%\usepackage{graphicx}

%\usepackage{tikz}
%\usetikzlibrary {arrows.meta}
%\usepackage{wrapfig}	% required for `\wrapfigure' (yatex added)
%\usepackage{bm}	% required for `\bm' (yatex added)

% ルビを振る
%\usepackage{luatexja-ruby}	% required for `\ruby'

%% 核Ker 像Im Hom を定義
%\newcommand{\Img}{\mathop{\mathrm{Im}}\nolimits}
%\newcommand{\Ker}{\mathop{\mathrm{Ker}}\nolimits}
%\newcommand{\Hom}{\mathop{\mathrm{Hom}}\nolimits}

%\DeclareMathOperator{\Rot}{rot}
%\DeclareMathOperator{\Div}{div}
%\DeclareMathOperator{\Grad}{grad}
%\DeclareMathOperator{\arcsinh}{arcsinh}
%\DeclareMathOperator{\arccosh}{arccosh}
%\DeclareMathOperator{\arctanh}{arctanh}



%\usepackage{listings,url}
%
%\lstset{
%%プログラム言語(複数の言語に対応,C,C++も可)
%  language = Python,
%%  language = Lisp,
%%  language = C,
%  %背景色と透過度
%  %backgroundcolor={\color[gray]{.90}},
%  %枠外に行った時の自動改行
%  breaklines = true,
%  %自動改行後のインデント量(デフォルトでは20[pt])
%  breakindent = 10pt,
%  %標準の書体
%%  basicstyle = \ttfamily\scriptsize,
%  basicstyle = \ttfamily,
%  %コメントの書体
%%  commentstyle = {\itshape \color[cmyk]{1,0.4,1,0}},
%  %関数名等の色の設定
%  classoffset = 0,
%  %キーワード(int, ifなど)の書体
%%  keywordstyle = {\bfseries \color[cmyk]{0,1,0,0}},
%  %表示する文字の書体
%  %stringstyle = {\ttfamily \color[rgb]{0,0,1}},
%  %枠 "t"は上に線を記載, "T"は上に二重線を記載
%  %他オプション:leftline,topline,bottomline,lines,single,shadowbox
%  frame = TBrl,
%  %frameまでの間隔(行番号とプログラムの間)
%  framesep = 5pt,
%  %行番号の位置
%  numbers = left,
%  %行番号の間隔
%  stepnumber = 1,
%  %行番号の書体
%%  numberstyle = \tiny,
%  %タブの大きさ
%  tabsize = 4,
%  %キャプションの場所("tb"ならば上下両方に記載)
%  captionpos = t
%}



\begin{document}

\begin{enumerate}
 \item
      -- 連続全射な開写像 --

      連続な全射$f$が、
      開写像もしくは閉写像であるなら
      $f$は商写像であることを示せ。

      \dotfill

      $(X,\mathcal{O}_X),(Y,\mathcal{O}_Y)$を
      位相空間とし連続な全射を$f:X\to Y$とする。

%      $f$は連続写像であるので、
%      $O\in\mathcal{O}_Y \Rightarrow f^{-1}(O)\in\mathcal{O}_X$
%      である。
      $f$が開写像であれば
      $O\in\mathcal{O}_X \Rightarrow f(O)\in\mathcal{O}_Y$
      である。

      任意の集合$S\subset Y$に対して、
      $f^{-1}(S)\in\mathcal{O}_X$とする。
      $f$は開写像であるので、
      $f(f^{-1}(S))\in\mathcal{O}_Y$
      であるが$f$は全射であるので
      $f(f^{-1}(S))=S$となる。

      つまり、
      $f^{-1}(S)\in\mathcal{O}_X\Rightarrow S\in\mathcal{O}_Y$
      であるので、$f$は商写像である。

      $f$が閉写像であれば
      $C$が閉集合なら$f(C)$も閉集合となる。

      任意の集合$S\subset Y$に対して、
      $f^{-1}(S)$を閉集合とする。
      $f$は閉写像であるので、
      $f(f^{-1}(S))$は閉集合となるが
      $f$は全射であるので
      $f(f^{-1}(S))=S$となる。

      つまり、
      $f^{-1}(S)$が閉集合であれば
      $S$も閉集合となる為、
      $f$は商写像である。


      \hrulefill

 \item
      -- 商空間 --

      $(X,\mathcal{O})$を位相空間とし、
      $f:X\to Y$を全射とする。
      このとき、
      $(Y,\mathcal{O}_{Y})$を$f$における商空間、
      つまり$\mathcal{O}_{Y}=\{ U\subset Y \mid f^{-1}(U)\in\mathcal{O}\}$
      とすると、
      $(Y,\mathcal{O}_{Y})$は$f$を連続にする最強の位相であることを示せ。

      \dotfill

      $f$を連続とするような$Y$の任意の位相を$\mathcal{T}$とする。
      $G\in\mathcal{T}$であれば$f^{-1}(G)\in\mathcal{O}$であるため、
      $G\in\mathcal{O}_{Y}$となる。

      よって、$\mathcal{T}\subset\mathcal{O}_{Y}$となり、
      $\mathcal{O}_{Y}$が最も強い位相であることがわかる。

      \hrulefill
 \item
      -- 商写像 --

      商写像は連続であることを示せ。

      \dotfill

      写像$f:X\to Y$が商写像であるとは
      $f$は全射であり、$Y$の位相$\mathcal{O}_{Y}$が
      商位相となるときをいう。

      $\mathcal{O}_{Y}=\{ U\subset Y \mid f^{-1}(U)\in\mathcal{O}\}$
      であるので、
      \begin{equation}
       S\subset Y \text{が開集合}
        \Leftrightarrow
       f^{-1}(S)\subset X \text{が開集合}
      \end{equation}
      である。

      よって、$f$は連続写像である。

      \hrulefill

\end{enumerate}


\end{document}
