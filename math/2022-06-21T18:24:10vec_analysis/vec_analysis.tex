\documentclass[12pt,b5paper]{ltjsarticle}

%\usepackage[margin=15truemm, top=5truemm, bottom=5truemm]{geometry}
\usepackage[margin=15truemm]{geometry}

\usepackage{amsmath,amssymb}
%\pagestyle{headings}
\pagestyle{empty}

%\usepackage{listings,url}
%\renewcommand{\theenumi}{(\arabic{enumi})}

\usepackage{graphicx}

\usepackage{tikz}
\usetikzlibrary {arrows.meta}
\usepackage{wrapfig}	% required for `\wrapfigure' (yatex added)
\usepackage{bm}	% required for `\bm' (yatex added)
\usepackage{luatexja-ruby}	% required for `\ruby'
%% 像Im を定義
%\newcommand{\Img}{\mathop{\mathrm{Im}}\nolimits}

\begin{document}

\hrulefill

$a<b,\,c<d$とする。
写像$\phi:[a,b]\to[c,d]$が次の条件を満たすとする。
\begin{enumerate}
 \item $\phi$は全単射
 \item $\phi$は$C^1$-級
 \item 逆写像$\phi^{-1}$も$C^1$-級
 \item ${}^{\forall}t\in[a,b]$において$\phi^{\prime}(t)>0$
\end{enumerate}
この時、$\phi(a)=c$かつ$\phi(b)=d$であることを示せ。

\dotfill

$\phi$と$\phi^{-1}$は$C^1$-級である為、連続である。
$\phi^\prime(t)>0$より$\phi$は単調増加である。

これにより
$\alpha,\beta \in [a,b]$において
$\alpha < \beta \Rightarrow \phi(\alpha) < \phi(\beta)$
である。

$\phi$は全単射であるから
$c,d\in [c,d]$に対応する点が$[a,b]$にだた一つだけ存在する。

この為、
$\phi(a)=c,\,\phi(b)=d$であることが分かる。

\hrulefill

写像$C_1,\,C_2$を次のように定める。
\begin{align}
 C_1:& [1,4] \to \mathbb{R}^2 \quad t \mapsto (t,\sin t)\\
 C_2:& [1,2] \to \mathbb{R}^2 \quad t \mapsto (t^2,\sin t^2)
\end{align}

この時、$C_1,\,C_2$は向きまで込めて$C^1$-級同値となることを示せ。

\dotfill

写像$f:[1,2]\to[1,4]$を$f(x)=x^2$とすると、
$f$は全単射であり、$C^{\infty}$-級である。

$C_2$は$C_1$と$f$の合成関数である。
つまり、$C_2=C_1\circ f$である。
$f$は単調増加であるので$C_1,\,C_2$の向きは同じとなる。

$(t,\sin t)^\prime = (1,\cos t)$であるので、
$C_1$は$C^1$-級である。
$C_2=C_1\circ f$であり、$f$は$C^\infty$-級であるので、
$C_2$は$C^1$-級である。

\hrulefill

$C:[a,b]\to\mathbb{R}^n$を$C^1$-級曲線とし、
$\check{C}:[-b,-a]\to\mathbb{R}^n$を$C$の逆向きの曲線とする。
この時、
\begin{equation}
 \int_{(C,\check{C})}\bm{f} =0
\end{equation}
となることを示せ。

\dotfill

$\check{C}$が$C$と逆向きであるので、
${}^{\forall}t\in [-b,-a]$において、
$\check{C}(t)=C(-t)$となる。
この為、
\begin{equation}
 \int_C \bm{f} = -\int_{\check{C}} \bm{f}
\end{equation}
となる。
これにより次のように積分値が$0$となる。
\begin{equation}
 \int_{(C,\check{C})}\bm{f} = \int_C \bm{f} + \int_{\check{C}} \bm{f}
  = -\int_{\check{C}} \bm{f} + \int_{\check{C}} \bm{f} =0
\end{equation}



\end{document}
