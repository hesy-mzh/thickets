\documentclass[12pt,b5paper]{ltjsarticle}

%\usepackage[margin=15truemm, top=5truemm, bottom=5truemm]{geometry}
\usepackage[margin=15truemm]{geometry}

\usepackage{amsmath,amssymb}
\pagestyle{empty}

\usepackage{listings,url}

\usepackage{bm}	% required for `\bm' (yatex added)
\begin{document}

反復法を用いて、次の連立1次方程式の近似解を与えるアルゴリズムを考察する。
\begin{equation}
 \begin{cases}
  5x_1 - x_2 =& 4\\
 x_1 + 5x_2 =& 6
 \end{cases}\label{154004_21Dec21}
\end{equation}
\dotfill

連立方程式は次のように行列で表現できます。
\begin{equation}
 \begin{pmatrix}
  5 & -1\\ 1 & 5
 \end{pmatrix}
 \begin{pmatrix}
  x_1\\ x_2
 \end{pmatrix}
 =
 \begin{pmatrix}
  4\\6
 \end{pmatrix}
\end{equation}

それぞれの行列を次のようにし、
連立方程式は$A\bm{x}=\bm{b}$となるようにします。
\begin{equation}
 A=
  \begin{pmatrix}
   5 & -1\\ 1 & 5
  \end{pmatrix}
  , \quad
  \bm{x}=
  \begin{pmatrix}
   x_1\\ x_2
  \end{pmatrix}
  , \quad
  \bm{b}=
  \begin{pmatrix}
   4\\6
  \end{pmatrix}
\end{equation}

また、行列$A$を分解し次のようにおきます。
\begin{equation}
 L=
  \begin{pmatrix}
   0 & 0\\ 1 & 0
  \end{pmatrix}
  , \quad
 D=
  \begin{pmatrix}
   5 & 0\\ 0 & 5
  \end{pmatrix}
  , \quad
 U=
  \begin{pmatrix}
   0 & -1\\ 0 & 0
  \end{pmatrix}
\end{equation}

この時、$A=L+D+U$であり、
$A\bm{x}=(L+D+U)\bm{x}=\bm{b}$となります。

\hrulefill
\begin{enumerate}\renewcommand{\theenumi}{(\roman{enumi})}
 \item 方程式(\ref{154004_21Dec21})に対して、ガウス・ザイデルの反復法による反復過程を与えよ。
       \label{153528_21Dec21}

       \dotfill

       $(L+D+U)\bm{x}=\bm{b}$より次のように変形できます。
       \begin{align}
        (L+D+U)\bm{x}=&\bm{b}\\
        (L+D)\bm{x}+U\bm{x}=&\bm{b}\\
        (L+D)\bm{x}=&-U\bm{x}+\bm{b}
       \end{align}

       この時、右辺の$\bm{x}$を$k+1$回目の近似解$\bm{x}^{(k+1)}$を、
       左辺の$\bm{x}$は$k$回目の近似解$\bm{x}^{(k)}$を表すとして
       書き直すと次のようになります。
       \begin{align}
        (L+D)\bm{x}^{(k+1)}=&-U\bm{x}^{(k)}+\bm{b}\\
        \bm{x}^{(k+1)}=& -(L+D)^{-1}U\bm{x}^{(k)}+(L+D)^{-1}\bm{b}\label{213753_21Dec21}
       \end{align}
       行列$L+D$は三角行列なので$\det D\ne0$であれば逆行列が存在します。

       $\bm{x}^{(k+1)}={}^{t}(x_1^{(k+1)},x_2^{(k+1)}),
       \bm{x}^{(k)}={}^{t}(x_1^{(k)},x_2^{(k)})$として、
       上の式を計算します。
       \begin{align}
        \begin{pmatrix}x_1^{(k+1)}\\x_2^{(k+1)}\end{pmatrix}
        =&
        - \begin{pmatrix} 5 & 0 \\ 1 & 5 \end{pmatrix}^{-1}
        \begin{pmatrix}0&-1\\0&0\end{pmatrix}
        \begin{pmatrix}x_1^{(k)}\\x_2^{(k)}\end{pmatrix}
        +\begin{pmatrix} 5 & 0 \\ 1 & 5 \end{pmatrix}^{-1}
        \begin{pmatrix}4\\6\end{pmatrix}\\
        =& -\begin{pmatrix} \frac{1}{5} & 0 \\ -\frac{1}{25} & \frac{1}{5} \end{pmatrix}
        \begin{pmatrix}-x_2^{(k)}\\0\end{pmatrix}
        +\begin{pmatrix} \frac{1}{5} & 0 \\ -\frac{1}{25} & \frac{1}{5} \end{pmatrix}
        \begin{pmatrix}4\\6\end{pmatrix}\\
        =& \begin{pmatrix} \frac{1}{5}x_2^{(k)} \\ -\frac{1}{25}x_2^{(k)} \end{pmatrix}
        +\begin{pmatrix} \frac{4}{5} \\ \frac{26}{25} \end{pmatrix}
        = \begin{pmatrix}
          \frac{1}{5}x_2^{(k)} + \frac{4}{5} \\
          -\frac{1}{25}x_2^{(k)} +\frac{26}{25}
          \end{pmatrix}
       \end{align}

       これにより次の式が得られます。
       \begin{equation}
        x_1^{(k+1)}=\frac{1}{5}x_2^{(k)} + \frac{4}{5}
         , \quad
         x_2^{(k+1)}=-\frac{1}{25}x_2^{(k)} +\frac{26}{25}\label{192158_21Dec21}
       \end{equation}

       \dotfill

       連立方程式を変形することでも上の式が得られます。
       \begin{equation}
        5x_1-x_2=4 \quad \longrightarrow \quad x_1=\frac{1}{5}(x_2+4)
       \end{equation}
       この変形した式の左辺を$k+1$回目の近似値$x_1^{(k+1)}$とし、
       右辺を$k$回目$x_2^{(k)}$とします。
       これにより次の式が得られます。
       \begin{equation}
        x_1^{(k+1)}=\frac{1}{5}(x_2^{(k)}+4)\label{191817_21Dec21}
       \end{equation}

       同様に二つ目の式も変形します。
       \begin{equation}
        x_1 + 5x_2 = 6 \quad \longrightarrow \quad x_2 = \frac{1}{5}(-x_1+6)
       \end{equation}
       左辺を$k+1$回目の近似値$x_2^{(k+1)}$をします。
       右辺は先ほど求めた$k+1$番目の近似値$x_1^{(k+1)}$を使います。
       (ヤコビ法では$k$番目を使います。)
       \begin{equation}
        x_2^{(k+1)} = \frac{1}{5}(-x_1^{(k+1)}+6)\label{191834_21Dec21}
       \end{equation}
       (\ref{191817_21Dec21})と(\ref{191834_21Dec21})の式を使って
       近似解を求めていきます。
       (\ref{191817_21Dec21})の式を(\ref{191834_21Dec21})に代入することで
       (\ref{192158_21Dec21})の式が得られます。

       \hrulefill
       
 \item 初期値を$\bm{x}^{(0)}=\left(x_1^{(0)}, \ x_2^{(0)}\right)=(0,0)$とする時、
       $m=1,2,3$に対して解の近似列 $\bm{x}^{(m)}$を求めよ。(有効数字4桁)

       \dotfill

       (\ref{192158_21Dec21})の式を利用し求めます。
       \begin{equation}
        x_1^{(k+1)}=\frac{1}{5}x_2^{(k)} + \frac{4}{5}
         , \quad
         x_2^{(k+1)}=-\frac{1}{25}x_2^{(k)} +\frac{26}{25}
       \end{equation}

       まず、$x_1^{(1)}, x_2^{(1)}$を計算します。
       \begin{align}
        x_1^{(1)} = \frac{4}{5} = 0.8000\\
        x_2^{(1)} = \frac{26}{25} = 1.040
       \end{align}
       これを利用し$x_1^{(2)}, x_2^{(2)}$を計算します。
       \begin{align}
        x_1^{(2)} = \frac{126}{125} = 1.008\\
        x_2^{(2)} = \frac{624}{625} = 0.9984
       \end{align}
       最後に$x_1^{(3)}, x_2^{(3)}$を計算します。
       \begin{align}
        x_1^{(3)} = \frac{3124}{3125} = 0.99968 \fallingdotseq 0.9996\\
        x_2^{(3)} = \frac{15626}{15625} = 1.000064 \fallingdotseq 1.000
       \end{align}

       \hrulefill
 \item \ref{153528_21Dec21}で与えた反復過程が$\|\cdot\|_{\infty}$について
       収束することを示せ。

       \dotfill

       (\ref{213753_21Dec21})より
       \begin{equation}
        \bm{x}^{(k+1)}= -(L+D)^{-1}U\bm{x}^{(k)}+(L+D)^{-1}\bm{b}
       \end{equation}
       であるが、$M=-(L+D)^{-1}U$とおくと次のようになる。
       \begin{equation}
        \bm{x}^{(k+1)}= M\bm{x}^{(k)}+(L+D)^{-1}\bm{b}
       \end{equation}
       連立方程式の解を$\hat{\bm{x}}$とおくと次のような等号である。
       \begin{equation}
        \hat{\bm{x}}= M\hat{\bm{x}}+(L+D)^{-1}\bm{b}
       \end{equation}
       2つの式の差を考えると
       \begin{equation}
        \hat{\bm{x}}-\bm{x}^{(k+1)}= M\hat{\bm{x}}-M\bm{x}^{(k)}=M(\hat{\bm{x}}-\bm{x}^{(k)})
       \end{equation}
       同様に$k$を一つ減らしても同じような式が得られる。
       \begin{equation}
        \hat{\bm{x}}-\bm{x}^{(k)}=M(\hat{\bm{x}}-\bm{x}^{(k-1)})
       \end{equation}
       これらを繰り返し求め、順に代入していくと次の式が得られます。
       \begin{equation}
        \hat{\bm{x}}-\bm{x}^{(k)}=M^k(\hat{\bm{x}}-\bm{x}^{(0)})
       \end{equation}

       ここで、
       \begin{align}
        M=&-(L+D)^{-1}U\\
             =& - \begin{pmatrix} 5 & 0 \\ 1 & 5 \end{pmatrix}^{-1}
        \begin{pmatrix}0&-1\\0&0\end{pmatrix}\\
             =& - \begin{pmatrix} \frac{1}{5} & 0 \\ -\frac{1}{25} & \frac{1}{5} \end{pmatrix}
        \begin{pmatrix}0&-1\\0&0\end{pmatrix}
           &= \begin{pmatrix}0&\frac{1}{5}\\0&-\frac{1}{25}\end{pmatrix}
       \end{align}
       であり、$\| M \|_{\infty}=\frac{1}{5}<1$となる。
       $\|M^k\|_{\infty} \rightarrow 0 \quad (k\rightarrow\infty)$となるから、
       初期値$\bm{x}^{(0)}$によらず次の様になる。
       \begin{equation}
        \|\hat{\bm{x}}-\bm{x}^{(k)}\|_{\infty}
         =\|M^k(\hat{\bm{x}}-\bm{x}^{(0)})\|_{\infty}
         \rightarrow 0 \quad (k\rightarrow\infty)
       \end{equation}

       この為、$k$が十分に大きい時$\bm{x}^{(k)}$は解$\hat{\bm{x}}$に収束する。



\end{enumerate}



\hrulefill

SageMathCell

\url{https://sagecell.sagemath.org/}

計算用コード
\begin{lstlisting}[language=Python,basicstyle={\small},frame=single,numbers=left]
 a, b = var('a b')

 L=Matrix([[0,0],[1,0]])
 D=Matrix([[5,0],[0,5]])
 U=Matrix([[0,-1],[0,0]])

 x=Matrix([a,b]).transpose()
 y=Matrix([4,6]).transpose()

 x0=Matrix([0,0]).transpose()

 print("==L+D の逆行列==")
 print((L+D).inverse())
 print("==計算式==")
 print((L+D).inverse()*(-U*x+y))
 print("==1回目==")
 x1=(L+D).inverse()*(-U*x0+y)
 print(x1)
 print("==2回目==")
 x2=(L+D).inverse()*(-U*x1+y)
 print(x2)
 print("==3回目==")
 x3=(L+D).inverse()*(-U*x2+y)
 print(x3)
 print("==収束性==")
 M=-(L+D).inverse()*U
 print(M)
\end{lstlisting}

実行結果

\begin{lstlisting}[language=Python,basicstyle={\small},frame=single,numbers=left]
 ==L+D の逆行列==
[  1/5     0]
[-1/25   1/5]
==計算式==
[    1/5*b + 4/5]
[-1/25*b + 26/25]
==1回目==
[  4/5]
[26/25]
==2回目==
[126/125]
[624/625]
==3回目==
[  3124/3125]
[15626/15625]
==収束性==
[    0   1/5]
[    0 -1/25]
\end{lstlisting}


\end{document}
