\documentclass[10pt,b5paper]{ltjsarticle}

\usepackage[margin=15truemm, top=5truemm, bottom=5truemm]{geometry}

\usepackage{amsmath}
\usepackage{amssymb}
\pagestyle{empty}


\begin{document}



\begin{enumerate}
 \item %\hrulefill
       $\mathbb{F}_{13}$における等式$5^{123}=x$をみたす$x$を
       $0,1,2,\dots,12$の中から選べ。

       \dotfill

       写像$f$を次のように自然な対応で定めると、準同型写像である。
       \begin{equation}
        f: \mathbb{Z} \rightarrow \mathbb{Z}/13\mathbb{Z} = \mathbb{F}_{13}
       \end{equation}
       $f(5)=5, f(5^2)=-1=12, f(5^3)=-5=8, f(5^4)=1$となる。
       これを利用すると次のように計算できる。
       \begin{align}
        5^{123}=(5^4)^{30}\times 5^3\\
        f(5^{123})=f(5^3)=8
       \end{align}

       答えは \underline{ $8$ }

       \dotfill

       写像$f$は整数に対し、$13$で割った余りを対応させる写像になっていて、
       その余りを集めた集合$\mathbb{Z}/13\mathbb{Z}=\mathbb{F}_{13}$は
       体となっている。

       $\mathbb{F}_{13}$の中では$-1$と$12$は同じ数字を表し、
       $(-1)^2=1$は$12^2=13\times 11 +1$よりも明らか。

       $f$は準同型であるので
       $f(5^4)=f(5^2)\times f(5^2)=(-1)\times (-1)=1$である。
       $f(5^{123})=f((5^4)^30\times 5^3)=(f(5^4))^{30}\times f(5^3)$
       となるので、答えは計算してもめられる。
       
       \hrulefill
 \item
      $f(x)=x^3 + 2x +2$について以下を確認する。
      \begin{enumerate}
       \item $\mathbb{F}_{3}[x]$の元として既約かどうか。

             \underline{$f(x) \in \mathbb{F}_{3}[x]$は既約}
       \item $\mathbb{F}_{5}[x]$の元として既約かどうか。

             \underline{$f(x) \in \mathbb{F}_{5}[x]$は可約}
             \begin{align}
              x^3 + 2x +2 =& (x-1)(x^2+x+3)\\
              =& (x-1)(x-1)(x+2) = (x+4)^2(x+2)
             \end{align}

       \item $\mathbb{F}_{7}[x]$の元として既約かどうか。

             \underline{$f(x) \in \mathbb{F}_{7}[x]$は可約}
             \begin{align}
              x^3 + 2x +2 =& (x-2)(x^2+2x+6)\\
               =& (x-2)(x-2)(x+4) = (x+5)^2(x+4)
             \end{align}
      \end{enumerate}

      \dotfill

      $f(x)=x^3+2x+2$の$x$に様々な数を代入し、
      $f(a)=0$となる$a$を見つければ
      $x-a$で式を分解できる。
      $f(0)=2, f(1)=5, f(2)=14$である。

      $\mathbb{F}_{5}$上では$f(1)=5=0$であるので、
      $\mathbb{F}_{5}[x]$上では$f(x)$は$(x-1)=(x+4)$で分解できる。

      $\mathbb{F}_{7}$上では$f(2)=14=0$であるので、
      $\mathbb{F}_{7}[x]$上では$f(x)$は$(x-2)=(x+5)$で分解できる。

      \hrulefill
 \item
      $\mathbb{F}_{3}[x]$のモニックな3次多項式の個数を求めよ。
      既約可約は問わない。


      モニックな3次多項式は$x^3+ax^2+bx+c\in\mathbb{F}_{3}[x]$の形をしている。
      $(a,b,c)\in\mathbb{F}_{3}^3$の組合せは$3^3=27$である。
      そこでモニック多項式の個数は\underline{ 27 }である。

      \hrulefill
 \item
      $\mathbb{F}_{2}[x]$の元として既約なものを選べ。
      \begin{enumerate}
       \item $x^5+1$
       \item $x^5+x+1$
       \item $x^5+x^2+1$
       \item $x^5+x^3+1$
       \item $x^5+x^4+1$
      \end{enumerate}
      $\mathbb{F}_2=\{0,1\}$なのでこれらを代入して$0$であれば可約である。
      $x^5+1$のみ$2=0$となるので、これ以外が既約。
\end{enumerate}

\end{document}
