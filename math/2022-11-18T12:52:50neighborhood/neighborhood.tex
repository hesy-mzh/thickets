\documentclass[12pt,b5paper]{ltjsarticle}

%\usepackage[margin=15truemm, top=5truemm, bottom=5truemm]{geometry}
\usepackage[margin=10truemm]{geometry}

\usepackage{amsmath,amssymb}
%\pagestyle{headings}
\pagestyle{empty}

%\usepackage{listings,url}
%\renewcommand{\theenumi}{(\arabic{enumi})}

%\usepackage{graphicx}

%\usepackage{tikz}
%\usetikzlibrary {arrows.meta}
%\usepackage{wrapfig}	% required for `\wrapfigure' (yatex added)
%\usepackage{bm}	% required for `\bm' (yatex added)

% ルビを振る
%\usepackage{luatexja-ruby}	% required for `\ruby'

%% 核Ker 像Im Hom を定義
%\newcommand{\Img}{\mathop{\mathrm{Im}}\nolimits}
%\newcommand{\Ker}{\mathop{\mathrm{Ker}}\nolimits}
%\newcommand{\Hom}{\mathop{\mathrm{Hom}}\nolimits}

%\DeclareMathOperator{\Rot}{rot}
%\DeclareMathOperator{\Div}{div}
%\DeclareMathOperator{\Grad}{grad}
%\DeclareMathOperator{\arcsinh}{arcsinh}
%\DeclareMathOperator{\arccosh}{arccosh}
%\DeclareMathOperator{\arctanh}{arctanh}



%\usepackage{listings,url}
%
%\lstset{
%%プログラム言語(複数の言語に対応,C,C++も可)
%  language = Python,
%%  language = Lisp,
%%  language = C,
%  %背景色と透過度
%  %backgroundcolor={\color[gray]{.90}},
%  %枠外に行った時の自動改行
%  breaklines = true,
%  %自動改行後のインデント量(デフォルトでは20[pt])
%  breakindent = 10pt,
%  %標準の書体
%%  basicstyle = \ttfamily\scriptsize,
%  basicstyle = \ttfamily,
%  %コメントの書体
%%  commentstyle = {\itshape \color[cmyk]{1,0.4,1,0}},
%  %関数名等の色の設定
%  classoffset = 0,
%  %キーワード(int, ifなど)の書体
%%  keywordstyle = {\bfseries \color[cmyk]{0,1,0,0}},
%  %表示する文字の書体
%  %stringstyle = {\ttfamily \color[rgb]{0,0,1}},
%  %枠 "t"は上に線を記載, "T"は上に二重線を記載
%  %他オプション:leftline,topline,bottomline,lines,single,shadowbox
%  frame = TBrl,
%  %frameまでの間隔(行番号とプログラムの間)
%  framesep = 5pt,
%  %行番号の位置
%  numbers = left,
%  %行番号の間隔
%  stepnumber = 1,
%  %行番号の書体
%%  numberstyle = \tiny,
%  %タブの大きさ
%  tabsize = 4,
%  %キャプションの場所("tb"ならば上下両方に記載)
%  captionpos = t
%}



\begin{document}

\hrulefill
\textbf{定義}
\hrulefill

$(X,\mathcal{O})$を位相空間とする。

\textbf{近傍系}

$x\in X$の近傍系$\mathcal{N}(x)$
とは、
$x\in X$の近傍全体の集合族。
\begin{equation}
 \mathcal{N}(x) =
  \{ U \subset X \mid {}^{\exists}O \in\mathcal{O} \ s.t. \ x\in O \subset U \}
\end{equation}


\textbf{内点}

$a\in A$に対して
$a$の近傍$U\in\mathcal{N}(a)$が存在し
$U \subset A$となるとき、
$a$は$A$の内点であるという。


\textbf{触点}

集合$A$に対して
$x\in X$が$A$の触点であるとは
次を満たすときをいう。
\begin{center}
 ${}^{\forall}U\in\mathcal{N}(x)$に対して
 $U\cap A \ne\emptyset$
\end{center}

\textbf{開核、内部}

$A$の全ての内点の集合を開核や内部といい、
$A^{\circ}$と書く。
\begin{equation}
 A^{\circ}
  =
  \{
   a\in A \mid
   {}^{\exists} U \in \mathcal{N}(a) \ s.t. \ U \subset A
  \}
\end{equation}


\textbf{閉包}

$A$の触点全体の集合を$A$の閉包といい
$\bar{A}$と書く。
\begin{equation}
 \bar{A}=
  \{
    x\in X \mid
    {}^{\forall} U\in\mathcal{N}(x),\
    U\cap A \ne\emptyset
  \}
\end{equation}


\hrulefill
\textbf{問題}
\hrulefill


\begin{enumerate}
 \item
      $(X,\mathcal{O})$を位相空間とする。
      部分集合$A\subset X$に対して、
      $x\in A^{\circ}$であることと、
      ${}^{\exists} N \in \mathcal{N}(x)$に対して
      $N\subset A$であることは
      同値であることを示せ。

      \dotfill

      $x\in A^{\circ} \Rightarrow N\subset A$

      $x\in A^{\circ}$より
      $x\in U \subset A$となる
      近傍$U\in\mathcal{N}(x)$が
      存在する。
      よって、
      ${}^{\exists} N \in \mathcal{N}(x)$に対して
      $N\subset A$である。

      $x\in A^{\circ} \Leftarrow N\subset A$

      $N\subset A$となる
      $N \in \mathcal{N}(x)$が存在するとする。

      $N \in \mathcal{N}(x)$であれば、
      $N$は$x$の近傍であり、$x\in N$である。
      これが$N\subset A$となるので
      $x\in N \subset A$となり
      $x$は$A$の内点である。
      よって、
      $x\in A^{\circ}$
      である。

      \hrulefill

 \item
      $(X,\mathcal{O})$を位相空間とする。
      部分集合$A\subset X$に対して、
      $x\in \bar{A}$であることと、
      ${}^{\forall}N \in \mathcal{N}(x)$に対して
      $N\cap A \ne\emptyset$であることは
      同値であることを示せ。

      \dotfill

      $x\in \bar{A} \Rightarrow N\cap A \ne\emptyset$

      $x\in \bar{A}$であるので
      $x$は$A$の触点である。
      触点であれば、
      ${}^{\forall}N \in \mathcal{N}(x)$に対して
      $N\cap A \ne\emptyset$である。

      $x\in \bar{A} \Leftarrow N\cap A \ne\emptyset$

      ${}^{\forall}N \in \mathcal{N}(x)$に対して
      $N\cap A \ne\emptyset$であることは
      $x$が$A$の触点であることを示している。
      よって、
      $x\in\bar{A}$である。


      \hrulefill

 \item
      $(X,\mathcal{O})$を位相空間とする。
      部分集合$A\subset X$に対して、
      収束する$A$の点列$x_n$の収束点$x$は
      $A$の触点であること
      を示せ。

      \dotfill

      $x_n\in A\ (n\in\mathbb{N})$に対して
      $\lim_{n\rightarrow \infty}x_n = x$である。
      この極限は次のように書くことができる。
      \begin{equation}
       {}^{\forall}U\in\mathcal{N}(x),\
        {}^{\exists}N_0\in\mathbb{N}\
        \text{ s.t. }\
        N>N_0 \Rightarrow x_N\in U
      \end{equation}

      つまり、$N>N_0$となる数$N$に対して
      $x_N\in A$は$x_N\in U$である。
      ${}^{\forall}U\in\mathcal{N}(x)$と$A$について
      $x_N\in U\cap A$であるので
      $U\cap A \ne\emptyset$である。

      よって、$x$は$A$の触点である。


      \hrulefill

\end{enumerate}


\end{document}
