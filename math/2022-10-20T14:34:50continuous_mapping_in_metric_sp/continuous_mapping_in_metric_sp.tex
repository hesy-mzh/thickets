\documentclass[12pt,b5paper]{ltjsarticle}

%\usepackage[margin=15truemm, top=5truemm, bottom=5truemm]{geometry}
\usepackage[margin=10truemm]{geometry}

\usepackage{amsmath,amssymb}
%\pagestyle{headings}
\pagestyle{empty}

%\usepackage{listings,url}
%\renewcommand{\theenumi}{(\arabic{enumi})}

%\usepackage{graphicx}

%\usepackage{tikz}
%\usetikzlibrary {arrows.meta}
%\usepackage{wrapfig}	% required for `\wrapfigure' (yatex added)
%\usepackage{bm}	% required for `\bm' (yatex added)

% ルビを振る
%\usepackage{luatexja-ruby}	% required for `\ruby'

%% 核Ker 像Im Hom を定義
%\newcommand{\Img}{\mathop{\mathrm{Im}}\nolimits}
%\newcommand{\Ker}{\mathop{\mathrm{Ker}}\nolimits}
%\newcommand{\Hom}{\mathop{\mathrm{Hom}}\nolimits}

%\DeclareMathOperator{\Rot}{rot}
%\DeclareMathOperator{\Div}{div}
%\DeclareMathOperator{\Grad}{grad}
%\DeclareMathOperator{\arcsinh}{arcsinh}
%\DeclareMathOperator{\arccosh}{arccosh}
%\DeclareMathOperator{\arctanh}{arctanh}



\begin{document}

\hrulefill
\textbf{定義}
\hrulefill

\textbf{距離関数}

集合$X$上の実数値関数$d$が次を満たすとする。
\begin{gather}
 d: X\times X \to \mathbb{R}\\
 d(a,b)\geq 0\\
 d(a,b)= 0 \Leftrightarrow a=b\\
 d(a,b) = d(b,a)\\
 d(a,b) + d(b,c) \geq d(a,c)
\end{gather}
このとき、関数$d$を距離関数という。

集合$X$に距離関数$d$が定義される場合、
この2つの組合せ$(X,d)$を距離空間という。



点と集合の距離について

集合$X$の部分集合$A$について
点$x\in X$と集合$A$の距離$d(x,A)$を
次のように定義する。
\begin{equation}
 d(x,A) = \min \{ d(x,a) \mid a\in A\}
\end{equation}


\textbf{集積点}

位相空間$X$とその部分集合$A$について、
$x\in X$の任意の近傍$U$が
$(A\backslash \{x\})\cap U \ne \emptyset$となるとき
$x\in X$を$A$の集積点という。


\textbf{連続写像}

2つの距離空間$(X,d_X),(Y,d_Y)$上で定義された
写像$f:X\to Y$が連続写像であるとは
${}^{\forall}a\in X$に対して次が成り立つときをいう。
\begin{equation}
 {}^{\forall}\varepsilon>0,\ {}^{\exists}\delta>0\
  s.t.\ x\in X ,\ d_X(a,x)<\delta \Rightarrow d_Y(f(a),f(x))<\varepsilon
\end{equation}




\hrulefill
\textbf{問題}
\hrulefill

\begin{enumerate}
 \item
      距離空間$(X,d)$とその部分集合$A$に於いて、
      $\lvert d(x,A) - d(y,A) \rvert \leq d(x,y)$
      が成り立つことを示せ。

\dotfill

%      三角不等式$d(a,b) + d(b,c) \geq d(a,c)$より
%      $d(a,b) \geq \lvert d(a,c) - d(b,c) \rvert$

      $x,y\in X$とする。
      $d(x,A) = \min \{ d(x,a) \mid a\in A\}$であるので、
      $d(y,A)=d(y,a_y)$となる$a_y\in A$を取ってくる。
      これにより次の三角不等式が成り立つ。
      \begin{equation}
       d(x,y)+d(y,a_y) \geq d(x,a_y)
      \end{equation}

      $d(x,a_y)\geq d(x,A)$であるので、
      上記不等式は次のように書き換えられる。
      \begin{equation}
       d(x,y)+d(y,A) \geq d(x,A)
      \end{equation}

      これにより次の式が得られる。
      \begin{equation}
       d(x,y) \geq d(x,A) - d(y,A)
      \end{equation}

      $x,y$を入れ替えて同様の議論を行うことにより
      次式が得られる。
      \begin{equation}
       d(x,y) \geq d(y,A) - d(x,A)
      \end{equation}

      よって、次が成立する。
      \begin{equation}
       d(x,y) \geq \lvert d(x,A) - d(y,A) \rvert
      \end{equation}

\hrulefill
 \item
      $A=\{1/n \mid n\in\mathbb{N} \}$とするとき、
      $0$は$A$の集積点であることを示せ。

\dotfill

      $A$は距離空間$(\mathbb{R},d)$の部分集合である。

      ${}^{\forall}\varepsilon>0$として、
      $0\in \mathbb{R}$の近傍を$U=(-\varepsilon,\varepsilon)$
      とする。

      $n_\varepsilon \in \mathbb{N}$を
      $n_\varepsilon > \frac{1}{\varepsilon}$
      を満たすように一つ取ってくる。
      これは$\varepsilon > \frac{1}{n_\varepsilon}$である。
      よって、$\frac{1}{n_\varepsilon}\in U$である。

      また、
      $n_\varepsilon \in \mathbb{N}$であるので、
      $\frac{1}{n_\varepsilon} \in A$である。

      $\frac{1}{n_\varepsilon} \ne 0$であるが、
      $\frac{1}{n_\varepsilon} \in A\cap U$である為、
      $0\in\mathbb{R}$は$A$の集積点である。



\hrulefill
 \item
      $(X,d)$を距離空間とする。
      写像$f:X\to\mathbb{R}$が連続であることの必要十分条件は、
      任意の開区間$I=(a,b)\subset\mathbb{R}$の
      逆像$f^{-1}(I)$は$X$の開集合であることを示せ。

\dotfill

      \dotfill
      写像$f$が連続
      $\Rightarrow$
      $f^{-1}(I)$が開集合
      \dotfill

      開集合として$\emptyset$を考えるとき、
      $f^{-1}(\emptyset)=\emptyset$である。

      開集合として$\mathbb{R}$を考えるとき、
      $f^{-1}(\mathbb{R})=X$である。


      開区間$I=(a,b)\subset\mathbb{R}$
      に対して、${}^{\forall}p\in f^{-1}(I)$とする。
      $f(p)\in I$であるので、
      ${}^{\exists}\varepsilon >0$に対し
      $f(p)\in (f(p)-\varepsilon , f(p)+\varepsilon ) \subset I$
      である。

      $f$は連続写像であるため、
      ある$\delta >0$が存在し、$p\in X$の近傍$U_\delta$が
      $f(U_\delta) \subset (f(p)-\varepsilon , f(p)+\varepsilon)$
      となる。
      つまり、近傍$U_\delta$は$f(p)$の近傍の逆像に含まれる。
      \begin{equation}
       U_\delta \subset f^{-1}(f(p)-\varepsilon , f(p)+\varepsilon)
        \subset f^{-1}(I)
      \end{equation}

      ${}^{\forall}p\in f^{-1}(I)$に対して
      その近傍$U_\delta$が含まれる
      為、
      $U_\delta \subset f^{-1}(I)$は開集合である。


      \dotfill
      $f^{-1}(I)$が開集合
      $\Rightarrow$
      写像$f$が連続
      \dotfill


      $p\in X$とする。
      $\varepsilon>0$に対し
      開区間$I=(f(p)-\varepsilon , f(p)+\varepsilon)$
      を定める。

      条件から$f^{-1}(I)$は開集合である。
      つまり、$\delta>0$に対し、
      $p\in X$の近傍$U_\delta$が$U_\delta \subset f^{-1}(I)$を満たす。
      \begin{equation}
       U_\delta = \{ x\in X \mid d(p,x)<\delta \}
      \end{equation}

      $U_\delta \subset f^{-1}(I)$より
      \begin{equation}
       f(U_\delta) \subset I = (f(p)-\varepsilon , f(p)+\varepsilon)
      \end{equation}
      である。

      よって、$f$は連続写像である。





\hrulefill

\end{enumerate}

\hrulefill


\end{document}

