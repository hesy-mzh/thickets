\documentclass[12pt,b5paper]{ltjsarticle}

%\usepackage[margin=15truemm, top=5truemm, bottom=5truemm]{geometry}
\usepackage[margin=10truemm,left=15truemm]{geometry}

\usepackage{amsmath,amssymb}
%\pagestyle{headings}
\pagestyle{empty}

%\usepackage{listings,url}
%\renewcommand{\theenumi}{(\arabic{enumi})}

%\usepackage{graphicx}

%\usepackage{tikz}
%\usetikzlibrary {arrows.meta}
%\usepackage{wrapfig}	% required for `\wrapfigure' (yatex added)
\usepackage{bm}	% required for `\bm' (yatex added)

% ルビを振る
\usepackage{luatexja-ruby}	% required for `\ruby'

%% 核Ker 像Im Hom を定義
%\newcommand{\Img}{\mathop{\mathrm{Im}}\nolimits}
%\newcommand{\Ker}{\mathop{\mathrm{Ker}}\nolimits}
%\newcommand{\Hom}{\mathop{\mathrm{Hom}}\nolimits}

%\DeclareMathOperator{\Rot}{rot}
%\DeclareMathOperator{\Div}{div}
%\DeclareMathOperator{\Grad}{grad}
%\DeclareMathOperator{\arcsinh}{arcsinh}
%\DeclareMathOperator{\arccosh}{arccosh}
%\DeclareMathOperator{\arctanh}{arctanh}



%\usepackage{listings,url}
%
%\lstset{
%%プログラム言語(複数の言語に対応,C,C++も可)
%  language = Python,
%%  language = Lisp,
%%  language = C,
%  %背景色と透過度
%  %backgroundcolor={\color[gray]{.90}},
%  %枠外に行った時の自動改行
%  breaklines = true,
%  %自動改行後のインデント量(デフォルトでは20[pt])
%  breakindent = 10pt,
%  %標準の書体
%%  basicstyle = \ttfamily\scriptsize,
%  basicstyle = \ttfamily,
%  %コメントの書体
%%  commentstyle = {\itshape \color[cmyk]{1,0.4,1,0}},
%  %関数名等の色の設定
%  classoffset = 0,
%  %キーワード(int, ifなど)の書体
%%  keywordstyle = {\bfseries \color[cmyk]{0,1,0,0}},
%  %表示する文字の書体
%  %stringstyle = {\ttfamily \color[rgb]{0,0,1}},
%  %枠 "t"は上に線を記載, "T"は上に二重線を記載
%  %他オプション:leftline,topline,bottomline,lines,single,shadowbox
%  frame = TBrl,
%  %frameまでの間隔(行番号とプログラムの間)
%  framesep = 5pt,
%  %行番号の位置
%  numbers = left,
%  %行番号の間隔
%  stepnumber = 1,
%  %行番号の書体
%%  numberstyle = \tiny,
%  %タブの大きさ
%  tabsize = 4,
%  %キャプションの場所("tb"ならば上下両方に記載)
%  captionpos = t
%}



\begin{document}

\hrulefill
\textbf{問題}
\hrulefill


\begin{enumerate}
 \item [系 2,12]
       
       $K$-ベクトル空間$V$から$W$への線形写像$f$があり、
       $V$の基底を$e_1,e_2,\dots,e_n$とする。
       この時、次が同値である。
       \begin{enumerate}
        \item $f$が同型写像
        \item $f(e_1),f(e_2),\dots,f(e_n)$が$W$の基底
       \end{enumerate}

 \item [命題2,13]
       
       $K$-ベクトル空間$V$から$W$への線形写像$f$が同型ならば、
       逆写像$f^{-1}:W\to V$も線形写像であり、同型である。

       \dotfill

       \textbf{命題の証明}

       $V$の基底を$e_1,e_2,\dots,e_n$とする。
       $f$が同型なので、上の系より
       $f(e_1),\dots,f(e_n)$
       が$W$の基底である。

       $V$の元$v=k_1e_1+\cdots+k_ne_n$は
       $f(v)\in W$に対応する。
       \begin{equation}
        f(v)=f(k_1e_1+\cdots+k_ne_n)
         =k_1f(e_1)+\cdots+k_nf(e_n)
       \end{equation}

       これにより任意の元
       $w=k_1f(e_1)+\cdots+k_nf(e_n)\in W$に対し
       $k_1e_1+\cdots+k_ne_n$を対応させる写像が存在する。
       \begin{equation}
        f^{-1} : W\to V
         ,\quad
         k_1f(e_1)+\cdots+k_nf(e_n) \mapsto k_1e_1+\cdots+k_ne_n
       \end{equation}

       $k_1,\dots,k_n$の内、$k_i=1$としそれ以外を$0$とすれば、
       $f^{-1}(f(e_i))=e_i$となる。
       その為、次のように$f^{-1}$は線形写像であることがわかる。
       \begin{align}
        f^{-1}(k_1f(e_1)+\cdots+k_nf(e_n))
         &=k_1e_1+\cdots+k_ne_n\\
         &=k_1f^{-1}(f(e_1))+\cdots+f^{-1}(f(k_ne_n))
       \end{align}

       また、$f^{-1}$は
       $W$の基底$f(e_i)$を
       $V$の基底$e_i$にうつすので
       同型写像であることもわかる。

       \hrulefill

 \item [問 2.4-1]

       線形写像$t:K[x]_2 \to K[x]_4$を
       $f(x)\mapsto f((x+1)^2)$で定める。

       $K[x]_2$の基底$a_1=1,a_2=x,a_3=x^2$と、
       $K[x]_4$の基底$b_1=1,b_2=x,b_3=x^2,b_4=x^3,b_5=x^4$に
       関する$t$の表現行列を求めよ。

       \dotfill

       線形写像$t$で$a_1,a_2,a_3$を移すと次のようになる。

       \begin{align}
        t(a_1) =& t(1) = 1
        \quad = b_1\\
        t(a_2) =& t(x) = (x+1)^2 = x^2+2x+1
        \quad = b_3 + 2b_2 + b_1\\
        t(a_3) =& t(x^2) = ((x+1)^2)^2
        = x^4 + 4x^3 + 6x^2 + 4x + 1\\
        =& b_5 + 4b_4 + 6b_3 + 4b_2 + b_1
       \end{align}

       これらをベクトルとして並べ、
       行列の積で表す。
       \begin{equation}
        \begin{pmatrix} t(a_1) & t(a_2) & t(a_3) \end{pmatrix}
        =
         \begin{pmatrix} b_1 & b_2 & b_3 & b_4 & b_5 \end{pmatrix}
         \begin{pmatrix}
          1 & 1 & 1 \\
          0 & 2 & 4 \\
          0 & 1 & 6 \\
          0 & 0 & 4 \\
          0 & 0 & 1
         \end{pmatrix}
       \end{equation}

       よって、線形写像$t$の表現行列は次の行列である。
       \begin{equation}
         \begin{pmatrix}
          1 & 1 & 1 \\
          0 & 2 & 4 \\
          0 & 1 & 6 \\
          0 & 0 & 4 \\
          0 & 0 & 1
         \end{pmatrix}
       \end{equation}


       \hrulefill


 \item [問 2.4-2]

       $a_1,\dots,a_n$と$b_1,\dots,b_n$を
       それぞれ$K^n$の線形独立な元とし、
       これらを並べてできる$n$次行列を
       それぞれ$A$と$B$とする。

       この時、
       $K^n$の基底$a_1,\dots,a_n$から
       $b_1,\dots,b_n$への変換行列は$A^{-1}B$であることを示せ。

       \dotfill

       $n$次正方行列$A,B$は次のような行列である。
       \begin{equation}
        A=\begin{pmatrix} a_1 & a_2 & \cdots & a_n \end{pmatrix}
        ,\quad
        B=\begin{pmatrix} b_1 & b_2 & \cdots & b_n \end{pmatrix}
       \end{equation}

       変換行列$M$は次の式を満たすような行列である。
       \begin{equation}
        \begin{pmatrix} b_1 & b_2 & \cdots & b_n \end{pmatrix}
        =
         \begin{pmatrix} a_1 & a_2 & \cdots & a_n \end{pmatrix}
         M
       \end{equation}
       これは$B=AM$ということである。
       $a_1,\dots,a_n$は線形独立であるので$A$は正則である。
       そこで、
       両辺に左から$A^{-1}$をかけることで
       $A^{-1}B=M$となる。

       つまり、変換行列は$A^{-1}B$となる。

       \hrulefill


 \item [問 2.6-1]

       次の行列$A$に対し、
       $\mathrm{Im}L_A$
       と
       $\mathrm{Ker}L_A$
       の基底を求めよ。

       \begin{enumerate}
        \item
             $A=\begin{pmatrix} 1 & 2 & 0 \\ 0 & 2 & 1 \end{pmatrix}$

             \dotfill

             $L_A$とは次のような写像である。
             \begin{equation}
              L_A : K^3 \to K^2 ,\quad \bm{x} \to A\bm{x}
             \end{equation}

             $\bm{a}_i,\bm{x}$を次のように置く。
             \begin{equation}
              \bm{a}_1=\begin{pmatrix} 1 \\ 0 \end{pmatrix}
              ,\quad
              \bm{a}_2=\begin{pmatrix} 2 \\ 2 \end{pmatrix}
              ,\quad
              \bm{a}_3=\begin{pmatrix} 0 \\ 1 \end{pmatrix}
              ,\quad
              \bm{x}=\begin{pmatrix} x_1 \\ x_2 \\ x_3 \end{pmatrix}
             \end{equation}

             $x_i\in K$に対して$\mathrm{Im} L_A$
             を考える。
             \begin{equation}
              L_A(\bm{x})=A\bm{x}
                =\begin{pmatrix}
                 \bm{a}_1&\bm{a}_2&\bm{a}_3
                 \end{pmatrix}\bm{x}
                = x_1\bm{a}_1 + x_2\bm{a}_2 + x_3\bm{a}_3
                \label{form}
             \end{equation}

             $\bm{a}_2=2\bm{a}_1+2\bm{a}_3$であるので、
             $L_A(\bm{x})=(x_1+2x_2)\bm{a}_1+(2x_2+x_3)\bm{a}_3$
             である。
             $\bm{a}_1,\bm{a}_3$は線形独立であるので
             これが$\mathrm{Im}L_A$の基底となる。
             \begin{equation}
              \mathrm{Im}L_A
               = \langle \bm{a}_1,\bm{a}_3 \rangle
               =\left\langle
               \begin{pmatrix} 1 \\ 0 \end{pmatrix}
               ,
               \begin{pmatrix} 0 \\ 1 \end{pmatrix}
                \right\rangle
             \end{equation}

             $\mathrm{Ker}L_A$は
             $L_A(\bm{x})=0$を満たす
             $\bm{x}\in K^3$全体の集合である。

             式(\ref{form})より
             $x_1\bm{a}_1 + x_2\bm{a}_2 + x_3\bm{a}_3=\bm{0}$
             の解空間の基底を求める。
             左辺を計算すると次のようになる。
             \begin{equation}
              x_1\bm{a}_1 + x_2\bm{a}_2 + x_3\bm{a}_3
               = \begin{pmatrix} x_1 + 2x_2 \\ 2x_2+ x_3 \end{pmatrix}
             \end{equation}

             これは成分ごとに見ると
             $x_1+2x_2=0,\ 2x_2+x_3=0$
             となるので、$\alpha = x_2$と置くと、
             $x_1=-2\alpha,\ x_3=-2\alpha$である。
             よって、$\bm{x}$は次のようになる。
             \begin{equation}
              \bm{x}
                =\begin{pmatrix}
                 -2\alpha \\ \alpha \\ -2\alpha
                 \end{pmatrix}
                 =\alpha \begin{pmatrix} -2 \\ 1 \\ -2 \end{pmatrix}
             \end{equation}

             これにより$\bm{x}$は1次元空間となり、
             $\mathrm{Ker}L_A$は次のように生成される。
             \begin{equation}
              \mathrm{Ker}L_A
               = \left\langle
                  \begin{pmatrix} -2 \\ 1 \\ -2 \end{pmatrix}
                  \right\rangle
             \end{equation}



             \hrulefill

        \item
             $A=\begin{pmatrix} 0 & 1 \\ 1 & 2 \end{pmatrix}$

             \dotfill

             $A$による線形写像$L_A$は次のような写像である。
             \begin{equation}
              L_A : K^2 \to K^2 ,\quad \bm{x} \to A\bm{x}
             \end{equation}

             行列$A$を列ベクトルに分ける。
             \begin{equation}
              \bm{a}_1= \begin{pmatrix} 0 \\ 1 \end{pmatrix}
              ,\quad
              \bm{a}_2= \begin{pmatrix} 1 \\ 2 \end{pmatrix}
              ,\quad
               A=\begin{pmatrix} \bm{a}_1 & \bm{a}_2 \end{pmatrix}
             \end{equation}

             $\bm{a}_1,\bm{a}_2$は一次独立なので$A$は正則行列であり、
             $\mathrm{Im}L_A$は2次元、$\mathrm{Ker}L_A$は0次元である。
             つまり、次のような基底で表すことが出来る。
             \begin{equation}
              \mathrm{Im}L_A
               =\left\langle \bm{a}_1,\bm{a}_2 \right\rangle
               =\left\langle
                 \begin{pmatrix} 0 \\ 1 \end{pmatrix}
                 ,
                 \begin{pmatrix} 1 \\ 2 \end{pmatrix}
                \right\rangle
               ,\quad
              \mathrm{Ker}L_A
              =\langle 0 \rangle
             \end{equation}


             \hrulefill

       \end{enumerate}


 \item [問 3.2-3]

       計量ベクトル空間$V$の任意の基底を
       $\bm{v}_1,\dots,\bm{v}_n$とする。

       \ruby{Gram}{グラム}-\ruby{Schmidt}{シュミット}の正規直交化法
       を用いて基底$\bm{e}_1,\dots,\bm{e}_n$を定めた時、
       $\bm{e}_1,\dots,\bm{e}_n$は互いに直交することを示せ。
       \begin{equation}
        \bm{e}_1=\bm{v}_1,\quad
        \bm{e}_j
         =\bm{v}_j
         - \sum_{i=1}^{j-1}
         \frac{(\bm{v}_j,\bm{e}_i)}{\|\bm{e}_i\|^2}\bm{e}_i
         \quad (j>1)
       \end{equation}

       $e_i,e_j$が直交するとは
       $i\ne j$の時$(e_i,e_j)=0$となることである。

       証明は
       $e_1,\dots,e_j$が互いに直交するなら
       $(e_{j+1},e_1)=(e_{j+1},e_2)=\cdots =(e_{j+1},e_j)=0$
       となることを帰納的に示す。

       \dotfill

       $\bm{e}_1,\bm{e}_2$は次のようなベクトルである。
       \begin{equation}
       \bm{e}_1=\bm{v}_1
        ,\
        \bm{e}_2=\bm{v}_2
        -\frac{(\bm{v}_2,\bm{e}_1)}{\|\bm{e}_1\|^2}\bm{e}_1
       \end{equation}

       この2つの内積$(\bm{e}_1,\bm{e}_2)$を求める。
       \begin{align}
        (\bm{e}_1,\bm{e}_2)
        =& (\bm{v}_1
        ,
        \bm{v}_2-\frac{(\bm{v}_2,\bm{e}_1)}{\|\bm{e}_1\|^2}\bm{e}_1)\\
        =& (\bm{v}_1,
        \bm{v}_2-\frac{(\bm{v}_2,\bm{v}_1)}{\|\bm{v}_1\|^2}\bm{v}_1)\\
        =& (\bm{v}_1,\bm{v}_2)
        -\frac{(\bm{v}_2,\bm{v}_1)}{\|\bm{v}_1\|^2} (\bm{v}_1,\bm{v}_1)\\
        =& (\bm{v}_1,\bm{v}_2)
        -\frac{(\bm{v}_2,\bm{v}_1)}{\|\bm{v}_1\|^2} \|\bm{v}_1\|^2\\
        =& 0
       \end{align}

       $(\bm{e}_1,\bm{e}_2)=0$より
       $\bm{e}_1,\bm{e}_2$は直交している。

       $\bm{e}_1,\dots,\bm{e}_j\ (1<j<n)$が直交していると仮定し、
       これらと$\bm{e}_{j+1}$が直交していることを確認する。

       $1\leq k \leq j$として
       内積$(\bm{e}_{j+1},\bm{e}_{k})$を計算する。
       \begin{align}
        (\bm{e}_{j+1},\bm{e}_{k})
        =& \left(
        \bm{v}_{j+1}
        - \sum_{i=1}^{j} \frac{(\bm{v}_{j+1},\bm{e}_{i})}{\|\bm{e}_{i}\|^2}\bm{e}_{i}
        ,\ \bm{e}_{k} \right)\\
        =& (\bm{v}_{j+1},\bm{e}_{k})
        - \left( \sum_{i=1}^{j} \frac{(\bm{v}_{j+1},\bm{e}_{i})}{\|\bm{e}_{i}\|^2}\bm{e}_{i},\bm{e}_{k} \right)\\
        =& (\bm{v}_{j+1},\bm{e}_{k})
        - \left( \frac{(\bm{v}_{j+1},\bm{e}_{k})}{\|\bm{e}_{k}\|^2}\bm{e}_{k},\bm{e}_{k} \right)
        \quad ( i\ne k \text{ の時 }(\bm{e}_{i},\bm{e}_{k})=0 )\\
        =& (\bm{v}_{j+1},\bm{e}_{k})
        - \frac{(\bm{v}_{j+1},\bm{e}_{k})}{\|\bm{e}_{k}\|^2} (\bm{e}_{k},\bm{e}_{k} )\\
        =& 0
       \end{align}

       これにより$\bm{e}_{1},\dots,\bm{e}_{j}$が
       互いに直交していれば
       このベクトルと$\bm{e}_{j+1}$は直交する。

       よって、
       $\bm{e}_1,\dots,\bm{e}_n$は互いに直交する。

       \hrulefill

\end{enumerate}


\end{document}
