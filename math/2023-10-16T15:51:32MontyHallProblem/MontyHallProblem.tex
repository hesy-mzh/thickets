\documentclass[12pt,b5paper]{ltjsarticle}

%\usepackage[margin=15truemm, top=5truemm, bottom=5truemm]{geometry}
%\usepackage[margin=10truemm,left=15truemm]{geometry}
\usepackage[margin=10truemm]{geometry}

\usepackage{amsmath,amssymb}
%\pagestyle{headings}
\pagestyle{empty}

%\usepackage{listings,url}
%\renewcommand{\theenumi}{(\arabic{enumi})}

%\usepackage{graphicx}

%\usepackage{tikz}
%\usetikzlibrary {arrows.meta}
%\usepackage{wrapfig}
%\usepackage{bm}

% ルビを振る
\usepackage{luatexja-ruby}	% required for `\ruby'

%% 核Ker 像Im Hom を定義
%\newcommand{\Img}{\mathop{\mathrm{Im}}\nolimits}
%\newcommand{\Ker}{\mathop{\mathrm{Ker}}\nolimits}
%\newcommand{\Hom}{\mathop{\mathrm{Hom}}\nolimits}

%\DeclareMathOperator{\Rot}{rot}
%\DeclareMathOperator{\Div}{div}
%\DeclareMathOperator{\Grad}{grad}
%\DeclareMathOperator{\arcsinh}{arcsinh}
%\DeclareMathOperator{\arccosh}{arccosh}
%\DeclareMathOperator{\arctanh}{arctanh}

\usepackage{url}

%\usepackage{listings}
%
%\lstset{
%%プログラム言語(複数の言語に対応,C,C++も可)
%  language = Python,
%%  language = Lisp,
%%  language = C,
%  %背景色と透過度
%  %backgroundcolor={\color[gray]{.90}},
%  %枠外に行った時の自動改行
%  breaklines = true,
%  %自動改行後のインデント量(デフォルトでは20[pt])
%  breakindent = 10pt,
%  %標準の書体
%%  basicstyle = \ttfamily\scriptsize,
%  basicstyle = \ttfamily,
%  %コメントの書体
%%  commentstyle = {\itshape \color[cmyk]{1,0.4,1,0}},
%  %関数名等の色の設定
%  classoffset = 0,
%  %キーワード(int, ifなど)の書体
%%  keywordstyle = {\bfseries \color[cmyk]{0,1,0,0}},
%  %表示する文字の書体
%  %stringstyle = {\ttfamily \color[rgb]{0,0,1}},
%  %枠 "t"は上に線を記載, "T"は上に二重線を記載
%  %他オプション:leftline,topline,bottomline,lines,single,shadowbox
%  frame = TBrl,
%  %frameまでの間隔(行番号とプログラムの間)
%  framesep = 5pt,
%  %行番号の位置
%  numbers = left,
%  %行番号の間隔
%  stepnumber = 1,
%  %行番号の書体
%%  numberstyle = \tiny,
%  %タブの大きさ
%  tabsize = 4,
%  %キャプションの場所("tb"ならば上下両方に記載)
%  captionpos = t
%}

%\usepackage{cancel}
%\usepackage{bussproofs}
%\usepackage{proof}

\begin{document}

\hrulefill

\begin{enumerate}
 %\setcounter{enumi}{1}
 \item
      ($n=3$のとき)
      $A_{1},A_{2},A_{3}$に対して次が成立するとき独立
      \begin{gather}
       P(A_{1} \cap A_{2} \cap A_{3}) = P(A_{1}) P(A_{2}) P(A_{3}) \label{eq_Conditions1} \\
       \begin{cases}
        P(A_{1} \cap A_{2}) = P(A_{1}) P(A_{2})\\
        P(A_{1} \cap A_{3}) = P(A_{1}) P(A_{3})\\
        P(A_{2} \cap A_{3}) = P(A_{2}) P(A_{3})
       \end{cases} \label{eq_Conditions2}
      \end{gather}

      コインを3回投げる試行において、
      次の条件を満たす事象$A_{3}$を与えよ。
      \begin{quotation}
       $A_{1}$:1回目に表、
       $A_{2}$:2回目に裏、
       $A_{3}$:?
       $\Longrightarrow$
       \eqref{eq_Conditions1}は成立しないが、
       \eqref{eq_Conditions2}は成立する
      \end{quotation}


      \dotfill

      $A_{3}$を「3回連続同じではない」とする。
      つまり、(表2回、裏1回)または(表1回、裏2回)である。
      このとき、$P(A_{3})=3/4$である。

      また、$A_{1}\cap A_{3}$は「1回目が表で、2回目と3回目が表表ではない」
      ということなので、
      $P(A_{1}\cap A_{3}) = 3/8$である。
      よって、$P(A_{1}\cap A_{3})=P(A_{1})P(A_{3})$となる。

      同様に、$P(A_{2}\cap A_{3})=P(A_{2})P(A_{3})$である。

      $A_{1}\cap A_{2}\cap A_{3}$は
      1回目に表、2回目に裏が出て、3回目はどちらでも良い事を意味するので、
      $P(A_{1}\cap A_{2}\cap A_{3})=1/4$である。

      一方、
      $P(A_{1})P(A_{3})P(A_{3})=1/2 \times 1/2 \times 3/4 = 3/16$であるので、
      $P(A_{1}\cap A_{2}\cap A_{3}) \ne P(A_{1})P(A_{3})P(A_{3})$である。

      \hrulefill


 \item
      つぎの真偽を確かめよ。ただし、$P(A)>0, \; P(B)>0$とする。
      \begin{enumerate}
        \item $A$と$B$が独立ならば、$A$と$B$は排反である。

              \dotfill

              $A$と$B$が独立ならば、$P(A\cap B)=P(A)P(B)$である。
              $P(A)>0, \; P(B)>0$より、
              $P(A\cap B)>0$である。
              よって、$A\cap B \ne \emptyset$となるので
              $A$と$B$は排反ではない。

              2回コイントスをする場合、
              事象$A$を1回目が表、
              事象$B$を2回目が表
              とする。
              このとき、$A$と$B$は独立であるが、
              $A\cap B$は2回連続で表が出る事象となるので
              排反ではない。

              \hrulefill
        \item $A$と$B$が独立ならば、$A^{c}$と$B^{c}$も独立である。

              \dotfill

              $P(A^{c}) = 1 - P(A)$、
              $P(B^{c}) = 1 - P(B)$である。
              よって、次の式が成り立つ。
              \begin{equation}
               P(A^{c})P(B^{c})
                =1- P(A)- P(B) + P(A)P(B)
                =1- P(A)- P(B) + P(A\cap B)
              \end{equation}

              また、次の式が成り立つ。
              \begin{align}
               P(A^{c}\cap B^{c})
               &= P( (A \cup B)^{c})
               = 1 - P(A\cup B)\\
               &= 1 - \left( P(A) + P(B) - P(A\cap B)\right)
              \end{align}

              よって、
              $P(A^{c}\cap B^{c}) = P(A^{c})P(B^{c})$であるから
              $A^{c}$と$B^{c}$も独立である。

              \hrulefill
      \end{enumerate}
 \item
      事象$A,B,C$に対して、
      $P(A\cap B \mid C) = P(A \mid C)P(B \mid C)$ならば、
      $A$と$B$は互いに独立になるか?

      \dotfill

      $P(A\cap B \mid C) = P(A \mid C)P(B \mid C)$のそれぞれの確率は
      次のように変形できる。
      \begin{gather}
       P(A\cap B \mid C) = \frac{P(A\cap B \cap C)}{P(C)}\\
       P(A \mid C) = \frac{P(A \cap C)}{P(C)},\quad
       P(B \mid C) = \frac{P(B \cap C)}{P(C)}
      \end{gather}

      よって、
      $P(A\cap B \cap C) = P(A\cap C)P(B \cap C)$
      である。
      $A\cap B \cap C = (A\cap C) \cap (B \cap C)$であるので、
      $A\cap C$ と $B\cap C$は独立である。

      $1$から$9$が書かれた9枚のカードがあるとする。
      この9枚のカードから2枚を選んで2桁の整数を作る。
      このとき、事象$A,B,C$を次のように定める。
      \begin{itemize}
       \item [A] 十の位が偶数となる
       \item [B] 一の位が素数となる
       \item [C] 選んだ2つの数はどちらも5以上
      \end{itemize}

      このとき、$A\cap C$ と $B\cap C$は独立であるが、
      偶素数$2$があるので、$A$と$B$は独立ではない。



      \hrulefill

 \item
      Monty Hall 問題 について、設定を以下のように一部変える。
      \begin{itemize}
       \item プレイヤーの前に5枚のカードがある。
             内、当たりは1枚、質問者は答えを知っている。
       \item プレイヤーは1枚カードを選び、
             質問者が残りの4枚のカードから
             3枚の外れを教える。
             プレイヤーは残ったいるカードに変更できる。
      \end{itemize}
      この場合、プレイヤーは残っているカードに変更したほうが良いか?

      \dotfill

      Monty Hall 問題は、
      最初の選択とその選択から外れた集団のどちらを選ぶべきかが問われる問題である。
      選んだ1枚だけの集団と選ばなかった残りの集団の内、
      当たりが含まれるのはどちらのほうが確率が高いかということになる。

      つまり、\textbf{変更をするべき}が確率上正しい。

      \dotfill


      5枚のカードを順に番号を振り、
      $A_{i}\;(i=1,\dots,5)$は対応するカードが当たる事象とする。
      このとき、確率は$P(A_{i})=1/5$である。

      プレイヤーが1のカードを選択した場合を考える。

      事象$B$は
      質問者(司会者)が残りの2から5のカードから2以外をオープンにする事象とする。
      \begin{gather}
       P(B\mid A_{1})=\frac{1}{4}
        ,\qquad
        P(B\mid A_{2})=1\\
        P(B\mid A_{3})=P(B\mid A_{4})=P(B\mid A_{5})=0
      \end{gather}

      ここから$P(B)$を求める。
      \begin{equation}
       P(B) = \sum_{i=1}^{5} P(B\mid A_{i}) P(A_{i})
        = \frac{1}{20} + \frac{1}{5} + 0 + 0 + 0 = \frac{1}{4}
      \end{equation}


      プレイヤーが1を選択した後、司会者が3,4,5を開いた時に
      1のカードが当たりの確率$P(A_{1} \mid B)$と、
      2のカードが当たりの確率$P(A_{2} \mid B)$を計算する。
      \begin{align}
       P(A_{1} \mid B)
       &= P(A_{1}) \times \frac{P(B\mid A_{1})}{P(B)}\\
       &= \frac{1}{5} \times \frac{\frac{1}{4}}{\frac{1}{4}}
       =\frac{1}{5}\\
       P(A_{2} \mid B)
       &= P(A_{2}) \times \frac{P(B\mid A_{2})}{P(B)}\\
       &= \frac{1}{5} \times \frac{1}{\frac{1}{4}}
       =\frac{4}{5}
      \end{align}

      よって、
      カード1を選択した後、2に変更した時の方が当たる確率が高い。

      選択したカードが他のカードであっても同様に計算できるので、
      変更したほうが良いことがわかる。



      \hrulefill
\end{enumerate}

\hrulefill

\end{document}
