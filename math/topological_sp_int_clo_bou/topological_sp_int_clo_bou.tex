\documentclass[12pt,b5paper]{ltjsarticle}

%\usepackage[margin=15truemm, top=5truemm, bottom=5truemm]{geometry}
\usepackage[margin=10truemm]{geometry}

\usepackage{amsmath,amssymb}
%\pagestyle{headings}
\pagestyle{empty}

%\usepackage{listings,url}
%\renewcommand{\theenumi}{(\arabic{enumi})}

%\usepackage{graphicx}

%\usepackage{tikz}
%\usetikzlibrary {arrows.meta}
%\usepackage{wrapfig}	% required for `\wrapfigure' (yatex added)
\usepackage{bm}	% required for `\bm' (yatex added)

% ルビを振る
%\usepackage{luatexja-ruby}	% required for `\ruby'

%% 核Ker 像Im Hom を定義
%\newcommand{\Img}{\mathop{\mathrm{Im}}\nolimits}
%\newcommand{\Ker}{\mathop{\mathrm{Ker}}\nolimits}
%\newcommand{\Hom}{\mathop{\mathrm{Hom}}\nolimits}

%\DeclareMathOperator{\Rot}{rot}
%\DeclareMathOperator{\Div}{div}
%\DeclareMathOperator{\Grad}{grad}
%\DeclareMathOperator{\arcsinh}{arcsinh}
%\DeclareMathOperator{\arccosh}{arccosh}
%\DeclareMathOperator{\arctanh}{arctanh}



%\usepackage{listings,url}
%
%\lstset{
%%プログラム言語(複数の言語に対応,C,C++も可)
%%  language = Python,
%%  language = Lisp,
%  language = C,
%  %背景色と透過度
%  %backgroundcolor={\color[gray]{.90}},
%  %枠外に行った時の自動改行
%  breaklines = true,
%  %自動改行後のインデント量(デフォルトでは20[pt])
%  breakindent = 10pt,
%  %標準の書体
%%  basicstyle = \ttfamily\scriptsize,
%  basicstyle = \ttfamily,
%  %コメントの書体
%%  commentstyle = {\itshape \color[cmyk]{1,0.4,1,0}},
%  %関数名等の色の設定
%  classoffset = 0,
%  %キーワード(int, ifなど)の書体
%%  keywordstyle = {\bfseries \color[cmyk]{0,1,0,0}},
%  %表示する文字の書体
%  %stringstyle = {\ttfamily \color[rgb]{0,0,1}},
%  %枠 "t"は上に線を記載, "T"は上に二重線を記載
%  %他オプション:leftline,topline,bottomline,lines,single,shadowbox
%  frame = TBrl,
%  %frameまでの間隔(行番号とプログラムの間)
%  framesep = 5pt,
%  %行番号の位置
%  numbers = left,
%  %行番号の間隔
%  stepnumber = 1,
%  %行番号の書体
%%  numberstyle = \tiny,
%  %タブの大きさ
%  tabsize = 4,
%  %キャプションの場所("tb"ならば上下両方に記載)
%  captionpos = t
%}



\begin{document}

\hrulefill
\textbf{定義}
\hrulefill

\textbf{補有限位相}

集合$X$に対して部分集合族$\mathcal{O}_{cf}$を定める。
\begin{equation}
 \mathcal{O}_{cf} = \{\emptyset\}\cup\{O\subset X \mid X\backslash O \text{ は有限集合}\}
\end{equation}
$\mathcal{O}_{cf}$を補有限位相といい、
$(X,\mathcal{O}_{cf})$を補有限位相空間という。


\textbf{距離関数}

集合$X$上の実数値関数$d$が次を満たすとする。
\begin{gather}
 d: X\times X \to \mathbb{R}\\
 d(a,b)\geq 0\\
 d(a,b)= 0 \Leftrightarrow a=b\\
 d(a,b) = d(b,a)\\
 d(a,b) + d(b,c) \geq d(a,c)
\end{gather}
このとき、関数$d$を距離関数という。

集合$X$に距離関数$d$が定義される場合、
この2つの組合せ$(X,d)$を距離空間という。

\textbf{ノルム}

\begin{equation}
 \lvert \lvert \bm{x} \rvert \rvert
\end{equation}


\hrulefill
\textbf{問題}
\hrulefill

\begin{enumerate}
 \item
      \,[補有限位相空間]

      集合$X$上の補有限位相空間$(X,\mathcal{O}_{cf})$が
      位相空間であることを確かめよ。

 \item
      \,[補有限位相空間]

      $(\mathbb{R},\mathcal{O}_{cf})$において
      $a<b$なる任意の$a,b\in\mathbb{R}$に対して
      開区間$(a,b)$は開集合ではないことを示せ。

 \item
      \,[$\mathbb{R}$上の開集合]

      $(\mathbb{R},\mathcal{O}_{d_1})$における
      $\mathbb{R}$において、
      $[a,b)$および$[a,b]$は開集合でないことを示せ。
\end{enumerate}

\end{document}
