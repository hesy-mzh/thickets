\documentclass[12pt,b5paper]{ltjsarticle}

%\usepackage[margin=15truemm, top=5truemm, bottom=5truemm]{geometry}
\usepackage[margin=15truemm]{geometry}

\usepackage{amsmath,amssymb}
%\pagestyle{headings}
\pagestyle{empty}

%\usepackage{listings,url}
\renewcommand{\theenumi}{(\arabic{enumi})}

\usepackage{graphicx}

\usepackage{tikz}
\usetikzlibrary {arrows.meta}
\usepackage{wrapfig}	% required for `\wrapfigure' (yatex added)

%% 像Im を定義
%\newcommand{\Img}{\mathop{\mathrm{Im}}\nolimits}

\begin{document}

\textbf{問題}
\begin{enumerate}
 \item $f(x)=0.1x + 8$の時の$f(10)$の値
 \item $f(x)=\frac{2}{x}$の時の$f(x-\Delta x)$の値
 \item $y=3x$ を $x$軸方向に$5$、$y$軸方向に$-2$平行移動した式
 \item $y=-\frac{5}{2x}$ を $x$軸方向に$-5$、$y$軸方向に$8$平行移動した式
\end{enumerate}

\hrulefill

\begin{enumerate}
 \item $f(x)=0.1x + 8$の時の$f(10)$の値

       $f(10)$とは、$f(x)$に$x=10$を代入の意味

       \begin{equation}
        f(10) = 0.1 \times 10 + 8 =9
       \end{equation}

 \item $f(x)=\frac{2}{x}$の時の$f(x-\Delta x)$の値

       \begin{equation}
        f(x-\Delta x) = 2(x-\Delta x) = 2x -2\Delta x
       \end{equation}

 \item $y=3x$ を $x$軸方向に$5$、$y$軸方向に$-2$平行移動した式

       変数$x$を$x-5$、変数$y$を$y-(-2)$に置き換えると
       問題のような平行移動が出来る

       \begin{gather}
        y-(-2) = 3(x-5)\\
        y = 3x-17
       \end{gather}

 \item $y=-\frac{5}{2x}$ を $x$軸方向に$-5$、$y$軸方向に$8$平行移動した式

       \begin{gather}
        y -8 =-\frac{5}{2(x-(-5))}\\
        y =\frac{16x+75}{2x+10}
       \end{gather}

\end{enumerate}




\end{document}
