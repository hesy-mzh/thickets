\documentclass[12pt,b5paper]{ltjsarticle}

%\usepackage[margin=15truemm, top=5truemm, bottom=5truemm]{geometry}
\usepackage[margin=10truemm]{geometry}

\usepackage{amsmath,amssymb}
%\pagestyle{headings}
\pagestyle{empty}

%\usepackage{listings,url}
%\renewcommand{\theenumi}{(\arabic{enumi})}

%\usepackage{graphicx}

%\usepackage{tikz}
%\usetikzlibrary {arrows.meta}
%\usepackage{wrapfig}	% required for `\wrapfigure' (yatex added)
%\usepackage{bm}	% required for `\bm' (yatex added)

% ルビを振る
%\usepackage{luatexja-ruby}	% required for `\ruby'

%% 核Ker 像Im Hom を定義
%\newcommand{\Img}{\mathop{\mathrm{Im}}\nolimits}
%\newcommand{\Ker}{\mathop{\mathrm{Ker}}\nolimits}
%\newcommand{\Hom}{\mathop{\mathrm{Hom}}\nolimits}

%\DeclareMathOperator{\Rot}{rot}
%\DeclareMathOperator{\Div}{div}
%\DeclareMathOperator{\Grad}{grad}
%\DeclareMathOperator{\arcsinh}{arcsinh}
%\DeclareMathOperator{\arccosh}{arccosh}
%\DeclareMathOperator{\arctanh}{arctanh}



\begin{document}

\textbf{複素数の大きさ}

\hrulefill

複素数$z$は
2つの実数$a,b$を用いて
$z=a+bi$と書ける。


ここで、複素数$z$の大きさを次のように定める。
\begin{equation}
 | z | = \sqrt{a^2 + b^2}
\end{equation}
これは三平方の定理から斜辺の長さを求める式と同じである。


大きさを2乗すると
\begin{equation}
 | z |^2 = a^2 + b^2
\end{equation}
である。


$z=a+bi$に対して
$\bar{z}=a-bi$を
$z$の共役な複素数という。
これらの積は次のようになる。
\begin{equation}
 z\times \bar{z} = (a+bi)(a-bi) = a^2 + b^2
\end{equation}


これにより次の式が成り立つ。
\begin{equation}
 |z|^2 = z\times \bar{z}
\end{equation}


\hrulefill

2つの複素数$\alpha,\beta$に対して
$|\alpha - \beta|^2$を計算する。
\begin{align}
 |\alpha - \beta|^2
 =& (\alpha - \beta)(\overline{\alpha - \beta})\\
 =& (\alpha - \beta)(\bar{\alpha} - \bar{\beta})\\
 =& \alpha\bar{\alpha} - \bar{\alpha}\beta - \alpha\bar{\beta} - \beta\bar{\beta}\\
 =& |\alpha|^2 - \bar{\alpha}\beta - \alpha\bar{\beta} - |\beta|^2
\end{align}

\end{document}

