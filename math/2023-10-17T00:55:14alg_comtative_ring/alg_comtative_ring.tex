\documentclass[12pt,b5paper]{ltjsarticle}

%\usepackage[margin=15truemm, top=5truemm, bottom=5truemm]{geometry}
%\usepackage[margin=10truemm,left=15truemm]{geometry}
\usepackage[margin=10truemm]{geometry}

\usepackage{amsmath,amssymb}
%\pagestyle{headings}
\pagestyle{empty}

%\usepackage{listings,url}
%\renewcommand{\theenumi}{(\arabic{enumi})}

%\usepackage{graphicx}

%\usepackage{tikz}
%\usetikzlibrary {arrows.meta}
%\usepackage{wrapfig}
%\usepackage{bm}

% ルビを振る
%\usepackage{luatexja-ruby}	% required for `\ruby'

%% 核Ker 像Im Hom を定義
%\newcommand{\Img}{\mathop{\mathrm{Im}}\nolimits}
%\newcommand{\Ker}{\mathop{\mathrm{Ker}}\nolimits}
%\newcommand{\Hom}{\mathop{\mathrm{Hom}}\nolimits}

%\DeclareMathOperator{\Rot}{rot}
%\DeclareMathOperator{\Div}{div}
%\DeclareMathOperator{\Grad}{grad}
%\DeclareMathOperator{\arcsinh}{arcsinh}
%\DeclareMathOperator{\arccosh}{arccosh}
%\DeclareMathOperator{\arctanh}{arctanh}

%\usepackage{url}

%\usepackage{listings}
%
%\lstset{
%%プログラム言語(複数の言語に対応,C,C++も可)
%  language = Python,
%%  language = Lisp,
%%  language = C,
%  %背景色と透過度
%  %backgroundcolor={\color[gray]{.90}},
%  %枠外に行った時の自動改行
%  breaklines = true,
%  %自動改行後のインデント量(デフォルトでは20[pt])
%  breakindent = 10pt,
%  %標準の書体
%%  basicstyle = \ttfamily\scriptsize,
%  basicstyle = \ttfamily,
%  %コメントの書体
%%  commentstyle = {\itshape \color[cmyk]{1,0.4,1,0}},
%  %関数名等の色の設定
%  classoffset = 0,
%  %キーワード(int, ifなど)の書体
%%  keywordstyle = {\bfseries \color[cmyk]{0,1,0,0}},
%  %表示する文字の書体
%  %stringstyle = {\ttfamily \color[rgb]{0,0,1}},
%  %枠 "t"は上に線を記載, "T"は上に二重線を記載
%  %他オプション:leftline,topline,bottomline,lines,single,shadowbox
%  frame = TBrl,
%  %frameまでの間隔(行番号とプログラムの間)
%  framesep = 5pt,
%  %行番号の位置
%  numbers = left,
%  %行番号の間隔
%  stepnumber = 1,
%  %行番号の書体
%%  numberstyle = \tiny,
%  %タブの大きさ
%  tabsize = 4,
%  %キャプションの場所("tb"ならば上下両方に記載)
%  captionpos = t
%}

%\usepackage{cancel}
%\usepackage{bussproofs}
%\usepackage{proof}

\begin{document}

\hrulefill

$(A,+,\cdot)$を可換環とし、
$I$を$A$のイデアルとする。
この時、
$\mathrm{Jac}(I)$とは
全ての$I$を含む$A$の極大イデアル
の共通部分である。

\begin{enumerate}
 \item
      $\mathrm{Jac}(I)$が
      $A$のイデアルであることを示せ。

      \dotfill

      $\mathrm{Jac}(I)$は次のような集合である。
      \begin{equation}
       \mathrm{Jac}(I)
        = \bigcap_{m\in M_{I}} m
        ,\qquad
        M_{I}=\{m \subset A \mid m \supset I となる極大イデアル\}
      \end{equation}


      $a,b\in\mathrm{Jac}(I)$とすると
      全ての$m \subset M_{I}$に対して、
      $a,b\in m$である。
      よって、
      $-a,a+b,ab\in m$となるので、
      $-a,a+b,ab\in\mathrm{Jac}(I)$である。

      また、$I \subset \mathrm{Jac}(I)$より、
      $0\in \mathrm{Jac}(I)$である。

      $c \in A$について$ca\in m$である為、
      $ca\in \mathrm{Jac}(I)$である。

      これにより$\mathrm{Jac}(I)$は$A$のイデアルである。

      \hrulefill

 \item
      $n=p_{1}^{a_{1}}\dots p_{k}^{a_{k}}$を
      $n\in \mathbb{Z}_{>1}$の素因数分解とする。
      ただし、$a_{i}\in\mathbb{Z}_{\geq 1}\; (1\leq i \leq k)$とする。
      $\mathbb{Z}$のイデアル$(n)=n\mathbb{Z}$に対して、
      $\mathrm{Jac}(n\mathbb{Z})$を求めよ。

      \dotfill

      環$\mathbb{Z}$の極大イデアルは
      素数$p$によって生成されるイデアル$(p)=p\mathbb{Z}$である。

      イデアル$(n)$を含む極大イデアルは素数から生成されるイデアルなので、
      $(p_{1}),\dots,(p_{k})$である。

      よって、$\mathrm{Jac}(n\mathbb{Z})$は次のようなイデアルとなる。
      \begin{equation}
       \mathrm{Jac}(n\mathbb{Z})
        = \bigcap_{i=1}^{k} (p_{i})
        = \bigcap_{i=1}^{k} p_{i}\mathbb{Z}
      \end{equation}

      \hrulefill

\end{enumerate}




\hrulefill


\end{document}
