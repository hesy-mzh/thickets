\documentclass[12pt,b5paper]{ltjsarticle}

%\usepackage[margin=15truemm, top=5truemm, bottom=5truemm]{geometry}
%\usepackage[margin=10truemm,left=15truemm]{geometry}
\usepackage[margin=10truemm]{geometry}

\usepackage{amsmath,amssymb}
%\pagestyle{headings}
\pagestyle{empty}

%\usepackage{listings,url}
%\renewcommand{\theenumi}{(\arabic{enumi})}

\usepackage{graphicx}

%\usepackage{tikz}
%\usetikzlibrary {arrows.meta}
%\usepackage{wrapfig}	% required for `\wrapfigure' (yatex added)
%\usepackage{bm}	% required for `\bm' (yatex added)

% ルビを振る
%\usepackage{luatexja-ruby}	% required for `\ruby'

%% 核Ker 像Im Hom を定義
%\newcommand{\Img}{\mathop{\mathrm{Im}}\nolimits}
%\newcommand{\Ker}{\mathop{\mathrm{Ker}}\nolimits}
%\newcommand{\Hom}{\mathop{\mathrm{Hom}}\nolimits}

%\DeclareMathOperator{\Rot}{rot}
%\DeclareMathOperator{\Div}{div}
%\DeclareMathOperator{\Grad}{grad}
%\DeclareMathOperator{\arcsinh}{arcsinh}
%\DeclareMathOperator{\arccosh}{arccosh}
%\DeclareMathOperator{\arctanh}{arctanh}



%\usepackage{listings,url}
%
%\lstset{
%%プログラム言語(複数の言語に対応,C,C++も可)
%  language = Python,
%%  language = Lisp,
%%  language = C,
%  %背景色と透過度
%  %backgroundcolor={\color[gray]{.90}},
%  %枠外に行った時の自動改行
%  breaklines = true,
%  %自動改行後のインデント量(デフォルトでは20[pt])
%  breakindent = 10pt,
%  %標準の書体
%%  basicstyle = \ttfamily\scriptsize,
%  basicstyle = \ttfamily,
%  %コメントの書体
%%  commentstyle = {\itshape \color[cmyk]{1,0.4,1,0}},
%  %関数名等の色の設定
%  classoffset = 0,
%  %キーワード(int, ifなど)の書体
%%  keywordstyle = {\bfseries \color[cmyk]{0,1,0,0}},
%  %表示する文字の書体
%  %stringstyle = {\ttfamily \color[rgb]{0,0,1}},
%  %枠 "t"は上に線を記載, "T"は上に二重線を記載
%  %他オプション:leftline,topline,bottomline,lines,single,shadowbox
%  frame = TBrl,
%  %frameまでの間隔(行番号とプログラムの間)
%  framesep = 5pt,
%  %行番号の位置
%  numbers = left,
%  %行番号の間隔
%  stepnumber = 1,
%  %行番号の書体
%%  numberstyle = \tiny,
%  %タブの大きさ
%  tabsize = 4,
%  %キャプションの場所("tb"ならば上下両方に記載)
%  captionpos = t
%}



\begin{document}

\hrulefill

\textbf{3者秘密計算}

\hrulefill

2桁の数$10a+b$から
秘密情報$(x_1,x_2)=(a+1,b+1)$を作る。
3人$P_1,P_2,P_3$で分散計算を行った結果を公開し、
公開された3つの情報から秘密情報の積$x_1\times x_2$を取り出す。

積$x_1\times x_2$を考えるため、
$x_1,x_2$はそれぞれ$0$以外であるようにする為、
$(x_1,x_2)=(a+1,b+1)$と1を加えている。
計算は有限体$\mathbb{F}_{11}$上で考える。

\dotfill

$P_1,P_2$の二人にそれぞれ$x_1,x_2$を渡す。

$P_1$は$x_1$以外を知ることはなく
$P_2$は$x_2$以外を知ることはない。
$P_3$は$x_1,x_2$を知ることはない。

$P_1,P_2$の二人は
それぞれ独自に1次多項式$f_{i}(X)=x_i + c_i X$を作る。
$c_i$は$P_1,P_2$の二人が勝手に決めほかに漏らすことはない。
この多項式を使い、
$P_1$に$f_1(1),f_2(1)$を渡し、
$P_2$に$f_1(2),f_2(2)$を渡し、
$P_3$に$f_1(3),f_2(3)$を渡す。

各$P_i$は
それぞれで$y_i=f_1(i)\times f_2(i)$
を計算し
$y_1,y_2,y_3$がシェアとして公開される。

\hrulefill

\textbf{復号}

公開された$y_1,y_2,y_3$から復号を考える。

多項式$f(X)=s+a_1 X +a_2 X^2$とし、
連立方程式$y_i=f(i)$を作り、
その解$s,a_1,a_2$を求める。
この時の$s$が$x_1\times x_2$である。

\dotfill

具体的な復号手順

$y_1,y_2,y_3$を得て次の連立方程式を作る。
\begin{equation}
 \begin{cases}
  y_1 = s+a_1  +a_2\\
  y_2 = s+2 a_1 + 4 a_2\\
  y_3 = s+3 a_1 + 9 a_2
 \end{cases}
 \Leftrightarrow
 \begin{pmatrix}
  y_1 \\ y_2 \\ y_3
 \end{pmatrix}
 =
 \begin{pmatrix}
  1 & 1 & 1 \\
  1 & 2 & 4 \\
  1 & 3 & 9
 \end{pmatrix}
 \begin{pmatrix}
  s \\ a_1 \\ a_2
 \end{pmatrix}
\end{equation}

方程式を解くと
$s, \ a_1, \ a_2$が求まるので、
この$s$が$x_1\times x_2$である。

\dotfill


復号のために$f(X)$を2次多項式としたが、
これは$P_1,P_2$が作った多項式$f_1(X),f_2(X)$が1次式であり、
各$y_i$は$y_i=f_1(i)\times f_2(i)$と積をとったためである。
この為、シェア$y_i$は3つ集めないと連立方程式を解けない。

実際に、各$y_i$は次のような計算で作られる。
\begin{align}
 y_i =& f_1(i)\times f_2(i)\\
 =& (x_1 + c_1i)(x_2 + c_2i)\\
 =& x_1x_2 + x_1c_2i + c_1ix_2 + c_1c_2i^2\\
 =& x_1x_2 + (x_1c_2 + c_1x_2)i + (c_1c_2)i^2
\end{align}

その為、連立方程式の解は次の値となる。
\begin{equation}
 (s,a_1,a_2)=(x_1x_2, x_1c_2 + c_1x_2, c_1c_2)
\end{equation}
$c_i$は勝手に決めた値なので$a_1,a_2$は何になるかわからないが、
$f_i(X)$の定数項を$x_i$と置けば
必ず$s=x_1x_2$となる。


\hrulefill

\end{document}
