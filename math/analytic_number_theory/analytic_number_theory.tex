\documentclass[12pt,b5paper]{ltjsarticle}

%\usepackage[margin=15truemm, top=5truemm, bottom=5truemm]{geometry}
%\usepackage[margin=10truemm,left=15truemm]{geometry}
\usepackage[margin=10truemm]{geometry}

\usepackage{amsmath,amssymb}
%\pagestyle{headings}
\pagestyle{empty}

%\usepackage{listings,url}
%\renewcommand{\theenumi}{(\arabic{enumi})}

\usepackage{graphicx}

%\usepackage{tikz}
%\usetikzlibrary {arrows.meta}
%\usepackage{wrapfig}
%\usepackage{bm}

% ルビを振る
%\usepackage{luatexja-ruby}	% required for `\ruby'

%% 核Ker 像Im Hom を定義
%\newcommand{\Img}{\mathop{\mathrm{Im}}\nolimits}
%\newcommand{\Ker}{\mathop{\mathrm{Ker}}\nolimits}
%\newcommand{\Hom}{\mathop{\mathrm{Hom}}\nolimits}

%\DeclareMathOperator{\Rot}{rot}
%\DeclareMathOperator{\Div}{div}
%\DeclareMathOperator{\Grad}{grad}
%\DeclareMathOperator{\arcsinh}{arcsinh}
%\DeclareMathOperator{\arccosh}{arccosh}
%\DeclareMathOperator{\arctanh}{arctanh}



%\usepackage{listings,url}
%
%\lstset{
%%プログラム言語(複数の言語に対応,C,C++も可)
%  language = Python,
%%  language = Lisp,
%%  language = C,
%  %背景色と透過度
%  %backgroundcolor={\color[gray]{.90}},
%  %枠外に行った時の自動改行
%  breaklines = true,
%  %自動改行後のインデント量(デフォルトでは20[pt])
%  breakindent = 10pt,
%  %標準の書体
%%  basicstyle = \ttfamily\scriptsize,
%  basicstyle = \ttfamily,
%  %コメントの書体
%%  commentstyle = {\itshape \color[cmyk]{1,0.4,1,0}},
%  %関数名等の色の設定
%  classoffset = 0,
%  %キーワード(int, ifなど)の書体
%%  keywordstyle = {\bfseries \color[cmyk]{0,1,0,0}},
%  %表示する文字の書体
%  %stringstyle = {\ttfamily \color[rgb]{0,0,1}},
%  %枠 "t"は上に線を記載, "T"は上に二重線を記載
%  %他オプション:leftline,topline,bottomline,lines,single,shadowbox
%  frame = TBrl,
%  %frameまでの間隔(行番号とプログラムの間)
%  framesep = 5pt,
%  %行番号の位置
%  numbers = left,
%  %行番号の間隔
%  stepnumber = 1,
%  %行番号の書体
%%  numberstyle = \tiny,
%  %タブの大きさ
%  tabsize = 4,
%  %キャプションの場所("tb"ならば上下両方に記載)
%  captionpos = t
%}



\begin{document}


\hrulefill

\textbf{定義}


\hrulefill

\textbf{問題}
\begin{enumerate}
 \item
      ある関数$R:[1,+\infty) \to \mathbb{R}$が存在して、
      実数$x\geq 1$に対して、次が成立することを示せ。
      \begin{equation}
       \sum_{n\leq x}\log{n} = x\log{x}-x + R(x)
        \quad \text{かつ}\quad
        \lvert R(x) \rvert \leq \log{x}+1
      \end{equation}

      \dotfill

      問題の式を変形することで次が得られる。
      \begin{equation}
       R(x) =
       \sum_{n\leq x}\log{n} - x\log{x} +x
      \end{equation}


      \begin{equation}
       \sum_{n\leq x}\log{n} - x(\log{x} -1)
      \end{equation}

      $x=0$における$\log{(x+1)}$のテイラー展開
      \begin{equation}
       \log{(x+1)} = \sum_{n=1}^{\infty}(-1)^{n-1}\frac{x^n}{n}
      \end{equation}
      $x=1$における$\log{x}$のテイラー展開
      \begin{equation}
       \log{x} = \sum_{n=1}^{\infty}(-1)^{n-1}\frac{(x-1)^n}{n}
      \end{equation}

      \hrulefill
 \item
      ある定数$c$と関数$R:[1,+\infty) \to \mathbb{R}$が存在して、
      実数$x\geq 1$に対して、次が成立することを示せ。
      \begin{equation}
       \sum_{n\leq x} \frac{1}{\sqrt{n}}
        = 2\sqrt{x}+c +R(x)
        \quad \text{かつ}\quad
        \lvert R(x) \rvert \leq \frac{1}{\sqrt{x}}
      \end{equation}


 \item
      ある定数$c$と関数$R:[1,+\infty) \to \mathbb{R}$が存在して、
      実数$x\geq e$に対して、次が成立することを示せ。
      \begin{equation}
       \sum_{n\leq x} \frac{\log{n}}{n} = \frac{1}{2}(\log{x})^2 +c+R(x)
        \quad \text{かつ}\quad
        \lvert R(x) \rvert \leq \frac{\log{x}}{x}
      \end{equation}


 \item
      実数$x,k\geq 1$に対して、次を示せ。
      \begin{equation}
       \sum_{n\leq x}\left( \log{\frac{x}{n}} \right)^k \leq k! x
      \end{equation}

      (ヒント:微分積分学の基本定理を
      \begin{equation}
       \left( \log{\frac{x}{n}} \right)^{k}
        = k\int_{1}^{\frac{x}{n}} \frac{(\log{u})^{k-1}}{u} \mathrm{d}u
      \end{equation}
      という形で用いる。
      )


\end{enumerate}


\hrulefill

\end{document}
