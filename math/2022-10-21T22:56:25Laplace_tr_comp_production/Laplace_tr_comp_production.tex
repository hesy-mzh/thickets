\documentclass[12pt,b5paper]{ltjsarticle}

%\usepackage[margin=15truemm, top=5truemm, bottom=5truemm]{geometry}
\usepackage[margin=10truemm]{geometry}

\usepackage{amsmath,amssymb}
%\pagestyle{headings}
\pagestyle{empty}

%\usepackage{listings,url}
%\renewcommand{\theenumi}{(\arabic{enumi})}

%\usepackage{graphicx}

%\usepackage{tikz}
%\usetikzlibrary {arrows.meta}
%\usepackage{wrapfig}	% required for `\wrapfigure' (yatex added)
%\usepackage{bm}	% required for `\bm' (yatex added)

% ルビを振る
%\usepackage{luatexja-ruby}	% required for `\ruby'

%% 核Ker 像Im Hom を定義
%\newcommand{\Img}{\mathop{\mathrm{Im}}\nolimits}
%\newcommand{\Ker}{\mathop{\mathrm{Ker}}\nolimits}
%\newcommand{\Hom}{\mathop{\mathrm{Hom}}\nolimits}

%\DeclareMathOperator{\Rot}{rot}
%\DeclareMathOperator{\Div}{div}
%\DeclareMathOperator{\Grad}{grad}
%\DeclareMathOperator{\arcsinh}{arcsinh}
%\DeclareMathOperator{\arccosh}{arccosh}
%\DeclareMathOperator{\arctanh}{arctanh}



\begin{document}

\hrulefill
\textbf{定義等}
\hrulefill

\textbf{ラプラス変換の線形性}
\begin{equation}
 \mathcal{L}[af(t)+bg(t)]
  = a\mathcal{L}[f(t)] + b\mathcal{L}[g(t)]
\end{equation}


ラプラス変換は積を分解できない。
\begin{equation}
 \mathcal{L}[f(t)g(t)] \ne \mathcal{L}[f(t)]\mathcal{L}[g(t)]
\end{equation}
そこで、
$\mathcal{L}[f(t)]\mathcal{L}[g(t)]=\mathcal{L}[h(t)]$となるような
式$h(t)$を次のように定義する。

\textbf{合成積、畳み込み}

$f(t),g(t)$の合成積$f(t)*g(t)$を次のように定義する。
\begin{equation}
 f(t)*g(t) = \int_{0}^{t}f(\tau)g(t-\tau)\mathrm{d}\tau
\end{equation}


性質
\begin{equation}
 f(t)*g(t)=g(t)*f(t)
  ,\quad
 (af(t))*g(t)=f(t)*(ag(t))
\end{equation}


ラプラス変換
\begin{equation}
 \mathcal{L}[f(t)*g(t)]
  =\mathcal{L}[f(t)]\mathcal{L}[g(t)]
\end{equation}




三角関数の積と和の公式
\begin{align}
 \sin \alpha \cos \beta
 =& \frac{1}{2}\left( \sin(\alpha+\beta)+\sin(\alpha-\beta) \right)\\
 \sin \alpha \sin \beta
 =& -\frac{1}{2}\left( \cos(\alpha+\beta)-\cos(\alpha-\beta) \right)\\
 \cos \alpha \cos \beta
 =& \frac{1}{2}\left( \cos(\alpha+\beta)+\cos(\alpha-\beta) \right)
\end{align}



\hrulefill
問題
\hrulefill

相乗定理を用いて、
次を逆ラプラス変換せよ。
\begin{equation}
 \frac{1}{(s^2+9)^2}
\end{equation}


\dotfill


$\sin 3t$のラプラス変換から$\frac{1}{s^2+9}$が得られる。
\begin{equation}
 \mathcal{L}[\sin 3t] = \frac{3}{s^2+3^2}
  ,\quad
 \mathcal{L}\left[ \frac{1}{3}\sin 3t \right] = \frac{1}{s^2+3^2}
\end{equation}

これにより問の式は次のように合成積のラプラス変換である。
\begin{align}
 \frac{1}{(x^2+9)^2}
  =&
 \frac{1}{s^2+3^2} \times \frac{1}{s^2+3^2}
 =
 \mathcal{L}\left[ \frac{1}{3}\sin 3t \right]\mathcal{L}\left[ \frac{1}{3}\sin 3t \right]\\
 =&
 \mathcal{L}\left[ \frac{1}{3}\sin 3t * \frac{1}{3}\sin 3t \right]
\end{align}

よって、逆変換は次のように得られる。
\begin{equation}
 \mathcal{L}^{-1}\left[ \frac{1}{(s^2+9)^2} \right]
  = \frac{1}{3}\sin 3t * \frac{1}{3}\sin 3t
  = \frac{1}{9}(\sin 3t * \sin 3t)
\end{equation}

合成積の定義に従い$\sin 3t * \sin 3t$を計算する。
\begin{align}
 \sin 3t * \sin 3t
 =& \int_{0}^{t}\sin3\tau \sin3(t-\tau)\mathrm{d}\tau\\
 =& -\frac{1}{2}\int_{0}^{t}\left(\cos 3t - \cos 3(-t+2\tau)\right)\mathrm{d}\tau\\
 =& -\frac{1}{2}\left[ \tau\cos 3t - \frac{1}{6}\sin (-3t+6\tau) \right]_{\tau=0}^{\tau=t}\\
 =& -\frac{1}{2}\left( t\cos 3t - \frac{1}{6}\sin 3t + \frac{1}{6}\sin (-3t) \right)\\
 =& \frac{1}{6}\left( \sin 3t - 3t\cos 3t \right)
\end{align}

よって、$\frac{1}{(s^2+9)^2}$の逆変換は次のように求まる。
\begin{equation}
 \mathcal{L}^{-1}\left[ \frac{1}{(s^2+9)^2} \right]
  = \frac{1}{54
}\left( \sin 3t - 3t\cos 3t \right)
\end{equation}


\hrulefill

\end{document}

