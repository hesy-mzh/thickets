\documentclass[12pt,b5paper]{ltjsarticle}

%\usepackage[margin=15truemm, top=5truemm, bottom=5truemm]{geometry}
%\usepackage[margin=10truemm,left=15truemm]{geometry}
\usepackage[margin=10truemm]{geometry}

\usepackage{amsmath,amssymb}
%\pagestyle{headings}
\pagestyle{empty}

%\usepackage{listings,url}
%\renewcommand{\theenumi}{(\arabic{enumi})}

%\usepackage{graphicx}

%\usepackage{tikz}
%\usetikzlibrary {arrows.meta}
%\usepackage{wrapfig}
%\usepackage{bm}

% ルビを振る
%\usepackage{luatexja-ruby}	% required for `\ruby'

%% 核Ker 像Im Hom を定義
%\newcommand{\Img}{\mathop{\mathrm{Im}}\nolimits}
%\newcommand{\Ker}{\mathop{\mathrm{Ker}}\nolimits}
%\newcommand{\Hom}{\mathop{\mathrm{Hom}}\nolimits}

%\DeclareMathOperator{\Rot}{rot}
%\DeclareMathOperator{\Div}{div}
%\DeclareMathOperator{\Grad}{grad}
%\DeclareMathOperator{\arcsinh}{arcsinh}
%\DeclareMathOperator{\arccosh}{arccosh}
%\DeclareMathOperator{\arctanh}{arctanh}



%\usepackage{listings,url}
%
%\lstset{
%%プログラム言語(複数の言語に対応,C,C++も可)
%  language = Python,
%%  language = Lisp,
%%  language = C,
%  %背景色と透過度
%  %backgroundcolor={\color[gray]{.90}},
%  %枠外に行った時の自動改行
%  breaklines = true,
%  %自動改行後のインデント量(デフォルトでは20[pt])
%  breakindent = 10pt,
%  %標準の書体
%%  basicstyle = \ttfamily\scriptsize,
%  basicstyle = \ttfamily,
%  %コメントの書体
%%  commentstyle = {\itshape \color[cmyk]{1,0.4,1,0}},
%  %関数名等の色の設定
%  classoffset = 0,
%  %キーワード(int, ifなど)の書体
%%  keywordstyle = {\bfseries \color[cmyk]{0,1,0,0}},
%  %表示する文字の書体
%  %stringstyle = {\ttfamily \color[rgb]{0,0,1}},
%  %枠 "t"は上に線を記載, "T"は上に二重線を記載
%  %他オプション:leftline,topline,bottomline,lines,single,shadowbox
%  frame = TBrl,
%  %frameまでの間隔(行番号とプログラムの間)
%  framesep = 5pt,
%  %行番号の位置
%  numbers = left,
%  %行番号の間隔
%  stepnumber = 1,
%  %行番号の書体
%%  numberstyle = \tiny,
%  %タブの大きさ
%  tabsize = 4,
%  %キャプションの場所("tb"ならば上下両方に記載)
%  captionpos = t
%}



\begin{document}

\hrulefill

$n\in\mathbb{N}$とするとき、次を示せ。
\begin{equation}
 \lim_{n\to\infty} n^{1/n} = 1
\end{equation}

\dotfill

$n^{1/n}$の対数を取れば
次の2つは同値である。
\begin{equation}
 \lim_{n\to\infty} n^{1/n} = 1
  \qquad \Longleftrightarrow \qquad
 \lim_{n\to\infty} \frac{\log{n}}{n} = 0
\end{equation}

そこで、
$\displaystyle \frac{\log{n}}{n}$
について考える。


$x=0$における$e^{x}$のテイラー展開が
$\displaystyle e^{x} = \sum_{k=0}^{\infty}\frac{x^{k}}{k!}$
である。

正の実数$x>0$とする。

$\displaystyle e^{x} = \sum_{k=0}^{\infty}\frac{x^{k}}{k!}$
であるが、
右辺の全ての項は正であるので、
これを途中で打ち切ると次の不等式を得る。
\begin{equation}
 e^{x} > 1 + x + \frac{x^{2}}{2}
\end{equation}

これを利用すると
次の不等式が得られる。
\begin{equation}
 \lim_{x\to\infty} \frac{x}{e^{x}}
  \leq
 \lim_{x\to\infty} \frac{x}{1 + x + \frac{x^{2}}{2}}
\end{equation}

$\displaystyle \lim_{x\to\infty} \frac{x}{1 + x + \frac{x^{2}}{2}} = 0$
であることから
$\displaystyle \lim_{x\to\infty} \frac{x}{e^{x}} = 0$
であることがわかる。

$n\in\mathbb{N}$として、
$x=\log{n}$とする。
\begin{equation}
 0
  = \lim_{x\to\infty} \frac{x}{e^{x}}
 = \lim_{n\to\infty} \frac{\log{n}}{e^{\log{n}}}
 = \lim_{n\to\infty} \frac{\log{n}}{n}
\end{equation}

これにより
$\displaystyle \lim_{n\to\infty} \frac{\log{n}}{n} = 0$
であり、
$\displaystyle \lim_{n\to\infty} n^{1/n} = 1$
であることが示せた。

\hrulefill

\end{document}
