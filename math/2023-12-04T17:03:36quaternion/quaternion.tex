\documentclass[12pt,b5paper]{ltjsarticle}

%\usepackage[margin=15truemm, top=5truemm, bottom=5truemm]{geometry}
%\usepackage[margin=10truemm,left=15truemm]{geometry}
\usepackage[margin=10truemm]{geometry}

\usepackage{amsmath,amssymb}
%\pagestyle{headings}
\pagestyle{empty}

%\usepackage{listings,url}
%\renewcommand{\theenumi}{(\arabic{enumi})}

%\usepackage{graphicx}

%\usepackage{tikz}
%\usetikzlibrary {arrows.meta}
%\usepackage{wrapfig}
%\usepackage{bm}

% ルビを振る
%\usepackage{luatexja-ruby}	% required for `\ruby'

%% 核Ker 像Im Hom を定義
%\newcommand{\Img}{\mathop{\mathrm{Im}}\nolimits}
%\newcommand{\Ker}{\mathop{\mathrm{Ker}}\nolimits}
%\newcommand{\Hom}{\mathop{\mathrm{Hom}}\nolimits}

%\DeclareMathOperator{\Rot}{rot}
%\DeclareMathOperator{\Div}{div}
%\DeclareMathOperator{\Grad}{grad}
%\DeclareMathOperator{\arcsinh}{arcsinh}
%\DeclareMathOperator{\arccosh}{arccosh}
%\DeclareMathOperator{\arctanh}{arctanh}

%\usepackage{url}

%\usepackage{listings}
%
%\lstset{
%%プログラム言語(複数の言語に対応,C,C++も可)
%  language = Python,
%%  language = Lisp,
%%  language = C,
%  %背景色と透過度
%  %backgroundcolor={\color[gray]{.90}},
%  %枠外に行った時の自動改行
%  breaklines = true,
%  %自動改行後のインデント量(デフォルトでは20[pt])
%  breakindent = 10pt,
%  %標準の書体
%%  basicstyle = \ttfamily\scriptsize,
%  basicstyle = \ttfamily,
%  %コメントの書体
%%  commentstyle = {\itshape \color[cmyk]{1,0.4,1,0}},
%  %関数名等の色の設定
%  classoffset = 0,
%  %キーワード(int, ifなど)の書体
%%  keywordstyle = {\bfseries \color[cmyk]{0,1,0,0}},
%  %表示する文字の書体
%  %stringstyle = {\ttfamily \color[rgb]{0,0,1}},
%  %枠 "t"は上に線を記載, "T"は上に二重線を記載
%  %他オプション:leftline,topline,bottomline,lines,single,shadowbox
%  frame = TBrl,
%  %frameまでの間隔(行番号とプログラムの間)
%  framesep = 5pt,
%  %行番号の位置
%  numbers = left,
%  %行番号の間隔
%  stepnumber = 1,
%  %行番号の書体
%%  numberstyle = \tiny,
%  %タブの大きさ
%  tabsize = 4,
%  %キャプションの場所("tb"ならば上下両方に記載)
%  captionpos = t
%}

%\usepackage{cancel}
%\usepackage{bussproofs}
%\usepackage{proof}

\begin{document}

\hrulefill

$A$を可換環とする。

$A \setminus \{0\}$が乗法群である時、
$A$を体という。

体は2つの演算それぞれで可換である。

\dotfill

環$A$に対して、
$A\setminus\{0\}$が乗法群である時、
$A$を斜体 または 可除環という。

斜体は乗法の可換を仮定しない。

\dotfill

$F$を体とする。
直積集合$F^{4}$に成分毎の加法を定義することにより
$(F^{4},+)$は加法群となる。

$p,q \in F \setminus \{0\}$とする。
$(H_{F}(p,q),+)$を加法群$F^{4}$とする。

$H_{F}(p,q)$に演算$\cdot$を定義する。

\begin{itemize}
 \item[] $F^{4}$の基底を$\{1,i,j,k\}$とする。
         $(a_{0},a_{1},a_{2},a_{3})=a_{0}+a_{1}i + a_{2}j+a_{3}k$
 \item[] $i^{2}=i\cdot i=p,\; j^{2}=j\cdot j=q,\; i\cdot j= -j\cdot i=k$とする。
         \begin{align}
          & (a_{0}+a_{1}i + a_{2}j+a_{3}k) \cdot (b_{0}+b_{1}i + b_{2}j+b_{3}k)\\
          = & a_{0} (b_{0} +b_{1}i +b_{2}j +b_{3}k)
            + a_{1}i (b_{0} +b_{1}i +b_{2}j +b_{3}k)\\
          & + a_{2}j (b_{0} +b_{1}i +b_{2}j +b_{3}k)
            + a_{3}k (b_{0} +b_{1}i +b_{2}j +b_{3}k)\\
          = & a_{0}b_{0} +a_{1}b_{1}i^{2} +a_{2}b_{2}j^{2} +a_{3}b_{3}k^{2}
            + (a_{0}b_{1} +a_{1}b_{0})i
            + (a_{0}b_{2} +a_{2}b_{0})j\\
          & + (a_{0}b_{3} +a_{3}b_{0})k
            + (a_{1}b_{2} -a_{2}b_{1})ij
            + (a_{1}b_{3} -a_{3}b_{1})ik
            + (a_{2}b_{3} -a_{3}b_{2})jk\\
          = & a_{0}b_{0} +a_{1}b_{1}p +a_{2}b_{2}q -a_{3}b_{3}pq
            + (a_{0}b_{1} +a_{1}b_{0})i
            + (a_{0}b_{2} +a_{2}b_{0})j\\
          & + (a_{0}b_{3} +a_{3}b_{0})k
            + (a_{1}b_{2} -a_{2}b_{1})k
            + (a_{1}b_{3} -a_{3}b_{1})pj
            - (a_{2}b_{3} -a_{3}b_{2})qi\\
          = & a_{0}b_{0} +a_{1}b_{1}p +a_{2}b_{2}q -a_{3}b_{3}pq
            + (a_{0}b_{1} +a_{1}b_{0} -a_{2}b_{3}q -a_{3}b_{2}q)i\\
           & + (a_{0}b_{2} +a_{2}b_{0} +a_{1}b_{3}p -a_{3}b_{1}p)j
            + (a_{0}b_{3} +a_{3}b_{0} +a_{1}b_{2} -a_{2}b_{1})k
         \end{align}

\end{itemize}

2つの演算を定めた$(H_{F}(p,q),+,\cdot)$は環となる。

%${}^{\forall}a+bi+cj+dk \in H_{F}(p,q)\setminus\{0\}$
%に対して
%$\frac{a-bi-cj-dk}{a^{2}+b^{2}+c^{2}+d^{2}}$
%が逆元となるので

\hrulefill

$F$を体、
$p,q \in F^{\times}$とする。
$a^{2}=p$を満たす$a\in F$が存在するとき、
$H_{F}(p,q)$は斜体でないことを示せ。

\dotfill

$a^{2}=p \in F^{\times}$より
$a\in F^{\times}$である。
つまり、
$1+a^{-1}i \ne 0,\; 1-a^{-1}i \ne 0$
である。
\begin{equation}
 (1+a^{-1}i)\cdot (1-a^{-1}i)
  = 1 - (a^{-1}i)^{2}
  = 1 - (a^{-1})^{2}i^{2}
  = 1 - (a^{-1})^{2}a^{2}
  = 0
\end{equation}

これにより
$1+a^{-1}i,\; 1-a^{-1}i$
は零因子である。

よって、
$H_{F}(p,q)$は零因子を持つ環であり、
斜体ではない。

\hrulefill

\end{document}
