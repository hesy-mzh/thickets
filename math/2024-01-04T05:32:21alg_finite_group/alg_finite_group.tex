\documentclass[12pt,b5paper]{ltjsarticle}

%\usepackage[margin=15truemm, top=5truemm, bottom=5truemm]{geometry}
%\usepackage[margin=10truemm,left=15truemm]{geometry}
\usepackage[margin=10truemm]{geometry}

\usepackage{amsmath,amssymb}
%\pagestyle{headings}
\pagestyle{empty}

%\usepackage{listings,url}
%\renewcommand{\theenumi}{(\arabic{enumi})}

%\usepackage{graphicx}

%\usepackage{tikz}
%\usetikzlibrary {arrows.meta}
%\usepackage{wrapfig}
%\usepackage{bm}

% ルビを振る
%\usepackage{luatexja-ruby}	% required for `\ruby'

%% 核Ker 像Im Hom を定義
%\newcommand{\Img}{\mathop{\mathrm{Im}}\nolimits}
%\newcommand{\Ker}{\mathop{\mathrm{Ker}}\nolimits}
%\newcommand{\Hom}{\mathop{\mathrm{Hom}}\nolimits}

%\DeclareMathOperator{\Rot}{rot}
%\DeclareMathOperator{\Div}{div}
%\DeclareMathOperator{\Grad}{grad}
%\DeclareMathOperator{\arcsinh}{arcsinh}
%\DeclareMathOperator{\arccosh}{arccosh}
%\DeclareMathOperator{\arctanh}{arctanh}

\usepackage{url}

%\usepackage{listings}
%
%\lstset{
%%プログラム言語(複数の言語に対応,C,C++も可)
%  language = Python,
%%  language = Lisp,
%%  language = C,
%  %背景色と透過度
%  %backgroundcolor={\color[gray]{.90}},
%  %枠外に行った時の自動改行
%  breaklines = true,
%  %自動改行後のインデント量(デフォルトでは20[pt])
%  breakindent = 10pt,
%  %標準の書体
%%  basicstyle = \ttfamily\scriptsize,
%  basicstyle = \ttfamily,
%  %コメントの書体
%%  commentstyle = {\itshape \color[cmyk]{1,0.4,1,0}},
%  %関数名等の色の設定
%  classoffset = 0,
%  %キーワード(int, ifなど)の書体
%%  keywordstyle = {\bfseries \color[cmyk]{0,1,0,0}},
%  %表示する文字の書体
%  %stringstyle = {\ttfamily \color[rgb]{0,0,1}},
%  %枠 "t"は上に線を記載, "T"は上に二重線を記載
%  %他オプション:leftline,topline,bottomline,lines,single,shadowbox
%  frame = TBrl,
%  %frameまでの間隔(行番号とプログラムの間)
%  framesep = 5pt,
%  %行番号の位置
%  numbers = left,
%  %行番号の間隔
%  stepnumber = 1,
%  %行番号の書体
%%  numberstyle = \tiny,
%  %タブの大きさ
%  tabsize = 4,
%  %キャプションの場所("tb"ならば上下両方に記載)
%  captionpos = t
%}

%\usepackage{cancel}
%\usepackage{bussproofs}
%\usepackage{proof}

\begin{document}


\hrulefill

$G$を位数$4$の群とする。

このとき、
$G \cong \mathbb{Z}/4\mathbb{Z}$
または
$G \cong \mathbb{Z}/2\mathbb{Z} \times \mathbb{Z}/2\mathbb{Z}$
であることを示せ。

\dotfill

$G$は群であるので、単位元を$e$として、
次のようにかける。
\begin{equation}
 G=\{ e,\; a,\; b,\; c\}
\end{equation}

群に単位元は一つなので、
$a\cdot a \ne a$である。

そこで、
$a\cdot a$ が
$e,b,c$のどれかになるが、
$a\cdot a=b$ と
$a\cdot a=c$ は
$b$と$c$を入れ替える
ことで同じとなるので、
$a\cdot a=e$ と $a\cdot a=b$
の二つの場合に分ける。


\begin{description}
 \item [$a\cdot a=e$の場合]

             $a$の位数が2なので、
             $\{e,a\}$が部分群になる。

             次に
             $a\cdot b=a$とすると
             左から$a^{-1}$をかけると
             $b=e$となる。
             その為、
             $a\cdot b \ne a$である。

             $a\cdot b=b$とすると
             $b$が単位元となり、
             単位元が2つになるので
             $a\cdot b \ne b$である。

             $a\cdot b = e$ とすると
             $b = a^{-1}$となり、
             $a\cdot a=e$と矛盾するので、
             $a\cdot b \ne e$ である。

             よって、
             $a\cdot b = c$
             ということになる。

             ここまでをまとめると次のような関係がある。
             \begin{equation}
              \begin{array}{c|cccc}
                & e & a & b & c \\
               \hline
               e & e\cdot e = e & e\cdot a = a & e\cdot b=b & e\cdot c =c \\
               a & a\cdot e = a & a\cdot a = e & a\cdot b=c & a\cdot c = \\
               b & b\cdot e = b & b\cdot a =  & b\cdot b= & b\cdot c = \\
               c & c\cdot e = c & c\cdot a =  & c\cdot b= & c\cdot c = 
              \end{array}
             \end{equation}

             残りの箇所も埋める。

             $a\cdot c$は横の演算結果から$a\cdot c = b$ということになる。
             これは左から$a$を異なる元にかければ
             異なる結果となるためである。

             次に$b\cdot a$を考える。
             もし、$b\cdot a = b$とすれば$a$が単位元であることになり矛盾する。
             よって、$b\cdot a = c$である。

             これにより$c\cdot a = b$である。

             次に
             $b\cdot c$を考える。
             $b\cdot c =e$と仮定すれば
             $b= c^{-1}$となる。
             この為、
             $c\cdot b=e$となり、
             残った$b\cdot b, \; c\cdot c$はともに$a$となり矛盾する。
             よって、$b\cdot c =a$である。

             残りも同様に考えると次の表ができる。
             
             \begin{equation}
              \begin{array}{c|cccc}
                & e & a & b & c \\
               \hline
               e & e\cdot e = e & e\cdot a = a & e\cdot b=b & e\cdot c =c \\
               a & a\cdot e = a & a\cdot a = e & a\cdot b=c & a\cdot c =b \\
               b & b\cdot e = b & b\cdot a = c & b\cdot b=e & b\cdot c = a\\
               c & c\cdot e = c & c\cdot a = b & c\cdot b=a & c\cdot c = e
              \end{array}
              \label{eq:z2z2}
             \end{equation}


 \item [$a\cdot a=b$の場合]

             $a\cdot b$ と $a\cdot c$ のどちらかが単位元となるが、
             $e\cdot c =c$であるので$a\cdot c\ne c$である。
             よって、$a\cdot b=c$ と $a\cdot c=e$ となる。

             同様に$b\cdot a=c$ と $c\cdot a=e$ となる。

             $a\cdot c =e$であるので$b\cdot c\ne e$である。
             よって、$b\cdot b=e$ と $b\cdot c=a$ となる。

             同様に $c\cdot b=a$ となる。

             残りは$c\cdot c = b$となる。

             これをまとめると次のようになる。
             \begin{equation}
              \begin{array}{c|cccc}
                & e & a & b & c \\
               \hline
               e & e\cdot e = e & e\cdot a = a & e\cdot b=b & e\cdot c =c \\
               a & a\cdot e = a & a\cdot a = b & a\cdot b=c & a\cdot c =e \\
               b & b\cdot e = b & b\cdot a =c  & b\cdot b=e & b\cdot c =a \\
               c & c\cdot e = c & c\cdot a =e  & c\cdot b=a & c\cdot c = b
              \end{array}
              \label{eq:z4}
             \end{equation}

\end{description}

\textbf{\eqref{eq:z2z2}の表が表す群について}

群$\mathbb{Z}/2 \mathbb{Z} \times  \mathbb{Z}/2 \mathbb{Z}$
次のような4つの元を持ち、
成分ごとの和でもって加法群となる。
\begin{equation}
 \mathbb{Z}/2 \mathbb{Z} \times  \mathbb{Z}/2 \mathbb{Z}
  =\{ (\bar{0},\bar{0}),(\bar{1},\bar{0}),(\bar{0},\bar{1}),(\bar{1},\bar{1}) \}
\end{equation}

この群に対して
写像$f:G\to \mathbb{Z}/2 \mathbb{Z} \times  \mathbb{Z}/2 \mathbb{Z}$
を次のように定義する。
\begin{equation}
 f(e)=(\bar{0},\bar{0}),\;
  f(a)=(\bar{1},\bar{0}),\;
  f(b)=(\bar{0},\bar{1}),\;
  f(b)=(\bar{1},\bar{1})
\end{equation}

この$f$は同型写像となることから
$G \cong \mathbb{Z}/2\mathbb{Z} \times \mathbb{Z}/2\mathbb{Z}$
となる。


\textbf{\eqref{eq:z4}の表が表す群について}

群$\mathbb{Z}/4\mathbb{Z}$
は次のような集合であり、
整数の和から自然に導入される加法により加法群となる。
\begin{equation}
 \mathbb{Z}/4\mathbb{Z}=
  \{ \bar{0},\; \bar{1},\; \bar{2},\; \bar{3} \}
\end{equation}

この群に対して
写像$g:G\to \mathbb{Z}/4 \mathbb{Z}$
を次のように定義する。
\begin{equation}
 g(e)=\bar{0},\;
  g(a)=\bar{1},\;
  g(b)=\bar{2},\;
  g(b)=\bar{3}
\end{equation}

この$g$は同型写像となることから
$G \cong \mathbb{Z}/4\mathbb{Z}$
となる。


\hrulefill

\end{document}
