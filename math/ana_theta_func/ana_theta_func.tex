\documentclass[12pt,b5paper]{ltjsarticle}

%\usepackage[margin=15truemm, top=5truemm, bottom=5truemm]{geometry}
%\usepackage[margin=10truemm,left=15truemm]{geometry}
\usepackage[margin=10truemm]{geometry}

\usepackage{amsmath,amssymb}
%\pagestyle{headings}
\pagestyle{empty}

%\usepackage{listings,url}
%\renewcommand{\theenumi}{(\arabic{enumi})}

%\usepackage{graphicx}

%\usepackage{tikz}
%\usetikzlibrary {arrows.meta}
%\usepackage{wrapfig}
%\usepackage{bm}

% ルビを振る
%\usepackage{luatexja-ruby}	% required for `\ruby'

%% 核Ker 像Im Hom を定義
%\newcommand{\Img}{\mathop{\mathrm{Im}}\nolimits}
%\newcommand{\Ker}{\mathop{\mathrm{Ker}}\nolimits}
%\newcommand{\Hom}{\mathop{\mathrm{Hom}}\nolimits}

%\DeclareMathOperator{\Rot}{rot}
%\DeclareMathOperator{\Div}{div}
%\DeclareMathOperator{\Grad}{grad}
%\DeclareMathOperator{\arcsinh}{arcsinh}
%\DeclareMathOperator{\arccosh}{arccosh}
%\DeclareMathOperator{\arctanh}{arctanh}

\usepackage{url}

%\usepackage{listings}
%
%\lstset{
%%プログラム言語(複数の言語に対応,C,C++も可)
%  language = Python,
%%  language = Lisp,
%%  language = C,
%  %背景色と透過度
%  %backgroundcolor={\color[gray]{.90}},
%  %枠外に行った時の自動改行
%  breaklines = true,
%  %自動改行後のインデント量(デフォルトでは20[pt])
%  breakindent = 10pt,
%  %標準の書体
%%  basicstyle = \ttfamily\scriptsize,
%  basicstyle = \ttfamily,
%  %コメントの書体
%%  commentstyle = {\itshape \color[cmyk]{1,0.4,1,0}},
%  %関数名等の色の設定
%  classoffset = 0,
%  %キーワード(int, ifなど)の書体
%%  keywordstyle = {\bfseries \color[cmyk]{0,1,0,0}},
%  %表示する文字の書体
%  %stringstyle = {\ttfamily \color[rgb]{0,0,1}},
%  %枠 "t"は上に線を記載, "T"は上に二重線を記載
%  %他オプション:leftline,topline,bottomline,lines,single,shadowbox
%  frame = TBrl,
%  %frameまでの間隔(行番号とプログラムの間)
%  framesep = 5pt,
%  %行番号の位置
%  numbers = left,
%  %行番号の間隔
%  stepnumber = 1,
%  %行番号の書体
%%  numberstyle = \tiny,
%  %タブの大きさ
%  tabsize = 4,
%  %キャプションの場所("tb"ならば上下両方に記載)
%  captionpos = t
%}

%\usepackage{cancel}
%\usepackage{bussproofs}
%\usepackage{proof}

\begin{document}

\hrulefill

$N=2$のとき、
$g,h\in\mathrm{End}(\mathbb{C}^{2})$は
$g^{2}=h^{2}=\mathrm{Id}_{\mathbb{C}^{2}}$
 $(g^{-1} = g, h^{-1}=h)$
 \begin{enumerate}
  \item
       $R_{8v}(0)=P$、
       $P(x\otimes y)=y\otimes x$ for any $x,y\in \mathbb{C}^{N}$
  \item
       $(g\otimes g)^{-1}R_{8v}(u)(g\otimes g)
       =(h\otimes h)^{-1}R_{8v}(u)(h\otimes h)
       =R_{8v}(u)$
  \item
       $R_{8v}(u+1)=\mathrm{e}[-1](g^{-1}\otimes 1)R_{8v}(u)(g\otimes 1)$
  \item
       $R_{8v}(u+\tau)
       =\mathrm{e}[-\tau-2(u+\frac{\eta}{2}+\frac{1}{2})]
       (h\otimes 1)R_{8v}(u)(h^{-1}\otimes 1)$
  \item
       $R_{8v}(u)$の各成分は
       $u$の正則関数である。
 \end{enumerate}

 \dotfill

 \begin{gather}
  R_{8v}(u)
  = \frac{1}{2}\sum_{\alpha,\beta=0}^{1} w_{\alpha,\beta}(u\mid \tau)
  h^{-\alpha}g^{\beta} \otimes g^{-\beta} h^{\alpha}
  \in \mathrm{End}(\mathbb{C}^{2}\otimes \mathbb{C}^{2})
  \\
  w_{\alpha,\beta}
  =\frac{\vartheta\begin{bmatrix}1/2+\alpha/2 \\ 1/2+\beta/2\end{bmatrix} (u+\frac{\eta}{2} \mid \tau)}
  {\vartheta\begin{bmatrix}1/2+\alpha/2 \\ 1/2+\beta/2\end{bmatrix} (\frac{\eta}{2} \mid \tau)}
 \end{gather}

\hrulefill

$R_{8v}(u)$がunitary関係式
\begin{equation}
 P R_{8v}(u) P R_{8v}(-u)
  = \rho(u) \mathrm{Id}_{\mathbb{C}^{2}\otimes \mathbb{C}^{2}}
\end{equation}
を満たすことを示せ。

\dotfill

$P,R_{8v} \in \mathrm{End}(\mathbb{C}^{2}\otimes \mathbb{C}^{2})$は
次の様に表せる。
\begin{gather}
 P = \begin{pmatrix}1 &&& \\ &0&1&\\  &1&0&\\  &&&1 \end{pmatrix}
 \in \mathrm{End}(\mathbb{C}^{2}\otimes \mathbb{C}^{2})
 %
 \\
 %
 R_{8v}(u)=
  \begin{pmatrix} a_{u} &&& d_{u} \\ &b_{u}&c_{u}&\\  &c_{u}&b_{u}&\\ d_{u}&&&a_{u} \end{pmatrix}
  \in \mathrm{End}(\mathbb{C}^{2}\otimes \mathbb{C}^{2})
\end{gather}

ただし、$a_{u},b_{u},c_{u},d_{u}$は次を表している。
\begin{gather}
 a_{u} = a(u\mid\tau)
 = \frac{\theta_{4}(u\mid 2\tau) \theta_{1}(u+\eta\mid 2\tau)}
 {\theta_{4}(0\mid 2\tau) \theta_{1}(\eta\mid 2\tau)}
 %
 \\
 %
 b_{u} = b(u\mid\tau)
 = \frac{\theta_{1}(u\mid 2\tau) \theta_{4}(u+\eta\mid 2\tau)}
 {\theta_{4}(0\mid 2\tau) \theta_{1}(\eta\mid 2\tau)}
 %
 \\
 %
 c_{u} = c(u\mid\tau)
 = \frac{\theta_{4}(u\mid 2\tau) \theta_{4}(u+\eta\mid 2\tau)}
 {\theta_{4}(0\mid 2\tau) \theta_{4}(\eta\mid 2\tau)}
 %
 \\
 %
 d_{u} = d(u\mid\tau)
 = \frac{\theta_{1}(u\mid 2\tau) \theta_{1}(u+\eta\mid 2\tau)}
 {\theta_{4}(0\mid 2\tau) \theta_{4}(\eta\mid 2\tau)}
\end{gather}

$P R_{8v}(u) P R_{8v}(-u)$を計算する。
\begin{align}
& P R_{8v}(u) P R_{8v}(-u) \\
  &=
  \begin{pmatrix}1 &&& \\ &0&1&\\  &1&0&\\  &&&1 \end{pmatrix}
  \begin{pmatrix} a_{u} &&& d_{u} \\ &b_{u}&c_{u}&\\  &c_{u}&b_{u}&\\ d_{u}&&&a_{u} \end{pmatrix}
  \begin{pmatrix}1 &&& \\ &0&1&\\  &1&0&\\  &&&1 \end{pmatrix}
  \begin{pmatrix} a_{-u} &&& d_{-u} \\ &b_{-u}&c_{-u}&\\  &c_{-u}&b_{-u}&\\ d_{-u}&&&a_{-u} \end{pmatrix}\\
 &=
 \begin{pmatrix}
 a_{u}a_{-u}+d_{u}d_{-u} & & & a_{u}d_{-u}+a_{-u}d_{u} \\
 & b_{u}b_{-u}+c_{u}c_{-u} & b_{u}c_{-u}+b_{-u}c_{u} & \\
 & b_{u}c_{-u}+b_{-u}c_{u} & b_{u}b_{-u}+c_{u}c_{-u} & \\
 a_{u}d_{-u}+a_{-u}d_{u} & & & a_{u}a_{-u}+d_{u}d_{-u}
 \end{pmatrix}
\end{align}

$\theta_{1}$は奇関数で、$\theta_{2},\theta_{3},\theta_{4}$は偶関数である。
これを用いて
$a_{u}d_{-u}+a_{-u}d_{u}$%と$b_{u}c_{-u}+b_{-u}c_{u}$は
は次の様に求まる。
\begin{align}
 a_{u}d_{-u}
  &=\frac{\theta_{4}(u\mid 2\tau) \theta_{1}(u+\eta\mid 2\tau)
     \theta_{1}(-u\mid 2\tau) \theta_{1}(-u+\eta\mid 2\tau)}
   {\theta_{4}(0\mid 2\tau) \theta_{1}(\eta\mid 2\tau)
     \theta_{4}(0\mid 2\tau) \theta_{4}(\eta\mid 2\tau)}\\
 &=-\frac{\theta_{4}(u\mid 2\tau) \theta_{1}(u+\eta\mid 2\tau)
     \theta_{1}(u\mid 2\tau) \theta_{1}(-u+\eta\mid 2\tau)}
   {\theta_{4}(0\mid 2\tau) \theta_{1}(\eta\mid 2\tau)
     \theta_{4}(0\mid 2\tau) \theta_{4}(\eta\mid 2\tau)}\\
 %
\\
%
a_{-u}d_{u}
  &=\frac{\theta_{4}(-u\mid 2\tau) \theta_{1}(-u+\eta\mid 2\tau)
     \theta_{1}(u\mid 2\tau) \theta_{1}(u+\eta\mid 2\tau)}
   {\theta_{4}(0\mid 2\tau) \theta_{1}(\eta\mid 2\tau)
 \theta_{4}(0\mid 2\tau) \theta_{4}(\eta\mid 2\tau)}\\
 &= \frac{\theta_{4}(u\mid 2\tau) \theta_{1}(-u+\eta\mid 2\tau)
     \theta_{1}(u\mid 2\tau) \theta_{1}(u+\eta\mid 2\tau)}
   {\theta_{4}(0\mid 2\tau) \theta_{1}(\eta\mid 2\tau)
 \theta_{4}(0\mid 2\tau) \theta_{4}(\eta\mid 2\tau)}
\end{align}

\begin{equation}
 a_{u}d_{-u} +a_{-u}d_{u} =0
\end{equation}

同様に
$b_{u}c_{-u}+b_{-u}c_{u} =0$
であることが求まる。


よって、次のような行列となる。
\begin{align}
& P R_{8v}(u) P R_{8v}(-u) \\
 &=
 \begin{pmatrix}
 a_{u}a_{-u}+d_{u}d_{-u} & & & \\
 & b_{u}b_{-u}+c_{u}c_{-u} & & \\
 &  & b_{u}b_{-u}+c_{u}c_{-u} & \\
  & & & a_{u}a_{-u}+d_{u}d_{-u}
 \end{pmatrix}
\end{align}

$a_{u}a_{-u}+d_{u}d_{-u}$と
$b_{u}b_{-u}+c_{u}c_{-u}$は
$u$の関数として偶関数である。
その為、
$P R_{8v}(u) P R_{8v}(-u)$
は偶関数で表すことが出来る。

\begin{equation}
 P R_{8v}(u) P R_{8v}(-u)
  =\rho(u) \mathrm{Id}_{\mathbb{C}^{2}\otimes \mathbb{C}^{2}}
\end{equation}


\dotfill

\begin{align}
 & a_{u}a_{-u}+d_{u}d_{-u}\\
  &= 
\frac{\theta_{4}(u\mid 2\tau) \theta_{1}(u+\eta\mid 2\tau)}
 {\theta_{4}(0\mid 2\tau) \theta_{1}(\eta\mid 2\tau)}
\frac{\theta_{4}(-u\mid 2\tau) \theta_{1}(-u+\eta\mid 2\tau)}
 {\theta_{4}(0\mid 2\tau) \theta_{1}(\eta\mid 2\tau)}
 \nonumber \\
 & \quad +
 \frac{\theta_{1}(u\mid 2\tau) \theta_{1}(u+\eta\mid 2\tau)}
 {\theta_{4}(0\mid 2\tau) \theta_{4}(\eta\mid 2\tau)}
 \frac{\theta_{1}(-u\mid 2\tau) \theta_{1}(-u+\eta\mid 2\tau)}
 {\theta_{4}(0\mid 2\tau) \theta_{4}(\eta\mid 2\tau)}
 \\
 &=
 \frac{\theta_{1}(u+\eta\mid 2\tau)\theta_{1}(-u+\eta\mid 2\tau)}{(\theta_{4}(0\mid 2\tau) \theta_{1}(\eta\mid 2\tau) \theta_{4}(\eta\mid 2\tau) )^{2}}
 ( (\theta_{4}(u\mid 2\tau) \theta_{4}(\eta\mid 2\tau) )^{2} - (\theta_{1}(u\mid 2\tau) \theta_{1}(\eta\mid 2\tau))^{2})
\end{align}


\begin{align}
 & b_{u}b_{-u}+c_{u}c_{-u}\\
 &=
  \frac{\theta_{1}(u\mid 2\tau) \theta_{4}(u+\eta\mid 2\tau)}
 {\theta_{4}(0\mid 2\tau) \theta_{1}(\eta\mid 2\tau)}
 \frac{\theta_{1}(-u\mid 2\tau) \theta_{4}(-u+\eta\mid 2\tau)}
 {\theta_{4}(0\mid 2\tau) \theta_{1}(\eta\mid 2\tau)} \nonumber \\
 & \quad +
  \frac{\theta_{4}(u\mid 2\tau) \theta_{4}(u+\eta\mid 2\tau)}
 {\theta_{4}(0\mid 2\tau) \theta_{4}(\eta\mid 2\tau)}
 \frac{\theta_{4}(-u\mid 2\tau) \theta_{4}(-u+\eta\mid 2\tau)}
 {\theta_{4}(0\mid 2\tau) \theta_{4}(\eta\mid 2\tau)}
 \\
 &=
 \frac{\theta_{4}(u+\eta\mid 2\tau)\theta_{4}(-u+\eta\mid 2\tau)}{(\theta_{4}(0\mid 2\tau) \theta_{1}(\eta\mid 2\tau) \theta_{4}(\eta\mid 2\tau) )^{2}}
 ( -(\theta_{1}(u\mid 2\tau) \theta_{4}(\eta\mid 2\tau) )^{2} + (\theta_{4}(u\mid 2\tau) \theta_{1}(\eta\mid 2\tau))^{2})
\end{align}



\dotfill


\begin{gather}
 \vartheta
  \begin{bmatrix}
   a \\ b
  \end{bmatrix}
  (u \mid \tau)
  =
  \sum_{n\in\mathbb{Z}}
  \mathbf{e}
  [(n + a)^{2} + 2(n + a)(u + b)]\\
 a,b\in\mathbb{R},\quad
 u\in\mathbb{C},\quad
 \tau\in\mathbb{H}=\{z\in\mathbb{C} \mid \mathrm{Im}z \geq 0\},\quad
 \mathbf{e}[x] \overset{\mathrm{def}}{=} \exp{(i\pi x)}
\end{gather}


\begin{align}
 \theta_{1}(u\mid\tau)
  &=
  -\vartheta
  \begin{bmatrix}
   1/2 \\ 1/2
  \end{bmatrix}
  (u +\tau \mid \tau)
 = \sum_{n\in\mathbb{Z}}
  \mathbf{e}
  [(n + 1/2)^{2} + 2(n + 1/2)(u + 1/2)]
 %
 \\
 %
 \theta_{4}(u\mid\tau)
  &=
  \vartheta
  \begin{bmatrix}
   0 \\ 1/2
  \end{bmatrix}
  (u +\tau \mid \tau)
 = \sum_{n\in\mathbb{Z}}
  \mathbf{e}
  [n^{2} + 2n(u + 1/2)]
\end{align}

\dotfill

\begin{align}
 & \theta_{1}(\alpha\mid\tau)
 \theta_{1}(\beta\mid\tau)\\
 &=
 \sum_{n\in\mathbb{Z}}
 \mathbf{e}[(n + 1/2)^{2} + 2(n + 1/2)(\alpha + 1/2)]
 \sum_{m\in\mathbb{Z}}
 \mathbf{e}[(m + 1/2)^{2} + 2(m + 1/2)(\beta + 1/2)]
 \\
 &=
 \sum_{n,m\in\mathbb{Z}}
 \mathbf{e}[
 (n + 1/2)^{2} + 2(n + 1/2)(\alpha + 1/2)
 +
 (m + 1/2)^{2} + 2(m + 1/2)(\beta + 1/2)
 ]\\
 &=
 \sum_{n,m\in\mathbb{Z}}
 \mathbf{e}[
 (n + 1/2)^{2} + (m + 1/2)^{2}
 + 2(n + 1/2)(\alpha + 1/2)
 +
 2(m + 1/2)(\beta + 1/2)
 ]\\
 &=
 \sum_{n,m\in\mathbb{Z}}
 \mathbf{e}[
 n^{2} + 2n + m^{2} + 2m
 + 2n\alpha  + 2m\beta
 + \alpha + \beta + 3/2
 ]
\end{align}

\begin{align}
 & \theta_{1}(\alpha\mid\tau)
 \theta_{4}(\beta\mid\tau)\\
 &= -
 \sum_{n\in\mathbb{Z}}
 \mathbf{e}[(n + 1/2)^{2} + 2(n + 1/2)(\alpha + 1/2)]
 \sum_{m\in\mathbb{Z}}
 \mathbf{e}[m^{2} + 2m(\beta + 1/2)]
 \\
 &=
 \sum_{n,m\in\mathbb{Z}}
 \mathbf{e}[
 (n + 1/2)^{2} + 2(n + 1/2)(\alpha + 1/2)
 +
 m^{2} + 2m(\beta + 1/2)
 ]\\
 &=
 \sum_{n,m\in\mathbb{Z}}
 \mathbf{e}[
 n^{2} + 2n + m^{2} + m
 + 2n\alpha  + 2m\beta
 + \alpha + 3/4
 ]
\end{align}


\begin{align}
 & \theta_{4}(\alpha\mid\tau)
 \theta_{4}(\beta\mid\tau)\\
 &= -
 \sum_{n\in\mathbb{Z}}
 \mathbf{e}[n^{2} + 2n(\alpha + 1/2)]
 \sum_{m\in\mathbb{Z}}
 \mathbf{e}[m^{2} + 2m(\beta + 1/2)]
 \\
 &=
 \sum_{n,m\in\mathbb{Z}}
 \mathbf{e}[
 n^{2} + 2n(\alpha + 1/2)
 +
 m^{2} + 2m(\beta + 1/2)
 ]\\
 &=
 \sum_{n,m\in\mathbb{Z}}
 \mathbf{e}[
 n^{2} + n + m^{2} + m
 + 2n\alpha  + 2m\beta
 ]
\end{align}


\hrulefill

\end{document}
