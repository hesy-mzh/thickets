\documentclass[12pt,b5paper]{ltjsarticle}

%\usepackage[margin=15truemm, top=5truemm, bottom=5truemm]{geometry}
%\usepackage[margin=10truemm,left=15truemm]{geometry}
\usepackage[margin=10truemm]{geometry}

\usepackage{amsmath,amssymb}
%\pagestyle{headings}
\pagestyle{empty}

%\usepackage{listings,url}
%\renewcommand{\theenumi}{(\arabic{enumi})}

%\usepackage{graphicx}

%\usepackage{tikz}
%\usetikzlibrary {arrows.meta}
%\usepackage{wrapfig}	% required for `\wrapfigure' (yatex added)
%\usepackage{bm}	% required for `\bm' (yatex added)

% ルビを振る
%\usepackage{luatexja-ruby}	% required for `\ruby'

%% 核Ker 像Im Hom を定義
%\newcommand{\Img}{\mathop{\mathrm{Im}}\nolimits}
%\newcommand{\Ker}{\mathop{\mathrm{Ker}}\nolimits}
%\newcommand{\Hom}{\mathop{\mathrm{Hom}}\nolimits}

%\DeclareMathOperator{\Rot}{rot}
%\DeclareMathOperator{\Div}{div}
%\DeclareMathOperator{\Grad}{grad}
%\DeclareMathOperator{\arcsinh}{arcsinh}
%\DeclareMathOperator{\arccosh}{arccosh}
%\DeclareMathOperator{\arctanh}{arctanh}



%\usepackage{listings,url}
%
%\lstset{
%%プログラム言語(複数の言語に対応,C,C++も可)
%  language = Python,
%%  language = Lisp,
%%  language = C,
%  %背景色と透過度
%  %backgroundcolor={\color[gray]{.90}},
%  %枠外に行った時の自動改行
%  breaklines = true,
%  %自動改行後のインデント量(デフォルトでは20[pt])
%  breakindent = 10pt,
%  %標準の書体
%%  basicstyle = \ttfamily\scriptsize,
%  basicstyle = \ttfamily,
%  %コメントの書体
%%  commentstyle = {\itshape \color[cmyk]{1,0.4,1,0}},
%  %関数名等の色の設定
%  classoffset = 0,
%  %キーワード(int, ifなど)の書体
%%  keywordstyle = {\bfseries \color[cmyk]{0,1,0,0}},
%  %表示する文字の書体
%  %stringstyle = {\ttfamily \color[rgb]{0,0,1}},
%  %枠 "t"は上に線を記載, "T"は上に二重線を記載
%  %他オプション:leftline,topline,bottomline,lines,single,shadowbox
%  frame = TBrl,
%  %frameまでの間隔(行番号とプログラムの間)
%  framesep = 5pt,
%  %行番号の位置
%  numbers = left,
%  %行番号の間隔
%  stepnumber = 1,
%  %行番号の書体
%%  numberstyle = \tiny,
%  %タブの大きさ
%  tabsize = 4,
%  %キャプションの場所("tb"ならば上下両方に記載)
%  captionpos = t
%}



\begin{document}

\hrulefill

\textbf{線形代数 直交行列}

\begin{equation}
 T:\text{直交行列}
 \Leftrightarrow
 T^2:\text{直交行列}
\end{equation}

\dotfill

$T:\text{直交行列}\Rightarrow T^2:\text{直交行列}$

$T$は直交行列であるので、
${}^{t}T=T^{-1}$である。

\begin{equation}
 {}^{t}(T^2)T^2
  = {}^{t}T{}^{t}TTT
  = {}^{t}TET
  = {}^{t}TT
  = E
\end{equation}

同様に$T^2\ {}^{t}(T^2)=E$であるので、
$T^2$は直交行列である。

\dotfill

$T:\text{直交行列} \Leftarrow T^2:\text{直交行列}$

$T^2$は直交行列であるので、
${}^{t}(T^2)=(T^2)^{-1}$である。

\begin{align}
 E
 =& {}^{t}(T^2)T^2
 = {}^{t}(TT)TT\\
 T^{-1}T^{-1}=& {}^{t}T {}^{t}T\\
 (T^{-1})^{2}=& ({}^{t}T)^2
\end{align}

正則行列全体の集合$GL(n)$上の写像$f$を次のように定める。
\begin{equation}
 f: GL(n) \to GL(n),
  \quad
  A\mapsto A^2
\end{equation}
この写像は
$f^{-1}(0)=\{0\}$
より単射である。
よって、
$(T^{-1})^{2}= ({}^{t}T)^2$
から
$T^{-1}= {}^{t}T$
が得られる。

つまり、
$T$は直交行列である。


\hrulefill

\end{document}
