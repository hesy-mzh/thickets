\documentclass[12pt,b5paper]{ltjsarticle}

%\usepackage[margin=15truemm, top=5truemm, bottom=5truemm]{geometry}
\usepackage[margin=15truemm]{geometry}

\usepackage{amsmath,amssymb}
%\pagestyle{headings}
\pagestyle{empty}

%\usepackage{listings,url}
%\renewcommand{\theenumi}{(\arabic{enumi})}

\usepackage{graphicx}

\usepackage{tikz}
\usetikzlibrary {arrows.meta}
\usepackage{wrapfig}	% required for `\wrapfigure' (yatex added)
\usepackage{bm}	% required for `\bm' (yatex added)
\usepackage{luatexja-ruby}	% required for `\ruby'
%% 像Im を定義
%\newcommand{\Img}{\mathop{\mathrm{Im}}\nolimits}

\begin{document}

区間$[a,b],\ [c,d]\subset\mathbb{R}$に対して、
領域$D\subset\mathbb{R}^2$を$D=[a,b]\times[c,d]$とする。

関数$f:D\rightarrow \mathbb{R}$が連続ならば、
$f$は$D$上積分可能である。

\hrulefill


領域$D$を小領域$D_i$に分割し、
$D_i$の面積を$S_i$とする。
$D_i$から任意の点$P_i$を取る。
$D_i$上の体積$f(P_i)\times S_i$とすれば、
$D$の全ての分割上の体積の和をリーマン和という。
\begin{equation}
 \sum_{i=1}^n f(P_i)\times S_i
\end{equation}

この分割を細かくし、小領域の個数$n$を無限に飛ばす。
リーマン和がある値に収束するときに積分可能であるという。

\dotfill

領域$D$を小領域に分割する。
$x_i\in [a,b] \, (i=0,\dots,n)$と
$y_i\in [c,d] \, (i=0,\dots,m)$を次のような範囲の値とする。
\begin{gather}
 a = x_0 < x_1 < \cdots < x_{n-1} < x_n = b\\
 c = y_0 < y_1 < \cdots < y_{m-1} < y_m = d
\end{gather}
これにより小領域$D_{ij}$を定める。
\begin{equation}
 D_{ij} = [x_{i-1},x_i]\times[y_{j-1},y_j]
\end{equation}

小領域$D_{ij}$の面積を$\lvert D_{ij}\lvert=(x_i-x_{i-1})(y_j-y_{j-1})$とし、
$D_{ij}$の任意の点を取り出し$P_{ij}$とする。
これによりリーマン和は次のようになる。
\begin{equation}
 \sum_{D_{ij}\subset D} f(P_{ij})\lvert D_{ij}\lvert
\end{equation}


%$f$が連続であれば、$D_{ij}$と$P_{ij}$の取り方によらずリーマン和の極限値が一つに定まる。


閉集合$D_{ij}$上の連続関数$f$は上限下限が$D_{ij}$に存在し、
これらを$\sup_{D_{ij}} f$、
$\inf_{D_{ij}} f$とする。
これによりリーマン和を次のように書き換える。
\begin{gather}
 \overline{S}=\sum_{i,j} \sup_{D_{ij}} f\lvert D_{ij}\lvert\\
 \underline{S}=\sum_{i,j} \inf_{D_{ij}} f\lvert D_{ij}\lvert
\end{gather}
これは次のような関係が成り立つ。
\begin{equation}
 \underline{S} \leq
  \sum_{D_{ij}\subset D} f(P_{ij})\lvert D_{ij}\lvert
  \leq \overline{S}
\end{equation}

各分割$D_{ij}$の面積の最大値を$m$とする。
\begin{equation}
 \delta = \max\{ \lvert D_{ij} \rvert \mid 1\leq x\leq n,\ 1\leq y\leq m \}
\end{equation}

この時、$x$と$y$の分割数$n,m$を大きくすることで$D_{ij}$の面積を小さくしていくと
次のような極限になる。
\begin{equation}
 \lim_{\delta\rightarrow 0} \underline{S}
  =
  \lim_{\delta\rightarrow 0} \overline{S}
\end{equation}

はさみうちの原理からリーマン和も極限値を持つので積分可能であることが分かる。


\end{document}
