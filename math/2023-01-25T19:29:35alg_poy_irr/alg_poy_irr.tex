\documentclass[12pt,b5paper]{ltjsarticle}

%\usepackage[margin=15truemm, top=5truemm, bottom=5truemm]{geometry}
%\usepackage[margin=10truemm,left=15truemm]{geometry}
\usepackage[margin=10truemm]{geometry}

\usepackage{amsmath,amssymb}
%\pagestyle{headings}
\pagestyle{empty}

%\usepackage{listings,url}
%\renewcommand{\theenumi}{(\arabic{enumi})}

%\usepackage{graphicx}

%\usepackage{tasks}

%\usepackage{tikz}
%\usetikzlibrary {arrows.meta}
%\usepackage{wrapfig}	% required for `\wrapfigure' (yatex added)
\usepackage{bm}	% required for `\bm' (yatex added)

% ルビを振る
%\usepackage{luatexja-ruby}	% required for `\ruby'

% カラー表示
\usepackage{color}

%% 核Ker 像Im Hom を定義
%\newcommand{\Img}{\mathop{\mathrm{Im}}\nolimits}
%\newcommand{\Ker}{\mathop{\mathrm{Ker}}\nolimits}
%\newcommand{\Hom}{\mathop{\mathrm{Hom}}\nolimits}

%\DeclareMathOperator{\Rot}{rot}
%\DeclareMathOperator{\Div}{div}
%\DeclareMathOperator{\Grad}{grad}
%\DeclareMathOperator{\arcsinh}{arcsinh}
%\DeclareMathOperator{\arccosh}{arccosh}
%\DeclareMathOperator{\arctanh}{arctanh}



%\usepackage{listings,url}
%
%\lstset{
%%プログラム言語(複数の言語に対応,C,C++も可)
%  language = Python,
%%  language = Lisp,
%%  language = C,
%  %背景色と透過度
%  %backgroundcolor={\color[gray]{.90}},
%  %枠外に行った時の自動改行
%  breaklines = true,
%  %自動改行後のインデント量(デフォルトでは20[pt])
%  breakindent = 10pt,
%  %標準の書体
%%  basicstyle = \ttfamily\scriptsize,
%  basicstyle = \ttfamily,
%  %コメントの書体
%%  commentstyle = {\itshape \color[cmyk]{1,0.4,1,0}},
%  %関数名等の色の設定
%  classoffset = 0,
%  %キーワード(int, ifなど)の書体
%%  keywordstyle = {\bfseries \color[cmyk]{0,1,0,0}},
%  %表示する文字の書体
%  %stringstyle = {\ttfamily \color[rgb]{0,0,1}},
%  %枠 "t"は上に線を記載, "T"は上に二重線を記載
%  %他オプション:leftline,topline,bottomline,lines,single,shadowbox
%  frame = TBrl,
%  %frameまでの間隔(行番号とプログラムの間)
%  framesep = 5pt,
%  %行番号の位置
%  numbers = left,
%  %行番号の間隔
%  stepnumber = 1,
%  %行番号の書体
%%  numberstyle = \tiny,
%  %タブの大きさ
%  tabsize = 4,
%  %キャプションの場所("tb"ならば上下両方に記載)
%  captionpos = t
%}



\begin{document}



\textbf{既約多項式}

\dotfill

$f\in R$が
単元ではない2つの元$a,b\in R$が存在し
$f=ab$とかけるとき、
$f$を可約といい、
そのような$a,b\in R$が存在しないとき
$f$を既約という。


例えば、単項式$2x$は$\mathbb{Z}[x]$において
$2x=2\times x$であり、$2,x\in\mathbb{Z}[x]$は単元ではない。
つまり、$2\times \alpha =1$や$x\times \beta =1$となる
$\alpha,\beta$は$\mathbb{Z}[x]$に存在しない。
その為、$2x$は$\mathbb{Z}[x]$において可約である。

同様に$2x\in\mathbb{Q}[x]$は単元でない物の積に表せない。
よって、$2x$は$\mathbb{Q}[x]$において既約である。

単元でないことが意味を持つのは
次のような無限の表現を持つものを排除するためである。
\begin{equation}
 2x \ = \ 2\cdot 2^{-1}\cdot 2 \cdot x\
  =\ 2\cdot 2^{-1}\cdot 2 \cdot 2^{-1}\cdot 2  x\
  = \cdots
\end{equation}


\hrulefill

\hrulefill

$f\in\mathbb{Z}[x]$を以下のように定める。
$f$が$\mathbb{Z}[x]$と$\mathbb{Q}[x]$それぞれにおいて
既約であるか否か判定せよ。

\begin{minipage}{200pt}
 \begin{enumerate}
  \item $-x-1$
  \item $4x$
  \item $2x^2+4$
  \item $x^2+2x+4$
  \item $x^3+5x+6$
  \item $7x^3+10x^2+20$
  \item $x^{13}-1$
  \item $\sum_{i=0}^{17}x^i$
  \item $\sum_{i=13}^{22}x^i$
  \item $187x^{80}-306x^{39}+204$
 \end{enumerate}
\end{minipage}
\begin{minipage}{100pt}
 \begin{tabular}{|c|c|c|}
  & $\mathbb{Z}[x]$ & $\mathbb{Q}[x]$ \\
  \hline
  $-x-1$ & 既約 & 既約 \\
  $4x$ & 可約 & 既約 \\
  $2x^2+4$ & 可約 & 既約 \\
  $x^2+2x+4$ & 既約 & 既約 \\
  $x^3+5x+6$ & 可約 & 可約 \\
  $7x^3+10x^2+20$ & 既約 & 既約 \\
  $x^{13}-1$ & 可約 & 可約 \\
  $\sum_{i=0}^{17}x^i$ & 可約 & 可約 \\
  $\sum_{i=13}^{22}x^i$ & 可約 & 可約 \\
  $187x^{80}-306x^{39}+204$ & 可約 & 既約
 \end{tabular}
\end{minipage}


\dotfill

\begin{enumerate}
 \item $-x-1$

       $-x-1$は1次式である為、
       $-x-1 = ab$となる$a,b$があれば
       $\deg{a}=0,\deg{b}=1$となる。
       $b=\alpha x + \beta$とすれば
       $a\alpha =-1$を満たすような$a$となるが、
       $a\times (-\alpha)=1$を満たす為に
       $a$は単元となる。

       よって、$-x-1$は$\mathbb{Z}[x]$において既約である。

       $-x-1$は1次式である為$\mathbb{Q}[x]$においても既約である。


 \item $4x$

       $4\in\mathbb{Z}$は単元ではないので、
       $4x=4\times x$である為、
       $4x$は$\mathbb{Z}[x]$において可約である。

       $4x$は1次式である為$\mathbb{Q}[x]$において既約である。

 \item $2x^2+4$

       $2x^2+4=2(x^2+2)$であるので、
       $2x^2+4$は$\mathbb{Z}[x]$において可約である。

       $x^2+2$がある$f,g\in\mathbb{Q}[x]$
       を用いて
       $x^2+2=fg$とかけたとする。
       $\deg{f}=\deg{g}=1$となる式があれば可約となる。

       最高次の項の係数が1である(モニック多項式)ので、
       $f=x+\alpha,g=x+\beta$となる式を用いれば
       $x^2+2=fg=(x+\alpha)(x+\beta)=x^2+(\alpha+\beta)x+\alpha\beta$
       である。
       つまり、$\alpha,\beta$は
       $\alpha+\beta=0,\alpha\beta=2$の2式を満たす。

       しかし、これを満たす有理数は存在しないので
       $x^2+2$は$\mathbb{Q}[x]$において既約である。
       これより、$2x^2+4$も$\mathbb{Q}[x]$において既約である。


 \item $x^2+2x+4$

       $x^2+2x+4$はモニック多項式であり、
       2次式であるので
       $x^2+2x+4=fg$となる多項式が$\deg{f}=0,\deg{g}=2$となるのであれば
       $f$は単元となる。
       この場合、可約とはならない。

       $x^2+2x+4=(x+\alpha)(x+\beta)$となる場合を考える。
       $(x+\alpha)(x+\beta)=x^2+(\alpha+\beta)x+\alpha\beta$であるので
       $\alpha,\beta$は$\alpha+\beta=2,\alpha\beta=4$を満たす。

       $\mathbb{Z}$で考えれば、
       $\alpha\beta=4$より
       $(\alpha,\beta)=(1,4),(2,2),(4,1)$とその$-1$倍したものとなるが、
       これらはすべて$\alpha+\beta=2$を満たさない。
       よって、$x^2+2x+4=(x+\alpha)(x+\beta)$となることはなく、
       $x^2+2x+4$は$\mathbb{Z}[x]$において既約である。

       有理数の範囲で考えても
       $\alpha,\beta$は
       $\alpha+\beta=2,\alpha\beta=4$を満たすので、
       $\beta=2-\alpha$を$\alpha\beta=4$に代入して出来る
       $\alpha^2-2\alpha+4=0$を満たす。
       この方程式の解は$\alpha=1\pm\sqrt{3}i$となり有理数ではない。
       よって、
       $x^2+2x+4$は$\mathbb{Q}[x]$においても既約である。
       
 \item $x^3+5x+6$

       $x^3+5x+6 ={\left(x + 1\right)}{\left(x^{2} - x + 6\right)}$
       である。

       $x + 1,\ x^{2} - x + 6 \in\mathbb{Z}[x]$
       であり、
       $x + 1,\ x^{2} - x + 6 \in\mathbb{Q}[x]$
       であるので、
       $x^3+5x+6$は
       $\mathbb{Z}[x]$でも
       $\mathbb{Q}[x]$でも
       可約である。

 \item $7x^3+10x^2+20$

       7は5で割り切れないが、10と20は5で割り切れる。
       また、20は$5^2$で割り切れない。
       アイゼンシュタインの既約判定法により
       $7x^3+10x^2+20$は
       $\mathbb{Q}[x]$で既約である。

       $7,10,20$は互いに素であるので、
       $7x^3+10x^2+20$は原始多項式である。
       原始多項式が
       $\mathbb{Z}[x]$で可約なら
       $\mathbb{Q}[x]$で可約である。
       その為、その対偶をとれば
       $\mathbb{Q}[x]$で既約なら
       $\mathbb{Z}[x]$で既約である。
       よって、
       $7x^3+10x^2+20$は
       $\mathbb{Q}[x]$で既約であるので、
       $\mathbb{Z}[x]$で既約である。


 \item $x^{13}-1$

       $x^{13}-1 = (x - 1){\left( \sum_{i=0}^{12}x^i\right)}$
       である。
       $x - 1$や${\left( \sum_{i=0}^{12}x^i\right)}$は
       $\mathbb{Z}[x]$や$\mathbb{Q}[x]$の元であるので、
       $x^{13}-1$は
       $\mathbb{Z}[x]$でも
       $\mathbb{Q}[x]$でも
       可約である。

 \item $\sum_{i=0}^{17}x^i$

       $x=-1$のとき、
       $\sum_{i=0}^{17}x^i=0$であるので、
       $x+1$で割り切れる。
       具体的には次のようになる。
       \begin{equation}
        \sum_{i=0}^{17}x^i
         =(x+1)\left( \sum_{i=0}^{8}x^{2i} \right)
       \end{equation}

       よって、
       $\sum_{i=0}^{17}x^i$は
       $\mathbb{Z}[x]$でも
       $\mathbb{Q}[x]$でも
       可約である。

 \item $\sum_{i=13}^{22}x^i$

       $\sum_{i=13}^{22}x^i = x^{13}\sum_{i=0}^{9}x^i$
       である。
       $x^{13}$や$\sum_{i=0}^{9}x^i$は
       $\mathbb{Z}[x]$や$\mathbb{Q}[x]$の元であるので、
       $\sum_{i=13}^{22}x^i$は
       $\mathbb{Z}[x]$でも
       $\mathbb{Q}[x]$でも
       可約である。


 \item $187x^{80}-306x^{39}+204$

       $187x^{80}-306x^{39}+204 = 17(11x^{80}-18x^{39}+12)$
       である。
       よって、
       $187x^{80}-306x^{39}+204$は
       $\mathbb{Z}[x]$において可約である。

       $11x^{80}-18x^{39}+12$
       は
       $18$や$12$が$3$で割り切れるが、$11$は$3$で割り切れない。
       また、$12$は$3^2$でも割り切れない。
       アイゼンシュタインの既約判定法
       より、
       $11x^{80}-18x^{39}+12$
       は
       $\mathbb{Q}[x]$において可約である。

\end{enumerate}

\hrulefill\

\begin{center}
%  \fbox{
    %\begin{minipage}{400pt}
    定数倍を除けば
    $\mathbb{Z}[x]$と
    $\mathbb{Q}[x]$の
    既約、可約は一致すると思われる。
     %\end{minipage}
%  }
\end{center}



\end{document}
