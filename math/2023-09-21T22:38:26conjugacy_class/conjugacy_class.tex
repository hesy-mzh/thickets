\documentclass[12pt,b5paper]{ltjsarticle}

%\usepackage[margin=15truemm, top=5truemm, bottom=5truemm]{geometry}
%\usepackage[margin=10truemm,left=15truemm]{geometry}
\usepackage[margin=10truemm]{geometry}

\usepackage{amsmath,amssymb}
%\pagestyle{headings}
\pagestyle{empty}

%\usepackage{listings,url}
%\renewcommand{\theenumi}{(\arabic{enumi})}

%\usepackage{graphicx}

%\usepackage{tikz}
%\usetikzlibrary {arrows.meta}
%\usepackage{wrapfig}
%\usepackage{bm}

% ルビを振る
%\usepackage{luatexja-ruby}	% required for `\ruby'

%% 核Ker 像Im Hom を定義
%\newcommand{\Img}{\mathop{\mathrm{Im}}\nolimits}
%\newcommand{\Ker}{\mathop{\mathrm{Ker}}\nolimits}
%\newcommand{\Hom}{\mathop{\mathrm{Hom}}\nolimits}

%\DeclareMathOperator{\Rot}{rot}
%\DeclareMathOperator{\Div}{div}
%\DeclareMathOperator{\Grad}{grad}
%\DeclareMathOperator{\arcsinh}{arcsinh}
%\DeclareMathOperator{\arccosh}{arccosh}
%\DeclareMathOperator{\arctanh}{arctanh}

%\usepackage{url}

%\usepackage{listings}
%
%\lstset{
%%プログラム言語(複数の言語に対応,C,C++も可)
%  language = Python,
%%  language = Lisp,
%%  language = C,
%  %背景色と透過度
%  %backgroundcolor={\color[gray]{.90}},
%  %枠外に行った時の自動改行
%  breaklines = true,
%  %自動改行後のインデント量(デフォルトでは20[pt])
%  breakindent = 10pt,
%  %標準の書体
%%  basicstyle = \ttfamily\scriptsize,
%  basicstyle = \ttfamily,
%  %コメントの書体
%%  commentstyle = {\itshape \color[cmyk]{1,0.4,1,0}},
%  %関数名等の色の設定
%  classoffset = 0,
%  %キーワード(int, ifなど)の書体
%%  keywordstyle = {\bfseries \color[cmyk]{0,1,0,0}},
%  %表示する文字の書体
%  %stringstyle = {\ttfamily \color[rgb]{0,0,1}},
%  %枠 "t"は上に線を記載, "T"は上に二重線を記載
%  %他オプション:leftline,topline,bottomline,lines,single,shadowbox
%  frame = TBrl,
%  %frameまでの間隔(行番号とプログラムの間)
%  framesep = 5pt,
%  %行番号の位置
%  numbers = left,
%  %行番号の間隔
%  stepnumber = 1,
%  %行番号の書体
%%  numberstyle = \tiny,
%  %タブの大きさ
%  tabsize = 4,
%  %キャプションの場所("tb"ならば上下両方に記載)
%  captionpos = t
%}

%\usepackage{cancel}
%\usepackage{bussproofs}
%\usepackage{proof}

\begin{document}

\hrulefill

$G$を群とする。

\textbf{共役類(conjugacy class)}

$a\in G$に対して、
$a$を含む共役類を$K(a)$を書く。
\begin{equation}
 K(a) = \{xax^{-1} \mid x\in G \}
\end{equation}

$a,b \in G$について$b=xax^{-1}$となる$x\in G$が存在するとき、
$b$は$a$に共役(conjugate)であるという。
ここでは共役であるとき$a\sim b$と書く。
共役は同値関係である。

\dotfill

\textbf{中心}

$G$の任意の元と可換な元全体の集合を$Z(G)$とかく。
\begin{equation}
 Z(G) = \{ a\in G \mid ab = ba \; ({}^{\forall}b\in G) \}
\end{equation}

$Z(G)$は$G$の正規部分群である。
この$Z(G)$を$G$の中心という。

\dotfill

\textbf{中心化群}

群$G$の部分集合$S$の中心化群$Z(S)$は次で定義される。
\begin{equation}
 Z(S) = \{ g\in G \mid gs=sg \; ({}^{\forall}s\in S)\}
\end{equation}

特に部分群が要素一つだけの集合$\{a\}$であるとき、
中心化群は$Z(a)$と書く。
\begin{equation}
 Z(a) = \{ g\in G \mid ga=ag \}
\end{equation}

\dotfill

\textbf{同値関係}

集合$S$において
次の3つの性質をすべて満たす関係を同値関係という
%任意の元$a,b,c\in S$について
\begin{itemize}
 \item \textbf{反射律} $a\sim a$
 \item \textbf{対称律} $a\sim b$ならば$b\sim a$
 \item \textbf{推移律} $a\sim b,\; b\sim c$ならば$a\sim c$
\end{itemize}


\dotfill

\textbf{正規部分群}

部分集合$N \subset G$について、
$gNg^{-1} \subset N \; ({}^{\forall}g\in G)$
が成り立つとき、$N$を$G$の正規部分といい、
$N\triangleleft G$と書く。


\hrulefill

$G$は群とする。
\begin{enumerate}
 \item 共役が同値関係であることを示せ。

       \dotfill

       $G$を群とし、$a,b\in G$とする。

       $a$と$b$が共役
       $\Leftrightarrow$
       ${}^{\forall}x\in G$
       $b=xax^{-1}$

       同値関係の3条件(反射律、対称律、推移律)を確認する。

       \textbf{反射律}

       $e\in G$を単位元とする。
       \begin{align}
        eae^{-1}
        & = eae
        = a
       \end{align}
       よって、$a\sim a$である。

       \textbf{対称律}
       \begin{align}
        a\sim b
        & \Rightarrow b=xax^{-1} \quad
         \Rightarrow x^{-1}bx = x^{-1}xax^{-1}x \quad
         \Rightarrow x^{-1}bx = a \\
        & \Rightarrow a = x^{-1}bx \quad
         \Rightarrow a = (x^{-1})b(x^{-1})^{-1} \\
        & \Rightarrow a = y b y^{-1} \quad (y=x^{-1}\in G) \quad
         \Rightarrow b\sim a
       \end{align}
       よって、$a\sim b$ ならば $b\sim a$である。

       \textbf{推移律}
       \begin{align}
        a\sim b, \; b\sim c
        & \Rightarrow b=xax^{-1}, \; c=xbx^{-1} &
        & \Rightarrow c = xxax^{-1}x^{-1} \\
        & \Rightarrow c = (xx)a(xx)^{-1} &
        & \Rightarrow a \sim c
       \end{align}
       よって、$a\sim b,\; b\sim c$ ならば $a\sim c$である。

       以上により、
       共役 $\sim$ は同値関係である。

       \hrulefill

 \item 次を示せ。
      \begin{enumerate}
       \item $a\in Z(G) \Rightarrow K(a)=\{a\}$。特に$K(e)=\{e\}$

             \dotfill

             $Z(G)$は群$G$の中心である。
             \begin{equation}
              Z(G) = \{ a\in G \mid ab = ba \; ({}^{\forall}b\in G) \}
             \end{equation}

             $a$の共役類$K(a)$は次のような集合である。
             \begin{equation}
              K(a) = \{xax^{-1} \mid x\in G \}
             \end{equation}

             $a\in Z(G)$より、
             $a$は$G$の任意の元と可換である。
             よって、$xax^{-1} = xx^{-1}a=ea=a$
             であるので、
             $K(a)$の元は$a$のみになる。

             単位元$e\in G$も$e\in Z(G)$であるから
             同様に$K(e)=\{e\}$である。

             \hrulefill

       \item $G \triangleright N \Rightarrow N$は
             $G$の共役類のいくつかの合併集合である。

             \dotfill

%             $n \in N$ とすると $n \in K(n)$である。
%             つまり、$\{n\} \subset K(n)$であるから
             ${}^{\forall} n\in N$に対して、
             ${}^{\exists}n^{\prime}\in N$が
             $n^{\prime} = xnx^{-1} \quad (x\in G)$となる。
             これに左から$x^{-1}$、右から$x$をかけると
             $(x^{-1})n^{\prime}(x^{-1})^{-1} = n$となり、
             $n \in K(n^{\prime})$となる。
             つまり、
             正規部分群の元は正規部分群の共役類のどれかに含まれる。
             \begin{equation}
              N \subset \bigcup_{n\in N} K(n)
             \end{equation}

             $n\in N$の共役類$K(n)$の定義は
             \begin{equation}
              K(n) = \{xnx^{-1} \mid x\in G \}
             \end{equation}
             であるから
             正規部分群$N$に含まれ、
             $K(n) \subset N$を満たす。

             つまり、
             \begin{equation}
              \bigcup_{n\in N} K(n) \subset N
             \end{equation}
             である。

             よって、次の式を満たす。
             \begin{equation}
              N= \bigcup_{n\in N} K(n)
             \end{equation}

             \hrulefill

      \end{enumerate}
 \item $a\in G$に対し、$f:G\to K(a)$を$f(x)=xax^{-1}$とする。
       このとき、次を示せ。
       \begin{enumerate}
        \item $f$は全射

              \dotfill

              任意の$K(a)$の元は$gag^{-1} \; (g\in G)$という形をしている。

              よって、$f(g)=gag^{-1}$となる$g\in G$が存在する為、
              $f$は全射である。

              \hrulefill

        \item $f(x)=f(y) \Leftrightarrow$ $x,y$は$G/Z(a)$の同じ類に属する
              \label{cll}

              \dotfill


              $f(x)=f(y)$とする。

              $f(x) = xax^{-1}, \; f(y)=yay^{-1}$より
              $xax^{-1}=yay^{-1}$である。

              右から$x$、左から$y^{-1}$をかけると
              $y^{-1}xa = ay^{-1}x$である。

              $y^{-1}xa = ay^{-1}x$より
              $y^{-1}x \in Z(a)$である。

              よって、
              $G/Z(a)$上で$y^{-1}x = e$である為、
              $x=y$である。

              これを逆にたどると
              $x,y$が$G/Z(a)$同じ類に属することから$f(x)=f(y)$を示せる。


              \hrulefill

       \end{enumerate}

%%%%%
 \item
      定理 1.7.1 より $|K(a)| = (G:Z(a))$ が示せたので、これを$h_{i}$と置いた。
      

      $(G:Z(a))$は $G$の元の数$|G|$を
      $Z(a)$の元の数$|Z(a)|$で割った値を意味するので、
      $(G:Z(a)) = |G| \div |Z(a)|$である。
      このとき、必ず割り切れるようになっている。

      $h_{i}=(G:Z(a_{i}))$と置けば、
      $h_{i}\times |Z(a_{i})| = |G|$であるので、
      $h_{i}$は$|G|$を割り切る数である。
%%%%%

      \hrulefill

 \item $g=|G|,\; h_{i}=K(a_{i})$とする。

       集合の直和$G = \bigsqcup_{i=1}^{t} K(a_{i})$から
       $g=h_{1}+\dots + h_{t}$である。

       \dotfill

       集合の直和$G = \bigsqcup_{i=1}^{t} K(a_{i})$は
       $G = \bigcup_{i} K(a_{i})$かつ
       任意の2つの元$a_{i},a_{j}$について
       $K(a_{i}) \cap K(a_{j}) = \emptyset$
       である。

       $G$の位数は
       $|K(a_{i})|$の和になっているので、
       $g=h_{1}+\dots + h_{t}$である。

       \hrulefill

 \item $h_{1}=|K(e)|$とすれば、$h_{1}=1$である。

       \dotfill

       $K(a)$は次のように定義されている。
       \begin{equation}
        K(a) = \{xax^{-1} \mid x\in G \}
       \end{equation}

       単位元$e\in G$は$G$の任意の元を常に可換である。
       つまり、${}^{\forall}g\in G$に対して
       $geg^{-1}=gg^{-1}e=ee=e$である。

       これにより
       $K(e)=\{e\}$
       であるから$|K(e)|=1$である。

       \hrulefill

\end{enumerate}




\textbf{同値関係について}

$a$と$b$が共役であることとは次のような定義です。
\begin{quote}
 群$G$の2つの元$a,b$に対して、
 $b=gag^{-1}$を満たす$g\in G$が存在する。
\end{quote}
共役であることを$a\sim b$とここでは書くことにします。
なお、$\sim$は同値関係でよく利用される記号です。


\textbf{対称律}
$a \sim b \Rightarrow b \sim a$

$a$と$b$が共役であれば
$b$と$a$が共役であるときに対称律を満たすという。
\begin{align}
 a \sim b \; \overset{\mathrm{def}}{\Longleftrightarrow} & \;
  \exists g \in G \;\; \mathrm{s.t.} \;\; b=gag^{-1}\\
 b \sim a \; \overset{\mathrm{def}}{\Longleftrightarrow} & \;
  \exists h \in G \;\; \mathrm{s.t.} \;\; a=hbh^{-1}
\end{align}

上記式が共役の定義からわかるので、
右側の関係性を調べる。
\begin{equation}
 b=gag^{-1}
  \quad \Rightarrow g^{-1}bg = g^{-1}gag^{-1}g
  \quad \Rightarrow g^{-1}bg = a
  \quad \Rightarrow a = g^{-1}bg
\end{equation}

$g\in G$より$g^{-1}\in G$であるから
$h=g^{-1}$と置くと
$a=hbh^{-1}$が得られる。

$h\in G$であるから$b$と$a$は共役($b\sim a$)である事がわかる。



\textbf{推移律}
$a \sim b \land b\sim c \Rightarrow a \sim c$

%$a \sim b$と$b\sim c$を満たすとき、$a \sim c$が得られるとき
$a$と$b$が共役であり、$b$と$c$が共役である時、
$a$と$c$が共役になる時に推移律を満たすという。
\begin{align}
 a \sim b \; \overset{\mathrm{def}}{\Longleftrightarrow} & \;
  \exists g \in G \;\; \mathrm{s.t.} \;\; b=gag^{-1}\\
 b \sim c \; \overset{\mathrm{def}}{\Longleftrightarrow} & \;
  \exists h \in G \;\; \mathrm{s.t.} \;\; c=hbh^{-1}\\
 a \sim c \; \overset{\mathrm{def}}{\Longleftrightarrow} & \;
  \exists i \in G \;\; \mathrm{s.t.} \;\; c=iai^{-1}
\end{align}

上記の上2つから3つ目を導き出せれば推移律を満たすことがわかる。

$b=gag^{-1},\; c=hbh^{-1}$から$b$を代入すると
$c=hgag^{-1}h^{-1}$が得られる。

$i=hg$と置くと$i^{-1}=(hg)^{-1} = g^{-1}h^{-1}$であるので、
$c=hgag^{-1}h^{-1} = iai^{-1}$となる。

$g,h\in G$より$gh, (gh)^{-1}\in G$であるので、
$i\in G$となる。

よって、$a\sim c$であり、$a$と$c$は共役であることがわかる。

\dotfill

\eqref{cll}の解説

\begin{align}
 f(x)=f(y) \quad
 & \Longleftrightarrow \quad xax^{-1} = yay^{-1}\\
 & \Longleftrightarrow \quad y^{-1}xa(y^{-1}x)^{-1} = a\\
 & \Longleftrightarrow \quad y^{-1}x \in Z(a)\\
 & \Longleftrightarrow \quad x,y \; は \; G/Z(a) \; の同じ類に属する
\end{align}

$f(x)=f(y) \Longleftrightarrow xax^{-1} = yay^{-1}$
\begin{quote}
 写像$f$の定義より$f(x)=xax^{-1},\;f(y)=yay^{-1}$である。

 これより$xax^{-1}=yay^{-1}$が得られる。

 逆に$xax^{-1}=yay^{-1}$であれば、
 写像$f$の定義より$f(x)=f(y)$である。
\end{quote}

$xax^{-1} = yay^{-1} \Longleftrightarrow y^{-1}xa(y^{-1}x)^{-1} = a$
\begin{quote}
 $xax^{-1} = yay^{-1}$の両辺に左から$y^{-1}$をかけると
 $y^{-1}xax^{-1} = ay^{-1}$となる。

 これに右から$y$をかけると
 $y^{-1}xax^{-1}y = a$となる。

 ここで、$x^{-1}y = (y^{-1}x)^{-1}$であるので、
 $y^{-1}xa(y^{-1}x)^{-1} = a$となる。

 逆に、$y^{-1}xa(y^{-1}x)^{-1} = a$であるとする。
 左から$y$、右から$y^{-1}$をかけることで
 $xax^{-1} = yay^{-1}$となる。
\end{quote}

$y^{-1}xa(y^{-1}x)^{-1} = a \Longleftrightarrow y^{-1}x \in Z(a)$
\begin{quote}
 $Z(a)$とは$a$と可換な$G$の元の全体の集合である。
 \begin{equation}
  Z(a) = \{ g\in G \mid ga=ag\}
 \end{equation}

 $g=y^{-1}x$と置くと、
 $y^{-1}xa(y^{-1}x)^{-1} = gag^{-1}$であるので、
 $gag^{-1}=a$を得る。
 これに右から$g$をかけると$ga=ag$となる。
 つまり、$g\in Z(a)$である。

 よって、$y^{-1}x \in Z(a)$である。

 逆に、$y^{-1}x \in Z(a)$であれば、
 $Z(a)$の定義より、
 $y^{-1}xa = ay^{-1}x$である。
 右から$(y^{-1}x)^{-1}$をかけることで、
 $y^{-1}xa(y^{-1}x)^{-1} = a$が得られる。
\end{quote}

$y^{-1}x \in Z(a) \Longleftrightarrow x,y \; は \; G/Z(a) \; の同じ類に属する$
\begin{quote}
 $G/Z(a)$は$G$の元を$Z(a)$を基準にして分割した集合であり、
 分割した集合を同値類と呼んだりする。

 $Z(a)$を基準にして分割するということは、
 $g,h\in G$について$g^{-1}h \in Z(a)$なら$g,h$は同じグループに属し、
 $g^{-1}h\not\in Z(a)$なら$g,h$は異なるグループに属するという分け方をする。

 $y^{-1}x \in Z(a)$であれば、
 $x,y$は$G/Z(a)$で同じ類に属することになる。

 逆に、
 $x,y$は$G/Z(a)$で同じ類に属するのであれば、
 $G/Z(a)$の定義から
 $y^{-1}x \in Z(a)$であることが言える。
\end{quote}

\hrulefill

\end{document}
