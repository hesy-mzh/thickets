\documentclass[12pt,b5paper]{ltjsarticle}

%\usepackage[margin=15truemm, top=5truemm, bottom=5truemm]{geometry}
%\usepackage[margin=10truemm,left=15truemm]{geometry}
\usepackage[margin=10truemm]{geometry}

\usepackage{amsmath,amssymb}
%\pagestyle{headings}
\pagestyle{empty}

%\usepackage{listings,url}
%\renewcommand{\theenumi}{(\arabic{enumi})}

%\usepackage{graphicx}

%\usepackage{tikz}
%\usetikzlibrary {arrows.meta}
%\usepackage{wrapfig}	% required for `\wrapfigure' (yatex added)
%\usepackage{bm}	% required for `\bm' (yatex added)

% ルビを振る
%\usepackage{luatexja-ruby}	% required for `\ruby'

%% 核Ker 像Im Hom を定義
%\newcommand{\Img}{\mathop{\mathrm{Im}}\nolimits}
%\newcommand{\Ker}{\mathop{\mathrm{Ker}}\nolimits}
%\newcommand{\Hom}{\mathop{\mathrm{Hom}}\nolimits}

%\DeclareMathOperator{\Rot}{rot}
%\DeclareMathOperator{\Div}{div}
%\DeclareMathOperator{\Grad}{grad}
%\DeclareMathOperator{\arcsinh}{arcsinh}
%\DeclareMathOperator{\arccosh}{arccosh}
%\DeclareMathOperator{\arctanh}{arctanh}



%\usepackage{listings,url}
%
%\lstset{
%%プログラム言語(複数の言語に対応,C,C++も可)
%  language = Python,
%%  language = Lisp,
%%  language = C,
%  %背景色と透過度
%  %backgroundcolor={\color[gray]{.90}},
%  %枠外に行った時の自動改行
%  breaklines = true,
%  %自動改行後のインデント量(デフォルトでは20[pt])
%  breakindent = 10pt,
%  %標準の書体
%%  basicstyle = \ttfamily\scriptsize,
%  basicstyle = \ttfamily,
%  %コメントの書体
%%  commentstyle = {\itshape \color[cmyk]{1,0.4,1,0}},
%  %関数名等の色の設定
%  classoffset = 0,
%  %キーワード(int, ifなど)の書体
%%  keywordstyle = {\bfseries \color[cmyk]{0,1,0,0}},
%  %表示する文字の書体
%  %stringstyle = {\ttfamily \color[rgb]{0,0,1}},
%  %枠 "t"は上に線を記載, "T"は上に二重線を記載
%  %他オプション:leftline,topline,bottomline,lines,single,shadowbox
%  frame = TBrl,
%  %frameまでの間隔(行番号とプログラムの間)
%  framesep = 5pt,
%  %行番号の位置
%  numbers = left,
%  %行番号の間隔
%  stepnumber = 1,
%  %行番号の書体
%%  numberstyle = \tiny,
%  %タブの大きさ
%  tabsize = 4,
%  %キャプションの場所("tb"ならば上下両方に記載)
%  captionpos = t
%}



\begin{document}


\hrulefill
\textbf{各種情報}
\hrulefill

\textbf{開集合の公理}

集合$X$の部分集合族$\mathcal{O}$について
次の3条件を開集合の公理という。
\begin{enumerate}
 \item $\emptyset\in\mathcal{O},\ X\in\mathcal{O}$
 \item $O_1,O_2\in\mathcal{O}$ならば$O_1\cap O_2\in\mathcal{O}$
 \item $O_{\lambda}\in\mathcal{O} \ (\lambda\in\Lambda)$ならば
       $\bigcup_{\lambda\in\Lambda}O_{\lambda}\in\mathcal{O}$
\end{enumerate}


\hrulefill
\textbf{問題}
\hrulefill

\begin{enumerate}
 \item
      \textbf{相対位相}
      
      $(X,\mathcal{O})$が位相空間とする。
      $A\subset X$を部分集合とする。
      $\mathcal{O}_A = \{ A\cap U\mid U\in\mathcal{O} \}$
      とする時、
      $(A,\mathcal{O}_A)$は位相空間であることを確かめよ。

      \dotfill

      $\mathcal{O}_A$に対して開集合の公理を確認すればよい。

      $\emptyset, X\in\mathcal{O}$より
      \begin{equation}
       \emptyset = A\cap \emptyset \in\mathcal{O}_A,\
       A = A\cap X \in\mathcal{O}_A
      \end{equation}

      ${}^{\forall}U_1,U_2\in\mathcal{O}_A$に対して
      $U_1=A\cap O_1,\ U_2=A\cap O_2$となる
      $O_1,O_2\in\mathcal{O}$が存在する。
      \begin{equation}
       U_1\cap U_2 = (A\cap O_1) \cap (A\cap O_2)
        = A\cap (O_1 \cap O_2) \in\mathcal{O}_A
      \end{equation}


      ${}^{\forall}U_\lambda\in\mathcal{O}_A\ (\lambda\in\Lambda)$
      に対して
      $U_\lambda=A\cap O_\lambda$となる
      $O_\lambda\in\mathcal{O}$が存在する。
      \begin{equation}
       \bigcup_{\lambda\in\Lambda}U_\lambda
        = \bigcup_{\lambda\in\Lambda}(A\cap O_\lambda)
        = A\cap \left( \bigcup_{\lambda\in\Lambda} O_\lambda \right)
        \in\mathcal{O}_A
      \end{equation}

      以上により$\mathcal{O}_A$は開集合の公理を満たす。
      よって、$(A,\mathcal{O}_A)$は位相空間である。


      \hrulefill

 \item
      $X=\mathbb{Z},\
      \mathcal{L} = \{
        \{ [3n-1,\infty) \cap X \mid n\in\mathbb{Z}\},\
        \{ (-\infty,2n] \cap X \mid n\in\mathbb{Z} \}
      \}$
      とする。
      $\mathcal{L}$によって生成される
      $X$上の位相は離散位相か?

      \dotfill

      離散位相は全ての部分集合が開集合となる位相空間である。
      \begin{align}
       [-1,\infty) \cap (\infty,0] =& \{-1,0\}\\
       [2,\infty) \cap (\infty,0] =& \emptyset\\
       [-1,\infty) \cap (\infty,-2] =& \emptyset
      \end{align}

      上記のように$\{0\}$という集合は作ることができない。
      この為、$\mathcal{L}$は離散位相を生成しない。


      \hrulefill


 \item
      $X=\{1,2,3,4\},\ \mathcal{T}=\{ \{1,2\},\{2,3\},4\}$
      とする。
      $\mathcal{T}$によって生成される集合$X$の開集合系を求めよ。

      \dotfill

      $\mathcal{T}=\{ \{1,2\},\{2,3\},4\}$とあるが、
      $\mathcal{T}=\{ \{1,2\},\{2,3\},\{4\}\}$
      ではないかと思うのでこちらで記述を行う。

      \dotfill

      \begin{align}
       \{1,2\}\cap \{4\} =& \emptyset\\
       \{1,2\}\cup \{2,3\}\cup\{4\} =& X\\
       \{1,2\}\cap \{2,3\} =& \{2\}\\
       \{1,2\}\cup \{2,3\} =& \{1,2,3\}\\
       \{1,2\}\cup \{4\} =& \{1,2,4\}\\
       \{2,3\}\cup \{4\} =& \{2,3,4\}\\
       \{2\}\cup \{4\} =& \{2,4\}\\
      \end{align}

      \begin{equation}
       \mathcal{O}_{\mathcal{T}}
        =\{
        \emptyset,X,
        \{2\},\{4\},
        \{1,2\},\{2,3\},\{2,4\},
        \{1,2,3\},\{1,2,4\},\{2,3,4\}
        \}
      \end{equation}



      \hrulefill

\end{enumerate}

\end{document}
