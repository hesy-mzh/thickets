\documentclass[12pt,b5paper]{ltjsarticle}

\pagestyle{empty}

\begin{document}

2曲線が直行するためには次の2つを確認する必要がある。
\begin{itemize}
 \item 共有点が存在する
 \item 共有点における接線が直交する
\end{itemize}

通常は、共有点の存在性を確認する事も含めて、
共有点を求め、
その共有点での接線の傾きの積が-1であることを確認する。

\quad

共有点の座標は答えに必要ではなく、
$y=\sqrt{x}$と$y=e^{-2x}$の共有点はわかりやすい数字ではないので、
これを求めずに考えたい。
具体的な共有点は$(0.30054,0.54821)$ぐらい。

\quad

この例の場合は、
\underline{共有点の存在を仮定}し、
先に接線の傾きを求める方法を取っている。
「共有点の存在」を仮定すると
接線が直交する場合が求まるので、
本当にこの場合に共有点の存在するのかを
最後に確認している。

\end{document}
