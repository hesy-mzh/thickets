\documentclass[12pt,b5paper]{ltjsarticle}

%\usepackage[margin=15truemm, top=5truemm, bottom=5truemm]{geometry}
%\usepackage[margin=10truemm,left=15truemm]{geometry}
\usepackage[margin=10truemm]{geometry}

\usepackage{amsmath,amssymb}
%\pagestyle{headings}
\pagestyle{empty}

%\usepackage{listings,url}
%\renewcommand{\theenumi}{(\arabic{enumi})}

%\usepackage{graphicx}

%\usepackage{tikz}
%\usetikzlibrary {arrows.meta}
%\usepackage{wrapfig}	% required for `\wrapfigure' (yatex added)
\usepackage{bm}	% required for `\bm' (yatex added)

% ルビを振る
%\usepackage{luatexja-ruby}	% required for `\ruby'

%% 核Ker 像Im Hom を定義
%\newcommand{\Img}{\mathop{\mathrm{Im}}\nolimits}
%\newcommand{\Ker}{\mathop{\mathrm{Ker}}\nolimits}
%\newcommand{\Hom}{\mathop{\mathrm{Hom}}\nolimits}

%\DeclareMathOperator{\Rot}{rot}
%\DeclareMathOperator{\Div}{div}
%\DeclareMathOperator{\Grad}{grad}
%\DeclareMathOperator{\arcsinh}{arcsinh}
%\DeclareMathOperator{\arccosh}{arccosh}
%\DeclareMathOperator{\arctanh}{arctanh}



%\usepackage{listings,url}
%
%\lstset{
%%プログラム言語(複数の言語に対応,C,C++も可)
%  language = Python,
%%  language = Lisp,
%%  language = C,
%  %背景色と透過度
%  %backgroundcolor={\color[gray]{.90}},
%  %枠外に行った時の自動改行
%  breaklines = true,
%  %自動改行後のインデント量(デフォルトでは20[pt])
%  breakindent = 10pt,
%  %標準の書体
%%  basicstyle = \ttfamily\scriptsize,
%  basicstyle = \ttfamily,
%  %コメントの書体
%%  commentstyle = {\itshape \color[cmyk]{1,0.4,1,0}},
%  %関数名等の色の設定
%  classoffset = 0,
%  %キーワード(int, ifなど)の書体
%%  keywordstyle = {\bfseries \color[cmyk]{0,1,0,0}},
%  %表示する文字の書体
%  %stringstyle = {\ttfamily \color[rgb]{0,0,1}},
%  %枠 "t"は上に線を記載, "T"は上に二重線を記載
%  %他オプション:leftline,topline,bottomline,lines,single,shadowbox
%  frame = TBrl,
%  %frameまでの間隔(行番号とプログラムの間)
%  framesep = 5pt,
%  %行番号の位置
%  numbers = left,
%  %行番号の間隔
%  stepnumber = 1,
%  %行番号の書体
%%  numberstyle = \tiny,
%  %タブの大きさ
%  tabsize = 4,
%  %キャプションの場所("tb"ならば上下両方に記載)
%  captionpos = t
%}



\begin{document}

3次正方行列$A$を次のようにおく。
\begin{equation}
 A=
 \begin{pmatrix}
  1 & -1 & 1 \\ -1 & 1 & -1 \\ 1 & -1 & 1
 \end{pmatrix}
\end{equation}

\begin{enumerate}
 \item
      行列$A$の固有多項式を求めよ。
      但し、途中式なども書くこと。
      答えは$x^3-3x^2$

      \dotfill

      固有多項式とは固有値を根に持つ多項式である。

      固有値とは$A\bm{x}=\lambda\bm{x}$となる
      スカラー$\lambda$のことで、
      この式を満たすベクトル$\bm{x}$を
      固有値$\lambda$の固有ベクトルという。

      $A\bm{x}=\lambda\bm{x}$を移項し、
      $A\bm{x}-\lambda\bm{x}=\bm{0}$となるので、
      左辺をまとめると$(A-\lambda E)\bm{x}=0$となる。
      $\bm{x}\ne\bm{0}$の時、この式が成り立つためには
      $A-\lambda E$の行列式が$0$であることが必要十分である。
      この$A-\lambda E$の行列式を固有多項式という。

      移項を逆に行い$\lambda\bm{x}-A\bm{x}=\bm{0}$とした場合、
      $\lambda E -A$が得られる。
      $\lambda E -A$の行列式も固有多項式という。

      実際に固有値を求めるには
      固有方程式$\det(\lambda E -A)=0$を解く為、
      $\det(\lambda E -A)=0$でも$\det(A-\lambda E)=0$でも
      同じ固有値が求まる。
      

      \dotfill

      $E$を単位行列として、
      $\lambda E-A$を計算する。
      \begin{align}
       & \det(\lambda E-A)\\
        =&\det\left(
        \lambda
        \begin{pmatrix}
         1 & 0 & 0 \\ 0 & 1 & 0 \\ 0 & 0 & 1
        \end{pmatrix}
       -
        \begin{pmatrix}
         1 & -1 & 1 \\ -1 & 1 & -1 \\ 1 & -1 & 1
        \end{pmatrix}
        \right)
        = \det
        \begin{pmatrix}
         \lambda-1 & 1 & -1 \\ 1 & \lambda-1 & 1 \\ -1 & 1 & \lambda-1
        \end{pmatrix}\\
        = & (\lambda-1)^3+1^2\cdot(-1)+1^2\cdot(-1)
       -1^2(\lambda-1)-1^2(\lambda-1)-(-1)^2(\lambda-1)\\
       = & (\lambda-1)^3+2-3(\lambda-1)
       = \lambda^3-3\lambda^2
      \end{align}

      この$\lambda$を$x$に置き換えると
      $x^3-3x$が得られる。

      \hrulefill

 \item
      行列$A$の固有値を求めよ。

      \dotfill


      固有方程式$x^3-3x^2=0$を解くと$x=0,3$である。
      よって固有値は$0$と$3$である。

      \hrulefill

 \item
      行列$A$の各固有値に対する固有空間を求めよ。

      \dotfill

      固有空間とは固有ベクトルの存在する空間であり、
      連立方程式$(\lambda E - A)\bm{x}=\bm{0}$の
      ベクトル$\bm{x}$の解空間を指す。

      \dotfill

      \textbf{固有値が$3$の場合}
      
      連立方程式$(3E-A)\bm{x}=\bm{0}$を考える。
      \begin{align}
       3E-A =
        \begin{pmatrix}
         2 & 1 & -1 \\ 1 & 2 & 1 \\ -1 & 1 & 2
        \end{pmatrix}
       \to
        \begin{pmatrix}
         0 & -3 & -3 \\ 1 & 2 & 1 \\ 0 & 3 & 3
        \end{pmatrix}
       \to
        \begin{pmatrix}
         0 & 0 & 0 \\ 1 & 2 & 1 \\ 0 & 3 & 3
        \end{pmatrix}
      \end{align}

      $\bm{x}$の成分を順に$x_1,x_2,x_3$とすると
      $(\lambda E - A)\bm{x}=\bm{0}$は上記行列の変形から
      次のように表記できる。
      \begin{equation}
        \begin{pmatrix}
         0 & 0 & 0 \\ 1 & 2 & 1 \\ 0 & 3 & 3
        \end{pmatrix}
        \begin{pmatrix}
         x_1\\ x_2\\ x_3
        \end{pmatrix}
        =
        \begin{pmatrix}
         0\\ 0\\ 0
        \end{pmatrix}
      \end{equation}

      これを計算すると次の2つの式が得られる。
      \begin{equation}
       x_1 + 2x_2 + x_3 = 0,
        \qquad
        3x_2 + 3x_3 = 0
      \end{equation}
      この式を変形して
      $x_1=-2x_2 - x_3$と$x_2=-x_3$
      が得られる。
      これを$\bm{x}$に当てはめる。
      \begin{equation}
       \bm{x}=
       \begin{pmatrix}
         x_1\\ x_2\\ x_3
       \end{pmatrix}
       =
       \begin{pmatrix}
         -2x_2-x_3\\ x_2\\ x_3
       \end{pmatrix}
       =
       \begin{pmatrix}
         2x_3-x_3\\ -x_3\\ x_3
       \end{pmatrix}
       =
       x_3
       \begin{pmatrix}
         1\\ -1\\ 1
       \end{pmatrix}
      \end{equation}

      つまり、
      連立方程式$(3E-A)\bm{x}=\bm{0}$をみたす$\bm{x}$は
      スカラー$t$を用いて
      $\bm{x}=t\begin{pmatrix}1\\ -1\\ 1 \end{pmatrix}$
      となることがわかる。
      %
      この1次元空間$t\begin{pmatrix}1\\ -1\\ 1 \end{pmatrix}$
      が固有値3の固有空間である。



      \textbf{固有値が$0$の場合}

      連立方程式$(0E-A)\bm{x}=\bm{0}$の解空間を求める。
      \begin{equation}
       0E-A =
        \begin{pmatrix}
         -1 & 1 & -1 \\ 1 & -1 & 1 \\ -1 & 1 & -1
        \end{pmatrix}
        \to
        \begin{pmatrix}
         -1 & 1 & -1 \\ 0 & 0 & 0 \\ 0 & 0 & 0
        \end{pmatrix}
      \end{equation}

      この行列の変形から連立方程式は
      次のようになる。
      \begin{equation}
        \begin{pmatrix}
         -1 & 1 & -1 \\ 0 & 0 & 0 \\ 0 & 0 & 0
        \end{pmatrix}
        \begin{pmatrix}
         x_1\\ x_2\\ x_3
        \end{pmatrix}
        =
        \begin{pmatrix}
         0\\ 0\\ 0
        \end{pmatrix}
      \end{equation}

      ここから
      $-x_1 + x_2 -x_3=0$が得られる。
      変形をして$x_1=x_2 -x_3$であるので、
      $\bm{x}$に当てはめる。
      \begin{equation}
       \bm{x}=
       \begin{pmatrix}
         x_1\\ x_2\\ x_3
       \end{pmatrix}
       =
       \begin{pmatrix}
         x_2-x_3\\ x_2\\ x_3
       \end{pmatrix}
       =
       \begin{pmatrix}
         x_2\\ x_2\\ 0
       \end{pmatrix}
       +
       \begin{pmatrix}
         -x_3\\ 0\\ x_3
       \end{pmatrix}
       =
       x_2
       \begin{pmatrix}
         1\\ 1\\ 0
       \end{pmatrix}
       + x_3
       \begin{pmatrix}
         -1\\ 0\\ 1
       \end{pmatrix}
      \end{equation}

      つまり、連立方程式の解は
      2つのスカラー$s,t$を用いて
      $\bm{x}= s\begin{pmatrix} 1\\ 1\\ 0 \end{pmatrix}
       + t \begin{pmatrix}-1\\ 0\\ 1\end{pmatrix}$
      となることがわかる。

      この2次元空間
      $s\begin{pmatrix} 1\\ 1\\ 0 \end{pmatrix}
       + t \begin{pmatrix}-1\\ 0\\ 1\end{pmatrix}$
      が固有値0の固有空間である。

      \hrulefill

 \item
      行列$A$を直交行列を用いて対角化せよ。

      \dotfill

      直交行列とは逆行列と転置行列が等しくなる行列

      \dotfill

      固有空間の基底となるベクトルを並べた行列を用いることで
      対角化は可能である。
      先ほど求めた固有空間の基底となるベクトルを
      次のように$\bm{a}\bm{b},\bm{c}$とする。
      \begin{equation}
       \bm{a}=\begin{pmatrix}1\\ -1\\ 1 \end{pmatrix},
       \quad
       \bm{b}=\begin{pmatrix} 1\\ 1\\ 0 \end{pmatrix},
       \quad
       \bm{c}=\begin{pmatrix}-1\\ 0\\ 1\end{pmatrix}
      \end{equation}

      このベクトル$\bm{a},\bm{b},\bm{c}$を並べた
      3次正方行列で$A$は対角化可能であるが、
      $\bm{a},\bm{b},\bm{c}$は互いに直交していないため、
      直交行列ではない。

      直交行列であるためには内積が
      $\bm{a}\cdot\bm{b}=\bm{a}\cdot\bm{c}=\bm{b}\cdot\bm{c}=0$
      でなければならず、
      $\bm{a}\cdot\bm{b}=\bm{a}\cdot\bm{c}=0$
      ではあるが、
      $\bm{b}\cdot\bm{c}\ne0$である。

      直交行列であるためには各列ベクトル(または 行ベクトル)が
      単位ベクトルであり、互いに直交していなければならない。

      そこで、ベクトル$\bm{a},\bm{b},\bm{c}$に
      シュミットの直交化法を用いて
      正規直交基底を求め、
      これらを並べた行列が直交行列となる。


      $\bm{a}$は$\bm{b},\bm{c}$と直交しているので
      単位ベクトル化だけをおこなう。
      \begin{equation}
       \bm{a^\prime}=\frac{\bm{a}}{\lvert\bm{a}\rvert}
        =\frac{1}{\sqrt{3}}\begin{pmatrix}1\\ -1\\ 1 \end{pmatrix}
        =\begin{pmatrix}\frac{1}{\sqrt{3}}\\ \frac{-1}{\sqrt{3}}\\ \frac{1}{\sqrt{3}} \end{pmatrix}
      \end{equation}

      $\bm{b},\bm{c}$は直交していないので
      シュミットの直交化法により正規直交基底を求める。
      まず、$\bm{b}$を正規化する。
      \begin{equation}
       \bm{b^\prime}=\frac{\bm{b}}{\lvert\bm{b}\rvert}
        =\frac{1}{\sqrt{2}}\begin{pmatrix}1\\ 1\\ 0 \end{pmatrix}
        =\begin{pmatrix}\frac{1}{\sqrt{2}}\\ \frac{1}{\sqrt{2}}\\ 0 \end{pmatrix}
      \end{equation}

      $\bm{b^\prime}$を用いて$\bm{c}$を直交化する。
      \begin{equation}
       \bm{c^\prime}=\bm{c}-(\bm{b^\prime}\cdot\bm{c})\bm{b^\prime}
        =
        \begin{pmatrix}-1\\ 0\\ 1\end{pmatrix}
        -
         \left(
          \begin{pmatrix}\frac{1}{\sqrt{2}}\\ \frac{1}{\sqrt{2}}\\ 0 \end{pmatrix}
          \cdot
          \begin{pmatrix}-1\\ 0\\ 1\end{pmatrix}
         \right)
         \begin{pmatrix}\frac{1}{\sqrt{2}}\\ \frac{1}{\sqrt{2}}\\ 0 \end{pmatrix}
         =
          \begin{pmatrix}-\frac{1}{2}\\ \frac{1}{2}\\ 1\end{pmatrix}
      \end{equation}

      $\lvert\bm{c^\prime}\rvert=\frac{\sqrt{6}}{2}$であるので
      $\bm{c^\prime}$を正規化する。
      \begin{equation}
       \bm{c^{\prime\prime}}
        = \frac{\bm{c^\prime}}{\lvert\bm{c^\prime}\rvert}
        =\frac{2}{\sqrt{6}}\begin{pmatrix}-\frac{1}{2}\\ \frac{1}{2}\\ 1\end{pmatrix}
        =\begin{pmatrix}-\frac{1}{\sqrt{6}}\\ \frac{1}{\sqrt{6}}\\ \frac{2}{\sqrt{6}}\end{pmatrix}
      \end{equation}

      $\bm{a^\prime},\bm{b^\prime},\bm{c^{\prime\prime}}$を用いて
      直交行列$P$は次のようになる。
      \begin{equation}
       P=
        \begin{pmatrix}
         \frac{1}{\sqrt{3}} & \frac{1}{\sqrt{2}} & -\frac{1}{\sqrt{6}}\\
         -\frac{1}{\sqrt{3}} & \frac{1}{\sqrt{2}} & \frac{1}{\sqrt{6}}\\
         \frac{1}{\sqrt{3}} & 0 & \frac{2}{\sqrt{6}}\end{pmatrix}
      \end{equation}

      $P$を用いて$A$は次のように対角化できる。
      \begin{equation}
       {}^{t}P A P =
        \begin{pmatrix}
         3 & 0 & 0\\
         0 & 0 & 0\\
         0 & 0 & 0
        \end{pmatrix}
      \end{equation}
      

      \hrulefill

\end{enumerate}


\hrulefill

実数を係数とする高々2次の多項式全体を
$\mathbb{R}[x]_{2}$と書く。
また、写像$T:\mathbb{R}[x]_{2}\to\mathbb{R}[x]_{2}$を
$T(f(x))=f^\prime(x)$(微分)で定義する。

\begin{enumerate}
 \item
      $\mathbb{R}[x]_{2}$
      は$\mathbb{R}$上のベクトル空間であることを示せ。

      \dotfill

      ベクトル空間であるということは
      空間が加法で閉じている
      ($f,g\in\mathbb{R}[x]_{2}\Rightarrow f+g\in\mathbb{R}[x]_{2}$)
      ことと、
      スカラー倍が存在する
      ($a\in\mathbb{R},f\in\mathbb{R}[x]_{2}\Rightarrow af\in\mathbb{R}[x]_{2}$)
      上で、次の性質を満たすということである。
      \begin{enumerate}
       \item $f,g\in\mathbb{R}[x]_{2}$に対して$f+g=g+f$
       \item $f,g,h\in\mathbb{R}[x]_{2}$に対して$(f+g)+h=f+(g+h)$
       \item $0\in\mathbb{R}[x]_{2}$が存在し、
             $f\in\mathbb{R}[x]_{2}$に対して$f+0=0+f=f$
       \item $a,b\in\mathbb{R},f\in\mathbb{R}[x]_{2}$に対して
             $a(bf)=(ab)f$
       \item $a,b\in\mathbb{R},f\in\mathbb{R}[x]_{2}$に対して
             $(a+b)f=af+bf$
       \item $a\in\mathbb{R},f,g\in\mathbb{R}[x]_{2}$に対して
             $a(f+g)=af+ag$
       \item $1\in\mathbb{R}$が存在し、
             $f\in\mathbb{R}[x]_{2}$に対して$1f=f$
       \item $0\in\mathbb{R}$が存在し、
             $f\in\mathbb{R}[x]_{2}$に対して$0f=0$
      \end{enumerate}

      \dotfill

      加法とスカラー倍は次のように成り立つ。
      \begin{enumerate}
       \item
            実数係数の多項式同士の和はまた実数係数の多項式になる。
            また、多項式の和により次数は同じか小さくなる為、
            2次以下の多項式同士の和は2次以下の多項式となる。
            これにより$\mathbb{R}[x]_2$は加法で閉じている。

       \item
            多項式の実数倍はまた同じ次数の多項式となるため、
            $\mathbb{R}[x]_2$にスカラー倍が存在する。
      \end{enumerate}

      そこで、性質が成り立つか確認する。

      多項式の和は係数同士の和で定義されるので、
      $f,g,h\in\mathbb{R}[x]_2$に対して
      $f+g=g+f,(f+g)+h=f+(g+h)$となる。
      
      次数0で係数0の多項式が$0\in\mathbb{R}[x]_2$である。
      これは$f\in\mathbb{R}[x]_2$に対して$f+0=0+f=f$である。

      実数と多項式の積は実数と係数の積で定義されるので、
      $a,b\in\mathbb{R}$と$f\in\mathbb{R}[x]_2$について
      $a(bf)=(ab)f$である。

      $x,x^2$は分配法則により$a,b\in\mathbb{R}$に対して
      $(a+b)x=ax+bx,(a+b)x^2=ax^2+bx^2$となる。
      その為、多項式$f\in\mathbb{R}[x]_2$において
      $(a+b)f=af+bf$となる。

      多項式$f,g\in\mathbb{R}[x]_2$において
      実数倍は係数との積であり、$f+g$は係数同士の和で定義されるので、
      $a(f+g)=af+ag$が成り立つ。

      $1\in\mathbb{R}$が存在し、係数との積を考えることにより
      $1f=f$である。
      また、$0\in\mathbb{R}$が存在し、
      全ての係数が0の多項式になるということで、
      $0f=0$となる。


      よって、$\mathbb{R}[x]_2$は$\mathbb{R}$上のベクトル空間である。

      \hrulefill

 \item
      $T$が$\mathbb{R}$上の線形変換であることを示せ。

      \dotfill

      $T$が線形変換であるとは次を満たす事をいう。
      \begin{enumerate}
       \item $f,g\in\mathbb{R}[x]_2$に対して
             $T(f+g)=T(f)+T(g)$
       \item $a\in\mathbb{R},f\in\mathbb{R}[x]_2$に対して
             $T(af)=aT(f)$
      \end{enumerate}

      \dotfill

      $f,g\in\mathbb{R}[x]_2$を
      $f=f_0+f_1x+f_2x^2,\ g=g_0+g_1x+g_2x^2\ (f_i,g_i\in\mathbb{R})$とする。
      \begin{gather}
       T(f)=f_1+2f_2x,
       T(g)=g_1+2g_2x\\
       T(f+g) = T( (f_0+g_0)+(f_1+g_1)x+(f_2+g_2)x^2 ) = (f_1+g_1)+2(f_2+g_2)x
      \end{gather}
      よって、$T(f+g)=T(f)+T(g)$である。

      また、$a\in\mathbb{R}$に対して
      \begin{equation}
       T(af)=T(af_0+af_1x+af_2x^2)=af_1+2af_2x
      \end{equation}
      であるので、$T(af)=aT(f)$である。

      以上により$T$は線形変換である。

      \hrulefill

 \item
      $\{1,x,x^2\}$が
      $\mathbb{R}[x]_{2}$
      の基底であることを示せ。

      \dotfill

      $\{1,x,x^2\}$が基底であるとは
      \begin{enumerate}
       \item $1,x,x^2$が一次独立である
       \item $\mathbb{R}[x]_{2}$の任意の元が
             $1,x,x^2$が一次結合で書ける
      \end{enumerate}
      を満たすときをいう。

      \dotfill

      $a_i\in\mathbb{R}$とする。
      $a_0+a_1x+a_2x^2=0$となるのは
      $(a_0,a_1,a_2)=(0,0,0)$のときのみである。
      これは、$1,x,x^2$に実数をかけても
      それぞれの次数が変化しないためである。

      また、$\mathbb{R}[x]_{2}$の任意の元は
      $f=f_0+f_1x+f_2x^2$と書けるため
      全ての元が$1,x,x^2$の一次結合である。

      よって、
      $\{1,x,x^2\}$は
      $\mathbb{R}[x]_{2}$
      の基底である。

      \hrulefill

 \item
      基底$\{1,x,x^2\}$に関する
      $T$の表現行列を求めよ。

      \dotfill

      $T(x^2)=2x,T(x)=1,T(1)=0$である。
      この為、基底は次のように対応が取れる。
      \begin{equation}
       \begin{pmatrix}
        x^2\\x\\1
       \end{pmatrix}
       \mapsto
       \begin{pmatrix}
        2x\\1\\0
       \end{pmatrix}
      \end{equation}

      これを行列で表すと次のようになる。
      \begin{equation}
       \begin{pmatrix}
        0 & 2 & 0\\
        0 & 0 & 1\\
        0 & 0 & 0
       \end{pmatrix}
       \begin{pmatrix}
        x^2\\x\\1
       \end{pmatrix}
       =
       \begin{pmatrix}
        2x\\1\\0
       \end{pmatrix}
      \end{equation}

      よって、表現行列は次の様になる。
      \begin{equation}
       \begin{pmatrix}
        0 & 2 & 0\\
        0 & 0 & 1\\
        0 & 0 & 0
       \end{pmatrix}
      \end{equation}

      \hrulefill

\end{enumerate}





\end{document}
