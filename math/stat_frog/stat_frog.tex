\documentclass[12pt,b5paper]{ltjsarticle}

\usepackage{amssymb}
\pagestyle{empty}

\usepackage{amsmath}	% required for `\align' (yatex added)
\begin{document}


母集団は正規分布$N(\mu,2)$である。
母平均$\mu$、母分散$2$

標本として50個取り出す。
取り出した個体の平均は$10$である。

信頼係数$0.95$における$\mu$の信頼区間を算出



\begin{enumerate}\renewcommand{\theenumi}{(\arabic{enumi})}
 \item
      取り出した50個の標本$X_i (i=1,\dots,50)$に対し、
      次のように
      標本平均$\overline{X_{50}}$を定め、
      分散$V(\overline{X_{50}})$を求める。
      \[
       \overline{X_{50}}=\frac{1}{50}\sum_{i=1}^{50}X_i
      \]

      $X_i (i=1,\dots,50)$が独立同分布(平均$\mu$、分散$2$)であれば
      正規分布でなくても$V(\overline{X_{50}}) = \frac{2}{50}$である。
      \begin{align}
       V(\overline{X_{50}}) & = V\left(\frac{1}{50}\sum_{i=1}^{50}X_i\right)\\
       & = \frac{1}{50^2}\sum_{i=1}^{50}V(X_i)\\
       & = \frac{1}{50^2}\sum_{i=1}^{50}2 & V(X_i)=2 i=1,\dots,50\\
       & = \frac{2}{50}
      \end{align}
 \item
      \[
       \int_{-1.96}^{1.96}\frac{1}{\sqrt{2\pi}}e^{-\frac{x^2}{2}}dx = 0.95
      \]
      この式は正規分布(平均 $0$、分散 $1$)の空間で
      $-1.96$から$1.96$に含まれる部分が全体の$95\%$であることを意味している。

      
\end{enumerate}


\end{document}
