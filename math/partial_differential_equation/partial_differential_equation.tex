\documentclass[12pt,b5paper]{ltjsarticle}

%\usepackage[margin=15truemm, top=5truemm, bottom=5truemm]{geometry}
%\usepackage[margin=10truemm,left=15truemm]{geometry}
\usepackage[margin=10truemm]{geometry}

\usepackage{amsmath,amssymb}
%\pagestyle{headings}
\pagestyle{empty}

%\usepackage{listings,url}
%\renewcommand{\theenumi}{(\arabic{enumi})}

%\usepackage{graphicx}

%\usepackage{tikz}
%\usetikzlibrary {arrows.meta}
%\usepackage{wrapfig}
%\usepackage{bm}

% ルビを振る
\usepackage{luatexja-ruby}	% required for `\ruby'

%% 核Ker 像Im Hom を定義
%\newcommand{\Img}{\mathop{\mathrm{Im}}\nolimits}
%\newcommand{\Ker}{\mathop{\mathrm{Ker}}\nolimits}
%\newcommand{\Hom}{\mathop{\mathrm{Hom}}\nolimits}

%\DeclareMathOperator{\Rot}{rot}
%\DeclareMathOperator{\Div}{div}
%\DeclareMathOperator{\Grad}{grad}
%\DeclareMathOperator{\arcsinh}{arcsinh}
%\DeclareMathOperator{\arccosh}{arccosh}
%\DeclareMathOperator{\arctanh}{arctanh}



%\usepackage{listings,url}
%
%\lstset{
%%プログラム言語(複数の言語に対応,C,C++も可)
%  language = Python,
%%  language = Lisp,
%%  language = C,
%  %背景色と透過度
%  %backgroundcolor={\color[gray]{.90}},
%  %枠外に行った時の自動改行
%  breaklines = true,
%  %自動改行後のインデント量(デフォルトでは20[pt])
%  breakindent = 10pt,
%  %標準の書体
%%  basicstyle = \ttfamily\scriptsize,
%  basicstyle = \ttfamily,
%  %コメントの書体
%%  commentstyle = {\itshape \color[cmyk]{1,0.4,1,0}},
%  %関数名等の色の設定
%  classoffset = 0,
%  %キーワード(int, ifなど)の書体
%%  keywordstyle = {\bfseries \color[cmyk]{0,1,0,0}},
%  %表示する文字の書体
%  %stringstyle = {\ttfamily \color[rgb]{0,0,1}},
%  %枠 "t"は上に線を記載, "T"は上に二重線を記載
%  %他オプション:leftline,topline,bottomline,lines,single,shadowbox
%  frame = TBrl,
%  %frameまでの間隔(行番号とプログラムの間)
%  framesep = 5pt,
%  %行番号の位置
%  numbers = left,
%  %行番号の間隔
%  stepnumber = 1,
%  %行番号の書体
%%  numberstyle = \tiny,
%  %タブの大きさ
%  tabsize = 4,
%  %キャプションの場所("tb"ならば上下両方に記載)
%  captionpos = t
%}



\begin{document}


\hrulefill


\textbf{勾配 (gradient)}

関数$f$に対して、
$f$の勾配 (\ruby{gradient}{グラディエント})
$Df(=\nabla f)$
\begin{gather}
 f:\mathbb{R}^3 \to \mathbb{R},
  \quad
 (x_1,x_2,x_3)\mapsto y \qquad\
 Df = \nabla f = \left(
  \frac{\partial f}{\partial x_1},
  \frac{\partial f}{\partial x_2},
  \frac{\partial f}{\partial x_3}
  \right)
\end{gather}


\hrulefill


\textbf{偏微分方程式 ( partial differential equation )}

輸送方程式 ( transport equation )

\hrulefill

$u=u(x,t)\ (x\in\mathbb{R}^n,\ t\in\mathbb{R})$
$u: \mathbb{R}^n\times [ 0, \infty ) \to \mathbb{R}$

$Du=(u_{x_{1}},u_{x_{2}},\dots,u_{x_{n}})$

$b\in\mathbb{R}^n$




\hrulefill

\textbf{Report 1.1}

関数$u=u(x,t)\ (x\in\mathbb{R}^n,\ t\in\mathbb{R})$は
$u: \mathbb{R}^n\times [ 0, \infty ) \to \mathbb{R}$
とし、
$b\in\mathbb{R}^n$とする。
\begin{equation}
 u_{t} + b \cdot Du = 0
  \quad
  \text{in}\
 \mathbb{R}^{n} \times (0,\infty)
 \qquad
 z(s) = u(x+sb,t+s)
 \quad
 (s\in\mathbb{R})
 \label{eq001}
\end{equation}


この時、次の式が成り立つ。

\begin{equation}
 \dot{z}(s) = Du(x+sb, t+s)\cdot b + u_{t}(x+sb,t+s) =0
\end{equation}



\dotfill

$u:\mathbb{R}^{n}\times [0,\infty)\to\mathbb{R}$

合成関数の微分を用いて$z(s)$を$s$で微分する。
\begin{align}
 \frac{d}{d s}z(s)
  =& \frac{\partial z}{\partial x}\frac{d x}{d s}
   + \frac{\partial z}{\partial t}\frac{d t}{d s}\\
%  =& \frac{\partial u}{\partial x}(x+sb,t+s)\cdot\frac{d (x+sb)}{d s}
%   + \frac{\partial u}{\partial t}(x+sb,t+s)\cdot\frac{d (t+s)}{d s}\\
%  =& \frac{\partial u}{\partial x}(x+sb,t+s)\cdot\frac{d (x+sb)}{d s}
%   + u_{t}
\end{align}

第2項$\frac{\partial z}{\partial t}\frac{d t}{d s}$は次のように計算できる。
\begin{equation}
 \frac{\partial z}{\partial t}\frac{d t}{d s}
  = \frac{\partial u}{\partial t}(x+sb,t+s)
  = u_{t}(x+sb,t+s)
\end{equation}

第1項$\frac{\partial z}{\partial x}\frac{d x}{d s}$は
多変数関数の微分であるので、
$x=(x_1,\dots,x_n)\in\mathbb{R}^n$より
次のようになる。
\begin{align}
 \frac{\partial z}{\partial x}\frac{d x}{d s}
  =& \sum_{i=1}^{n}\frac{\partial z}{\partial x_i}\frac{d x_i}{d s}
  = \sum_{i=1}^{n}\frac{\partial u}{\partial x_i}(x_i+sb_i)\cdot\frac{d (x_i+sb_i)}{d s}\\
  =& \sum_{i=1}^{n} u_{x_i}(x_{i}+sb_{i})\cdot b_{i}
  = Du(x+sb)\cdot b
\end{align}

よって、
$z(s)$を$s$で微分すると
$Du(x+sb, t+s)\cdot b + u_{t}(x+sb,t+s)$が得られる。

式\eqref{eq001}より
$\mathbb{R}^{n} \times (0,\infty)$上で、
$u_{t} + b \cdot Du = 0$
であるので、
$t+s>0$において
$Du(x+sb, t+s)\cdot b + u_{t}(x+sb,t+s)=0$となる。


\hrulefill

\textbf{Report 1.2}

$b\in\mathbb{R}^{n}$
,\quad
$g:\mathbb{R}^{n}\to\mathbb{R}$



\begin{equation}
 \begin{cases}
  u_{t} + b\cdot Du = 0 & \text{in} \ \mathbb{R}^{n}\times (0,\infty)\\
  u=g & \text{on} \ \mathbb{R}^{n}\times \{ t=0 \}
 \end{cases}
 \label{eq02b}
\end{equation}

\begin{equation}
 u(x,t) = g(x-tb) \quad (x\in\mathbb{R}^{n},\ t\geq 0)
 \label{eq02a}
\end{equation}

\eqref{eq02a}で定義される$u(x,t)$は
\eqref{eq02b}を満たすことを示せ。

\dotfill

$u(x,t) = g(x-tb)$より
$t=0$の時は
$u(x,0) = g(x)$である。

$t>0$において、
$u_{t}$を計算する。
これは$t$で偏微分を行うので、
$g(x-tb)$を偏微分する。
\begin{equation}
 u_{t}
  = \frac{\partial}{\partial t}g(x-tb)
  = \frac{\partial g}{\partial x_{1}}(x-tb)\times (-b_{1})+\cdots+\frac{\partial g}{\partial x_{n}}(x-tb)\times (-b_{n})
\end{equation}

同様に$b\cdot Du$も計算する。
\begin{align}
 b\cdot Du
 =& (b_{1},\dots,b_{n})\cdot(u_{x_{1}},\dots,u_{x_{n}})\\
 =& (b_{1},\dots,b_{n})\cdot\left(\frac{\partial g}{\partial x_{1}}(x-tb),\dots,\frac{\partial g}{\partial x_{n}}(x-tb) \right)\\
 =& b_{1}\frac{\partial g}{\partial x_{1}}(x-tb)+\cdots b_{n}\frac{\partial g}{\partial x_{n}}(x-tb)
\end{align}

よって、$u_{t}+b\cdot Du =0$となる。

\hrulefill

\textbf{Report 1.3}


\dotfill

\hrulefill

\textbf{Report 1.4}


\dotfill

\hrulefill

\end{document}
