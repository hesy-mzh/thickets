\documentclass[12pt,b5paper]{ltjsarticle}

%\usepackage[margin=15truemm, top=5truemm, bottom=5truemm]{geometry}
\usepackage[margin=10truemm]{geometry}

\usepackage{amsmath,amssymb}
%\pagestyle{headings}
\pagestyle{empty}

%\usepackage{listings,url}
%\renewcommand{\theenumi}{(\arabic{enumi})}

%\usepackage{graphicx}

%\usepackage{tikz}
%\usetikzlibrary {arrows.meta}
%\usepackage{wrapfig}	% required for `\wrapfigure' (yatex added)
%\usepackage{bm}	% required for `\bm' (yatex added)

% ルビを振る
\usepackage{luatexja-ruby}	% required for `\ruby'

%% 核Ker 像Im Hom を定義
%\newcommand{\Img}{\mathop{\mathrm{Im}}\nolimits}
%\newcommand{\Ker}{\mathop{\mathrm{Ker}}\nolimits}
%\newcommand{\Hom}{\mathop{\mathrm{Hom}}\nolimits}

%\DeclareMathOperator{\Rot}{rot}
%\DeclareMathOperator{\Div}{div}
%\DeclareMathOperator{\Grad}{grad}
%\DeclareMathOperator{\arcsinh}{arcsinh}
%\DeclareMathOperator{\arccosh}{arccosh}
%\DeclareMathOperator{\arctanh}{arctanh}



%\usepackage{listings,url}
%
%\lstset{
%%プログラム言語(複数の言語に対応,C,C++も可)
%%  language = Python,
%%  language = Lisp,
%  language = C,
%  %背景色と透過度
%  %backgroundcolor={\color[gray]{.90}},
%  %枠外に行った時の自動改行
%  breaklines = true,
%  %自動改行後のインデント量(デフォルトでは20[pt])
%  breakindent = 10pt,
%  %標準の書体
%%  basicstyle = \ttfamily\scriptsize,
%  basicstyle = \ttfamily,
%  %コメントの書体
%%  commentstyle = {\itshape \color[cmyk]{1,0.4,1,0}},
%  %関数名等の色の設定
%  classoffset = 0,
%  %キーワード(int, ifなど)の書体
%%  keywordstyle = {\bfseries \color[cmyk]{0,1,0,0}},
%  %表示する文字の書体
%  %stringstyle = {\ttfamily \color[rgb]{0,0,1}},
%  %枠 "t"は上に線を記載, "T"は上に二重線を記載
%  %他オプション:leftline,topline,bottomline,lines,single,shadowbox
%  frame = TBrl,
%  %frameまでの間隔(行番号とプログラムの間)
%  framesep = 5pt,
%  %行番号の位置
%  numbers = left,
%  %行番号の間隔
%  stepnumber = 1,
%  %行番号の書体
%%  numberstyle = \tiny,
%  %タブの大きさ
%  tabsize = 4,
%  %キャプションの場所("tb"ならば上下両方に記載)
%  captionpos = t
%}



\begin{document}

\hrulefill
\textbf{定義}
\hrulefill

領域$D$上の関数$f(z)$が正則であるとは、
$f(z))$が$D$の任意の点に於いて微分可能である
ときにいう。


\textbf{定理}

関数$f(z)$が点$a+bi$で微分可能であることと
次の条件が必要十分である。

$z=x+yi$ とすると $f(z)=u(x,y)+iv(x,y)$と書ける時、
$u(x,y), \ v(x,y)$が共に点$a+bi$で全微分可能であり、
次の\ruby{Cauchy}{コーシー}-\ruby{Riemann}{リーマン}の関係式が
成り立つ。
\begin{equation}
 \frac{\partial u}{\partial x}(a,b)
  =
 \frac{\partial v}{\partial y}(a,b),\quad
 \frac{\partial u}{\partial y}(a,b)
  = -
 \frac{\partial v}{\partial x}(a,b)
\end{equation}

\hrulefill
\textbf{問題}
\hrulefill

$z=x+yi \ (x,y\in\mathbb{R}),\ f(z)=u(x,y)+iv(x,y)$
($u,v$は実数値関数)とし、
$f(z)$は$\mathbb{C}$上で正則であるとする。
\begin{enumerate}
 \item\label{p1}
      $u(x,y)=4x^3y-4xy^3$のとき、
      $v(x,y)$が満たす
      全微分方程式$P(x,y)dx + Q(x,y)dy =0$
      を求めよ。

 \item
      上記 \ref{p1} を用いて、$v(x,y)$を求めよ。
\end{enumerate}

\dotfill

$f(z)$は$\mathbb{C}$上で正則である為、
$\mathbb{C}$の任意の点で
\ruby{Cauchy}{コーシー}-\ruby{Riemann}{リーマン}の関係式
が成り立つ。
\begin{equation}
 \frac{\partial u}{\partial x}(x,y)
  =
 \frac{\partial v}{\partial y}(x,y),\quad
 \frac{\partial u}{\partial y}(x,y)
  = -
 \frac{\partial v}{\partial x}(x,y)
\end{equation}

\begin{enumerate}
 \item
      $v(x,y)$の全微分は
      \begin{equation}
       dv
        =\frac{\partial v}{\partial x}dx
        + \frac{\partial v}{\partial y}dy
      \end{equation}
       であるが、
       $u(x,y)=4x^3y-4xy^3$と
       上記関係式より
       \begin{equation}
        \frac{\partial v}{\partial x}
         = -\frac{\partial u}{\partial y}
         = -4x^3 +12xy^2
         ,\quad
         \frac{\partial v}{\partial y}
         = \frac{\partial u}{\partial x}
         = 12x^2y-4y^3
       \end{equation}
       である。
       この為、求めるべき方程式は次の式である。
       \begin{equation}
        (-x^3+3xy^2)dx + (3x^2y-y^3)dy =0
       \end{equation}

 \item
      $v(x,y)$の全微分が0となる式$dv=0$が得られたので、
      次の式を利用し$v(x,y)$を計算する。
      \begin{equation}
        \frac{\partial v}{\partial x}
         = -4x^3 +12xy^2
         ,\quad
         \frac{\partial v}{\partial y}
         = 12x^2y-4y^3
      \end{equation}

      $\frac{\partial v}{\partial x}= -4x^3 +12xy^2$より
      $x$で積分を行うと次のような式が得られる。
      \begin{equation}
       v = \int (-4x^3 +12xy^2)dx +C(y)
        = -x^4 + 6x^2y^2 +C(y)
        \label{sol}
      \end{equation}
      $C(y)$は$x$を含まない$y$の関数である。
      これを$y$で偏微分する。
      \begin{equation}
       \frac{\partial v}{\partial y}
        =\frac{\partial}{\partial y}(-x^4 + 6x^2y^2 +C(y))
        = 12x^2y +\frac{\partial}{\partial y}C(y)
      \end{equation}

      $\frac{\partial v}{\partial y}= 12x^2y-4y^3$より
      \begin{equation}
       12x^2y +\frac{\partial}{\partial y}C(y)
        = 12x^2y-4y^3
      \end{equation}
      であるから
      $\frac{\partial}{\partial y}C(y)=-4y^3$
      であり、
      積分をすることで$C(y)=-y^4$であることが分かる。

      $dv=0$であるので定数$C$を用いて$v(x,y)=C$となる。
      よって、式(\ref{sol})より
      \begin{equation}
       v(x,y) = -x^4 + 6x^2y^2 -y^4 =C
      \end{equation}
      であることが分かる。

\end{enumerate}


\end{document}
