\documentclass[12pt,b5paper]{ltjsarticle}

%\usepackage[margin=15truemm, top=5truemm, bottom=5truemm]{geometry}
%\usepackage[margin=10truemm,left=15truemm]{geometry}
\usepackage[margin=10truemm]{geometry}

\usepackage{amsmath,amssymb}
%\pagestyle{headings}
\pagestyle{empty}

%\usepackage{listings,url}
%\renewcommand{\theenumi}{(\arabic{enumi})}

%\usepackage{graphicx}

%\usepackage{tikz}
%\usetikzlibrary {arrows.meta}
%\usepackage{wrapfig}	% required for `\wrapfigure' (yatex added)
%\usepackage{bm}	% required for `\bm' (yatex added)

% ルビを振る
%\usepackage{luatexja-ruby}	% required for `\ruby'

%% 核Ker 像Im Hom を定義
%\newcommand{\Img}{\mathop{\mathrm{Im}}\nolimits}
%\newcommand{\Ker}{\mathop{\mathrm{Ker}}\nolimits}
%\newcommand{\Hom}{\mathop{\mathrm{Hom}}\nolimits}

%\DeclareMathOperator{\Rot}{rot}
%\DeclareMathOperator{\Div}{div}
%\DeclareMathOperator{\Grad}{grad}
%\DeclareMathOperator{\arcsinh}{arcsinh}
%\DeclareMathOperator{\arccosh}{arccosh}
%\DeclareMathOperator{\arctanh}{arctanh}



%\usepackage{listings,url}
%
%\lstset{
%%プログラム言語(複数の言語に対応,C,C++も可)
%  language = Python,
%%  language = Lisp,
%%  language = C,
%  %背景色と透過度
%  %backgroundcolor={\color[gray]{.90}},
%  %枠外に行った時の自動改行
%  breaklines = true,
%  %自動改行後のインデント量(デフォルトでは20[pt])
%  breakindent = 10pt,
%  %標準の書体
%%  basicstyle = \ttfamily\scriptsize,
%  basicstyle = \ttfamily,
%  %コメントの書体
%%  commentstyle = {\itshape \color[cmyk]{1,0.4,1,0}},
%  %関数名等の色の設定
%  classoffset = 0,
%  %キーワード(int, ifなど)の書体
%%  keywordstyle = {\bfseries \color[cmyk]{0,1,0,0}},
%  %表示する文字の書体
%  %stringstyle = {\ttfamily \color[rgb]{0,0,1}},
%  %枠 "t"は上に線を記載, "T"は上に二重線を記載
%  %他オプション:leftline,topline,bottomline,lines,single,shadowbox
%  frame = TBrl,
%  %frameまでの間隔(行番号とプログラムの間)
%  framesep = 5pt,
%  %行番号の位置
%  numbers = left,
%  %行番号の間隔
%  stepnumber = 1,
%  %行番号の書体
%%  numberstyle = \tiny,
%  %タブの大きさ
%  tabsize = 4,
%  %キャプションの場所("tb"ならば上下両方に記載)
%  captionpos = t
%}



\begin{document}


\begin{enumerate}
 \item
      \textbf{---ユークリッド空間の直積位相---}
      
      $(\mathbb{R}^n,\mathcal{O}_{d_{n}})$は
      $(\mathbb{R}^n,\mathcal{O}_{d_{1}}^n)$と
      同値であることを示せ。

      \dotfill

      距離関数$d_{1},d_{n}$は次のような関数である。
      \begin{align}
       d_{1}:&\mathbb{R}\times\mathbb{R} \to \mathbb{R}
       & (x,y) & \mapsto \sqrt{(x-y)^2}=\lvert x-y \rvert\\
       d_{n}:&\mathbb{R}^n\times\mathbb{R}^n \to \mathbb{R}
       & ((x_1,\dots,x_n),(y_1,\dots,y_n))
       & \mapsto \sqrt{\sum_{i=1}^n(x_i-y_i)^2}
      \end{align}

      位相$\mathcal{O}_{d_{1}},\mathcal{O}_{d_{n}}$は
      それぞれの距離関数より導入される位相である。

      $\mathcal{O}_{d_{1}}^n
      =\mathcal{O}_{d_{1}}\times\cdots\times\mathcal{O}_{d_{1}}$は
      $\mathcal{O}_{d_1}$の開集合の直積を要素とする。

      \dotfill

      $x=(x_1,\dots,x_n)\in\mathbb{R}^n$とする。
      ある$\varepsilon$に対して、
      $x\in\mathbb{R}^n$の$\varepsilon$-近傍$N_{d_n}(x,\varepsilon)$
      と
      $x_i\in\mathbb{R}$の$\varepsilon$-近傍$N_{d_1}(x_i,\varepsilon)$
      において次のような包含関係が成り立つ。
      \begin{equation}
       N_{d_n}(x,\varepsilon)
        \subset N_{d_1}(x_1,\varepsilon)
        \times\cdots\times N_{d_1}(x_n,\varepsilon)
      \end{equation}
      $N_{d_n}(x,\varepsilon)$は中心$x$で半径$\varepsilon$の球の内部
      であり、
      $N_{d_1}(x_i,\varepsilon)$は
      開区間$(x_i-\varepsilon , x_i+\varepsilon)$を指す。


      また、$N_{d_n}(x,\varepsilon)$の半径を広げ
      $N_{d_n}(x,\sqrt{n}\varepsilon)$とすると
      次のような包含関係が成り立つ。
      \begin{equation}
        N_{d_1}(x_1,\varepsilon)
        \times\cdots\times N_{d_1}(x_n,\varepsilon)
        \subset N_{d_n}(x,\sqrt{n}\varepsilon)
      \end{equation}

      2つをまとめると次のようになる。
      \begin{equation}
       N_{d_n}(x,\varepsilon)
        \subset N_{d_1}(x_1,\varepsilon)
        \times\cdots\times N_{d_1}(x_n,\varepsilon)
        \subset N_{d_n}(x,\sqrt{n}\varepsilon)
      \end{equation}

      $U\subset\mathbb{R}^n$について
      任意の$U$の点の近傍が$U$に含まれる時
      $U$は開集合である。
      上の包含関係より次の関係がわかる。
      \begin{align}
       U \text{ は } (\mathbb{R}^n,\mathcal{O}_{d_{n}}) \text{ で開集合}
       & \Rightarrow
        U \text{ は } (\mathbb{R}^n,\mathcal{O}_{d_{1}}^n) \text{ で開集合}\\
       & \Rightarrow
       U \text{ は } (\mathbb{R}^n,\mathcal{O}_{d_{n}}) \text{ で開集合}
      \end{align}

      よって、
      $(\mathbb{R}^n,\mathcal{O}_{d_{1}}^n)$
      と
      $ (\mathbb{R}^n,\mathcal{O}_{d_{n}})$
      の開集合が一致する事がわかる。

      \hrulefill

 \item
      \textbf{---連続単射写像---}

      $f:\mathbb{R}\to\mathbb{R}$を連続写像とする。
      この時、
      $F:\mathbb{R}\to\mathbb{R}^2$を
      $F:x\mapsto (x,f(x))$とすると、
      $F$は単射な連続写像であることを示せ。

      \dotfill

      $x,y\in\mathbb{R}$が$x\ne y$とする。
      $x\ne y$であれば$(x,f(x))\ne (y,f(y))$であるので、
      写像$F$は単射である。

      射影$p_i$を次のように定める。
      \begin{equation}
       p_{1}:\mathbb{R}^2\to\mathbb{R}, \ (a,b)\mapsto a
        ,\qquad
       p_{2}:\mathbb{R}^2\to\mathbb{R}, \ (a,b)\mapsto b
      \end{equation}

      これらの射影と$F$の合成は次のように連続写像となる。
      \begin{equation}
       p_{1}\circ F =\mathrm{id}_{\mathbb{R}}
        ,\qquad
        p_{2}\circ F = f
      \end{equation}

      $\mathbb{R}^2$の開集合は
      開集合$U_1,U_2\subset\mathbb{R}$の積$U_1\times U_2$を開基とする。
      写像$F$が連続写像であるためには、
      $F^{-1}(U_1\times U_2)$が開集合であることを示せばよい。
      \begin{align}
       F^{-1}(U_1\times U_2)
        =& F^{-1}(p_{1}^{-1}(U_1)\cap p_{2}^{-1}(U_2))\\
        =& F^{-1}(p_{1}^{-1}(U_1))\cap F^{-1}(p_{2}^{-1}(U_2))\\
        =& (p_{1}\circ F)^{-1}(U_1)\cap (p_{2}\circ F)^{-1}(U_2)\\
        =& \mathrm{id}_{\mathbb{R}}^{-1}(U_1)\cap f^{-1}(U_2)
      \end{align}

      $\mathrm{id}_{\mathbb{R}}^{-1}(U_1), \ f^{-1}(U_2)$
      はそれぞれ開集合であるので、
      $F^{-1}(U_1\times U_2)$は開集合となり、
      $F$は連続写像であることがわかる。

      \hrulefill

 \item
      \textbf{---直積位相---}

      $X,Y$を位相空間とし、
      $X\times Y$に直積位相を与えておく。
      $X\times Y$に対して
      $B$が$(x,y)\in X\times Y$の近傍であるとは、
      ある$x\in X$の開集合$U$と
      $y\in Y$の開集合$V$が存在して、
      $U\times V \subset B \subset X\times Y$
      となることを示せ。


      \dotfill

      $B$が$(x,y)\in X\times Y$の近傍であれば、
      $(x,y)\in O \subset B \subset X\times Y$
      となる開集合$O$が存在する。

      $X\times Y$に直積位相が入っているため、
      開集合$O$は開基$U\times V$が存在し
      $(x,y)\in U\times V \subset O$
      を満たす。
      この$U,V$はそれぞれ$X,Y$の開集合であり、
      $x\in U \subset X, \ y\in V \subset Y$である。

      つまり、次のような関係がある。
      \begin{equation}
       (x,y)\in U\times V \subset O \subset B \subset X\times Y
        \qquad (x\in U,\ y\in V)
      \end{equation}

      \hrulefill

\end{enumerate}



\end{document}
