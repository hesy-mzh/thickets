\documentclass[12pt,b5paper]{ltjsarticle}

%\usepackage[margin=15truemm, top=5truemm, bottom=5truemm]{geometry}
%\usepackage[margin=10truemm,left=15truemm]{geometry}
\usepackage[margin=10truemm]{geometry}

\usepackage{amsmath,amssymb}
%\pagestyle{headings}
\pagestyle{empty}

%\usepackage{listings,url}
\renewcommand{\theenumi}{(\arabic{enumi})}

%\usepackage{graphicx}

%\usepackage{tikz}
%\usetikzlibrary {arrows.meta}
%\usepackage{wrapfig}
\usepackage{bm}

% ルビを振る
%\usepackage{luatexja-ruby}	% required for `\ruby'

%% 核Ker 像Im Hom を定義
%\newcommand{\Img}{\mathop{\mathrm{Im}}\nolimits}
%\newcommand{\Ker}{\mathop{\mathrm{Ker}}\nolimits}
%\newcommand{\Hom}{\mathop{\mathrm{Hom}}\nolimits}

%\DeclareMathOperator{\Rot}{rot}
%\DeclareMathOperator{\Div}{div}
%\DeclareMathOperator{\Grad}{grad}
%\DeclareMathOperator{\arcsinh}{arcsinh}
%\DeclareMathOperator{\arccosh}{arccosh}
%\DeclareMathOperator{\arctanh}{arctanh}

%\usepackage{url}

%\usepackage{listings}
%
%\lstset{
%%プログラム言語(複数の言語に対応,C,C++も可)
%  language = Python,
%%  language = Lisp,
%%  language = C,
%  %背景色と透過度
%  %backgroundcolor={\color[gray]{.90}},
%  %枠外に行った時の自動改行
%  breaklines = true,
%  %自動改行後のインデント量(デフォルトでは20[pt])
%  breakindent = 10pt,
%  %標準の書体
%%  basicstyle = \ttfamily\scriptsize,
%  basicstyle = \ttfamily,
%  %コメントの書体
%%  commentstyle = {\itshape \color[cmyk]{1,0.4,1,0}},
%  %関数名等の色の設定
%  classoffset = 0,
%  %キーワード(int, ifなど)の書体
%%  keywordstyle = {\bfseries \color[cmyk]{0,1,0,0}},
%  %表示する文字の書体
%  %stringstyle = {\ttfamily \color[rgb]{0,0,1}},
%  %枠 "t"は上に線を記載, "T"は上に二重線を記載
%  %他オプション:leftline,topline,bottomline,lines,single,shadowbox
%  frame = TBrl,
%  %frameまでの間隔(行番号とプログラムの間)
%  framesep = 5pt,
%  %行番号の位置
%  numbers = left,
%  %行番号の間隔
%  stepnumber = 1,
%  %行番号の書体
%%  numberstyle = \tiny,
%  %タブの大きさ
%  tabsize = 4,
%  %キャプションの場所("tb"ならば上下両方に記載)
%  captionpos = t
%}

%\usepackage{cancel}
%\usepackage{bussproofs}
%\usepackage{proof}

\begin{document}

\hrulefill

$2$次元空間における
次の閉曲線$C$上の
線積分$\displaystyle \int_{C} \bm{A}(\bm{r})\cdot \mathrm{d}\bm{r}$
を計算せよ。

ここで、
閉曲線$C$は原点を中心とする半径$a$の円周で、
$(a,0)$を始点に反時計回りに一周する。
$\bm{A}(\bm{r})$は次式で与えられる。
\begin{equation}
 \bm{A}(\bm{r})
  =\frac{-y}{\sqrt{x^{2}+y^{2}}}\bm{e}_{x}
  +\frac{x}{\sqrt{x^{2}+y^{2}}}\bm{e}_{y}
  ,\quad
  (\bm{r}=x\bm{e}_{x}+y\bm{e}_{y})
\end{equation}

曲線$C$上では、$\bm{r},\mathrm{d}\bm{r} \: (0\leq \theta \leq 2\pi)$
は次の式で表される。
\begin{equation}
 \bm{r}=a(\cos{\theta}\bm{e}_{x}+\sin{\theta}\bm{e}_{y}),\quad
 \mathrm{d}\bm{r}=a(-\sin{\theta}\bm{e}_{x}+\cos{\theta}\bm{e}_{y})\mathrm{d}\theta
\end{equation}

\hrulefill

$\bm{r}=x\bm{e}_{x}+y\bm{e}_{y}$は
曲線$C$上で極座標表示をすると次のようになる。この時、$a>0$である。
\begin{equation}
 \bm{r}=x\bm{e}_{x}+y\bm{e}_{y}
 = a(\cos{\theta}\bm{e}_{x}+\sin{\theta}\bm{e}_{y})
\end{equation}

ここから、
$x=a\cos{\theta}, \; y=a\sin{\theta}$
であることがわかる。
そこで、$\bm{A}(\bm{r})$の各成分を極座標表示に書き換える。
\begin{align}
 \frac{-y}{\sqrt{x^{2}+y^{2}}}
 &=\frac{-a\sin{\theta}}{\sqrt{(a\cos{\theta})^{2}+(a\sin{\theta})^{2}}}
 =-\sin{\theta}\\
%
  \frac{x}{\sqrt{x^{2}+y^{2}}}
 &=
  \frac{a\cos{\theta}}{\sqrt{(a\cos{\theta})^{2}+(a\sin{\theta})^{2}}}
 =\cos{\theta}
\end{align}

$\bm{A}(\bm{r})\cdot \mathrm{d}\bm{r}$
は内積であるので、
各成分ごとの積を計算することで次のような式となる。
$\theta$は円周一周りなので、$0$から$2\pi$の積分範囲となる。
\begin{align}
 \int_{C} \bm{A}(\bm{r})\cdot \mathrm{d}\bm{r}
  &=
  \int_{0}^{2\pi} (-\sin{\theta}\bm{e}_{x}+\cos{\theta}\bm{e}_{y})
  \cdot
(a(-\sin{\theta}\bm{e}_{x}+\cos{\theta}\bm{e}_{y})\mathrm{d}\theta)\\
  &= \int_{0}^{2\pi} a(\sin^{2}{\theta}+\cos^{2}{\theta})\mathrm{d}\theta
  = \int_{0}^{2\pi} a\mathrm{d}\theta
  = 2\pi a
\end{align}



\hrulefill
\end{document}
