\documentclass[12pt,b5paper]{ltjsarticle}

%\usepackage[margin=15truemm, top=5truemm, bottom=5truemm]{geometry}
%\usepackage[margin=10truemm,left=15truemm]{geometry}
\usepackage[margin=10truemm]{geometry}

\usepackage{amsmath,amssymb}
%\pagestyle{headings}
\pagestyle{empty}

%\usepackage{listings,url}
%\renewcommand{\theenumi}{(\arabic{enumi})}

%\usepackage{graphicx}

%\usepackage{tikz}
%\usetikzlibrary {arrows.meta}
%\usepackage{wrapfig}
%\usepackage{bm}

% ルビを振る
%\usepackage{luatexja-ruby}	% required for `\ruby'

%% 核Ker 像Im Hom を定義
%\newcommand{\Img}{\mathop{\mathrm{Im}}\nolimits}
%\newcommand{\Ker}{\mathop{\mathrm{Ker}}\nolimits}
%\newcommand{\Hom}{\mathop{\mathrm{Hom}}\nolimits}

%\DeclareMathOperator{\Rot}{rot}
%\DeclareMathOperator{\Div}{div}
%\DeclareMathOperator{\Grad}{grad}
%\DeclareMathOperator{\arcsinh}{arcsinh}
%\DeclareMathOperator{\arccosh}{arccosh}
%\DeclareMathOperator{\arctanh}{arctanh}

%\usepackage{url}

%\usepackage{listings}
%
%\lstset{
%%プログラム言語(複数の言語に対応,C,C++も可)
%  language = Python,
%%  language = Lisp,
%%  language = C,
%  %背景色と透過度
%  %backgroundcolor={\color[gray]{.90}},
%  %枠外に行った時の自動改行
%  breaklines = true,
%  %自動改行後のインデント量(デフォルトでは20[pt])
%  breakindent = 10pt,
%  %標準の書体
%%  basicstyle = \ttfamily\scriptsize,
%  basicstyle = \ttfamily,
%  %コメントの書体
%%  commentstyle = {\itshape \color[cmyk]{1,0.4,1,0}},
%  %関数名等の色の設定
%  classoffset = 0,
%  %キーワード(int, ifなど)の書体
%%  keywordstyle = {\bfseries \color[cmyk]{0,1,0,0}},
%  %表示する文字の書体
%  %stringstyle = {\ttfamily \color[rgb]{0,0,1}},
%  %枠 "t"は上に線を記載, "T"は上に二重線を記載
%  %他オプション:leftline,topline,bottomline,lines,single,shadowbox
%  frame = TBrl,
%  %frameまでの間隔(行番号とプログラムの間)
%  framesep = 5pt,
%  %行番号の位置
%  numbers = left,
%  %行番号の間隔
%  stepnumber = 1,
%  %行番号の書体
%%  numberstyle = \tiny,
%  %タブの大きさ
%  tabsize = 4,
%  %キャプションの場所("tb"ならば上下両方に記載)
%  captionpos = t
%}

%\usepackage{cancel}
%\usepackage{bussproofs}
%\usepackage{proof}

\begin{document}


\hrulefill

\begin{align}
 V(X) &= E(X^{2})-(E(X))^{2}\\
 V(Y) &= E(Y^{2})-(E(Y))^{2}\\
 Cov(X,Y) &= E( (X-E(X)(Y-E(Y)) )\\
 &= E(XY)-E(XE(Y))-E(E(X)Y)-E(X)E(Y)\\
 &= E(XY)-E(X)E(Y)
\end{align}


\begin{align}
 r &= \frac{s_{xy}}{s_xs_y} \\
 &= \frac{Cov(X,Y)}{\sqrt{V(X)}\sqrt{V(Y)}}\\
 &=
 \frac{\frac{1}{n}\sum_{i=1}^n(x_i-\overline{x})(y_i-\overline{y})}
 {\sqrt{\frac{1}{n}\sum_{i=1}^n(x_i-\overline{x})^2}
 \sqrt{\frac{1}{n}\sum_{i=1}^n(y_i-\overline{y})^2}}
\end{align}


$X,Y$が独立である場合
\begin{gather}
 E(XY)=E(X)E(Y),\quad Cov(XY)=0\\
 V(X+Y)=V(X)+2Cov(XY)+V(Y)=V(X)+V(Y)
\end{gather}


\hrulefill

\begin{enumerate}
 \item
      $E(X_{k})=0,\; E(Y_{k})=0 \; (k=1,\dots,n)$とする。

      $n$個の確率ベクトル$(X_{k},Y_{k})$が
      共分散$\sigma_{X,Y}=E(X_{k}Y_{k})$を持つ
      (2変量の)分布からの
      無作為標本とする。
      この時、
      \begin{equation}
       \hat{\sigma}_{X,Y}
        =\frac{1}{n-1}\sum_{k=1}^{n}(X_{k}-\bar{X})(Y_{k}-\bar{Y})
        \quad
        (ただし、\bar{X}=\frac{1}{n}\sum_{k=1}^{n}X_{k},
        \bar{Y}=\frac{1}{n}\sum_{k=1}^{n}Y_{k})
      \end{equation}
      が$\sigma_{X,Y}$の不偏推定量であるか調べよ。
      不偏推定量でない場合は
      不偏推定量になるように修正せよ。

      (HINT:$k\ne k^{\prime}$ならば$X_{k}$と$Y_{k^{\prime}}$は独立なので
      $E(X_{k}Y_{k^{\prime}})=0$となる)

      \dotfill

      条件をまとめると次の式となる。
      \begin{gather}
       E(X_{1})=\cdots=E(X_{n})=0,\quad E(Y_{1})=\cdots=E(Y_{n})=0\\
       \sigma_{X,Y}=E(X_{1}Y_{1})=\cdots=E(X_{n}Y_{n})\\
       k\ne k^{\prime} \Rightarrow E(X_{k}Y_{k^{\prime}})=0
      \end{gather}


%      \begin{gather}
%       \mathrm{V}[X+Y] = \mathrm{V}[X] + 2\mathrm{Cov}[X,Y] + \mathrm{V}[Y]\\
%       \mathrm{Cov}[X,Y]=\frac{1}{2}\left( \mathrm{V}[X+Y] - \mathrm{V}[X] - \mathrm{V}[Y] \right)\\
%       \mathrm{V}[X] = \mathrm{E}[X^{2}]-(\mathrm{E}[X])^{2},\quad
%       \mathrm{V}[Y] = \mathrm{E}[Y^{2}]-(\mathrm{E}[Y])^{2}\\
%       \mathrm{V}[X+Y] = \mathrm{E}[(X+Y)^{2}]-(\mathrm{E}[X+Y])^{2}
%      \end{gather}

      $E(\hat{\sigma}_{X,Y})$を計算する。
      \begin{align}
       E(\hat{\sigma}_{X,Y})
        =&\; E\left(\frac{1}{n-1}\sum_{k=1}^{n}(X_{k}-\bar{X})(Y_{k}-\bar{Y})\right)\\
        =&\; \frac{1}{n-1}\sum_{k=1}^{n}(E(X_{k}Y_{k}) -E(\bar{X}Y_{k})-E(X_{k}\bar{Y}) + E(\bar{X}\bar{Y}) )
      \end{align}

      $E(\bar{X}Y_{k})$は次のように計算できる。
      \begin{gather}
       E(\bar{X}Y_{k})
       = E \left( \frac{1}{n}\sum_{i=1}^{n}X_{i}Y_{k} \right)
       = \frac{1}{n}\sum_{i=1}^{n} E( X_{i}Y_{k} )
       = \frac{1}{n} E( X_{k}Y_{k} )
      \end{gather}
      同様に、$E(X_{k}\bar{Y})=\frac{1}{n} E(X_{k}Y_{k})$である。

      $E(\bar{X}\bar{Y})$を計算する。
      \begin{align}
       E(\bar{X}\bar{Y})
       =&\; E \left( \left(\frac{1}{n}\sum_{i=1}^{n}X_{i}\right) \left(\frac{1}{n}\sum_{j=1}^{n}Y_{j} \right) \right)
       = \frac{1}{n^{2}} E \left( \sum_{i=1}^{n}\sum_{j=1}^{n} X_{i}Y_{j} \right)\\
       =&\; \frac{1}{n^{2}} \sum_{i=1}^{n}\sum_{j=1}^{n} E (X_{i}Y_{j} )
       = \frac{1}{n^{2}} \sum_{i=1}^{n} E(X_{i}Y_{i})
      \end{align}

      \begin{align}
       E(\hat{\sigma}_{X,Y})
        =&\; \frac{1}{n-1}\sum_{k=1}^{n}(E(X_{k}Y_{k}) -E(\bar{X}Y_{k})-E(X_{k}\bar{Y}) + E(\bar{X}\bar{Y}) )\\
       =&\; \frac{1}{n-1}\sum_{k=1}^{n}(E(X_{k}Y_{k})
       -\frac{1}{n} E( X_{k}Y_{k} )
       -\frac{1}{n} E(X_{k}Y_{k})
       + \frac{1}{n^{2}} \sum_{i=1}^{n} E(X_{i}Y_{i}) )\\
       =&\; \frac{1}{n-1}\left( \sum_{k=1}^{n}\frac{n-2}{n}E(X_{k}Y_{k})
       +\frac{n}{n^{2}} \sum_{i=1}^{n} E(X_{i}Y_{i}) \right)\\
       =&\; \frac{1}{n-1}\left(\frac{n-2}{n} +\frac{1}{n}\right)\sum_{k=1}^{n}E(X_{k}Y_{k})\\
       ~=&\; \frac{1}{n} \sum_{k=1}^{n}E(X_{k}Y_{k}) = \sigma_{X,Y}
      \end{align}

      よって、
      $\hat{\sigma}_{X,Y}$は$\sigma_{X,Y}$の不偏推定量である。

      \hrulefill

 \item
      $X\sim Po(\lambda_{1}),\; Y\sim Po(\lambda_{2})$で
      $X$と$Y$が独立とする($\lambda_{1},\lambda_{2}>0$)。
      その時、
      $U=X+Y$の従う分布を求める。

      \dotfill

      \begin{gather}
       E(X)=V(X)=\lambda_{1},\quad E(Y)=V(Y)=\lambda_{2}\\
       f_{X}(x)=P(X=x)=\frac{\lambda_{1}^{x}}{x!}\cdot e^{-\lambda_{1}}
       ,\quad f_{Y}(y)=P(Y=y)=\frac{\lambda_{2}^{y}}{y!}\cdot e^{-\lambda_{2}}\\
       \sum_{x=0}^{\infty}f_{X}(x)=1,\quad \sum_{y=0}^{\infty}f_{Y}(y)=1\\
       M_{X}(t)=E(e^{tX})
       =\sum_{k=0}^{\infty}e^{tk}\frac{\lambda_{1}^{k}e^{-\lambda_{1}}}{k!}
       =e^{\lambda_{1}(e^{t}-1)}\\
       M_{Y}(t)=E(e^{tY})
       =\sum_{k=0}^{\infty}e^{tk}\frac{\lambda_{2}^{k}e^{-\lambda_{2}}}{k!}
       =e^{\lambda_{2}(e^{t}-1)}
      \end{gather}

      \hrulefill

      \begin{enumerate}
       \item
            積率母関数を用いて求めよ。

            \dotfill

            \begin{align}
             M_{U}(t)
             =&\; E(e^{tU})
             = \sum_{k=0}^{\infty} \frac{t^{k}}{k!} E(U^{k})
             = \sum_{k=0}^{\infty} \frac{t^{k}}{k!} E((X+Y)^{k})\\
             =&\; \sum_{k=0}^{\infty} \frac{t^{k}}{k!} E\left(\sum_{i=0}^{k}X^{i}Y^{k-i}\right)
             = \sum_{k=0}^{\infty} \sum_{i=0}^{k} \frac{t^{k}}{k!} E(X^{i}Y^{k-i})\\
             =&\; \left(\sum_{k=0}^{\infty} \frac{t^{k}}{k!} E(X^{k})\right)
             \left(\sum_{l=0}^{\infty} \frac{t^{l}}{l!} E(Y^{l})\right)\\
             =&\; M_{X}(t) M_{Y}(t)
             = e^{\lambda_{1}(e^{t}-1)}e^{\lambda_{2}(e^{t}-1)}\\
             =&\; e^{(\lambda_{1}+\lambda_{2})(e^{t}-1)}
            \end{align}

            よって、
            $U=X+Y \sim Po(\lambda_{1}+\lambda_{2})$
            となり、
            ポアソン分布に従うことが分かる。

            \hrulefill

       \item
            畳み込みを用いて求めよ。

            (HINT:$f_{U}(u)=\sum_{x=0}^{u} f_{X}(x)f_{Y}(u-x)$を計算。
            $f_{X},f_{Y}$は$X,Y$のそれぞれの確率質量関数)

            \dotfill

            \begin{align}
             f_{U}(u)
             =&\; \sum_{x=0}^{u} f_{X}(x) f_{Y}(u-x)
             = \sum_{x=0}^{u}
             \frac{\lambda_{1}^{x}e^{-\lambda_{1}}}{x!}
             \frac{\lambda_{2}^{u-x}e^{-\lambda_{2}}}{(u-x)!}\\
             =&\; e^{-\lambda_{1}-\lambda_{2}}
             \sum_{x=0}^{u} \frac{\lambda_{1}^{x} \lambda_{2}^{u-x}}{x!(u-x)!}
             = \frac{e^{-\lambda_{1}-\lambda_{2}}}{u!}
             \sum_{x=0}^{u} \frac{u!}{x!(u-x)!}\lambda_{1}^{x} \lambda_{2}^{u-x}\\
             =&\; \frac{e^{-(\lambda_{1}+\lambda_{2})}}{u!}
             \sum_{x=0}^{u} {}_{u}C_{x} \lambda_{1}^{x} \lambda_{2}^{u-x}
             = \frac{e^{-(\lambda_{1}+\lambda_{2})}}{u!}
             (\lambda_{1}+\lambda_{2})^{u}\\
             =&\; \frac{(\lambda_{1}+\lambda_{2})^{u}}{u!}e^{-(\lambda_{1}+\lambda_{2})}
            \end{align}

            これにより、
            $U=X+Y \sim Po(\lambda_{1}+\lambda_{2})$
            であることが分かる。



            \hrulefill

      \end{enumerate}

 \item
      確率変数 $X_{1},\dots,X_{n}$が互いに独立に
      それぞれ$N(\mu,\sigma^{2})$に従うとする。
      \begin{enumerate}
       \item
            $Y_{i}=X_{i}-\bar{X} \;(i=1,\dots,n)$が従う分布を求めよ。

            (HINT:正規分布の再生性の性質を使う)

            \dotfill

            各 $i$ について、
            $X_{i} \sim N(\mu,\sigma^{2})$である。

            正規分布の再生性により
            $X_{i}+X_{j}$
            も正規分布である。
            よって、
            $\bar{X} = \frac{1}{n}\sum_{i=1}^{n}X_{i}$
            も正規分布である。

            つまり、
            $Y_{i} = X_{i}-\bar{X}$
            も正規分布である。

            \hrulefill


       \item
            $Y_{i}$と$\bar{X}$の相関係数を求め
            $Y_{i}$と$\bar{X}$の関係を考察せよ。

            \dotfill

            \begin{align}
             E(\bar{X})
             &=E\left( \frac{1}{n}\sum_{i=1}^{n}X_{i} \right)
              = \frac{1}{n}\sum_{i=1}^{n}E(X_{i})
              =\mu\\
             E(Y_{i})
             &= E(X_{i}-\bar{X})
             = E(X_{i})-E(\bar{X})
             =0\\
             V(\bar{X})
             &= E((\bar{X})^{2}) - (E(\bar{X}))^{2}
             %= \frac{1}{n^{2}} E\left( \left( \sum_{i=1}^{n}X_{i}\right)^{2} \right) - \mu^{2}\\
             %&= \frac{1}{n^{2}} E\left( \sum_{j=1}^{n} \sum_{i=1}^{n} X_{j}X_{i} \right) - \mu^{2}
             %= \frac{1}{n} \sum_{j=1}^{n} \sum_{i=1}^{n} \mu^{2} - \mu^{2}\\
             %&= (n-1)\mu^{2}
             \\
             V(Y_{i})
             &= E(Y_{i}^{2}) - (E(Y_{i}))^{2}
             = E((X_{i}-\bar{X})^{2})\\
             &= E(X_{i}^{2}-2X_{i}\bar{X}+\bar{X}^{2})\\
             &= E(X_{i}^{2})-2E(X_{i}\bar{X})+E(\bar{X}^{2})
            \end{align}

            $V(X_{i})=E(X_{i}^{2})-(E(X_{i}))^{2}$より
            $E(X_{i}^{2}) = \sigma^{2} + \mu^{2}$
            である。
            また、$i\ne j$において$X_{i}$と$X_{j}$は独立であるので、
            $E(X_{i}X_{j})=E(X_{i})E(X_{j})$である。
            これらを用いて$E(X_{i}\bar{X})$と$E(\bar{X}^{2})$を計算する。
            \begin{align}
             E(X_{i}\bar{X})
             &= E \left( X_{i} \frac{1}{n}\sum_{j=1}^{n}X_{j} \right)
             = \frac{1}{n} \sum_{j=1}^{n} E \left( X_{i}X_{j} \right)\\
             &= (n-1)\mu^{2} + \sigma^{2} + \mu^{2}
             = n\mu^{2} + \sigma^{2}
             \\
             E(\bar{X}^{2})
             &= E \left( \left( \sum_{j=1}^{n}X_{j}\right) \left( \sum_{i=1}^{n}X_{i} \right) \right)
             = \sum_{j=1}^{n} \sum_{i=1}^{n} E(X_{j}X_{i})\\
             &= (n^{2}-n)\mu^{2} + n(\mu^{2} + \sigma^{2})
             = n^{2}\mu^{2} + n\sigma^{2}
            \end{align}

            これらを用いて$V(Y_{i}),V(\bar{X})$を求める。
            \begin{align}
             V(Y_{i})
             &= E(X_{i}^{2})-2E(X_{i}\bar{X})+E(\bar{X}^{2})\\
             &= (\sigma^{2}+\mu^{2}) -2(n\mu^{2} + \sigma^{2}) + (n^{2}\mu^{2} + n\sigma^{2})\\
             &= (n-1)\sigma^{2} + (n-1)^{2}\mu^{2}\\
             V(\bar{X})
             &= E((\bar{X})^{2}) - (E(\bar{X}))^{2}
             = (n^{2}\mu^{2} + n\sigma^{2}) - \mu^{2}
            \end{align}

            共分散を求める。
            \begin{equation}
             Cov(Y_{i},\bar{X}) = E(Y_{i}\bar{X}) - E(Y_{i})E(\bar{X})
            \end{equation}

            このため、期待値$E(Y_{i}\bar{X})$を計算する。
            \begin{align}
             E(Y_{i}\bar{X})
              &= E\left( (X_{i}-\bar{X})\bar{X} \right)
              = E(X_{i}\bar{X})-E(\bar{X}^{2})\\
              &= (n\mu^{2} + \sigma^{2} ) - (n^{2}\mu^{2} + n\sigma^{2})
              = n(1-n)\mu^{2}+(1-n)\sigma^{2}
            \end{align}

            よって、共分散は次のようになる。
            \begin{equation}
             Cov(Y_{i},\bar{X}) = n(1-n)\mu^{2}+(1-n)\sigma^{2}
            \end{equation}

            ここから
            相関係数
            を求める。
            \begin{align}
             & \frac{Cov(Y_{i},\bar{X})}{\sqrt{V(Y_{i})}\sqrt{V(\bar{X})}}\\
             &=
             \frac{n(1-n)\mu^{2}+(1-n)\sigma^{2}}{\sqrt{(n-1)\sigma^{2} + (n-1)^{2}\mu^{2}}\sqrt{(n^{2}-1)\mu^{2}+n\sigma^{2}}}\\
             &= \frac{(1/n-1)\mu^{2}+(1/n^{2}-1/n)\sigma^{2}}{\sqrt{(1/n-1/n^{2})\sigma^{2} + (1-1/n)^{2}\mu^{2}}\sqrt{(1-1/n^{2})\mu^{2}+(1/n)\sigma^{2}}}\\
             & \rightarrow 1 (n\rightarrow \infty)
            \end{align}

            このため、確率変数を多く取ると
            $Y_{i}$と$\bar{X}$は正比例する。


            \hrulefill

      \end{enumerate}


 \item
      平均$\mu_{1}$、分散$\sigma_{1}^{2} (>0)$の分布に従う確率変数$X$と、
      平均$\mu_{2}$、分散$\sigma_{2}^{2} (>1)$の分布に従う確率変数$Y$とおく。
      \begin{enumerate}
       \item
            $Y=(\sigma_{2}/\sigma_{1})(X-\mu_{1})+\mu_{2}$とかける時、
            $X$と$Y$の相関を求めよ。
            ($\sigma_{i}=\sqrt{\sigma_{i}^{2}} \quad i=1,2$)

            \dotfill

            \begin{gather}
             \sigma_{1}^{2} = E(X^{2})-\mu_{1}^{2},\quad
             \sigma_{2}^{2} = E(Y^{2})-\mu_{2}^{2}\\
             Cov(X,Y) = E(XY) - \mu_{1}\mu_{2}
            \end{gather}

            相関係数
            \begin{align}
             \frac{Cov(X,Y)}{\sqrt{V(X)}\sqrt{V(Y)}}
             = \frac{ E(XY) - \mu_{1}\mu_{2} }{ \sigma_{1} \sigma_{2} }
            \end{align}

            \begin{align}
             E(XY) &= E\left( X \left( \frac{\sigma_{2}}{\sigma_{1}}(X-\mu_{1})+\mu_{2} \right) \right)\\
             &= E \left(
              \frac{\sigma_{2}}{\sigma_{1}}X^{2}-\frac{\sigma_{2}}{\sigma_{1}}\mu_{1}X +\mu_{2}X
             \right)\\
             &=
             \frac{\sigma_{2}}{\sigma_{1}}E(X^{2})-\frac{\sigma_{2}}{\sigma_{1}}\mu_{1}E(X) +\mu_{2}E(X)\\
             &=
             \frac{\sigma_{2}}{\sigma_{1}}(\sigma_{1}^{2} +\mu_{1}^{2})-\frac{\sigma_{2}}{\sigma_{1}}\mu_{1}^{2} +\mu_{1}\mu_{2}\\
             &= \sigma_{1}\sigma_{2} + \mu_{1}\mu_{2}
            \end{align}


            \begin{gather}
             \frac{Cov(X,Y)}{\sqrt{V(X)}\sqrt{V(Y)}}
             =1
            \end{gather}
            これにより相関係数は$1$であることがわかる。

            つまり、$X$と$Y$は正比例する。

            \hrulefill
       \item
            $X$と独立な平均0、分散1の確率変数$Z$を用いて、
            $Y=a(X-\mu_{1})+\mu_{2}+Z$と書ける時、
            $X$と$Y$の相関と$a (\geq 0)$の値を求めよ。

            \dotfill

            \begin{align}
             E(XY)
             &= E\left( X (a(X-\mu_{1})+\mu_{2} +Z) \right)\\
             &= E \left( aX^{2} -a\mu_{1}X +\mu_{2}X +XZ \right)\\
             &= E (aX^{2}) - E(a\mu_{1}X) + E(\mu_{2}X) + E(XZ)\\
             &= aE (X^{2}) - a\mu_{1}^{2} + \mu_{1}\mu_{2} + E(X)E(Z)\\
             &= a(\sigma_{1}^{2}+\mu_{1}^{2}) - a\mu_{1}^{2} + \mu_{1}\mu_{2}\\
             &= a\sigma_{1}^{2} + \mu_{1}\mu_{2}
            \end{align}

            \begin{align}
             \frac{Cov(X,Y)}{\sqrt{V(X)}\sqrt{V(Y)}}
             &= \frac{ E(XY) - \mu_{1}\mu_{2} }{ \sigma_{1} \sigma_{2} }\\
             &= \frac{ a\sigma_{1}^{2} + \mu_{1}\mu_{2} - \mu_{1}\mu_{2} }{ \sigma_{1} \sigma_{2} }\\
             &= \frac{\sigma_{1}}{\sigma_{2}}a
            \end{align}

            相関係数は$-1$から$1$までの値をとるので、
            $-1\leq \frac{\sigma_{1}}{\sigma_{2}}a \leq 1$である。
            ここから、
            $-\frac{\sigma_{2}}{\sigma_{1}} \leq a \leq \frac{\sigma_{2}}{\sigma_{1}}$
            がわかる。

            $a\geq0$より、
            $0\leq a \leq \frac{\sigma_{2}}{\sigma_{1}}$
            である。

            $a$が$0$に近くなれば
            $X$と$Y$は無相関となり、
            $a$が$\frac{\sigma_{2}}{\sigma_{1}}$に近くなれば
            正の相関となる。

            \hrulefill

      \end{enumerate}

\end{enumerate}

\hrulefill

\end{document}
