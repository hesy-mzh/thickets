\documentclass[10pt,b5paper]{ltjsarticle}

\usepackage[margin=15truemm]{geometry}
\pagestyle{empty}

\usepackage{amssymb}
\usepackage{amsmath}	% required for `\align' (yatex added)

\begin{document}

留数定理(Residue theorem)

曲線$C$上で正則な関数$f(z)$がその内部の
有限個の点$\alpha_{1},\dots ,\alpha_{n}$を除いた領域でも
正則である場合、
曲線上の積分は留数の和によって求められる。

\begin{equation}
 \frac{1}{2\pi i}\int_{C}f(z)dz = \sum_{k=1}^{n}\mathrm{Res}(f,\alpha_{k})
\end{equation}

\hrulefill

留数(Residue)

複素関数$f(z)$の点$\alpha$付近でのローラン展開を次のように表す。
\begin{equation}
 f(z) = \sum_{k=-\infty}^{\infty} c_k (z-\alpha)^k
\end{equation}
この時、係数$c_{-1}$のことを関数$f(z)$の点$\alpha$での留数という。
$\mathrm{Res}(f,\alpha)$等の記号で表す。
つまり、$c_{-1} = \mathrm{Res}(f,\alpha)$である。

\hrulefill

ローラン級数(Laurent series)

次のように$-\infty$から$\infty$までを利用した級数を点$\alpha$まわりのローラン級数という。
\begin{equation}
 \sum_{k=-\infty}^{\infty} c_k (z-\alpha)^k
\end{equation}
複素関数をローラン級数に直すことをローラン展開という。


上のローラン級数について、
ある負の整数$p$において$c_{p}\ne 0$であり、
$p$より小さな全ての整数$\hat{p} ({}^\forall \hat{p}<p)$において
$c_{\hat{p}}=0$である時、
点$\alpha$を$-p$位の極、または位数が$-p$であるという。

整数$p$を$p<0$とし、$c_p\ne 0$とすると次のような式になる。
 $\sum_{k=p}^{\infty} c_k (z-\alpha)^k$


\hrulefill

留数の求め方

複素関数がローラン級数で表せていなければ
次の方法で求めることができる。

複素関数$f(z)$が点$\alpha$にて$p$位の極を持つとする。
\begin{equation}
 f(z)=\frac{c_{-p}}{(z-\alpha)^p}+\cdots +\frac{c_{-1}}{(z-\alpha)}
  + \sum_{k=0}^{\infty}c_k(z-\alpha)^k
\end{equation}
これに$(z-\alpha)^{p}$をかけると右辺の分母が払われる。
\begin{equation}
 (z-\alpha)^{p}f(z)=c_{-p}+ \cdots +c_{-1}(z-\alpha)^{p-1}
  + \sum_{k=0}^{\infty}c_k(z-\alpha)^{k+p}
\end{equation}
この右辺を$p-1$回微分すると$c_{-1}$が取り出せる。
式で表すと次のようになる。
\begin{equation}
 \lim_{z\rightarrow\alpha} \frac{d^{(p-1)}}{dz^{(p-1)}}(z-\alpha)^{p}f(z)
  = (p-1)!\cdot c_{-1}
\end{equation}

\hrulefill

積分値を求める手順は次の通り

\begin{enumerate}
 \item 閉曲線内に極が存在するかを調べる
 \item それぞれの極の留数を求める
\end{enumerate}

\begin{enumerate}\renewcommand{\theenumi}{(\arabic{enumi})}
 \item
      \begin{equation}
       I = \frac{1}{2\pi i}\int_{\lvert z \rvert = 2}\frac{e^z}{z^2-2z+5}dz
      \end{equation}

      $f(z)=\frac{e^z}{z^2-2z+5}$とする。
      $z^2-2z+5=0$を解くと$z=1\pm 2i$より、次のような式になる。
      \begin{equation}
       f(z)=\frac{e^z}{z^2-2z+5}=\frac{e^z}{(z-(1+2i))(z-(1-2i))}
      \end{equation}
      $z=1\pm 2i$はそれぞれ$1$位の極である。
      これらの絶対値は$\lvert 1+2i\lvert = \rvert 1-2i\lvert = \sqrt{5}>2$であり、
      閉曲線$\lvert z \rvert =2$内に存在しない。

      閉曲線内に極がない為、積分値は$0$になる。
      \begin{equation}
       I = \frac{1}{2\pi i}\int_{\lvert z \rvert = 2}\frac{e^z}{z^2-2z+5}dz =0
      \end{equation}
 \item
      \begin{equation}
       I = \frac{1}{2\pi i}\int_{\lvert z-i \rvert = 2}\frac{z^2}{z^2-2z+5}dz
      \end{equation}

      $f(z)=\frac{z^2}{z^2-2z+5}$とする。
      \begin{equation}
       f(z)=\frac{z^2}{z^2-2z+5}=\frac{z^2}{(z-(1+2i))(z-(1-2i))}
      \end{equation}

      $z=1\pm 2i$はそれぞれ$1$位の極である。
      これらが閉曲線$\lvert z-i\lvert=2$内にあるかどうかを調べる。
      \begin{align}
       z=1+2iの場合 \qquad \lvert (1+2i) -i\lvert &= \lvert 1+i \rvert =\sqrt{2} < 2\\
       z=1-2iの場合 \qquad \lvert (1-2i) -i\lvert &= \lvert 1-3i \rvert =\sqrt{10} > 2
      \end{align}

      $1+2i$が閉曲線内に含まれるので、この極の留数を求める。
      1位の極であるので、微分は不要。
      \begin{align}
       \mathrm{Res}(f,1+2i) =& \lim_{z\rightarrow 1+2i}(z-(1+2i))f(z)\\
           =& \lim_{z\rightarrow 1+2i}(z-(1+2i))\frac{z^2}{(z-(1+2i))(z-(1-2i))}\\
           =& \lim_{z\rightarrow 1+2i}\frac{z^2}{z-(1-2i)}\\
           =& \frac{(1+2i)^2}{1+2i-(1-2i)} \quad = 1+\frac{3}{4}i
      \end{align}

      極がこの一つだけなので、積分値は$1+\frac{3}{4}i$である。
      \begin{equation}
       I = \frac{1}{2\pi i}\int_{\lvert z-i \rvert = 2}\frac{z^2}{z^2-2z+5}dz =1+\frac{3}{4}i
      \end{equation}

\end{enumerate}





\end{document}
