\documentclass[12pt,b5paper]{ltjsarticle}

%\usepackage[margin=15truemm, top=5truemm, bottom=5truemm]{geometry}
%\usepackage[margin=10truemm,left=15truemm]{geometry}
\usepackage[margin=10truemm]{geometry}

\usepackage{amsmath,amssymb}
%\pagestyle{headings}
\pagestyle{empty}

%\usepackage{listings,url}
%\renewcommand{\theenumi}{(\arabic{enumi})}

%\usepackage{graphicx}

%\usepackage{tikz}
%\usetikzlibrary {arrows.meta}
%\usepackage{wrapfig}
%\usepackage{bm}

% ルビを振る
%\usepackage{luatexja-ruby}	% required for `\ruby'

%% 核Ker 像Im Hom を定義
%\newcommand{\Img}{\mathop{\mathrm{Im}}\nolimits}
%\newcommand{\Ker}{\mathop{\mathrm{Ker}}\nolimits}
%\newcommand{\Hom}{\mathop{\mathrm{Hom}}\nolimits}

%\DeclareMathOperator{\Rot}{rot}
%\DeclareMathOperator{\Div}{div}
%\DeclareMathOperator{\Grad}{grad}
%\DeclareMathOperator{\arcsinh}{arcsinh}
%\DeclareMathOperator{\arccosh}{arccosh}
%\DeclareMathOperator{\arctanh}{arctanh}



%\usepackage{listings,url}
%
%\lstset{
%%プログラム言語(複数の言語に対応,C,C++も可)
%  language = Python,
%%  language = Lisp,
%%  language = C,
%  %背景色と透過度
%  %backgroundcolor={\color[gray]{.90}},
%  %枠外に行った時の自動改行
%  breaklines = true,
%  %自動改行後のインデント量(デフォルトでは20[pt])
%  breakindent = 10pt,
%  %標準の書体
%%  basicstyle = \ttfamily\scriptsize,
%  basicstyle = \ttfamily,
%  %コメントの書体
%%  commentstyle = {\itshape \color[cmyk]{1,0.4,1,0}},
%  %関数名等の色の設定
%  classoffset = 0,
%  %キーワード(int, ifなど)の書体
%%  keywordstyle = {\bfseries \color[cmyk]{0,1,0,0}},
%  %表示する文字の書体
%  %stringstyle = {\ttfamily \color[rgb]{0,0,1}},
%  %枠 "t"は上に線を記載, "T"は上に二重線を記載
%  %他オプション:leftline,topline,bottomline,lines,single,shadowbox
%  frame = TBrl,
%  %frameまでの間隔(行番号とプログラムの間)
%  framesep = 5pt,
%  %行番号の位置
%  numbers = left,
%  %行番号の間隔
%  stepnumber = 1,
%  %行番号の書体
%%  numberstyle = \tiny,
%  %タブの大きさ
%  tabsize = 4,
%  %キャプションの場所("tb"ならば上下両方に記載)
%  captionpos = t
%}



\begin{document}


\hrulefill

\textbf{$\sigma$-加法族}

集合$X$の集合族$\Sigma$が
「$\sigma$-加法族である」
とは次を満たすときをいう。
\begin{enumerate}
 \item $X \in \Sigma$
 \item $A \in \Sigma \Rightarrow A^{c}\in\Sigma$
 \item $A_{i}\in\Sigma \ (i\in\mathbb{N})
       \Rightarrow \bigcup_{i=1}^{\infty}A_{i}\in\Sigma$
\end{enumerate}

\textbf{生成される$\sigma$-加法族}

$X$の部分集合族$\mathcal{A}$について、
$\mathcal{A}$を含む最小の$\sigma$-加法族を
$\sigma_{X}(\mathcal{A})$と表す。
\begin{equation}
 \sigma_{X}(\mathcal{A})=
  \bigcap_{\substack{\mathcal{M}:\sigma\text{-加法族} \\ \mathcal{A}\subset \mathcal{M}}}\mathcal{M}
\end{equation}

\textbf{ボレル$\sigma$-加法族}

$(X,\mathcal{O})$を位相空間とする。
$\sigma_{X}(\mathcal{O})$を
$X$上のボレル$\sigma$-加法族といい、
$\mathcal{B}(X)$と表す。

$\mathcal{B}(X)$の元のことをボレル集合という。

\hrulefill

\textbf{演習問題 3.4.}

$\mathbb{R}$には通常の位相を入れるものとし、
$f : \mathbb{R}\to\mathbb{R}$を連続関数とする。
また、
$\mathbb{R}$の部分集合$A$に対し、
$f(A) = \{ f(x) \mid x\in A \}$とする。
\begin{enumerate}
 \item
      $K$を$\mathbb{R}$のコンパクト集合とするとき、
      $f(K)$も$\mathbb{R}$のコンパクト集合になることを示せ。
      
 \item
      $f(\mathbb{R})$は$\mathbb{R}$のボレル集合であることを示せ。
\end{enumerate}

\dotfill

\begin{enumerate}
 \item

      $f(K)$の
      任意の開被覆$\{U_{\lambda}\}_{\lambda\in\Lambda}$を
      とってくる。

      関数$f$は連続であるので、
      $f^{-1}(U_{\lambda})$は開集合である。
      よって、$\{ f^{-1}(U_{\lambda}) \}$は$K$の開被覆である。

      $K$はコンパクトであるので、
      この開被覆は有限個$\{ f( U_{\lambda_{k}} ) \}$を選ぶことが出来る。
      \begin{equation}
       K = \bigcup_{k=1}^{n} f( U_{\lambda_{k}} )
      \end{equation}

      これにより$\{ U_{\lambda_{k}} \}$が$f(K)$の有限開被覆となり、
      $f(K)$がコンパクトであることになる。

 \item




\end{enumerate}



\hrulefill

\textbf{演習問題 4.5.}

$(X,\mathcal{M})$を可測空間とし、
$f,g : X \to \mathbb{R}$は
$\mathcal{M}$-可測であるとする。

このとき、
$\{ x\in X \mid f(x) < g(x) \} \in \mathcal{M}$
であることを示せ。

\dotfill


区間$I_{\alpha} \subset \overline{\mathbb{R}}$を次のように定義する。
\begin{equation}
 I_{\alpha} = \{ \alpha\in \mathbb{R} \mid -\infty \leq x \leq \alpha  \}
\end{equation}

$f,g$は$\mathcal{M}$-可測であるので、
任意の$\alpha\in\mathbb{R}$に対して
$f^{-1}(I_{\alpha}), g^{-1}(I_{\alpha})\in\mathcal{M}$
である。



\hrulefill

\textbf{演習問題 4.6.}

$(X,\mathcal{M})$を可測空間とし、
$\{f_{n}\}_{n=1}^{\infty}$を
$X$上の$\overline{\mathbb{R}}$-値関数の列とする。
${}^{\forall}n\in\mathbb{N}$に対し、
$f_{n}$が$\mathcal{M}$-可測であることを仮定する。

\begin{equation}
 E =
  \left\{
   x\in X \mid
   \lim_{n\to\infty} f_{n}
   が
   (\overline{\mathbb{R}}内に)存在する
  \right\}
\end{equation}
とおくとき、
$E\in\mathcal{M}$であることを示せ。


\dotfill






\hrulefill

\end{document}
