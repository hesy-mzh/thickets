\documentclass[12pt,b5paper]{ltjsarticle}

%\usepackage[margin=15truemm, top=5truemm, bottom=5truemm]{geometry}
%\usepackage[margin=10truemm,left=15truemm]{geometry}
\usepackage[margin=10truemm]{geometry}

\usepackage{amsmath,amssymb}
%\pagestyle{headings}
\pagestyle{empty}

%\usepackage{listings,url}
%\renewcommand{\theenumi}{(\arabic{enumi})}

%\usepackage{graphicx}

%\usepackage{tikz}
%\usetikzlibrary {arrows.meta}
%\usepackage{wrapfig}	% required for `\wrapfigure' (yatex added)
%\usepackage{bm}	% required for `\bm' (yatex added)

% ルビを振る
%\usepackage{luatexja-ruby}	% required for `\ruby'

%% 核Ker 像Im Hom を定義
%\newcommand{\Img}{\mathop{\mathrm{Im}}\nolimits}
%\newcommand{\Ker}{\mathop{\mathrm{Ker}}\nolimits}
%\newcommand{\Hom}{\mathop{\mathrm{Hom}}\nolimits}

%\DeclareMathOperator{\Rot}{rot}
%\DeclareMathOperator{\Div}{div}
%\DeclareMathOperator{\Grad}{grad}
%\DeclareMathOperator{\arcsinh}{arcsinh}
%\DeclareMathOperator{\arccosh}{arccosh}
%\DeclareMathOperator{\arctanh}{arctanh}



%\usepackage{listings,url}
%
%\lstset{
%%プログラム言語(複数の言語に対応,C,C++も可)
%  language = Python,
%%  language = Lisp,
%%  language = C,
%  %背景色と透過度
%  %backgroundcolor={\color[gray]{.90}},
%  %枠外に行った時の自動改行
%  breaklines = true,
%  %自動改行後のインデント量(デフォルトでは20[pt])
%  breakindent = 10pt,
%  %標準の書体
%%  basicstyle = \ttfamily\scriptsize,
%  basicstyle = \ttfamily,
%  %コメントの書体
%%  commentstyle = {\itshape \color[cmyk]{1,0.4,1,0}},
%  %関数名等の色の設定
%  classoffset = 0,
%  %キーワード(int, ifなど)の書体
%%  keywordstyle = {\bfseries \color[cmyk]{0,1,0,0}},
%  %表示する文字の書体
%  %stringstyle = {\ttfamily \color[rgb]{0,0,1}},
%  %枠 "t"は上に線を記載, "T"は上に二重線を記載
%  %他オプション:leftline,topline,bottomline,lines,single,shadowbox
%  frame = TBrl,
%  %frameまでの間隔(行番号とプログラムの間)
%  framesep = 5pt,
%  %行番号の位置
%  numbers = left,
%  %行番号の間隔
%  stepnumber = 1,
%  %行番号の書体
%%  numberstyle = \tiny,
%  %タブの大きさ
%  tabsize = 4,
%  %キャプションの場所("tb"ならば上下両方に記載)
%  captionpos = t
%}



\begin{document}

\hrulefill


$t$を独立変数とする関数$x=x(t)$について、
$x$は非斉次二回常微分方程式の初期値問題
\begin{equation}
 x^{\prime\prime} + 4x^{\prime} + 13x = 9e^{-2t}
  ,\quad
  x(0)=0
  ,\quad
  x^{\prime}(0)=3
  \label{diff_eq01}
\end{equation}
の解であるとする。
次の各問に答えよ。
\begin{enumerate}
 \item
      微分方程式の初期値問題(\ref{diff_eq01})を
      逆演算子を利用して解け。

      \dotfill

      微分演算子を
      $D=\frac{\mathrm{d}}{\mathrm{d}t}$
      として方程式を書き換える。
      \begin{align}
       x^{\prime\prime} + 4x^{\prime} + 13x = 9e^{-2t}\\
       \frac{\mathrm{d}}{\mathrm{d}t}\frac{\mathrm{d}}{\mathrm{d}t}x
       +\frac{\mathrm{d}}{\mathrm{d}t}4x
       +13x =& 9e^{-2t}\\
       DDx+4Dx+13x =& 9e^{-2t}\\
       (D^2+4D+13)x =& 9e^{-2t}
      \end{align}

      よって、
      特殊解$x_1$は
      \begin{equation}
       x_1=
        \frac{1}{D^2+4D+13}9e^{-2t}
        =
        \frac{9}{(-2)^2+4(-2)+13}e^{-2t}
        =e^{-2t}
      \end{equation}

      また、特性方程式は
      $D^2+4D+13=0$より
      変数を$\lambda$とすると
      $\lambda^2+4\lambda+13 = 0$
      である。
      \begin{equation}
       \lambda^2+4\lambda+13 = 0
        ,\quad
        \lambda = -2 \pm 3i
      \end{equation}
      これにより
      一般解は
      \begin{equation}
       x = C_1 e^{- 2t} \sin{3t} + C_2 e^{- 2t} \cos{3t}
      \end{equation}
      である。
      微分方程式の解は
      \begin{equation}
       x = C_1 e^{- 2t} \sin{3t} + C_2 e^{- 2t} \cos{3t} + e^{-2t}
        \label{sol_eq}
      \end{equation}
      と表せる。
      これに初期値$x(0)=0,x^{\prime}(0)=3$を利用し
      $C_1,C_2$を求める。
      \begin{align}
       x(0)
       =& C_1 e^{0} \sin{0} + C_2 e^{0} \cos{0} + e^{0}
       = C_2 + 1 = 0\\
       C_2 =& - 1
      \end{align}

      \begin{equation}
       x^{\prime}(t)
       =  (C_1 e^{- 2t} \sin{3t})^{\prime}
       + (C_2 e^{- 2t} \cos{3t})^{\prime}
       + (e^{-2t})^{\prime}
      \end{equation}
      より
      \begin{align}
       (C_1 e^{- 2t} \sin{3t})^{\prime}
       =& C_1( -2e^{-2t} \sin 3t +3 e^{-2t} \cos 3t )\\
       (C_2 e^{- 2t} \cos{3t})^{\prime}
       =& C_2( -2e^{-2t} \cos 3t - 3e^{-2t} \sin 3t)\\
       (e^{-2t})^{\prime}
       =& -2 e^{-2t}
      \end{align}
      なので、
      \begin{equation}
       x^{\prime}(0)
       =  3C_1 -2C_2 -2 =3
      \end{equation}
      $C_2=-1$より$C_1=1$である。

      定数$C_1,C_2$を当てはめると方程式の解は次のようになる。
      \begin{equation}
       x = e^{- 2t} \sin{3t} - e^{- 2t} \cos{3t} + e^{-2t}
      \end{equation}

      \hrulefill

 \item
      微分方程式の初期値問題(\ref{diff_eq01})を
      定数変化法により解け。

      \dotfill

      特性方程式は
      変数を$\lambda$とすると
      $\lambda^2+4\lambda+13 = 0$
      である。
      \begin{equation}
       \lambda^2+4\lambda+13 = 0
        ,\quad
        \lambda
        = - 2 \pm 3i
      \end{equation}
      ここから微分方程式の基本解が次のように求まる。
      \begin{equation}
       x_1 = e^{- 2t} \cos{3t}
        ,\quad
       x_2 = e^{- 2t} \sin{3t}
      \end{equation}
      このため、
      解を次のように置く。
      \begin{equation}
       x=c_1(t)x_1 + c_2(t)x_2
        \label{sol_1st}
      \end{equation}

      この二つの基本解$x_1,x_2$から
      ロンスキアン$W(x_1,x_2)$を計算する。
       \begin{equation}
        W(x_1,x_2)
        =
        \begin{vmatrix}
         x_1 & x_2\\
         x_1^{\prime} & x_2^{\prime}
        \end{vmatrix}
        = x_1x_2^{\prime} - x_1^{\prime}x_2
        = 3e^{-4t}
        \ne 0
       \end{equation}

      そこで、次の連立方程式を解く。
      \begin{equation}
       \begin{cases}
        c_1^{\prime}(t)x_1 + c_2^{\prime}(t)x_2 =0\\
        c_1^{\prime}(t)x_1^{\prime} + c_2(t)x_2^{\prime} =9e^{-2t}
       \end{cases}
       \Longleftrightarrow
       \begin{pmatrix}
         x_1 & x_2\\
         x_1^{\prime} & x_2^{\prime}
       \end{pmatrix}
       \begin{pmatrix}
        c_1^{\prime}(t) \\ c_2^{\prime}(t)
       \end{pmatrix}
       =
       \begin{pmatrix}
        0 \\ 9e^{-2t}
       \end{pmatrix}
      \end{equation}

      ロンスキアンは$W(x_1,x_2)\ne0$であるので、
      この連立方程式は解をもつ。
      \begin{equation}
       \begin{pmatrix}
        c_1^{\prime}(t) \\ c_2^{\prime}(t)
       \end{pmatrix}
       =
       \frac{1}{W(x_1,x_2)}
       \begin{pmatrix}
         x_2^{\prime} & -x_2\\
         -x_1^{\prime} & x_1
       \end{pmatrix}
       \begin{pmatrix}
        0 \\ 9e^{-2t}
       \end{pmatrix}
       =
       \frac{1}{W(x_1,x_2)}
       \begin{pmatrix}
         -9e^{-2t}x_2\\
          9e^{-2t}x_1
       \end{pmatrix}
      \end{equation}

      $c_1^{\prime}(t),c_2^{\prime}(t)$は次のように求められる。
      \begin{align}
       c_1^{\prime}(t)
       =& \frac{-9e^{-2t}x_2}{W(x_1,x_2)}
       = \frac{-9e^{-2t}e^{- 2t} \sin{3t}}{3e^{-4t}}
       = -3\sin{3t}\\
       c_2^{\prime}(t)
       =& \frac{9e^{-2t}x_1}{W(x_1,x_2)}
      = \frac{9e^{-2t}e^{- 2t} \cos{3t}}{3e^{-4t}}
       = 3\cos{3t}
      \end{align}

      積分をすることにより$c_1(t),c_2(t)$を求める。
      \begin{align}
       c_1(t) =& \int (-3\sin{3t})\mathrm{d}t = \cos{3t} + C_1\\
       c_2(t) =& \int 3\cos{3t}\mathrm{d}t = \sin{3t} + C_2
      \end{align}

      これを式(\ref{sol_1st})に代入する。
      \begin{align}
       x =& c_1(t)x_1 + c_2(t)x_2\\
        =& (\cos{3t} + C_1)e^{- 2t} \cos{3t}
        +(\sin{3t} + C_2)e^{- 2t} \sin{3t}\\
       =& e^{- 2t} 
       +C_1e^{- 2t} \cos{3t} + C_2e^{- 2t} \sin{3t}
      \end{align}

      この解は微分演算子を用いた結果
      (\ref{sol_eq})
      と同じである。
      以降初期値を同様に求めると、
      \begin{equation}
       x = e^{- 2t} \sin{3t} - e^{- 2t} \cos{3t} + e^{-2t}
      \end{equation}
      が求まる。


      \hrulefill

\end{enumerate}


\hrulefill

\begin{equation}
 x^{\prime\prime}- 2x^{\prime}-3x=27t^2
  \label{diff_eq03}
\end{equation}

$t$を独立変数とする関数$x=x(t)$についての
微分方程式(\ref{diff_eq03})について、
一般解が次で与えられることを、
定数変化法により確認せよ。
\begin{equation}
 x(t)=-9t^2+12t-14+C_1e^{3t}+C_2e^{-t}
  \qquad
  (C_1,C_2\text{:const})
\end{equation}

\dotfill


基本解を求めるために
特性方程式$\lambda^2-2\lambda-3=0$を解く。
\begin{equation}
 \lambda^2-2\lambda-3=0
  \Leftrightarrow
  \lambda = -1,3
\end{equation}
これにより基本解は
$x_1=e^{-t}, \ x_2=e^{3t}$
である。

そこで微分方程式の解を次のように置く。
\begin{equation}
 x=c_1(t)x_1+c_2(t)x_2
  \label{sol_2nd}
\end{equation}

ロンスキアンを計算すると0にならないことが確認できる。
\begin{equation}
 W(x_1,x_2)=
  \begin{vmatrix}
   x_1 & x_2\\
   x_1^{\prime} & x_2^{\prime}
  \end{vmatrix}
  =x_1x_2^{\prime} - x_1^{\prime}x_2
  =e^{-t} \cdot 3e^{3t} + e^{-t} \cdot e^{3t}
  =4e^{2t} \ne 0
\end{equation}

そこで、次の連立方程式を解いて
$c_1^{\prime}(t), c_2^{\prime}(t)$
を求める。
\begin{equation}
 \begin{pmatrix}
   x_1 & x_2\\
   x_1^{\prime} & x_2^{\prime}
 \end{pmatrix}
 \begin{pmatrix}
  c_1^{\prime}(t) \\ c_2^{\prime}(t)
 \end{pmatrix}
 =
 \begin{pmatrix}
  0 \\ 27t^2
 \end{pmatrix}
\end{equation}

\begin{equation}
 \begin{pmatrix}
  c_1^{\prime}(t) \\ c_2^{\prime}(t)
 \end{pmatrix}
 =
 \frac{1}{W(x_1,x_2)}
 \begin{pmatrix}
   x_2^{\prime} & -x_2\\
   -x_1^{\prime} & x_1
 \end{pmatrix}
 \begin{pmatrix}
  0 \\ 27t^2
 \end{pmatrix}
 =
 27t^2
 \begin{pmatrix}
  -\frac{e^{t}}{4} \\ \frac{e^{-3t}}{4}
 \end{pmatrix}
\end{equation}

$c_1^{\prime}(t) = -\frac{27t^2e^{t}}{4} , c_2^{\prime}(t)=\frac{27t^2e^{-3t}}{4}$
を積分し、$c_1(t),c_2(t)$を求める。
\begin{align}
 c_1(t) =& \int \left(-\frac{27t^2e^{t}}{4}\right)\mathrm{d}t
 = -\frac{27}{4}e^{t} (t^{2} - 2t + 2) + C_1\\
 c_2(t) =& \int \frac{27t^2e^{-3t}}{4}\mathrm{d}t
 =
 -\frac{1}{4}e^{-3t} (9t^{2} + 6t + 2) +C_2
\end{align}

これを式(\ref{sol_2nd})に当てはめる。
\begin{align}
 x =& c_1(t)x_1+c_2(t)x_2\\
 =& \left( -\frac{27}{4}e^{t} (t^{2} - 2t + 2) +C_1 \right) \cdot e^{-t}
 + \left( -\frac{1}{4}e^{-3t} (9t^{2} + 6t + 2) +C_2 \right)\cdot e^{3t}\\
 =& -9t^2+12t-14+C_1e^{-t}+C_2e^{3t}
\end{align}

定数$C_1,C_2$の添え字を振りなおせば
問題の式になることがわかる。


\hrulefill

\begin{equation}
 x^{\prime\prime}+x^{\prime}+x=7e^{2t}
  \label{diff_eq04}
\end{equation}

$t$を独立変数とする関数$x=x(t)$についての
微分方程式(\ref{diff_eq04})について、
一般解が次で与えられることを、
定数変化法により確認せよ。

\begin{equation}
 x(t)=e^{2t}
  +C_1e^{-\frac{1}{2}t}\cos{\left(\frac{\sqrt{3}}{2}t\right)}
  +C_2e^{-\frac{1}{2}t}\sin{\left(\frac{\sqrt{3}}{2}t\right)}
  \qquad
  (C_1,C_2\text{:const})
\end{equation}


\dotfill

特性方程式$\lambda^2+\lambda+1=0$を解くと
$\displaystyle \lambda = \frac{-1}{2}\pm\frac{\sqrt{3}}{2}i$
となる。
これにより基本解$x_1,x_2$は次のようになる。
\begin{equation}
 x_1= e^{-\frac{1}{2}t}\cos{\left(\frac{\sqrt{3}}{2}t\right)}
  ,\quad
 x_2= e^{-\frac{1}{2}t}\sin{\left(\frac{\sqrt{3}}{2}t\right)}
\end{equation}

ロンスキアン$W(x_1,x_2)$を求める。
\begin{equation}
 W(x_1,x_2)
  = \frac{\sqrt{3}}{2}  e^{-t}\ne 0
\end{equation}

$W(x_1,x_2)\ne 0$より
次の連立方程式は解をもつ。
\begin{equation}
 \begin{pmatrix}
   x_1 & x_2\\
   x_1^{\prime} & x_2^{\prime}
 \end{pmatrix}
 \begin{pmatrix}
  c_1^{\prime}(t) \\ c_2^{\prime}(t)
 \end{pmatrix}
 =
 \begin{pmatrix}
  0 \\ 7e^{2t}
 \end{pmatrix}
\end{equation}

逆行列をかけることで
$c_1^{\prime}(t), c_2^{\prime}(t)$
を求める。

\begin{equation}
 \begin{pmatrix}
  c_1^{\prime}(t) \\ c_2^{\prime}(t)
 \end{pmatrix}
 =
 \frac{1}{W(x_1,x_2)}
 \begin{pmatrix}
   x_2^{\prime} & -x_2\\
   -x_1^{\prime} & x_1
 \end{pmatrix}
 \begin{pmatrix}
  0 \\ 7e^{2t}
 \end{pmatrix}
 =
 \frac{7e^{2t}}{\frac{\sqrt{3}}{2}e^{-t} }
 \begin{pmatrix}
  -e^{-\frac{1}{2}t}\sin{\left(\frac{\sqrt{3}}{2}t\right)} \\
  e^{-\frac{1}{2}t}\cos{\left(\frac{\sqrt{3}}{2}t\right)}
 \end{pmatrix}
\end{equation}

\begin{equation}
  c_1^{\prime}(t)=
  -\frac{14}{3} \sqrt{3} e^{\frac{5}{2} t} \sin\left(\frac{\sqrt{3}}{2}t\right)
  ,\quad
  c_2^{\prime}(t)=
  \frac{14}{3} \sqrt{3} e^{\frac{5}{2}t} \cos\left(\frac{\sqrt{3}}{2}t\right)
\end{equation}

これらを積分し$c_1(t),c_2(t)$を求める。
\begin{align}
 c_1(t) =& \int \left(
-\frac{14}{3} \sqrt{3} e^{\frac{5}{2} t} \sin\left(\frac{\sqrt{3}}{2}t\right)
\right)\mathrm{d}t\\
 =&
 e^{\frac{5}{2}t} \cos\left(\frac{\sqrt{3}}{2}t\right)
 - \frac{5}{3}\sqrt{3}\, e^{\frac{5}{2}t} \sin\left(\frac{\sqrt{3}}{2}t\right)
 + C_1\\
 c_2(t) =& \int \frac{14}{3} \sqrt{3} e^{\frac{5}{2}t} \cos\left(\frac{\sqrt{3}}{2}t\right) \mathrm{d}t\\
 =&
 e^{\frac{5}{2}t} \sin\left(\frac{\sqrt{3}}{2}t\right)
+ \frac{5}{3}\sqrt{3}\, e^{\frac{5}{2}t} \cos\left(\frac{\sqrt{3}}{2}t\right)
 +C_2
\end{align}

これを用いて一般解$x(t)=c_1(t)x_1+c_2(t)x_2$を求める。

\begin{align}
 x(t)=& c_1(t)x_1+c_2(t)x_2\\
 =&
  \left(
 e^{\frac{5}{2}t} \cos\left(\frac{\sqrt{3}}{2}t\right)
 - \frac{5}{3}\sqrt{3}\, e^{\frac{5}{2}t} \sin\left(\frac{\sqrt{3}}{2}t\right)
 + C_1
      \right)
  \cdot
  e^{-\frac{1}{2}t}\cos{\left(\frac{\sqrt{3}}{2}t\right)}
\\
  & +
  \left(
 e^{\frac{5}{2}t} \sin\left(\frac{\sqrt{3}}{2}t\right)
+ \frac{5}{3}\sqrt{3}\, e^{\frac{5}{2}t} \cos\left(\frac{\sqrt{3}}{2}t\right)
 +C_2
\right)
\cdot
e^{-\frac{1}{2}t}\sin{\left(\frac{\sqrt{3}}{2}t\right)}\\
 =&
 e^{2t}
 +C_1 e^{-\frac{1}{2}t}\cos{\left(\frac{\sqrt{3}}{2}t\right)}
 +C_2 e^{-\frac{1}{2}t}\sin{\left(\frac{\sqrt{3}}{2}t\right)}
\end{align}

よって、問題の式が得られることがわかる。

\hrulefill

\end{document}
