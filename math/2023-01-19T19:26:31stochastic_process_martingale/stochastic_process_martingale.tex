\documentclass[12pt,b5paper]{ltjsarticle}

%\usepackage[margin=15truemm, top=5truemm, bottom=5truemm]{geometry}
%\usepackage[margin=10truemm,left=15truemm]{geometry}
\usepackage[margin=10truemm]{geometry}

\usepackage{amsmath,amssymb}
%\pagestyle{headings}
\pagestyle{empty}

%\usepackage{listings,url}
%\renewcommand{\theenumi}{(\arabic{enumi})}

\usepackage{graphicx}

%\usepackage{tikz}
%\usetikzlibrary {arrows.meta}
%\usepackage{wrapfig}
%\usepackage{bm}

% ルビを振る
\usepackage{luatexja-ruby}

%% 核Ker 像Im Hom を定義
%\newcommand{\Img}{\mathop{\mathrm{Im}}\nolimits}
%\newcommand{\Ker}{\mathop{\mathrm{Ker}}\nolimits}
%\newcommand{\Hom}{\mathop{\mathrm{Hom}}\nolimits}

%\DeclareMathOperator{\Rot}{rot}
%\DeclareMathOperator{\Div}{div}
%\DeclareMathOperator{\Grad}{grad}
%\DeclareMathOperator{\arcsinh}{arcsinh}
%\DeclareMathOperator{\arccosh}{arccosh}
%\DeclareMathOperator{\arctanh}{arctanh}


\usepackage{url}
%\usepackage{listings,url}
%
%\lstset{
%%プログラム言語(複数の言語に対応,C,C++も可)
%  language = Python,
%%  language = Lisp,
%%  language = C,
%  %背景色と透過度
%  %backgroundcolor={\color[gray]{.90}},
%  %枠外に行った時の自動改行
%  breaklines = true,
%  %自動改行後のインデント量(デフォルトでは20[pt])
%  breakindent = 10pt,
%  %標準の書体
%%  basicstyle = \ttfamily\scriptsize,
%  basicstyle = \ttfamily,
%  %コメントの書体
%%  commentstyle = {\itshape \color[cmyk]{1,0.4,1,0}},
%  %関数名等の色の設定
%  classoffset = 0,
%  %キーワード(int, ifなど)の書体
%%  keywordstyle = {\bfseries \color[cmyk]{0,1,0,0}},
%  %表示する文字の書体
%  %stringstyle = {\ttfamily \color[rgb]{0,0,1}},
%  %枠 "t"は上に線を記載, "T"は上に二重線を記載
%  %他オプション:leftline,topline,bottomline,lines,single,shadowbox
%  frame = TBrl,
%  %frameまでの間隔(行番号とプログラムの間)
%  framesep = 5pt,
%  %行番号の位置
%  numbers = left,
%  %行番号の間隔
%  stepnumber = 1,
%  %行番号の書体
%%  numberstyle = \tiny,
%  %タブの大きさ
%  tabsize = 4,
%  %キャプションの場所("tb"ならば上下両方に記載)
%  captionpos = t
%}



\begin{document}

\hrulefill

\textbf{確率空間}

$(\Omega,\mathcal{F},P)$が確率空間であるとは、
\begin{itemize}
 \item $\Omega(\ne \emptyset)$ を集合(全体集合や全事象)
 \item $\mathcal{F}(\subset 2^{\Omega})$を$\sigma$-加法族
       ($A\in\mathcal{F}$を事象)
 \item $P$を確率測度
\end{itemize}
の組のことをいう。


\textbf{条件付き確率}
\begin{equation}
 P(X\mid A) = \frac{P(A\cap X)}{P(A)}
\end{equation}

\textbf{条件付き期待値}

確率変数が離散の場合。連続であれば積分。
\begin{equation}
 E[X\mid A] = \frac{E[X,A]}{P(A)}
  = \sum_{x}x\frac{P(X \cap A)}{P(A)}
\end{equation}

\textbf{条件付き分散}
\begin{equation}
 V[X\mid A]
  = E[X^2 \mid A] - (E[X\mid A])^2
\end{equation}

\textbf{
  \ruby{martingale}{マルチンゲール}
}

\begin{align}
 & \{M_{n}\}\text{ が }(\mathcal{F}_{n})\text{-マルチンゲール (martingale)}\\
  \stackrel{\mathrm{def}}{\Leftrightarrow} \ & \
  M_{n}\in L^1 \land E[ M_{n+1} \mid \mathcal{F}_{n}] =M_{n} \ a.s.\ {}^{\forall}n \geq 1
\end{align}
特に、
$E[M_{n+1}-M_{n} \mid \mathcal{F}_{n}]\leq 0$のとき
優マルチンゲール、
$E[M_{n+1}-M_{n} \mid \mathcal{F}_{n}]\geq 0$のとき
劣マルチンゲールという。


\textbf{almost surely (a.s.)}

a.s.(almost surely)とは「ほぼ確実に」という意味で、
確率が1であることを意味する。
極僅かな部分を除いて確実であるということで、
「ほとんど至る所で(almost everywhere \ a.e.)」
と同じような意味である。

\textbf{右連続左極限関数 c\`adl\`ag RCLL}

実数の部分集合$I(\subset\mathbb{R})$上で定義された関数$f$において
次を満たすとき、右連続左極限関数という。
\begin{center}
 $\displaystyle {}^{\forall}p\in I$について
左極限$\displaystyle \lim_{x\to p-0}f(x)$が存在し、
右連続($\displaystyle f(p)=\lim_{x\to p+0}f(x)$)である
\end{center}
フランス語の
continue \`a droite, limite \`a gauche
を省略し c\`adl\`ag
や、
英語の
right continuous with left limits
を省略し RCLL
などともいう。



\hrulefill

\textbf{問題}
\begin{enumerate}
 \item
      $(X_{t})_{t\geq0}$が
      $(\mathcal{F}_{t})$-劣マルチンゲール、
      c\`adl\`ag
      とする。
      次を証明せよ。
      \begin{equation}
       \lambda P\left( \inf_{0\leq s \leq t} X_{s} > \lambda \right)
        \leq E\left[ X_{t} \vee 0 \right] - E\left[ X_0 \right]
        ,\ ( {}^{\forall}\lambda >0, {}^{\forall}t \geq 0 )
      \end{equation}

      \dotfill

      $E[X_0]$を次のように正の部分と負の部分の二つに分ける。
      \begin{equation}
       E[X_0] = E[X_0>0] + E[X_0\leq 0]
      \end{equation}

      一つ目の問いから次の不等式が成り立つ。
      \begin{equation}
       \lambda P\left( \sup_{0\leq s \leq t} X_{s} > \lambda \right)
        \leq E\left[ X_{t} \lor 0 \right]
        ,\ ( {}^{\forall}\lambda >0, {}^{\forall}t \geq 0 )
%        \leq E\left[ X_{t} \lor 0 \right] - E\left[ X_0 \leq 0\right]
      \end{equation}

      $t=0$とすると上記式は次のようになる。
      \begin{equation}
       \lambda P\left( X_{0} > \lambda \right)
        \leq E\left[ X_{0} \lor 0 \right]
      \end{equation}


      \begin{equation}
       \lambda P\left( \inf_{0\leq s \leq t} X_{s} > \lambda \right)
        \leq
       \lambda P\left( \sup_{0\leq s \leq t} X_{s} > \lambda \right)
      \end{equation}
      より
      \begin{gather}
       \lambda P\left( \inf_{0\leq s \leq t} X_{s} > \lambda \right)
        + E[X_0>0]
        \leq
       \lambda P\left( \sup_{0\leq s \leq t} X_{s} > \lambda \right)\\
       \lambda P\left( \sup_{0\leq s \leq t} X_{s} > \lambda \right)
       \leq
       E\left[ X_{t} \lor 0 \right] - E[X_0\leq 0]
      \end{gather}
      が成り立てば
      \begin{equation}
       \lambda P\left( \inf_{0\leq s \leq t} X_{s} > \lambda \right)
        + E[X_0>0]
        \leq
        E\left[ X_{t} \lor 0 \right] - E[X_0\leq 0]
      \end{equation}
      より
      \begin{equation}
       \lambda P\left( \inf_{0\leq s \leq t} X_{s} > \lambda \right)
        \leq E\left[ X_{t} \lor 0 \right] - E[X_0]
      \end{equation}
      が得られる。


%      \dotfill
%
%      $E[X_{t} \lor 0]$は確率変数$X_{t}$が正の時の期待値
%      であるので、
%      \begin{equation}
%       E[X_{t} \lor 0] \geq E[X_{t}]
%      \end{equation}
%
%      マルコフの不等式($\lambda P(X_s \geq \lambda) \leq E[X_s]$)より
%      \begin{equation}
%       \lambda P\left(\inf_{0\leq s \leq t} X_{s} \geq \lambda \right) \leq E \left[\inf_{0\leq s \leq t} X_{s} \right]
%      \end{equation}
%      であり、
%      劣マルチンゲール性($E[X_s]\geq X_{s-1}$)より
%      \begin{equation}
%        E[X_t]\geq E\left[\inf_{0\leq s \leq t} X_{s} \right]
%      \end{equation}
%
%      これらを合わせると
%      \begin{align}
%       \lambda P\left(\inf_{0\leq s \leq t} X_{s} \geq  \lambda \right)
%       \leq & E \left[\inf_{0\leq s \leq t} X_{s} \right]\\
%       \leq & E[X_t]
%       \leq  E[X_t \lor 0]
%      \end{align}
%
%
%
%      \dotfill
%
%      \begin{equation}
%       \lambda P\left(
%                 \max_{m\in\mathcal{F}_{n}}X_m>\lambda
%                 \right) \leq E[X_t \lor 0]
%      \end{equation}
%
%
%      \begin{equation}
%       \lambda P\left(
%                 \max_{m\in\mathcal{F}_{n}}X_m>\lambda
%                 \right)
%       \geq
%       \lambda P\left(
%                 \inf_{0\leq m \leq t}X_m>\lambda
%                 \right) +E[X_0]
%      \end{equation}
%
%
%      \dotfill
%      
%      \textbf{参考} pp.32 Doobの不等式
%      \url{http://www.eng.niigata-u.ac.jp/~nagahata/lecture/2022/master/daigakuin-2022-3.pdf}
%
%      $\tau = \inf_{0\leq s \leq t} \{ s \mid X_{s} > \lambda \}$
%      とする。
%
%      $(X_t)_{t>0}$は劣マルチンゲールであるので、
%      $0<s<t$に対して次の不等式が成り立つ。
%      \begin{equation}
%       E[X_0] \leq E[X_s] \leq E[X_t]
%      \end{equation}
%
%      ここで、$E[X_s]$を次のように分ける。
%      \begin{equation}
%       E[X_s] = E[X_s,\{s\leq \tau\}] + E[X_s,\{ s > \tau\}]
%      \end{equation}
%
%      $\tau$は$X_{s} > \lambda$を満たす添え字の下限であり、
%      $s > \tau$であれば$t > \tau$であればであるので
%      \begin{equation}
%       E[X_s,\{s\leq \tau\}] + E[X_s,\{ s > \tau\}]
%        \leq
%       \lambda P(X_s,\{s\leq \tau\}) + E[X_t,\{ s > \tau\}]
%      \end{equation}
%      となる。
%
%      $\{ s > \tau\}$の余事象は
%      $\{ s \leq \tau\}
%      = \displaystyle \{ \inf_{0\leq s \leq t} X_{s} > \lambda \}$
%      であるため、
%      \begin{equation}
%       E[X_{t},\{ s > \tau\}]=
%        E[X_{t}]-E\left[X_{t},
%                   \left\{
%                    \inf_{0\leq s \leq t} X_{s} > \lambda \right\}\right]
%      \end{equation}


      \hrulefill

 \item
      $(X_{t})_{t\geq 0} : (\mathcal{F}_{t}) \text{ - 劣マルチンゲール (sub-martingale)}$
      、
      $\sup_{t\geq 0}E\left[ X_{t} \vee 0 \right] < \infty$
      とする。
      このとき、次が成り立つことを証明せよ。
      \begin{equation}
       {}^{\exists}X_{\infty}\in L^{1} \ s.t.\ \lim_{t\to\infty}X_{t}=X_{\infty} \ a.s.
      \end{equation}

      \dotfill

      次を利用して証明する。
      \begin{equation}
       E[v((X_s)_{0\leq s\leq t};a,b)]
        \leq \frac{1}{b-a}E[(X_{t}-a)\lor 0],
        \quad ({}^{\forall}a<b , a,b\in\mathbb{R})
        \label{2nd}
      \end{equation}

      \dotfill

      条件より
      $\sup_{t\geq 0}E\left[ X_{t} \vee 0 \right] < \infty$
      であるため、
      ある実数$a\in\mathbb{R}$を一つとってきたとき
      \begin{equation}
       \sup_{t\geq 0}E\left[ (X_{t} -a) \vee 0 \right]
        = \sup_{t\geq 0}E\left[ X_{t} \vee 0 \right] -a
        < \infty
      \end{equation}
      である。

      式(\ref{2nd})の右辺の分母は$a<b$より$b-a>0$である。
      $t\to\infty$とすると右辺は
      $\sup_{t\geq 0}E\left[ (X_{t} -a) \vee 0 \right]$
      が動くがこれが発散しないため
      $E[v((X_s)_{0\leq s\leq t};a,b)]$は存在する。

      よって
      $X_{t}\to X_{\infty} \ (t\to\infty)$
      は存在することがわかる。
      

      \hrulefill

\end{enumerate}


\hrulefill

\end{document}
