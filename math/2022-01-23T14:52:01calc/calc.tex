\documentclass[12pt,b5paper]{ltjsarticle}

%\usepackage[margin=15truemm, top=5truemm, bottom=5truemm]{geometry}
\usepackage[margin=15truemm]{geometry}

\usepackage{amsmath,amssymb}
\pagestyle{empty}

\begin{document}

\begin{enumerate}\setlength{\itemsep}{10pt}
 \item 不定積分
       \begin{enumerate}\setlength{\itemsep}{10pt}
        \item $\displaystyle\int\cos(5x-7)dx$
              $\displaystyle = \frac{1}{5}\sin(5x-7) +C$
        \item $\displaystyle\int\frac{dx}{6+x^2}$

              $x=\sqrt{6}\tan t$と置くと積分する式は
              \begin{equation}
               \frac{1}{6+x^2} = \frac{1}{6(1+\tan^2 t)}
                = \frac{\cos^2 t}{6}
              \end{equation}
              である。
              また、$\frac{dx}{dt}=\frac{\sqrt{6}}{\cos^2 t}$より
              $dx=\frac{\sqrt{6}}{\cos^2 t}dt$であるので、
              \begin{align}
               \int\frac{dx}{6+x^2} =&
               \int\frac{\cos^2 t}{6}\cdot\frac{\sqrt{6}}{\cos^2 t}dt\\
               =& \int\frac{\sqrt{6}}{6}dt = \frac{1}{\sqrt{6}}t +C
              \end{align}
              $x=\sqrt{6}\tan t$から$t=\arctan\frac{x}{\sqrt{6}}$
              であるので
              \begin{equation}
               \int\frac{dx}{6+x^2}
                = \frac{1}{\sqrt{6}}\arctan\frac{x}{\sqrt{6}} +C
              \end{equation}
        \item $\displaystyle\int\frac{3\log x +2}{x}dx$

              $t=3\log x +2$と置くと$dt=\frac{3}{x}dx$であるので、
              \begin{equation}
               \int\frac{3\log x +2}{x}dx = \int \frac{t}{3}dt
                = \frac{t^2}{6}+C
                = \frac{1}{6}(3\log x + 2)^2+C
              \end{equation}
        \item $\displaystyle\int (x^2+2)e^{x}dx$

              部分積分を利用し、
              \begin{align}
               \int (x^2+2)e^{x}dx =& (x^2+2)e^{x} - \int 2xe^{x}dx\\
               =& (x^2+2)e^{x} - \left(2xe^{x} - \int 2e^{x}dx \right)\\
               =& (x^2+2)e^{x} - \left(2xe^{x} - 2e^{x} \right)+C\\
               =& e^{x}(x^2-2x+4)+C
              \end{align}
        \item $\displaystyle\int \frac{2-9x}{(x+4)(x^2+3)}dx$

              被積分関数を3つの分数に分ける。
              \begin{equation}
               \frac{2-9x}{(x+4)(x^2+3)}
                = \frac{2}{x+4} - \frac{2x}{x^2+3} - \frac{1}{x^2+3}
              \end{equation}
              それぞれ別々に不定積分を行う。
              \begin{align}
               \int \frac{2}{x+4}dx =& 2\log(x+4)+C\\
               \int \frac{2x}{x^2+3}dx =& \log(x^2+3)+C
              \end{align}
               $\int \frac{1}{x^2+3}dx$ は $x=\sqrt{3}\tan t$と置き
              積分を行う。
              \begin{align}
               \int \frac{1}{x^2+3}dx
                =& \int \frac{\cos^2 t}{3}\cdot\frac{\sqrt{3}}{\cos^2t} dt\\
               =& \frac{1}{\sqrt{3}}t +C
               = \frac{1}{\sqrt{3}} \arctan\frac{x}{\sqrt{3}} +C
              \end{align}
              これらを合わせると
              \begin{equation}
               \int\frac{2-9x}{(x+4)(x^2+3)}dx
                = \log\frac{(x+4)^2}{x^2+3}
                -\frac{1}{\sqrt{3}} \arctan\frac{x}{\sqrt{3}}
              \end{equation}
       \end{enumerate}
 \item $\displaystyle\int_{-\infty}^{-2} \frac{dx}{\sqrt[5]{x^6}}$

       $x=-\frac{1}{t}$と置くと \;
        $\begin{matrix} \hline
          x: & -\infty & \rightarrow & -2\\ \hline
         t:&0&\rightarrow&1/2\\ \hline
         \end{matrix}$ \; となるので、
       \begin{equation}
        \int_{-\infty}^{-2} \frac{dx}{\sqrt[5]{x^6}}
         = \int_{0}^{\frac{1}{2}} t^{\frac{6}{5}} \cdot t^{-2}dt
         = \left[ 5t^{\frac{1}{5}}\right]_{0}^{\frac{1}{2}}
         = \frac{5}{\sqrt[5]{2}}
       \end{equation}
 \item $\displaystyle\lim_{x\rightarrow 1}\left(\sin\frac{\pi}{2}x\right)^{\frac{1}{x-1}}$

       \begin{equation}
        t=x-1 の時,\quad
         \left(\sin\frac{\pi}{2}x\right)^{\frac{1}{x-1}}
         = \left(\sin\frac{\pi}{2}(t+1)\right)^{\frac{1}{t}}
         = \left(\cos\frac{\pi}{2}t\right)^{\frac{1}{t}}
       \end{equation}

       % \begin{equation}
       %  \theta=\frac{\pi}{2}x,\quad
       %   \left(\sin\frac{\pi}{2}x\right)^{\frac{1}{x-1}}
       %   = \left(\sin\theta\right)^{\frac{\pi}{2\theta-\pi}}
       %   = \left(\cos\frac{\pi}{2}t\right)^{\frac{1}{t}}
       % \end{equation}

       % \begin{equation}
       %  e=\lim_{x\rightarrow 0}\left(1+x\right)^{\frac{1}{x}}
       % \end{equation}

       % \begin{equation}
       %  t=\tan\frac{\theta}{2}, \quad
       %   \sin\theta=\frac{2t}{1+t^2}, \quad
       %   \cos\theta=\frac{1-t^2}{1+t^2}, \quad
       %   \tan\theta=\frac{2t}{1-t^2}
       % \end{equation}

       $-1\leq t\leq1$において
       \begin{equation}
        0\leq \cos\frac{\pi}{2}t \leq 1
       \end{equation}
       であるから、
       \begin{equation}
        \lim_{t\rightarrow 0}0^{\frac{1}{t}}
         \leq \lim_{t\rightarrow 0} \left(\cos\frac{\pi}{2}t\right)^{\frac{1}{t}}
         \leq \lim_{t\rightarrow 0}1^{\frac{1}{t}}
       \end{equation}
       $\lim_{t\rightarrow 0}0^{\frac{1}{t}}=0, \
       \lim_{t\rightarrow 0}1^{\frac{1}{t}}=1$
       より
        \begin{equation}
        0\leq
         \lim_{t\rightarrow 0} \left(\cos\frac{\pi}{2}t\right)^{\frac{1}{t}}
         \leq 1
        \end{equation}

 \item $\displaystyle z=\log(3x+4y)$ の第2次偏導関数を求めよ。

       1階偏導関数は次の2つ。
       \begin{equation}
        \frac{\partial z}{\partial x} = \frac{3}{3x+4y} , \quad
        \frac{\partial z}{\partial y} = \frac{4}{3x+4y}
       \end{equation}

       これより2階偏導関数は次の3つ。
       \begin{equation}
        \frac{\partial}{\partial x}\frac{\partial}{\partial x}z
         = -\frac{9}{(3x+4y)^2} , \quad
         \frac{\partial}{\partial x}\frac{\partial}{\partial y}z
         = -\frac{12}{(3x+4y)^2} , \quad
         \frac{\partial}{\partial y}\frac{\partial}{\partial y}z
         = -\frac{16}{(3x+4y)^2}
       \end{equation}
 \item $\displaystyle z=x^2y^3 , x=\sin uv , y=\cos(u+v)$の時、
       $\displaystyle \frac{\partial z}{\partial u} , \frac{\partial z}{\partial v}$を求めよ。

       媒介変数の偏微分は次の式を満たす。
       \begin{equation}
        \frac{\partial z}{\partial u}
         =\frac{\partial z}{\partial x}\frac{\partial x}{\partial u}
         + \frac{\partial z}{\partial y}\frac{\partial y}{\partial u}
         , \quad
        \frac{\partial z}{\partial v}
        =\frac{\partial z}{\partial x}\frac{\partial x}{\partial v}
        + \frac{\partial z}{\partial y}\frac{\partial y}{\partial v}
       \end{equation}

       そこで、それぞれを求めると
       \begin{align}
        \frac{\partial z}{\partial x} =& 2xy^3 &
         \frac{\partial z}{\partial y} =& 3x^2y^2 &
         \frac{\partial x}{\partial u} =& v\cos uv\\
         \frac{\partial x}{\partial v} =& u\cos uv &
         \frac{\partial y}{\partial u} =& -\sin(u+v) &
        \frac{\partial y}{\partial v} =& -\sin(u+v)
       \end{align}

       これらを用いると
       \begin{align}
        \frac{\partial z}{\partial u}
        =& 2xy^3\cdot v\cos uv - 3x^2y^2\cdot \sin(u+v)\\
        \frac{\partial z}{\partial v}
        =& 2xy^3\cdot u\cos uv - 3x^2y^2\cdot \sin(u+v)
       \end{align}

 \item
      \begin{equation}
       \iint_{D}(4xy-y^3)dxdy , \quad D: 0\leq x \leq 1 , 0\leq y\leq 2x
      \end{equation}

      \begin{align}
       \iint_{D}(4xy-y^3)dxdy
       =& \int_{0}^{1}\int_{0}^{2x}(4xy-y^2)dydx\\
       =& \int_{0}^{1}\left[2xy^2-\frac{1}{3}y^3\right]_{y=0}^{y=2x}dx\\
       =& \int_{0}^{1}\left(8x^3-\frac{8}{3}x^3\right)dx\\
       =& \frac{4}{3}\left[x^4\right]_{x=0}^{x=1}
       = \frac{4}{3}
      \end{align}
\end{enumerate}


\end{document}
