\documentclass[12pt,b5paper]{ltjsarticle}

%\usepackage[margin=15truemm, top=5truemm, bottom=5truemm]{geometry}
\usepackage[margin=15truemm]{geometry}

\usepackage{amsmath,amssymb}
%\pagestyle{headings}
\pagestyle{empty}

%\usepackage{listings,url}
%\renewcommand{\theenumi}{(\arabic{enumi})}

\usepackage{graphicx}

\usepackage{tikz}
\usetikzlibrary {arrows.meta}
\usepackage{wrapfig}	% required for `\wrapfigure' (yatex added)
\usepackage{bm}	% required for `\bm' (yatex added)

% ルビを振る
%\usepackage{luatexja-ruby}	% required for `\ruby'

%% 核Ker 像Im Hom を定義
\newcommand{\Img}{\mathop{\mathrm{Im}}\nolimits}
\newcommand{\Ker}{\mathop{\mathrm{Ker}}\nolimits}
%\newcommand{\Hom}{\mathop{\mathrm{Hom}}\nolimits}

\begin{document}

\hrulefill

\textbf{冪零行列}

正方行列$A$が冪零行列であるとは、
ある自然数$k$が存在し、$A^k$が零行列であるときをいう。

%\begin{enumerate}
% \item $A$が冪零行列である
% \item 
%\end{enumerate}

\hrulefill

\begin{enumerate}
 \item
      次の複素正方行列は冪零であるか判別せよ。

      \begin{equation}
       A_1=
        \begin{pmatrix}
         a & a \\ -a & -a
        \end{pmatrix}
        ,\quad
        A_2=
        \begin{pmatrix}
         0 & -1 & 1 \\ 0 & 0 & 0 \\ 2 & 0 & 0
        \end{pmatrix}
        ,\quad
        A_3 =
        \begin{pmatrix}
         0 & 1 & 0 & 0 \\ 0 & 0 & -1 & 0 \\ 0 & 0 & 0 & 1 \\ - 1 & 0 & 0 & 0
        \end{pmatrix}
      \end{equation}

      \begin{equation}
       A_4=
        \begin{pmatrix}
         2 & 2 & 2 & 2 & -4\\
         7 & 1 & 1 & 1 & -5\\
         1 & 7 & 1 & 1 & -5\\
         1 & 1 & 7 & 1 & -5\\
         1 & 1 & 1 & 7 & -5
        \end{pmatrix}
        ,\quad
        A_5 =
        \begin{pmatrix}
         0 & 1 & 1 & 1 & 0 & 1 \\
         0 & 1 & 0 & 1 & 0 & 0 \\
         -1 & 0 & -1 & 1 & 0 & 0 \\
         1 & 1 & 0 & 2 & 1 & 0 \\
         -2 & 1 & 0 & 1 & 0 & -1 \\
         1 & -1 & 5 & 1 & -1 & -1
        \end{pmatrix}
      \end{equation}

      \dotfill

      $A_1$は$a\ne 0$の時、2回の積で零行列となる為、冪零行列である。
      \begin{equation}
       A_1^2=
        \begin{pmatrix}
         a & a \\ -a & -a
        \end{pmatrix}
        \begin{pmatrix}
         a & a \\ -a & -a
        \end{pmatrix}
        = 0
      \end{equation}

      $A_2$は偶数回の積と奇数回の積で異なるが、
      零行列にはならない為、冪零行列ではない。
      \begin{equation}
       A_2^{2n}=
        \begin{pmatrix}
         2^n & 0 & 0 \\ 0 & 0 & 0 \\ 0 & -2^n & 2^n
        \end{pmatrix}
        ,\quad
       A_2^{2n+1}=
        \begin{pmatrix}
         0 & -2^n & 2^n \\ 0 & 0 & 0 \\ 2^{n+1} & 0 & 0
        \end{pmatrix}
      \end{equation}


      $A_3$の列ベクトルは1次独立である為、
      $A_3$は正則行列である。
      よって、$A_3$は冪零行列ではない。

      $A_4$は$A_4^5=0$であるので冪零行列である。

      $A_5$は$\mathrm{tr} A_5 =1$より冪零行列ではない。


      \hrulefill

 \item
      \begin{equation}
       f_A : \mathbb{C}^3 \to \mathbb{C}^3 ,\ x\mapsto Ax
        \qquad
        A=\begin{pmatrix}
           2&2&-2\\5&1&-3\\1&5&-3
          \end{pmatrix}
        \qquad
        v=\begin{pmatrix}1\\0\\0\end{pmatrix}
      \end{equation}

      \begin{enumerate}
       \item
            $f_A^3 =0$を示せ。
       \item
            $\Ker f_A,\, \Ker f_A^2,\, \Ker f_A^3$の次元を求めよ。
       \item
            $\Ker f_A^2 = \Img f_A$と$\Ker f_A = \Img f_A^2$を示せ。
       \item
            $A^2v,\,Av,\,v$は$\mathbb{C}^3$の基底となることを示せ。
       \item
            $f_A$の$A^2v,\,Av,\,v$に関する表現行列を求めよ。
      \end{enumerate}

      \dotfill

      \begin{enumerate}
       \item
            $f_A^3 : \mathbb{C}^3\to\mathbb{C}^3, \ x \mapsto A^3x$
            であるので、$A^3$を求める。
            \begin{equation}
             A^3 =
              \begin{pmatrix}
               12 & -4 & -4 \\ 12 & -4 & -4 \\ 24 & -8 & -8
              \end{pmatrix}
              \begin{pmatrix}
               2&2&-2\\5&1&-3\\1&5&-3
              \end{pmatrix}
              = 0
            \end{equation}
            これにより$f_A^3$は零写像である。

       \item
            $f_A^3=0$より$\dim \Img f_A^3 =0$であるので、
            $\dim \Ker f_A^3 =3$である。
            よって、$\dim \Ker f_A =1,\,\dim \Ker f_A^2 =2$である。

            実際に階数を求めると
            \begin{equation}
             \mathrm{rank} A = 2,\quad
              \mathrm{rank} A^2 = 1,\quad
              \mathrm{rank} A^3 = 0
            \end{equation}
            である。

       \item
            $f_A^3=0$とは、次のような図で左端の$\mathbb{C}^3$元が
            $f_A$に3回移されると$0$になることを意味する。
            \begin{equation}
             \mathbb{C}^3
              \stackrel{f_A}{\longrightarrow}
              \mathbb{C}^3
              \stackrel{f_A}{\longrightarrow}
              \mathbb{C}^3
              \stackrel{f_A}{\longrightarrow}
              \mathbb{C}^3
            \end{equation}

            $\Ker f_A^2$と$\Img f_A$は2番目の$\mathbb{C}^3$の部分集合である。

            $f_A^2 \circ f_A=0$より
            $\Img f_A \subset \Ker f_A^2$である。
            \begin{align}
             f_A^2(f_A(\mathbb{C}^3))&=0\\
             f_A(\mathbb{C}^3) &\subset \Ker f_A^2\\
             \Img f_A &\subset \Ker f_A^2\\
            \end{align}
            また、$\dim \Img f_A = 3 - \dim \Ker f_A =2$であるので、
            $\Img f_A = \Ker f_A^2$である事がわかる。

            同様に、
            $\Img f_A^2 \subset \Ker f_A$であるが、
            両方とも次元が1であるので、
            $\Img f_A^2 = \Ker f_A$である。

       \item
            \begin{equation}
             a_0 v + a_1 Av + a_2 A^2v =0
              ,\qquad
            a_i \in \mathbb{C}
            \label{expr}
            \end{equation}
            とする。

            左から$A^2$をかけると
            \begin{equation}
             a_0 A^2v + a_1 A^3v + a_2 A^4v =0
            \end{equation}
            であるが、$A^3=A^4=0$であるので、
            $a_0 A^2v =0$となる。
            $A^2v \ne0$より$a_0=0$となる。

            左から$A$をかけ、$a_0=0$を代入すると
            $a_1 A^2v=0$となり、$a_1=0$が得られる。

            $a_0 = a_1 = 0$より$a_2=0$となるので、
            式(\ref{expr})を満たすとき$a_i=0$である。
            これによりベクトル$v,\,Av,\,A^2v$は1次独立であることが分かる。

            $\mathbb{C}^3$は3次元であるので、
            $v,\,Av,\,A^2v$は基底となる。

       \item
            $\mathbb{C}^3$の基底$A^2v,\,Av,\,v$を$f_A$で移した先のベクトルを
            $A^2v,\,Av,\,v$の一次結合で表す。
            \begin{align}
             f_A(A^2v)=&A^3v=0=0A^2v +0Av +0v = (A^2v \ Av \ v)\begin{pmatrix}0\\0\\0\end{pmatrix}\\
             f_A(Av)=&A^2v=1A^2v +0Av +0v = (A^2v \ Av \ v)\begin{pmatrix}1\\0\\0\end{pmatrix}\\
             f_A(v)=&Av=0A^2v +1Av +0v = (A^2v \ Av \ v)\begin{pmatrix}0\\1\\0\end{pmatrix}
            \end{align}

            これらをまとめると次のようになる。
            \begin{equation}
             f_A(A^2v \ Av \ v)=
              (A^2v \ Av \ v)
              \begin{pmatrix}0&1&0\\0&0&1\\0&0&0\end{pmatrix}
            \end{equation}

            よって、表現行列は次の行列である。
            \begin{equation}
             \begin{pmatrix}0&1&0\\0&0&1\\0&0&0\end{pmatrix}
            \end{equation}
      \end{enumerate}

      \hrulefill

\end{enumerate}


\end{document}
