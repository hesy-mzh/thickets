\documentclass[12pt,b5paper]{ltjsarticle}

%\usepackage[margin=15truemm, top=5truemm, bottom=5truemm]{geometry}
%\usepackage[margin=10truemm,left=15truemm]{geometry}
\usepackage[margin=10truemm]{geometry}

\usepackage{amsmath,amssymb}
%\pagestyle{headings}
\pagestyle{empty}

%\usepackage{listings,url}
%\renewcommand{\theenumi}{(\arabic{enumi})}

\usepackage{graphicx}

\usepackage{tikz}
%\usetikzlibrary {arrows.meta}
\usetikzlibrary{intersections,calc,arrows.meta}

%\usepackage{wrapfig}
%\usepackage{bm}

% ルビを振る
%\usepackage{luatexja-ruby}	% required for `\ruby'

%% 核Ker 像Im Hom を定義
%\newcommand{\Img}{\mathop{\mathrm{Im}}\nolimits}
%\newcommand{\Ker}{\mathop{\mathrm{Ker}}\nolimits}
%\newcommand{\Hom}{\mathop{\mathrm{Hom}}\nolimits}

%\DeclareMathOperator{\Rot}{rot}
%\DeclareMathOperator{\Div}{div}
%\DeclareMathOperator{\Grad}{grad}
%\DeclareMathOperator{\arcsinh}{arcsinh}
%\DeclareMathOperator{\arccosh}{arccosh}
%\DeclareMathOperator{\arctanh}{arctanh}



%\usepackage{listings,url}
%
%\lstset{
%%プログラム言語(複数の言語に対応,C,C++も可)
%  language = Python,
%%  language = Lisp,
%%  language = C,
%  %背景色と透過度
%  %backgroundcolor={\color[gray]{.90}},
%  %枠外に行った時の自動改行
%  breaklines = true,
%  %自動改行後のインデント量(デフォルトでは20[pt])
%  breakindent = 10pt,
%  %標準の書体
%%  basicstyle = \ttfamily\scriptsize,
%  basicstyle = \ttfamily,
%  %コメントの書体
%%  commentstyle = {\itshape \color[cmyk]{1,0.4,1,0}},
%  %関数名等の色の設定
%  classoffset = 0,
%  %キーワード(int, ifなど)の書体
%%  keywordstyle = {\bfseries \color[cmyk]{0,1,0,0}},
%  %表示する文字の書体
%  %stringstyle = {\ttfamily \color[rgb]{0,0,1}},
%  %枠 "t"は上に線を記載, "T"は上に二重線を記載
%  %他オプション:leftline,topline,bottomline,lines,single,shadowbox
%  frame = TBrl,
%  %frameまでの間隔(行番号とプログラムの間)
%  framesep = 5pt,
%  %行番号の位置
%  numbers = left,
%  %行番号の間隔
%  stepnumber = 1,
%  %行番号の書体
%%  numberstyle = \tiny,
%  %タブの大きさ
%  tabsize = 4,
%  %キャプションの場所("tb"ならば上下両方に記載)
%  captionpos = t
%}



\begin{document}

\hrulefill

\begin{equation}
 D^{2} = \{z\in\mathbb{C} \mid \lvert z \rvert <1\}
\end{equation}

\begin{equation}
 \mathbb{H} = \{ \tau \in \mathbb{C} \mid \mathrm{Im}\tau > 0 \}
\end{equation}

\begin{equation}
 \mathbf{e}_{n}(z) = \exp{(2\pi i n z)}
\end{equation}

\hrulefill

\textbf{等角写像}

写像
$F:D^{2} \subset \mathbb{C} \to\mathbb{R}^{2}$
が次の式を満たすとき等角写像という。
\begin{equation}
 \lvert F_{x} \rvert
= \lvert F_{y} \rvert
,\quad
(F_{x},F_{y})=0
\end{equation}

\hrulefill

テータ関数
\begin{gather}
 \theta(z,\tau)
  = \sum_{n\in\mathbb{Z}} \exp{(\pi i n^{2} \tau + 2\pi inz)}
  = \sum_{n\in\mathbb{Z}} \exp{(\pi i n^{2} \tau)} \mathbf{e}_{n}(z)
 \\
 \theta(z+a\tau+b,\tau)
 = \exp{(-\pi ia^{2}\tau-2\pi iaz)\theta(z,\tau)}
 ,\quad a,b\in\mathbb{Z}
 \\
 \theta(z+1,\tau)=\theta(z,\tau)
 ,\quad
 \theta(z+\tau,\tau)=\exp{(-\pi i\tau-2\pi iz)\theta(z,\tau)}
\end{gather}

\hrulefill

$\mathbb{C}\times \mathbb{H}$上の正則関数全体の集合を$\mathcal{H}$とする。
$\mathcal{H}$は$\mathbb{C}$上の線形空間である。

$a,b\in\mathbb{R}$とする。
$\mathcal{H}$から$\mathcal{H}$への線形写像$S_{b},T_{a}$を次のように定義する。
\begin{equation}
 S_{b}f(z,\tau) = f(z+b,\tau)
  ,\quad
 T_{a}f(z,\tau)
 = \exp{(\pi ia^{2}\tau + 2\pi i az)} f(z+a\tau,\tau)
\end{equation}

\hrulefill

\textbf{指標付きテータ関数}

$a,b\in \frac{1}{2}\mathbb{Z}$に対して
$\theta_{a,b}$を次のように定める。
\begin{equation}
 \theta_{a,b}
  = S_{b}(T_{a}\theta)
  = \exp{(2\pi i ab)} T_{a}(S_{b}\theta)
\end{equation}

\begin{equation}
 \theta_{a,b}(z,\tau)
  = \sum_{n\in\mathbb{Z}}\exp{(
  \pi i (n+a)^{2}\tau + 2\pi i (n+a)(z+b)
  )}
\end{equation}

性質

 $a,b,a^{\prime},b^{\prime}\in\frac{1}{2}\mathbb{Z}$
\begin{gather}
 \theta_{0,0}(z,\tau)
 = \theta(z,\tau)
 \\
 S_{b^{\prime}}\theta_{a,b}(z,\tau)
 = \theta_{a,b} (z+b^{\prime},\tau)
 = \theta_{a,b+b^{\prime}} (z,\tau)
 \\
 T_{a^{\prime}} \theta_{a,b}(z,\tau)
 =\exp{(\pi i {a^{\prime}}^{2} \tau +2\pi ia^{\prime}z)}
 \theta_{a,b}(z+a^{\prime}\tau,\tau)
 = \exp{(-2\pi i a^{\prime}b)} \theta_{a+a^{\prime},b}(z,\tau)
 \\
 \theta_{a+p,b+q}(z,\tau) = \exp{(2\pi iaq)} \theta_{a,b}(z,\tau) \quad p,q \in\mathbb{Z}
 \\
 \theta_{a,b}(z+2,\tau) = \theta_{a,b}(z,\tau)\\
 \theta_{a,b}(z+2\tau,\tau) = \exp{(-4\pi i(\tau+z))}\theta_{a,b}(z,\tau)
\end{gather}

\hrulefill

\begin{description}
 \item[第1回]
 \begin{description}
  \item[問題 1.2.1.]
             $(x,y)$について
            複素数値関数$f$を
            $f=a+bi$と表す。
            ただし、$a,b$は実数値関数とする。
            $\bar{f}$が
            $z=x+yi$についての
            正則関数とすると
            $F$は等角写像となることを示せ。

\dotfill


             $\bar{f}=a-bi$であるので、
             写像
             $F:D^{2}\to\mathbb{R}^{2}$
             は
             $F=\begin{pmatrix}a\\-b \end{pmatrix}$
             となる。

             Cauchy-Riemann 方程式から
             $a_{x}=-b_{y},\; a_{y}=b_{x}$
             である。

             \begin{gather}
              \lvert F_{x} \rvert
              = \lvert a_{x}^{2} + b_{x}^{2} \rvert
              = \lvert a_{y}^{2} + b_{y}^{2} \rvert
              = \lvert F_{y} \rvert
              \\
              (F_{x},F_{y})
              = a_{x}a_{y}+(-b_{x})(-b_{y})
              = a_{x}a_{y}+(-a_{y})(a_{x})
              = 0
             \end{gather}

             これにより
             $F$は等角写像である。

\hrulefill


 \item[問題 1.2.2.]
            $F_{x}$が$F_{y}$を$\pi/2$回転したベクトルとなるとき、
            $\bar{f}=a-bi$は
            $z=x+yi$についての
            正則関数となることを示せ。

\dotfill


\begin{equation}
 F=\begin{pmatrix} a \\ b \end{pmatrix}
 ,\quad
 F_{x}=\begin{pmatrix} a_{x} \\ b_{x} \end{pmatrix}
 ,\quad
 F_{y}=\begin{pmatrix} a_{y} \\ b_{y} \end{pmatrix}
\end{equation}

            $F_{x}$が$F_{y}$を$\pi/2$回転したベクトルとなる為、
            次の式が成り立つ。
            \begin{gather}
             F_{x}=
              \begin{pmatrix}
              \cos{\frac{\pi}{2}} & -\sin{\frac{\pi}{2}} \\
              \sin{\frac{\pi}{2}} & \cos{\frac{\pi}{2}}
              \end{pmatrix}
              F_{y}
             \\
             a_{x}=-b_{y}
             ,\quad
             b_{x}=a_{y}
            \end{gather}

%            直交することから$(F_{x},F_{y})=0$である。

            $\bar{f}=a-bi$は
            Cauchy-Riemann 方程式
            を満たすので
            $z=x+yi$について正則関数である。

\hrulefill

\end{description}


 \item[第3回]
 \begin{description}
  \item[問題 1.1.1.]

             関数$f$を次の様に定める。
             \begin{equation}
              f(z,\tau)
               = \theta(-z +\frac{1}{2}\tau - \frac{b}{2\pi i},\tau)
             \end{equation}
             
             この時、次の式を満たすことを示せ。
             \begin{equation}
              f(z+1,\tau) = f(z,\tau)
               ,\quad
               f(z+\tau,\tau)
               = \exp{(-2\pi iz-b)} f(z,\tau)
             \end{equation}

\dotfill

             $\theta(z+1,\tau) = \theta(z,\tau)$より
             \begin{equation}
              f(z+1,\tau)
               =\theta(-(z+1) +\frac{1}{2}\tau - \frac{b}{2\pi i},\tau)
               =\theta(-z +\frac{1}{2}\tau - \frac{b}{2\pi i},\tau)
               = f(z,\tau)
             \end{equation}

             $\theta(z+\tau,\tau)=\exp{(-\pi i\tau-2\pi iz)}\theta(z,\tau)$
             であるので、
             両辺に
             $\exp{(\pi i\tau+2\pi iz)}$
             をかけることで
             $\exp{(\pi i\tau+2\pi iz)} \theta(z+\tau,\tau)=\theta(z,\tau)$
             である。
             \begin{align}
              f(z+\tau,\tau)
               &= \theta(-(z+\tau) +\frac{1}{2}\tau - \frac{b}{2\pi i},\tau)\\
               &= \theta(-z +\frac{1}{2}\tau - \frac{b}{2\pi i} -\tau,\tau)\\
               &= \exp{(\pi i\tau+2\pi i(-z +\frac{1}{2}\tau - \frac{b}{2\pi i} -\tau))} \theta(-z +\frac{1}{2}\tau - \frac{b}{2\pi i},\tau)\\
              &= \exp{(-2\pi iz   -b)} f(z,\tau)
             \end{align}

\hrulefill

  \item[問題 1.1.1.]
             整数$a,b$に対し次が成り立つことを確認せよ。
             \begin{equation}
              T_{a}\theta(z,\tau)
               =\theta(z,\tau)
               ,\quad
               S_{b}\theta(z,\tau)
               =\theta(z,\tau)
             \end{equation}

\dotfill

             $b\in\mathbb{Z}$より
             $\theta(z+\tau,\tau)=\exp{(-\pi i\tau-2\pi iz)\theta(z,\tau)}$
             を繰り返し行うと次の式を得る。
             \begin{align}
              T_{a}\theta(z,\tau)
              &= \exp{(\pi i a^{2}\tau + 2\pi i a z)} \theta(z+a\tau,\tau)\\
              &= \exp{(\pi i a^{2}\tau + 2\pi i a z)} \exp{(-\pi i\tau-2\pi i(z+(a-1)\tau))}\theta(z+(a-1)\tau,\tau)\\
              &= \exp{(\pi i a^{2}\tau + 2\pi i a z)}
              \exp{(-a\pi i\tau-2a\pi iz- 2\pi i\tau \sum_{j=1}^{a}(a-j)))}\theta(z,\tau)\\
              &= \exp{(\pi i a^{2}\tau
              -a\pi i\tau - 2\pi i\tau a^{2} +2\pi i \tau \frac{1}{2}a(a+1)))}\theta(z,\tau)\\
              &= \exp{(0)}\theta(z,\tau) = \theta(z,\tau)
             \end{align}


             $a\in\mathbb{Z}$より
             $\theta(z+1,\tau)=\theta(z,\tau)$
             を繰り返し行うと次が得られる。
             \begin{align}
              S_{b}\theta(z,\tau)
              &= \theta(z+b,\tau)
              = \theta(z+(b-1),\tau)
              = \theta(z+(b-2),\tau)
              = \dots = \theta(z,\tau)
             \end{align}

\hrulefill

  \item[問題 1.2.1.]
             下記補題
             を用いて
             $R_{(0,1/2)}$と
             $R_{(1/2,0)}$を求めよ。
             さらに
             $R_{(0,1/2)}\cdot R_{(1/2,0)}=iR_{(1/2,0)}\cdot R_{(0,1/2)}$
             を直接計算により確認せよ。

             \textbf{補題}

             $\delta \in \Delta = \{0,1/2,1,3/2\}$
             \begin{gather}
              S_{\frac{1}{2}} \xi_{\delta} = \exp{(\pi i \delta)} \xi_{\delta}\\
              T_{\frac{1}{2}} \xi_{\delta} = \xi_{\delta + \frac{1}{2}}
             \end{gather}

\dotfill

\begin{equation}
 \xi_{\delta} = \sum_{m\in\mathbb{Z}} \exp{(\pi i(\delta+2m)\tau + 2\pi i(\delta+2m)z)}
\end{equation}



\hrulefill

  \item[問題 1.2.2.]
             $H_{2}$は
             $(1,0,0)$を単位元とする群であることを確かめよ。

\dotfill

ハイゼンベルク群$H_{2}$の定義は次の通り。
\begin{equation}
 H_{2}
  = \{ (\lambda,a,b) \mid \lambda\in\mu_{4},\;
  a,b\in\frac{1}{2}\mathbb{Z}/\mathbb{Z}\}
\end{equation}

$\frac{1}{2}\mathbb{Z}/\mathbb{Z}=\{0,1/2\}$は群である。

             ${}^{\forall}(\lambda,a,b),(\lambda^{\prime},a^{\prime},b^{\prime}) \in H_{2}$に対して
             \begin{equation}
              (\lambda,a,b)\cdot(\lambda^{\prime},a^{\prime},b^{\prime})
               =(\lambda\lambda^{\prime} \exp{(2\pi iba^{\prime})},a+a^{\prime},b+b^{\prime})
             \end{equation}




\hrulefill

 \end{description}

 \item[第4回]
 \begin{description}
  \item[問題 1.1.1.]
             次の事実を確認せよ。
             \begin{enumerate}
              \item
                   $\theta_{1/2,0}(z,\tau)$は
                   領域$\Omega(\tau)$に次の零点を持つ。
                   \begin{equation}
                    z=\frac{1}{2},\quad
                     \frac{3}{2},\quad
                     \frac{1}{2}+\tau,\quad
                     \frac{3}{2}+\tau
                   \end{equation}

              \item
                   $\theta_{0,1/2}(z,\tau)$は
                   領域$\Omega(\tau)$に次の零点を持つ。
                   \begin{equation}
                    z=\frac{1}{2}\tau,\quad
                     1+\frac{1}{2}\tau,\quad
                     \frac{3}{2}\tau,\quad
                     1+\frac{3}{2}\tau
                   \end{equation}

              \item
                   $\theta(z,\tau)=\theta_{0,0}(z,\tau)$は
                   領域$\Omega(\tau)$に次の零点を持つ。
                   \begin{equation}
                    z=\frac{1}{2}+\frac{1}{2}\tau,\quad
                     \frac{3}{2}+\frac{1}{2}\tau,\quad
                     \frac{1}{2}+\frac{3}{2}\tau,\quad
                     \frac{3}{2}+\frac{3}{2}\tau
                   \end{equation}
             \end{enumerate}

\dotfill



\hrulefill

\end{description}

\end{description}

\hrulefill

\end{document}
