\documentclass[12pt,b5paper]{ltjsarticle}

%\usepackage[margin=15truemm, top=5truemm, bottom=5truemm]{geometry}
%\usepackage[margin=10truemm,left=15truemm]{geometry}
\usepackage[margin=5truemm]{geometry}

\usepackage{amsmath,amssymb}
%\pagestyle{headings}
\pagestyle{empty}

%\usepackage{listings,url}
\renewcommand{\theenumi}{(\arabic{enumi})}

%\usepackage{graphicx}

%\usepackage{tikz}
%\usetikzlibrary {arrows.meta}
%\usepackage{wrapfig}
%\usepackage{bm}

% ルビを振る
%\usepackage{luatexja-ruby}	% required for `\ruby'

%% 核Ker 像Im Hom を定義
%\newcommand{\Img}{\mathop{\mathrm{Im}}\nolimits}
%\newcommand{\Ker}{\mathop{\mathrm{Ker}}\nolimits}
%\newcommand{\Hom}{\mathop{\mathrm{Hom}}\nolimits}

%\DeclareMathOperator{\Rot}{rot}
%\DeclareMathOperator{\Div}{div}
%\DeclareMathOperator{\Grad}{grad}
%\DeclareMathOperator{\arcsinh}{arcsinh}
%\DeclareMathOperator{\arccosh}{arccosh}
%\DeclareMathOperator{\arctanh}{arctanh}

%\usepackage{url}

%\usepackage{listings}
%
%\lstset{
%%プログラム言語(複数の言語に対応,C,C++も可)
%  language = Python,
%%  language = Lisp,
%%  language = C,
%  %背景色と透過度
%  %backgroundcolor={\color[gray]{.90}},
%  %枠外に行った時の自動改行
%  breaklines = true,
%  %自動改行後のインデント量(デフォルトでは20[pt])
%  breakindent = 10pt,
%  %標準の書体
%%  basicstyle = \ttfamily\scriptsize,
%  basicstyle = \ttfamily,
%  %コメントの書体
%%  commentstyle = {\itshape \color[cmyk]{1,0.4,1,0}},
%  %関数名等の色の設定
%  classoffset = 0,
%  %キーワード(int, ifなど)の書体
%%  keywordstyle = {\bfseries \color[cmyk]{0,1,0,0}},
%  %表示する文字の書体
%  %stringstyle = {\ttfamily \color[rgb]{0,0,1}},
%  %枠 "t"は上に線を記載, "T"は上に二重線を記載
%  %他オプション:leftline,topline,bottomline,lines,single,shadowbox
%  frame = TBrl,
%  %frameまでの間隔(行番号とプログラムの間)
%  framesep = 5pt,
%  %行番号の位置
%  numbers = left,
%  %行番号の間隔
%  stepnumber = 1,
%  %行番号の書体
%%  numberstyle = \tiny,
%  %タブの大きさ
%  tabsize = 4,
%  %キャプションの場所("tb"ならば上下両方に記載)
%  captionpos = t
%}

%\usepackage{cancel}
%\usepackage{bussproofs}
%\usepackage{proof}

\begin{document}

%\hrulefill
\begin{equation}
 \Gamma =
  \sqrt{
    \frac{
      \left(\frac{p}{p_{0}}\right)^{\frac{2}{\gamma}}
      -\left(\frac{p}{p_{0}}\right)^{\frac{\gamma+1}{\gamma}}
      }{
        \frac{\gamma-1}{\gamma+1}\left(\frac{2}{\gamma+1}\right)^{\frac{2}{\gamma-1}}
      }
  }
  ,\qquad
  \frac{p_{0}}{p}= \left(1+\frac{\gamma-1}{2}M^{2} \right)^{\frac{\gamma}{\gamma-1}}
\end{equation}
上記$\Gamma$に$\frac{p_{0}}{p}$を代入すると次の式が得られる。
\begin{equation}
 \Gamma=M\left[ \frac{2+(\gamma-1)M^{2}}{\gamma+1}\right]^{-\frac{\gamma+1}{2(\gamma-1)}}
\end{equation}
\hrulefill


根号の中の分子を計算する。
\begin{align}
  \left(\frac{p}{p_{0}}\right)^{\frac{2}{\gamma}}
 -\left(\frac{p}{p_{0}}\right)^{\frac{\gamma+1}{\gamma}}
 =&
 \left(\frac{p_{0}}{p}\right)^{-\frac{2}{\gamma}}
 -\left(\frac{p_{0}}{p}\right)^{-\frac{\gamma+1}{\gamma}}
 =
 \left(
 \left(\frac{p_{0}}{p}\right)^{\frac{\gamma-1}{\gamma}}
 -1\right)\left(\frac{p_{0}}{p}\right)^{-\frac{\gamma+1}{\gamma}}\\
 =&
 \left(
 \left(1+\frac{\gamma-1}{2}M^{2} \right)
 -1
 \right)
 \left(1+\frac{\gamma-1}{2}M^{2} \right)^{-\frac{\gamma+1}{\gamma-1}}\\
 =&
 \left(
 \frac{\gamma-1}{2}M^{2}
 \right)
 \cdot 2^{\frac{\gamma+1}{\gamma-1}} \left(2+(\gamma-1)M^{2} \right)^{-\frac{\gamma+1}{\gamma-1}}\\
 =&
 2^{\frac{2}{\gamma-1}}
  (\gamma-1)M^{2}\left(2+(\gamma-1)M^{2} \right)^{-\frac{\gamma+1}{\gamma-1}}
\end{align}

分母の変形を行う。
\begin{align}
 \frac{\gamma-1}{\gamma+1}\left(\frac{2}{\gamma+1}\right)^{\frac{2}{\gamma-1}}
 &=
 (\gamma-1)\frac{1}{\gamma+1}
 \cdot 2^{\frac{2}{\gamma-1}}\left(\frac{1}{\gamma+1}\right)^{\frac{2}{\gamma-1}}\\
 &=2^{\frac{2}{\gamma-1}}(\gamma-1)
 \left(\frac{1}{\gamma+1}\right)^{\frac{\gamma+1}{\gamma-1}}
 =2^{\frac{2}{\gamma-1}}(\gamma-1)(\gamma+1)^{-\frac{\gamma+1}{\gamma-1}}
\end{align}

これらを$\Gamma$に当てはめる。
\begin{align}
 \Gamma
 &=
   \sqrt{
    \frac{
      \left(\frac{p}{p_{0}}\right)^{\frac{2}{\gamma}}
      -\left(\frac{p}{p_{0}}\right)^{\frac{\gamma+1}{\gamma}}
      }{
        \frac{\gamma-1}{\gamma+1}\left(\frac{2}{\gamma+1}\right)^{\frac{2}{\gamma-1}}
      }
  }
=\sqrt{\frac{
 2^{\frac{2}{\gamma-1}}
  (\gamma-1)M^{2}\left(2+(\gamma-1)M^{2} \right)^{-\frac{\gamma+1}{\gamma-1}}
 }{
  2^{\frac{2}{\gamma-1}}(\gamma-1)(\gamma+1)^{-\frac{\gamma+1}{\gamma-1}}
 }}\\
 &=
 \sqrt{\frac{
   M^{2}\left(2+(\gamma-1)M^{2} \right)^{-\frac{\gamma+1}{\gamma-1}}
 }{
  (\gamma+1)^{-\frac{\gamma+1}{\gamma-1}}
 }}
 =
 \lvert M \rvert
 \sqrt{\left(\frac{
   2+(\gamma-1)M^{2} }{ \gamma+1 }\right)^{-\frac{\gamma+1}{\gamma-1}}} \\
 &=
  \lvert M \rvert
 \left(\frac{2+(\gamma-1)M^{2} }{ \gamma+1 }\right)^{-\frac{\gamma+1}{2(\gamma-1)}}
\end{align}

$M \geq 0$であれば、
$\Gamma = M \left(\frac{2+(\gamma-1)M^{2} }{ \gamma+1 }\right)^{-\frac{\gamma+1}{2(\gamma-1)}}$
である。



\end{document}
