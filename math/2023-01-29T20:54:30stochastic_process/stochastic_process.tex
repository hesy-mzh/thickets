\documentclass[12pt,b5paper]{ltjsarticle}

%\usepackage[margin=15truemm, top=5truemm, bottom=5truemm]{geometry}
%\usepackage[margin=10truemm,left=15truemm]{geometry}
\usepackage[margin=10truemm]{geometry}

\usepackage{amsmath,amssymb}
%\pagestyle{headings}
\pagestyle{empty}

%\usepackage{listings,url}
%\renewcommand{\theenumi}{(\arabic{enumi})}

\usepackage{graphicx}

%\usepackage{tikz}
%\usetikzlibrary {arrows.meta}
%\usepackage{wrapfig}	% required for `\wrapfigure' (yatex added)
%\usepackage{bm}	% required for `\bm' (yatex added)

% ルビを振る
%\usepackage{luatexja-ruby}	% required for `\ruby'

%% 核Ker 像Im Hom を定義
%\newcommand{\Img}{\mathop{\mathrm{Im}}\nolimits}
%\newcommand{\Ker}{\mathop{\mathrm{Ker}}\nolimits}
%\newcommand{\Hom}{\mathop{\mathrm{Hom}}\nolimits}

%\DeclareMathOperator{\Rot}{rot}
%\DeclareMathOperator{\Div}{div}
%\DeclareMathOperator{\Grad}{grad}
%\DeclareMathOperator{\arcsinh}{arcsinh}
%\DeclareMathOperator{\arccosh}{arccosh}
%\DeclareMathOperator{\arctanh}{arctanh}



%\usepackage{listings,url}
%
%\lstset{
%%プログラム言語(複数の言語に対応,C,C++も可)
%  language = Python,
%%  language = Lisp,
%%  language = C,
%  %背景色と透過度
%  %backgroundcolor={\color[gray]{.90}},
%  %枠外に行った時の自動改行
%  breaklines = true,
%  %自動改行後のインデント量(デフォルトでは20[pt])
%  breakindent = 10pt,
%  %標準の書体
%%  basicstyle = \ttfamily\scriptsize,
%  basicstyle = \ttfamily,
%  %コメントの書体
%%  commentstyle = {\itshape \color[cmyk]{1,0.4,1,0}},
%  %関数名等の色の設定
%  classoffset = 0,
%  %キーワード(int, ifなど)の書体
%%  keywordstyle = {\bfseries \color[cmyk]{0,1,0,0}},
%  %表示する文字の書体
%  %stringstyle = {\ttfamily \color[rgb]{0,0,1}},
%  %枠 "t"は上に線を記載, "T"は上に二重線を記載
%  %他オプション:leftline,topline,bottomline,lines,single,shadowbox
%  frame = TBrl,
%  %frameまでの間隔(行番号とプログラムの間)
%  framesep = 5pt,
%  %行番号の位置
%  numbers = left,
%  %行番号の間隔
%  stepnumber = 1,
%  %行番号の書体
%%  numberstyle = \tiny,
%  %タブの大きさ
%  tabsize = 4,
%  %キャプションの場所("tb"ならば上下両方に記載)
%  captionpos = t
%}



\begin{document}

\hrulefill

\begin{enumerate}
 \item
      $\sigma$を$0$でない定数とし、
      実数値弱定常過程
      $\varepsilon = \{ \varepsilon_{t} ; t\in\mathbb{Z}\}$
      はホワイトノイズ$WN(0,\sigma^2)$であるとする。
      この時、
      全ての$t,h \in\mathbb{Z}$に対して次が成り立つ。
      \begin{equation}
       E [ \varepsilon_{t+h}\varepsilon_{t} ]=
        \sigma^2\frac{\sin{\pi h}}{\pi h}
      \end{equation}

      \dotfill

      共分散は次の式より得られる。
      \begin{equation}
       Cov( \varepsilon_{t+h}, \varepsilon_{t} )
        = E[ \varepsilon_{t+h} \varepsilon_{t} ]
        - E[ \varepsilon_{t+h}] E[ \varepsilon_{t} ]
      \end{equation}

      $\varepsilon_{i}$はホワイトノイズであるので期待値は0、
      分散は$\sigma^2$である。
      よって、
      $Cov( \varepsilon_{t+h}, \varepsilon_{t} )
      =E[ \varepsilon_{t+h} \varepsilon_{t} ]$
      である。

      任意の$t\in\mathbb{Z}$に対し、
      $h\ne 0$である$h\in\mathbb{Z}$について
      $Cov( \varepsilon_{t+h}, \varepsilon_{t} ) =E[ \varepsilon_{t+h} \varepsilon_{t} ]=0$となる。
      また、$V[\epsilon_{t}]=\sigma^2$となるので、
      $E[ \varepsilon_{t} \varepsilon_{t} ]=\sigma^2$である。

      \begin{gather}
       E[ \varepsilon_{t} \varepsilon_{t} ]=\sigma^2
        =\sigma^2\cdot \lim_{h\rightarrow 0}\frac{\sin{(\pi h)}}{\pi h}
        =\sigma^2\frac{\sin{(\pi\cdot 0)}}{\pi\cdot 0}\\
       E[ \varepsilon_{t+h} \varepsilon_{t} ]=\sigma^2
        =0
        =\sigma^2\frac{ 0}{\pi h}
        =\sigma^2\frac{ \sin{\pi h}}{\pi h}
      \end{gather}

      上記のように$h$の値について式が成り立つので
      まとめると次を得る。
      \begin{equation}
       E [ \varepsilon_{t+h}\varepsilon_{t} ]=
        \sigma^2\frac{\sin{\pi h}}{\pi h}
        \qquad
        (t,h\in\mathbb{Z})
      \end{equation}


      \hrulefill

 \item
      $W=\{ W(t);t\geq0\}$をブラウン運動、
      つまり次の性質を満たす実数値確率過程とする。
      \begin{itemize}
       \item
            $W(0)=0$
       \item
            $W(t)$は$t$について連続
       \item
            全ての$n\in\mathbb{N}$と
            全ての$0=t_0<t_1<\cdots<t_n$に対して、
            実数値確率変数列
            $\{W(t_1),W(t_2)-W(t_1),\dots,W(t_n)-W(t_{n-1})\}$
            は独立であり、
            なおかつ各$i=1,\dots,n$に対して、
            実数値確率変数$W(t_i)-W(t_{i-1})$
            は正規分布$N(0,t_i-t_{i-1})$に従う。
      \end{itemize}

      このとき、次の問いに答えよ。
      \begin{enumerate}
       \item 各$t>0$に対して、平均$E[W(t)]$を求めよ。

             \dotfill

             $W(t_i)-W(t_{i-1})$
             は$N(0,t_i-t_{i-1})$に従うので、
             $E[W(t_i)-W(t_{i-1})]=0$である。
             $E[W(t_i)-W(t_{i-1})]=E[W(t_i)]-E[W(t_{i-1})]$
             であるから
             $E[W(t_i)]=E[W(t_{i-1})]$となる。

             $W(t_{1}),W(t_{0})$についても同様に
             $E[W(t_{1})]=E[W(t_{0})]$であるが
             $E[W(t_{0})]=E[W(0)]=E[0]=0$となるので、
             $E[W(t_{n})]=\cdots=E[W(t_{1})]=0$である。

             \hrulefill

       \item $0<s\leq t$に対して、
             $Cov[W(s),W(t)]=s$が成り立つことを示せ。

             \dotfill

             確率変数$W(s)$と$W(t)-W(s)$は独立であるので、
             $Cov(W(s),W(t)-W(s))=0$である。

             共分散は
             $Cov(X,Y) = E[XY]-E[X]E[Y]$
             と期待値で表せるので、
             \begin{align}
               & Cov(W(s),W(t)-W(s))\\
               = & E[W(s)(W(t)-W(s))]
               -E[W(s)]E[(W(t)-W(s))]
             \end{align}
             である。
             ここで、$E[W(s)]=0$となるので、
             次の式が得られる。
             \begin{align}
              0=&
              Cov(W(s),W(t)-W(s))\\
               =& E[W(s)(W(t)-W(s))]
               =E[W(s)W(t)]-E[(W(s))^2]
             \end{align}
             よって、$E[W(s)W(t)]=E[(W(s))^2]$である。

             確率変数$W(t_i)-W(t_{i-1})$
             は正規分布$N(0,t_i-t_{i-1})$に従うので、
             $V[W(s)-W(0)]=V[W(s)]=s$
             である。
             分散は
             $V[X] = E[X^2]-(E[X])^2$
             となるので、期待値で表現できる。
             \begin{align}
              V[W(s)] =& E[(W(s))^2] - (E[W(s)])^2\\
               =& E[(W(s))^2] - (0)^2
               = E[(W(s))^2]
             \end{align}

             これまでをまとめると次の式になる。
             \begin{equation}
              E[W(s)W(t)]
              = E[(W(s))^2]
              = V[W(s)]=s
             \end{equation}

             共分散$Cov(W(s),W(t))$を
             計算する。
             \begin{align}
              Cov(W(s),W(t))
               =& E[W(s)W(t)] - E[W(s)]E[W(t)]\\
               =& E[W(s)W(t)] =s
             \end{align}

             以上により
             $Cov(W(s),W(t))=s$
             が得られる。
             
%             \begin{align}
%              Cov(X,Y) =& E[XY]-E[X]E{Y}\\
%              V[X+Y] =& V{X} + V{Y} + 2Cov(X,Y)\\
%              V[X-Y] =& V{X} + V{Y} - 2Cov(X,Y)\\
%              V[X] =& E{X^2} + (E{X})^2\\
%              V[X] =& Cov(X,X)
%             \end{align}

             \hrulefill

      \end{enumerate}


 \item
      $\theta >0$および$\alpha \ne 0$はともに定数とし、
      確率過程$W$はブラウン運動であるとする。
      各$t\in\mathbb{Z}$に対して
      次のような時系列
      $\varepsilon=\{\epsilon_{t};t\in\mathbb{Z}\}$
      を考える。
      \begin{equation}
       \varepsilon_{t}=
        \frac{\alpha}{\sqrt{2\theta}} \left\{
         W \left( e^{2\theta t} \right)
         - W \left( e^{2\theta (t-1)} \right)
         \right\}
        e^{-\theta t}
      \end{equation}

      この時、次の問いに答えよ。
      \begin{enumerate}
       \item 各$t\in\mathbb{Z}$に対して
             $\varepsilon_{t}$の平均$E[\varepsilon_{t}]$を求めよ。

             \dotfill

             $\theta>0$より
             $e^{2\theta t}>e^{2\theta (t-1)}$である。
             
             ブラウン運動であるので、
             確率変数
             $W \left( e^{2\theta t} \right)- W \left( e^{2\theta (t-1)} \right)$
             は
             正規分布$N(0,e^{2\theta t}-e^{2\theta (t-1)})$に従う。

             よって、$E[\varepsilon_t]$は次のように求まる。
             \begin{align}
              E[\varepsilon_{t}] =&
              E\left[
              \frac{\alpha}{\sqrt{2\theta}} \left\{
              W \left( e^{2\theta t} \right)
              - W \left( e^{2\theta (t-1)} \right)
              \right\}
              e^{-\theta t}
              \right]\\
              =&
              \frac{\alpha}{\sqrt{2\theta}}
              E\left[
              W \left( e^{2\theta t} \right)
              - W \left( e^{2\theta (t-1)} \right)
              \right]
              e^{-\theta t}\\
              =&
              \frac{\alpha}{\sqrt{2\theta}}\cdot 0 \cdot e^{-\theta t}
              =0
             \end{align}


             \hrulefill

       \item 時系列$\varepsilon$は弱定常性を満たすことを示せ。

             \dotfill

             弱定常性ということは
             期待値が一定で共分散が離れ具合にのみ依存している
             ということである。
             \begin{equation}
              {}^{\forall}\varepsilon_t,\ \varepsilon_{t+h}\in \varepsilon
               \qquad
               E[\varepsilon_t]=E[\varepsilon_{t+h}]=\mu
               ,\quad
               Cov(\varepsilon_{t+h},\varepsilon_t)=\gamma_h
             \end{equation}

             $E[\varepsilon_{t}]=0$であるので期待値は一定となる。

             共分散は
             $Cov(aX,Y)=Cov(X,aY)=aCov(X,Y)$
             となるので、
             $\varepsilon_{t},\varepsilon_{t+h}$の定数部分も外に出せる。
             \begin{align}
              & Cov(\varepsilon_{t},\varepsilon_{t+h})\\
              =& \frac{\alpha^2}{2\theta}e^{-\theta t}e^{-\theta (t+h)}\\
             &\ \cdot Cov(
              W \left( e^{2\theta t} \right)-W \left( e^{2\theta (t-1)} \right),
              W \left( e^{2\theta (t+h)} \right)-W \left( e^{2\theta (t+h-1)} \right)
              )
             \end{align}

             $W \left( e^{2\theta t} \right)-W \left( e^{2\theta (t-1)} \right)$
             と
             $W \left( e^{2\theta (t+h)} \right)-W \left( e^{2\theta (t+h-1)} \right)$
             は独立である為、共分散は0となる。
             よって、
             $Cov(\varepsilon_{t},\varepsilon_{t+h}) =0$となる。

             つまり、$\varepsilon$は弱定常性を持つといえる。
             

             \hrulefill

      \end{enumerate}


 \item
      $\theta >0$および$\alpha \ne 0$はともに定数とし、
      確率過程$W$はブラウン運動であるとする。
      各$t\in\mathbb{Z}$に対して
      次のような時系列
      $Z=\{Z_{t};t\in\mathbb{Z}\}$
      を考える。
      \begin{equation}
       Z_{t}=\frac{\alpha}{\sqrt{2\theta}}e^{-\theta t}W(e^{2\theta t})
      \end{equation}
      $\varepsilon$を前問で定義した時系列とするとき、
      時系列$Z$はAR(1)モデルとなることを説明せよ。

      \dotfill

      $Z_{t}$の期待値は次のようにすべて0となる。
      \begin{equation}
       E[Z_{t}]=
        E\left[
          \frac{\alpha}{\sqrt{2\theta}}e^{-\theta t}W(e^{2\theta t})
         \right]
        =\frac{\alpha}{\sqrt{2\theta}}e^{-\theta t} E[W(e^{2\theta t})]
        =\frac{\alpha}{\sqrt{2\theta}}e^{-\theta t} \cdot 0
        =0
      \end{equation}

      $Z_{t}$の分散を計算する。
      \begin{equation}
       V[Z_{t}]=
        V\left[
          \frac{\alpha}{\sqrt{2\theta}}e^{-\theta t}W(e^{2\theta t})
         \right]
        =
        \frac{\alpha^2}{2\theta}e^{-2\theta t}
        V\left[ W(e^{2\theta t}) \right]
      \end{equation}

      $0<s\leq t$に対して$Cov(W(s),W(t))=s$であることに注意すると
      \begin{equation}
       V\left[ W(e^{2\theta t}) \right]
        = Cov\left( W(e^{2\theta t}), W(e^{2\theta t}) \right)
        = e^{2\theta t}
      \end{equation}
      であるので、分散が求まる。
      \begin{equation}
       V[Z_{t}]=\frac{\alpha^2}{2\theta}
      \end{equation}

%      $Z_{t+1}$は次の式で表される。
%      \begin{equation}
%       Z_{t+1}
%       = \frac{\alpha}{\sqrt{2\theta}}e^{-\theta (t+1)}W(e^{2\theta (t+1)})
%       = e^{-\theta}\frac{\alpha}{\sqrt{2\theta}}e^{-\theta t}W(e^{2\theta (t+1)})
%      \end{equation}


      期待値$E[Z_{t+h}-Z_{t}]$は
      $E[Z_{t+h}-Z_{t}]=E[Z_{t+h}]-E[Z_{t}]=0$となる。

      分散$V[Z_{t+h}-Z_{t}]$を求める。
      \begin{equation}
       V[Z_{t+h}-Z_{t}]=
        \left( \frac{\alpha}{\sqrt{2\theta}}e^{-\theta (t+h)} \right)^2
        V[W(e^{2\theta (t+h)})-W(e^{2\theta t})]
      \end{equation}
      定数部分をまとめるとブラウン運動の分散が残るので
      計算をする。
      \begin{equation}
       V[W(e^{2\theta (t+h)})-W(e^{2\theta t})]
        = e^{2\theta (t+h)} - e^{2\theta t}
        = e^{2\theta t}(e^{2\theta h} -1)
      \end{equation}

      これにより分散が次のように求まる。
      \begin{equation}
       V[Z_{t+h}-Z_{t}]=
        \frac{\alpha^2}{2\theta} (1-e^{-2\theta h})
      \end{equation}
%      同様に計算すると
%      $t$に依存しないことがわかる。
%      \begin{equation}
%       V[Z_{t+h}-Z_{t}]=
%        \frac{\alpha^2}{2\theta} (1-e^{-2h\theta})
%      \end{equation}



%      \begin{align}
%       & V[W(e^{2\theta (t+1)})-W(e^{2\theta t})]\\
%        =&
%        V[W(e^{2\theta (t+1)})] + V[W(e^{2\theta t})] - Cov(W(e^{2\theta (t+1)}),W(e^{2\theta t}))\\
%        = &
%        V[W(e^{2\theta (t+1)})] + V[W(e^{2\theta t})] - e^{2\theta t}
%      \end{align}

%      \begin{equation}
%       V[W(e^{2\theta (t+1)})] + V[W(e^{2\theta t})]
%        = e^{2\theta (t+1)}
%      \end{equation}


      各$Z_t$の期待値と分散は次のように一定となり、
      二つの差$Z_{t+h}-Z_{t}$の期待値と分散は次のようになる。
      \begin{equation}
       E[Z_{t}]=0
        , \quad
        V[Z_{t}]=\frac{\alpha^2}{2\theta}
        , \quad
       E[Z_{t+h}-Z_{t}]=0
        , \quad
        V[Z_{t+h}-Z_{t}]=
        \frac{\alpha^2}{2\theta} (1-e^{-2\theta h})
      \end{equation}

      確率変数$Z_{t}$は期待値、分散が$t$によらず決定し、
      $Z_{t+h}-Z_{t}$の分散が$h$にのみ依存して決まる。
      つまり、$Z_{t+h}$は過去の一つの$Z_{t}$から影響を受ける。
      この為、$\{Z_{t};t\in\mathbb{Z}\}$は
      AR(1)モデルといえる。


      \hrulefill

\end{enumerate}





\hrulefill

\end{document}
