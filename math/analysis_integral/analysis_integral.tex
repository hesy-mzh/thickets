\documentclass[12pt,b5paper]{ltjsarticle}

%\usepackage[margin=15truemm, top=5truemm, bottom=5truemm]{geometry}
%\usepackage[margin=10truemm,left=15truemm]{geometry}
\usepackage[margin=10truemm]{geometry}

\usepackage{amsmath,amssymb}
%\pagestyle{headings}
\pagestyle{empty}

%\usepackage{listings,url}
%\renewcommand{\theenumi}{(\arabic{enumi})}

%\usepackage{graphicx}

%\usepackage{tikz}
%\usetikzlibrary {arrows.meta}
%\usepackage{wrapfig}	% required for `\wrapfigure' (yatex added)
%\usepackage{bm}	% required for `\bm' (yatex added)

% ルビを振る
%\usepackage{luatexja-ruby}	% required for `\ruby'

%% 核Ker 像Im Hom を定義
%\newcommand{\Img}{\mathop{\mathrm{Im}}\nolimits}
%\newcommand{\Ker}{\mathop{\mathrm{Ker}}\nolimits}
%\newcommand{\Hom}{\mathop{\mathrm{Hom}}\nolimits}

%\DeclareMathOperator{\Rot}{rot}
%\DeclareMathOperator{\Div}{div}
%\DeclareMathOperator{\Grad}{grad}
%\DeclareMathOperator{\arcsinh}{arcsinh}
%\DeclareMathOperator{\arccosh}{arccosh}
%\DeclareMathOperator{\arctanh}{arctanh}



%\usepackage{listings,url}
%
%\lstset{
%%プログラム言語(複数の言語に対応,C,C++も可)
%  language = Python,
%%  language = Lisp,
%%  language = C,
%  %背景色と透過度
%  %backgroundcolor={\color[gray]{.90}},
%  %枠外に行った時の自動改行
%  breaklines = true,
%  %自動改行後のインデント量(デフォルトでは20[pt])
%  breakindent = 10pt,
%  %標準の書体
%%  basicstyle = \ttfamily\scriptsize,
%  basicstyle = \ttfamily,
%  %コメントの書体
%%  commentstyle = {\itshape \color[cmyk]{1,0.4,1,0}},
%  %関数名等の色の設定
%  classoffset = 0,
%  %キーワード(int, ifなど)の書体
%%  keywordstyle = {\bfseries \color[cmyk]{0,1,0,0}},
%  %表示する文字の書体
%  %stringstyle = {\ttfamily \color[rgb]{0,0,1}},
%  %枠 "t"は上に線を記載, "T"は上に二重線を記載
%  %他オプション:leftline,topline,bottomline,lines,single,shadowbox
%  frame = TBrl,
%  %frameまでの間隔(行番号とプログラムの間)
%  framesep = 5pt,
%  %行番号の位置
%  numbers = left,
%  %行番号の間隔
%  stepnumber = 1,
%  %行番号の書体
%%  numberstyle = \tiny,
%  %タブの大きさ
%  tabsize = 4,
%  %キャプションの場所("tb"ならば上下両方に記載)
%  captionpos = t
%}



\begin{document}

\hrulefill

次の積分を求めよ。
\begin{equation}
 \int_{0}^{\infty} \frac{\sqrt{x}}{x^3+1}\mathrm{d}x
\end{equation}

\dotfill

この広義積分を複素数上の積分として考える。
\begin{equation}
 \int_{0}^{\infty} \frac{\sqrt{z}}{z^3+1}\mathrm{d}z
  \qquad (z\in\mathbb{C})
\end{equation}

被積分関数$\frac{\sqrt{z}}{z^3+1}$に対して、
積分経路を
実軸上$0\to R$と
半径$R$の円周上$R\to Ri$と
虚軸上$Ri\to 0$の
3つの部分$C_1,C_2,C_3$からなる閉曲線$C$とする。

この時、
$C_1$上の積分$\int_{C_1} \frac{\sqrt{z}}{z^3+1}\mathrm{d}z$
は
$\int_{0}^{\infty} \frac{\sqrt{x}}{x^3+1}\mathrm{d}x$
と一致する。

$C$上の積分は次のような式となる。
\begin{equation}
 \int_{C} \frac{\sqrt{z}}{z^3+1}\mathrm{d}z
  =
 \int_{C_1} \frac{\sqrt{z}}{z^3+1}\mathrm{d}z
 +
 \int_{C_2} \frac{\sqrt{z}}{z^3+1}\mathrm{d}z
 +
 \int_{C_3} \frac{\sqrt{z}}{z^3+1}\mathrm{d}z
\end{equation}


$z^3+1=0$を満たす複素数は
$z^3=-1=e^{(2n+1)\pi i}$から
$z=e^{\frac{\pi}{3}i},e^{\pi i},e^{\frac{5\pi}{3}i}$
である。
これらは関数$\frac{\sqrt{z}}{z^3+1}$の極となりえるが、
$R$を十分大きな数とした時に$C$の内部に含まれるのは
$z=e^{\frac{\pi}{3}i}$
のみである。
留数定理により
$\int_{C} \frac{\sqrt{z}}{z^3+1}\mathrm{d}z$
は$z=e^{\frac{\pi}{3}i}$の留数から求まる。

\begin{equation}
 (z-e^{\frac{\pi}{3}i})\times \frac{\sqrt{z}}{z^3+1}
  =
   (z-e^{\frac{\pi}{3}i})\times \frac{\sqrt{z}}{(z-e^{\frac{\pi}{3}i})(z-e^{\pi i})(z-e^{\frac{5\pi}{3}i})}
   =
   \frac{\sqrt{z}}{(z-e^{\pi i})(z-e^{\frac{5\pi}{3}i})}
\end{equation}

上記の関数は$z=e^{\frac{\pi}{3}i}$の時に値を持つので、
%$\frac{\sqrt{z}}{z^3+1}$は
1位の極である。

この時の留数を求める。
\begin{equation}
 \lim_{z\to e^{\frac{\pi}{3}i}}\frac{\sqrt{z}}{z^3+1}
  =\lim_{z\to e^{\frac{\pi}{3}i}} \frac{1}{2}\frac{z^{-1/2}}{3z^2}
  =\frac{1}{6} e^{\frac{5\pi}{6}i}
\end{equation}

よって、$C$上の積分は次のようになる。
\begin{equation}
 \int_{C} \frac{\sqrt{z}}{z^3+1}\mathrm{d}z
  = 2\pi i \times \frac{1}{6} e^{\frac{5\pi}{6}i}
  = \frac{1}{3} \pi i e^{\frac{5\pi}{6}i}
\end{equation}

$C$は3つに分かれるためそれぞれの積分を考える。

$C_2$は半径$R$の円周上であるので、
$z=Re^{i\theta}$で$\theta$が$0\to \frac{1}{2}\pi$の範囲の
区間となる。
$\mathrm{d}z = iRe^{i\theta}\mathrm{d}\theta$
を利用し$C_2$上の積分を計算する。
\begin{align}
 \int_{C_2} \frac{\sqrt{z}}{z^3+1}\mathrm{d}z
 = \int_{0}^{\frac{\pi}{2}} \frac{\sqrt{Re^{i\theta}}}{(Re^{i\theta})^3+1} iRe^{i\theta}\mathrm{d}\theta
 = \int_{0}^{\frac{\pi}{2}} \frac{iR^{\frac{3}{2}}e^{\frac{3}{2}i\theta}}{R^3 e^{3i\theta}+1} \mathrm{d}\theta
\end{align}


\begin{align}
  \left\lvert \int_{0}^{\frac{\pi}{2}} \frac{iR^{\frac{3}{2}}e^{\frac{3}{2}i\theta}}{R^3 e^{3i\theta}+1} \mathrm{d}\theta \right\rvert
  \leq &
   \int_{0}^{\frac{\pi}{2}} \left\lvert \frac{iR^{\frac{3}{2}}e^{\frac{3}{2}i\theta}}{R^3 e^{3i\theta}+1}  \right\rvert \mathrm{d}\theta\\
   = &
   \int_{0}^{\frac{\pi}{2}}  \frac{\lvert iR^{\frac{3}{2}}e^{\frac{3}{2}i\theta} \rvert}{\lvert R^3 e^{3i\theta}+1 \rvert}   \mathrm{d}\theta
   =
   \int_{0}^{\frac{\pi}{2}}
   \frac{\lvert iR^{\frac{3}{2}}e^{\frac{3}{2}i\theta} \rvert}
   {\lvert e^{3i\theta} \rvert \lvert R^3 + e^{-3i\theta} \rvert}
   \mathrm{d}\theta
\end{align}

$\theta\in\mathbb{R}$について
$-1 \leq \lvert e^{-3i\theta} \rvert \leq 1$となるので
$R$が十分に大きい値であれば
$\lvert R^3 + e^{-3i\theta} \rvert \geq  R^3 -1$
である。
これを利用し上の式を計算する。
\begin{equation}
   \int_{0}^{\frac{\pi}{2}}
   \frac{\lvert iR^{\frac{3}{2}}e^{\frac{3}{2}i\theta} \rvert}
   {\lvert e^{3i\theta} \rvert \lvert R^3 + e^{-3i\theta} \rvert}
   \mathrm{d}\theta
   \leq
  \int_{0}^{\frac{\pi}{2}}\frac{R^{\frac{3}{2}}}{R^3-1}\mathrm{d}\theta
  = \frac{R^{\frac{3}{2}}}{R^3-1}\cdot\frac{\pi}{2}
  \to 0 \ (R\to\infty)
\end{equation}

つまり、$C_2$上の積分は$R\to\infty$において$0$に収束する。
\begin{equation}
 \lim_{R\to\infty}\int_{C_2} \frac{\sqrt{z}}{z^3+1}\mathrm{d}z=0
\end{equation}

次に$C_3$上の積分を考える。
$C_3$は虚軸上の$iR\to 0$の区間である。
そこで、$z=it$と置き、$t$が$R\to 0$に動く場合の積分として計算する。
このとき、$\frac{\mathrm{d}z}{\mathrm{d}t}=i$であるので、
$\mathrm{d}z=i\mathrm{d}t$である。
\begin{equation}
  \int_{C_3} \frac{\sqrt{z}}{z^3+1}\mathrm{d}z
   =
  \int_{R}^{0} \frac{\sqrt{it}}{(it)^3+1}i\mathrm{d}t
  =
  -i^{\frac{3}{2}} \int_{0}^{R} \frac{\sqrt{t}}{-it^3+1}\mathrm{d}t
\end{equation}




\hrulefill

\end{document}
