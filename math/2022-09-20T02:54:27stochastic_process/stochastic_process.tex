\documentclass[12pt,b5paper]{ltjsarticle}

%\usepackage[margin=15truemm, top=5truemm, bottom=5truemm]{geometry}
\usepackage[margin=10truemm]{geometry}

\usepackage{amsmath,amssymb}
%\pagestyle{headings}
\pagestyle{empty}

%\usepackage{listings,url}
%\renewcommand{\theenumi}{(\arabic{enumi})}

%\usepackage{graphicx}

%\usepackage{tikz}
%\usetikzlibrary {arrows.meta}
%\usepackage{wrapfig}	% required for `\wrapfigure' (yatex added)
%\usepackage{bm}	% required for `\bm' (yatex added)

% ルビを振る
%\usepackage{luatexja-ruby}	% required for `\ruby'

%% 核Ker 像Im Hom を定義
%\newcommand{\Img}{\mathop{\mathrm{Im}}\nolimits}
%\newcommand{\Ker}{\mathop{\mathrm{Ker}}\nolimits}
%\newcommand{\Hom}{\mathop{\mathrm{Hom}}\nolimits}

%\DeclareMathOperator{\Rot}{rot}
%\DeclareMathOperator{\Div}{div}
%\DeclareMathOperator{\Grad}{grad}
%\DeclareMathOperator{\arcsinh}{arcsinh}
%\DeclareMathOperator{\arccosh}{arccosh}
%\DeclareMathOperator{\arctanh}{arctanh}



\begin{document}

\textbf{確率過程}

\dotfill

$X = \{ X_t \mid t\in\mathbb{Z} \}$: 時系列

${}^\forall n \in \mathbb{N}$
,
$\ {}^\forall t_i \in \mathbb{Z} \ (1\leq i \leq n)$
,
${}^\forall k \in \mathbb{Z}$


\textbf{強定常性}

$(X_{t_1},\dots,X_{t_n})$と
$(X_{t_1+k},\dots,X_{t_n+k})$が
同じ分布を持つ時、
時系列$X$は強定常性を持つという。


\textbf{弱定常性}

\begin{itemize}
 \item
      ${}^\forall X_i \in X$に対して
      期待値$E[X_i]$が一定である。
      $E[X_i] = \mu$

 \item
      ${}^\forall X_i,X_{i+k} \in X$に対して
      共分散$Cov(X_i,X_{i+k})$は$k$についてのみ依存する。
      $Cov(X_i,X_{i+k})=\sigma_k$
\end{itemize}

上記の2つを満たす時、
時系列$X$は弱定常性を持つという。


\hrulefill

任意の確率過程${}^\forall X_i,X_j\in X$に対して
共分散$Cov[X_i,X_j]$が存在する場合、
時系列$X$が強定常性を持つのなら
弱定常性も持つ。

\dotfill


$X$が強定常性を持つので、
${}^\forall n \in\mathbb{N}$に対して
$(X_{t_1},\dots,X_{t_n})$と
$(X_{t_1+k},\dots,X_{t_n+k})$が
同じ分布である。

この為、$n=1$の場合、
$(X_{t_1})$と$(X_{t_1+k})$が同じ分布である。
これにより$E[X_{t_1}]=E[X_{t_1+k}]$であるが、
${}^\forall k\in\mathbb{Z}$であるので
全ての期待値が一致する。

$n=2$の場合、
$(X_{t_1},X_{t_2})$と$(X_{t_1+k},X_{t_2+k})$が同じ分布である。
この為、$Cov(X_{t_1},X_{t_2})=Cov(X_{t_1+k},X_{t_2+k})$である。
これにより$X_{\alpha},X_{\beta}$の共分散は
これを$k$だけスライドさせた$X_{\alpha+k},X_{\beta+k}$の共分散と一致する。
つまり、確率過程$X_{\alpha},X_{\beta}$の添字の差$\beta-\alpha$の値に対して
共分散が定まる。

これにより、期待値と共分散の条件が弱定常性を満たす。


\end{document}

