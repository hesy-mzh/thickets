\documentclass[12pt,b5paper]{ltjsarticle}

%\usepackage[margin=15truemm, top=5truemm, bottom=5truemm]{geometry}
\usepackage[margin=10truemm]{geometry}

\usepackage{amsmath,amssymb}
%\pagestyle{headings}
\pagestyle{empty}

%\usepackage{listings,url}
%\renewcommand{\theenumi}{(\arabic{enumi})}

%\usepackage{graphicx}

%\usepackage{tikz}
%\usetikzlibrary {arrows.meta}
%\usepackage{wrapfig}	% required for `\wrapfigure' (yatex added)
%\usepackage{bm}	% required for `\bm' (yatex added)

% ルビを振る
%\usepackage{luatexja-ruby}	% required for `\ruby'

%% 核Ker 像Im Hom を定義
%\newcommand{\Img}{\mathop{\mathrm{Im}}\nolimits}
%\newcommand{\Ker}{\mathop{\mathrm{Ker}}\nolimits}
%\newcommand{\Hom}{\mathop{\mathrm{Hom}}\nolimits}

%\DeclareMathOperator{\Rot}{rot}
%\DeclareMathOperator{\Div}{div}
%\DeclareMathOperator{\Grad}{grad}
%\DeclareMathOperator{\arcsinh}{arcsinh}
%\DeclareMathOperator{\arccosh}{arccosh}
%\DeclareMathOperator{\arctanh}{arctanh}



%\usepackage{listings,url}
%
%\lstset{
%%プログラム言語(複数の言語に対応,C,C++も可)
%  language = Python,
%%  language = Lisp,
%%  language = C,
%  %背景色と透過度
%  %backgroundcolor={\color[gray]{.90}},
%  %枠外に行った時の自動改行
%  breaklines = true,
%  %自動改行後のインデント量(デフォルトでは20[pt])
%  breakindent = 10pt,
%  %標準の書体
%%  basicstyle = \ttfamily\scriptsize,
%  basicstyle = \ttfamily,
%  %コメントの書体
%%  commentstyle = {\itshape \color[cmyk]{1,0.4,1,0}},
%  %関数名等の色の設定
%  classoffset = 0,
%  %キーワード(int, ifなど)の書体
%%  keywordstyle = {\bfseries \color[cmyk]{0,1,0,0}},
%  %表示する文字の書体
%  %stringstyle = {\ttfamily \color[rgb]{0,0,1}},
%  %枠 "t"は上に線を記載, "T"は上に二重線を記載
%  %他オプション:leftline,topline,bottomline,lines,single,shadowbox
%  frame = TBrl,
%  %frameまでの間隔(行番号とプログラムの間)
%  framesep = 5pt,
%  %行番号の位置
%  numbers = left,
%  %行番号の間隔
%  stepnumber = 1,
%  %行番号の書体
%%  numberstyle = \tiny,
%  %タブの大きさ
%  tabsize = 4,
%  %キャプションの場所("tb"ならば上下両方に記載)
%  captionpos = t
%}



\begin{document}

\hrulefill
\textbf{問題}
\hrulefill

\begin{enumerate}
 \item
      乗法群$(\mathbb{Z}/18\mathbb{Z})^{\times}$
      の元を全て列挙し、
      それぞれの位数を求めよ。

      \dotfill
      
      集合$\mathbb{Z}/18\mathbb{Z}$の要素は18個あり、
      次のようになる。
      \begin{equation}
       \mathbb{Z}/18\mathbb{Z}
        =\{ \bar{0}, \bar{1}, \bar{2}, \dots , \bar{17} \}
      \end{equation}

      乗法群$(\mathbb{Z}/18\mathbb{Z})^{\times}$
      は
      集合$\mathbb{Z}/18\mathbb{Z}$の要素から
      単元となるものだけを選んだ集合である。
      単元となるのは$18$と互いに素な数の元であるので
      具体的には次のような集合である。
      \begin{equation}
       (\mathbb{Z}/18\mathbb{Z})^{\times}
        =
        \{ \bar{1},\bar{5},\bar{7},\bar{11},\bar{13},\bar{17} \}
      \end{equation}

      それぞれの元の位数を計算する。

      \begin{align}
       \bar{5} \times \bar{5} =& \bar{7}\\
       \bar{5} \times \bar{5} \times \bar{5} =& \bar{17}\\
       \bar{5} \times \bar{5} \times \bar{5} \times \bar{5} =& \bar{13}\\
       \bar{5} \times \bar{5} \times \bar{5} \times \bar{5} \times \bar{5}=& \bar{11}\\
       \bar{5} \times \bar{5} \times \bar{5} \times \bar{5} \times \bar{5}\times \bar{5}=& \bar{1}
      \end{align}

      これにより
      $(\mathbb{Z}/18\mathbb{Z})^{\times} = \langle \bar{5} \rangle$
      であることが分かる。
      つまり、$\bar{5}$の位数は6である。

      同様に計算すると次のようになる。
      \begin{align}
       \bar{7}\times\bar{7}\times\bar{7}=\bar{1}\\
       \bar{11}\times\bar{11}\times\bar{11}\times\bar{11}\times\bar{11}\times\bar{11}=\bar{1}\\
       \bar{13}\times\bar{13}\times\bar{13}=\bar{1}\\
       \bar{17}\times\bar{17}=\bar{1}
      \end{align}

      これらをまとめると次のようになる。
      \begin{center}
       \begin{tabular}{c|cccccc}
        元 & $\bar{1}$ & $\bar{5}$ & $\bar{7}$ & $\bar{11}$ & $\bar{13}$ & $\bar{17}$ \\
        \hline
        位数 & 1 & 6 & 3 & 6 & 3 & 2 \\
       \end{tabular}
      \end{center}

      \hrulefill
 \item
      正三角形の対称群
      $D_{6}=\langle r , s \mid r^3=s^2 =e,\ rs=sr^{-1} \rangle$
      に対し、
      $H$を頂点1を動かさない対称変換からなる部分群とする。
      頂点1を通る線対称の反射を$s$と呼び、
      左$2\pi /3$回転を$r$と呼ぶ。

      この時、
      頂点1を頂点3に移す対象変換からなる集合は
      $D_{6}$の$H$に関する左剰余類であるか調べよ。

      \dotfill

      正三角形の頂点は左回りに1,2,3とする。

      対称群$D_{6}$と部分群$H$は次のような群である。
      \begin{align}
       D_{6} =& \{ e,r,r^2,s,rs,r^2s \}\\
       H =& \{ e, s\}
      \end{align}

      また、頂点1を頂点3に移す対称変換は$r^2,rs$の2つである。
      これを$S$とする。
      \begin{equation}
       S=\{ r^2,rs\}=\{r^2,sr^2\}
      \end{equation}

      $r^2\in D_{6}$を$H$に右からかけると$S$となるので、
      右剰余類である。
      \begin{equation}
       Hr^2=\{xr^2\mid x\in H\}=\{ er^2,sr^2\}=S
      \end{equation}

      しかし、$H$に左から何をかけても$S$とはならないので
      左剰余類ではない。


      \hrulefill

\end{enumerate}


\end{document}
