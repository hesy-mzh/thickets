\documentclass[12pt,b5paper]{ltjsarticle}

%\usepackage[margin=15truemm, top=5truemm, bottom=5truemm]{geometry}
\usepackage[margin=15truemm]{geometry}

\usepackage{amsmath,amssymb}
%\pagestyle{headings}
\pagestyle{empty}

%\usepackage{listings,url}
%\renewcommand{\theenumi}{(\arabic{enumi})}

\usepackage{graphicx}

\usepackage{tikz}
\usetikzlibrary {arrows.meta}
\usepackage{wrapfig}	% required for `\wrapfigure' (yatex added)
\usepackage{bm}	% required for `\bm' (yatex added)
\usepackage{luatexja-ruby}	% required for `\ruby'
%% 像Im を定義
%\newcommand{\Img}{\mathop{\mathrm{Im}}\nolimits}

\begin{document}
関数$f_1,f_2$を次のように定義する。
\begin{gather}
 A = \{ (x,y) \in \mathbb{R}^2 \mid x^2+y^2\leq 1\}\\
 %
 f_1 : A \rightarrow \mathbb{R}, \quad f_{1}(x,y)=\sqrt{4-x^2-y^2}\\
 f_2 : A \rightarrow \mathbb{R}, \quad f_{2}(x,y)=\sqrt{1-x^2-y^2}
\end{gather}
\begin{enumerate}
 \item $f_1$は$A$上$C^1$級であるか?
 \item $f_2$は$A$上$C^1$級であるか?
\end{enumerate}

\hrulefill

\textbf{$C^1$級}

関数$f$が$C^1$級
%$\stackrel{\mathrm{def.}}{\Longleftrightarrow}$
$\stackrel{\mathrm{def.}}{\Leftrightarrow}$
\begin{itemize}
 \item 定義域の任意の点において$f_x, \, f_y$が存在
 \item 定義域の任意の点において$f, \, f_x, \, f_y$が連続
\end{itemize}

\dotfill

\textbf{連続}

$S$: $\mathbb{R}^n$の部分集合

$f : S \rightarrow \mathbb{R}^m$ 関数

$a\in S$において$f$が連続
$\stackrel{\mathrm{def.}}{\Leftrightarrow}$

$\lim_{x\rightarrow a}f(x) = f(a)$

この定義は$\varepsilon - \delta$で書くと次のようになる。
( $B(a;\delta)$は 中心$a$、半径$\delta$の開球)
\begin{equation}
 {}^{\forall} \varepsilon >0 , \ {}^{\exists}\delta >0
  \ s.t. \
  {}^{\forall} x\in B(a;\delta) \cap S ,\ \lvert f(x)-f(a) \rvert < \varepsilon
\end{equation}

\dotfill

\textbf{偏微分}

$U$: $\mathbb{R}^n$の開集合

$f : U \rightarrow \mathbb{R}^m$ 関数

$a=(a_1,\dots,a_n)\in U$において$x_i$について$f$が偏微分可能
$\stackrel{\mathrm{def.}}{\Leftrightarrow}$
次の極限値が存在
\begin{equation}
 \lim_{h\rightarrow 0}\frac{f(a_1,\dots,a_{i-1},a_{i}+h,a_{i+1},\dots,a_n)-f(a_1,\dots,a_{i-1},a_{i},a_{i+1},\dots,a_n)}{h}
\end{equation}

\hrulefill

\begin{enumerate}
 \item 
次のように開集合$U$と関数$g$を定義する。

\begin{gather}
 U=\{ (x,y)\in\mathbb{R}^2 \mid x^2+y^2 < 4 \}\\
 g : U \rightarrow \mathbb{R} , \quad g(x,y)=\sqrt{4-x^2-y^2}
\end{gather}
\begin{itemize}
 \item 
$g$の偏微分可能性について

${}^{\forall} (a,b)\in U$とする。
\begin{align}
 \lim_{h\rightarrow 0}\frac{g(a+h,b)-g(a,b)}{h}
 =& \lim_{h\rightarrow 0}\frac{\sqrt{4-(a+h)^2-b^2}-\sqrt{4-a^2-b^2}}{h}\\
 =& \lim_{h\rightarrow 0}\frac{(4-(a+h)^2-b^2)-(4-a^2-b^2)}{h(\sqrt{4-(a+h)^2-b^2}+\sqrt{4-a^2-b^2})}\\
 =& \lim_{h\rightarrow 0}\frac{-2ah-h^2}{h(\sqrt{4-(a+h)^2-b^2}+\sqrt{4-a^2-b^2})}\\
 =& \frac{-a}{\sqrt{4-a^2-b^2}}\\
\end{align}
これにより関数$g$は点$(a,b)$において$x$で偏微分可能である。
同様の計算で$y$でも偏微分可能となる。

 \item
連続性

\begin{equation}
 g_x(x,y)= \frac{-x}{\sqrt{4-x^2-y^2}}, \quad g_y(x,y)= \frac{-y}{\sqrt{4-x^2-y^2}}
\end{equation}
とすると、$g,g_x,g_y$は$4-x^2-y^2>0$の範囲において連続である。
よって、$U$上連続となる。
\end{itemize}
これにより$g$は$U$上$C^1$級である。

$A\subset U$であり、$g$の定義域を$A$に制限した関数が$f_1$である為、
$f_1$も$C^1$級である。

 \item
      集合$A$は閉集合である。
      \begin{equation}
       A^{\circ} =\{ (x,y)\in\mathbb{R}^2 \mid x^2+y^2<1\}
      \end{equation}
      とすると、$A^{\circ}\subset A$は開集合となる。

      関数$f_2$を$A^{\circ}$上に制限すると
      $C^1$級である。

      ただ、$A$の境界上($x^2+y^2=1$上)での偏微分の定義によって
      $f_2$は$C^1$級かどうかが変わる。

      もし、境界上は定義域内においてのみ偏微分の極限が存在することで
      偏微分可能だとするのであれば
      $f_2$は$A$上で$C^1$級であるが、
      そうでない場合は$C^1$級ではない。
\end{enumerate}




\end{document}
