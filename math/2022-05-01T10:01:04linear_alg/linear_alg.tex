\documentclass[12pt,b5paper]{ltjsarticle}

%\usepackage[margin=15truemm, top=5truemm, bottom=5truemm]{geometry}
\usepackage[margin=15truemm]{geometry}

\usepackage{amsmath,amssymb}
%\pagestyle{headings}
\pagestyle{empty}

%\usepackage{listings,url}
\renewcommand{\theenumi}{(\arabic{enumi})}

\usepackage{graphicx}

\usepackage{tikz}
\usetikzlibrary {arrows.meta}
\usepackage{wrapfig}	% required for `\wrapfigure' (yatex added)
\usepackage{bm}	% required for `\bm' (yatex added)

%% 像Im を定義
%\newcommand{\Img}{\mathop{\mathrm{Im}}\nolimits}

\begin{document}


\hrulefill

線形写像に関する説明

\begin{equation}
 f: V \rightarrow W
\end{equation}

\begin{enumerate}
 \item $\mathrm{Ker}f=\{0\} \Leftrightarrow f:\text{単射}$
 \item $\mathrm{rank} f = \mathrm{dim}(\mathrm{Im}f)$
 \item $\mathrm{null}f = \mathrm{dim}(\mathrm{Ker} f)$
 \item $\mathrm{dim} V = \mathrm{rank} f \Leftrightarrow f:\text{単射}$
 \item $\mathrm{dim} W = \mathrm{rank} f \Leftrightarrow f:\text{全射}$
 \item $\mathrm{dim} V = \mathrm{dim}(\mathrm{Ker} f) +  \mathrm{dim}(\mathrm{Im}f)$
\end{enumerate}


\hrulefill

\begin{enumerate}
 \item 線形空間$V$から$V$への線形写像が全射であれば単射である。

       \dotfill

       $f$は自己線形写像。
       全射であれば$\mathrm{dim} V = \mathrm{rank} f$であるので
       単射となる。

       \hrulefill
 \item 線形空間$V$から$V$への線形写像$f$が $\mathrm{null} f = 0$であれば全射である。

       \dotfill

       $\mathrm{null}f=0$であれば、$f$は単射。
       単射であれば$\mathrm{dim} V = \mathrm{rank} f$であり、
       $\mathrm{dim} V = \mathrm{rank} f$であるなら全射となる。

       \hrulefill
 \item すべての要素が$1$の$m\times n$行列$A$により
       線形空間$K^n$から$K^m$への線形写像$f$を定義する。
       $\mathrm{null}f$はいくつになるか

       \dotfill

       線形写像$f$は次のような写像である。
       \begin{equation}
        f: K^n \rightarrow K^m , \ \bm{x} \mapsto A\bm{x}
       \end{equation}

       すべての行列成分が同じなので$\mathrm{rank}A =1$である。
       $\mathrm{rank}f=\mathrm{rank}A$より$\mathrm{rank}f=1$。

       $\mathrm{dim} V = \mathrm{dim}(\mathrm{Ker} f) +  \mathrm{dim}(\mathrm{Im}f)$
       より
       \begin{equation}
        \mathrm{dim}K^n = \mathrm{null}f + \mathrm{rank}f
       \end{equation}
       であるので、$n=\mathrm{null}f + 1$より$\mathrm{null}f = n-1$。

       \hrulefill
 \item 実数体上で$5\times 3$行列$A$と
       $5$次元単位ベクトル$\bm{e_1}$について、
       $3$元連立方程式$A\bm{x}=\bm{e_1}$の解の全体の集合はどのような形か。

       \dotfill

       この行列$A$を用いて線形写像を定義すると次のような$f$が出来る。
       \begin{equation}
        f: \mathbb{R}^3 \rightarrow \mathbb{R}^5, \ \bm{x} \mapsto A\bm{x}
       \end{equation}
       $A\bm{x}=\bm{e_1}$の解空間とは、$f^{-1}(\bm{e_1})$のことである。

       $\bm{e_1}$は単位ベクトルなので、長さが1のベクトル。$\bm{e_1}\ne \bm{0}$

       $f$は線形写像なので$\bm{0}\in\mathbb{R}^3$は
       $\bm{0}\in\mathbb{R}^5$に対応する。
       つまり、$\bm{0}\not\in f^{-1}(\bm{e_1})$である。

       行列$A$の階数$\mathrm{rank}A$は
       \begin{equation}
        0 \leq \mathrm{rank}A \leq 3
       \end{equation}
       である。

       \textbf{$\mathrm{rank}A=0$の場合}

       この場合、$A$はすべての成分が$0$であるので、
       ${}^\forall \bm{x} \in \mathbb{R}^3 , \ f(\bm{x})=\bm{0}$。
       つまり解空間は\textbf{空集合}となる。

       $\bm{e_1}\not\in\mathrm{Im}f$の時、解空間は\textbf{空集合}であるので、
       これ以降、$\bm{e_1}\in\mathrm{Im}f$の場合を考える。

       \textbf{$\mathrm{rank}A=1$の場合}

       $\bm{e_1}\in\mathrm{Im}f$の時、解空間の次元は$\mathrm{null}f=2$と等しい。
       よって、解空間は\textbf{原点を通らない平面}である。

       \textbf{$\mathrm{rank}A=2$の場合}

       $\mathrm{rank}f=2$なので、
       $\bm{e_1}\in\mathrm{Im}f$の時、解空間の次元は$\mathrm{null}f=1$と等しい。
       よって、解空間は\textbf{原点を通らない直線}である。

       \textbf{$\mathrm{rank}A=3$の場合}

       $\mathrm{rank}f=3$なので、
       $\bm{e_1}\in\mathrm{Im}f$の時、解空間の次元は$\mathrm{null}f=0$と等しい。
       よって、解空間は\textbf{原点ではない一点}である。

       \vspace*{10pt}

       まとめると、
       \begin{enumerate}
        \item $\mathrm{rank}A$の値によらず $A\bm{x}=\bm{e_1}$に解がない時、\textbf{空集合}
        \item $A\bm{x}=\bm{e_1}$に解がある時
              \begin{enumerate}
               \item $\mathrm{rank}A=1$の場合、\textbf{原点を通らない平面}
               \item $\mathrm{rank}A=2$の場合、\textbf{原点を通らない直線}
               \item $\mathrm{rank}A=3$の場合、\textbf{原点以外の1点}
              \end{enumerate}
       \end{enumerate}


       \hrulefill
\end{enumerate}



\end{document}
