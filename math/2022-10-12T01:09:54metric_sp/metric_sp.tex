\documentclass[12pt,b5paper]{ltjsarticle}

%\usepackage[margin=15truemm, top=5truemm, bottom=5truemm]{geometry}
\usepackage[margin=10truemm]{geometry}

\usepackage{amsmath,amssymb}
%\pagestyle{headings}
\pagestyle{empty}

%\usepackage{listings,url}
%\renewcommand{\theenumi}{(\arabic{enumi})}

%\usepackage{graphicx}

%\usepackage{tikz}
%\usetikzlibrary {arrows.meta}
%\usepackage{wrapfig}	% required for `\wrapfigure' (yatex added)
\usepackage{bm}	% required for `\bm' (yatex added)

% ルビを振る
\usepackage{luatexja-ruby}	% required for `\ruby'

%% 核Ker 像Im Hom を定義
%\newcommand{\Img}{\mathop{\mathrm{Im}}\nolimits}
%\newcommand{\Ker}{\mathop{\mathrm{Ker}}\nolimits}
%\newcommand{\Hom}{\mathop{\mathrm{Hom}}\nolimits}

%\DeclareMathOperator{\Rot}{rot}
%\DeclareMathOperator{\Div}{div}
%\DeclareMathOperator{\Grad}{grad}
%\DeclareMathOperator{\arcsinh}{arcsinh}
%\DeclareMathOperator{\arccosh}{arccosh}
%\DeclareMathOperator{\arctanh}{arctanh}




\begin{document}


\textbf{距離空間}

空間$X$に関数$d:X\times X \to \mathbb{R}$が定義され、
$d$が次の3つを満たすとする。
\begin{enumerate}
 \item
      ${}^{\forall}x,y\in X$に対し、
      $d(x,y)\geq 0$であり、
      $d(x,y)=0 \Leftrightarrow x=y$

 \item
      ${}^{\forall}x,y\in X$に対し、
      $d(x,y)=d(y,x)$

 \item
      ${}^{\forall}x,y,z\in X$に対し、
      $d(x,y)+d(y,z)\geq d(x,z)$
\end{enumerate}

この時、関数$d$を距離関数といい、
距離関数が定義された空間$X$を距離空間という。


\textbf{ユークリッド距離}

$\mathbb{R}^n$の元$x=(x_1,\dots,x_n), y=(y_1,\dots,y_n)$に対し、
\begin{equation}
 d(x,y)=\sqrt{\sum_{i=1}^{n}(x_i - y_i)^2}
\end{equation}
をユークリッド距離という。


\textbf{開集合}

集合$X$の部分集合$O$が開集合であるとは、
任意の点$x\in O$について
ある$\varepsilon$近傍$U_{\varepsilon}(x)$が存在し
$U_{\varepsilon}(x)\subset O$であるときをいう。


\textbf{閉集合}

集合$X$の部分集合$C$が閉集合であるとは、
$C$の補集合$X\backslash C$が開集合となるときをいう。



\hrulefill



\hrulefill


$\mathbb{R}^n$上の通常のユークリッド距離$d_n$に対して
$(\mathbb{R}^n,d_n)$は距離空間になることを示せ。

\dotfill

ユークリッド距離$d_n$が距離関数の3条件を満たすことを確認する。

${}^{\forall}\bm{x},\bm{y},\bm{z}\in\mathbb{R}^n$とする。

\begin{equation}
 d_n(\bm{x},\bm{y}) = \sqrt{\sum_{i=1}^{n}(x_i - y_i)^2} \geq 0
\end{equation}
ユークリッド距離は正の平方根であるため常に0以上である。
また、$d_n(\bm{x},\bm{y})=0$となる時$x_i=y_i$であり、
$\bm{x}=\bm{y}$であれば$d_n(\bm{x},\bm{y})=0$となる。
つまり、$d_n(\bm{x},\bm{y})=0 \Leftrightarrow \bm{x}=\bm{y}$である。


\begin{equation}
 d_n(\bm{x},\bm{y}) = \sqrt{\sum_{i=1}^{n}(x_i - y_i)^2}
 = \sqrt{\sum_{i=1}^{n}(y_i - x_i)^2} = d_n(\bm{y},\bm{x})
\end{equation}
根号の中は平方和であるので、$(x_i-y_i)^2 = (y_i-x_i)^2$となり、
$d_n(\bm{x},\bm{y})=d_n(\bm{y},\bm{x})$である。



三角不等式$d_n(\bm{x},\bm{y})+d_n(\bm{y},\bm{z})\geq d_n(\bm{x},\bm{z})$を示す。

最初に示した通り$d_n$は常に0以上である。
そこで、2乗の差が正となることを示せばよい。
\begin{equation}
 ( d_n(\bm{x},\bm{y})+d_n(\bm{y},\bm{z}) )^2 - ( d_n(\bm{x},\bm{z}) )^2 \geq 0
\end{equation}


\begin{align}
 & ( d_n(\bm{x},\bm{y})+d_n(\bm{y},\bm{z}) )^2 - ( d_n(\bm{x},\bm{z}) )^2\\
 = & \left( \sqrt{\sum_{i=1}^{n}(x_i - y_i)^2} + \sqrt{\sum_{i=1}^{n}(y_i - z_i)^2} \right)^2 - \left( \sqrt{\sum_{i=1}^{n}(x_i - z_i)^2} \right)^2\\
% = & 2\sqrt{ \left( \sum_{i=1}^{n}(x_i - y_i)^2 \right) \left(\sum_{i=1}^{n}(y_i - z_i)^2 \right) }
 % + \sum_{i=1}^{n}((x_i - y_i)^2 + (y_i - z_i)^2 - (x_i - z_i)^2) \\
 = & 2\left(\sqrt{ \left( \sum_{i=1}^{n}(x_i - y_i)^2 \right) \left(\sum_{i=1}^{n}(y_i - z_i)^2 \right) } - \sum_{i=1}^{n}(x_i - y_i)(y_i - z_i) \right) \label{schwarz}\\
 \geq & 2\left(\sqrt{ \left( \sum_{i=1}^{n}(x_i - y_i)(y_i - z_i) \right)^2 } - \sum_{i=1}^{n}(x_i - y_i)(y_i - z_i) \right) =0
\end{align}

式(\ref{schwarz})には
\ruby{Schwarz}{シュワルツ}の不等式
($
(\sum_{i=1}^{n}a_i^2)
(\sum_{i=1}^{n}b_i^2)
\geq
\left( \sum_{i=1}^{n}a_ib_i \right)^2
$)
を利用した。


三角不等式$d_n(\bm{x},\bm{y})+d_n(\bm{y},\bm{z})\geq d_n(\bm{x},\bm{z})$
が成り立つことが示せた。

これにより
ユークリッド距離$d_n$は$\mathbb{R}^n$上の距離関数となるので、
$(\mathbb{R}^n,d_n)$は距離空間である。



\hrulefill

$\mathbb{R}^n$の2つの元
$\bm{x}=(x_1,\dots,x_n),\bm{y}=(y_1,\dots,y_n)$
に対して
$d_n^{*}(\bm{x},\bm{y})=\max\{ \lvert x_i-y_i \lvert \in \mathbb{R} \mid i=1,\dots,n \}$
とすると
$(\mathbb{R}^n,d_n^{*})$は距離空間になることを示せ。

\dotfill

$d_n^*$の定義に$\lvert x_i-y_i \rvert$とあるので、
${}^{\forall}\bm{x},\bm{y}\in\mathbb{R}^n$に対し
$d_n^*(\bm{x},\bm{y})\geq 0$である。
また、$d_n^*(\bm{x},\bm{y})=0 \Rightarrow \bm{x}=\bm{y}$であり、
$d_n^*(\bm{x},\bm{x})=0$である。

次の式の通り
$d_n^*(\bm{x},\bm{y})= d_n^*(\bm{y},\bm{x})$
である。
\begin{align}
 d_n^*(\bm{x},\bm{y})
  =& \max\{ \lvert x_i-y_i \lvert \in\mathbb{R} \mid i=1,\dots,n \}\\
  =& \max\{ \lvert y_i-x_i \lvert \in\mathbb{R} \mid i=1,\dots,n \}
  =d_n^*(\bm{y},\bm{x})
\end{align}

三角不等式
$d_n^*(\bm{x},\bm{y}) + d_n^*(\bm{y},\bm{z}) \geq d_n^*(\bm{x},\bm{z})$
を示す。


$d_n^*(\bm{x},\bm{z})$は定義からある$k$が存在し
$d_n^*(\bm{x},\bm{z}) = \lvert x_{k}-z_{k} \rvert$となる。

実数の絶対値における三角不等式から
$\bm{y}$の$k$成分$y_{k}$を用いて次の不等式が成り立つ。
\begin{equation}
 \lvert x_{k} - y_{k} \rvert +   \lvert y_{k} - z_{k} \rvert
 \geq \lvert x_{k}-z_{k} \rvert
\end{equation}

$d_n^{*}$の定義より
$d_n^{*}(\bm{x},\bm{y}) \geq \lvert x_{k} - y_{k} \rvert$、
$d_n^{*}(\bm{y},\bm{z}) \geq \lvert y_{k} - z_{k} \rvert$
である。

\begin{equation}
 d_n^{*}(\bm{x},\bm{y}) + d_n^{*}(\bm{y},\bm{z})
 \geq \lvert x_{k} - y_{k} \rvert + \lvert y_{k} - z_{k} \rvert
 \geq \lvert x_{k} - z_{k} \rvert
 = d_n^*(\bm{x},\bm{z})
\end{equation}

これにより$d_n^{*}$は距離関数であり、
$(\mathbb{R}^n,d_n^{*})$は距離空間になる。



\hrulefill

閉集合の無限個の和集合が閉集合ではない開集合となる例をあげ、
それを証明せよ。

\dotfill


$\mathbb{R}$上の集合を考える。
\begin{gather}
 A_1=[0,0]=\{0\},\
 A_2=\left[-\frac{1}{2},\frac{1}{2}\right],\
 A_3=\left[-\frac{2}{3},\frac{2}{3}\right],\\
 A_4=\left[-\frac{3}{4},\frac{3}{4}\right],\
  \dots,\
  A_n=\left[-1+\frac{1}{n},1-\frac{1}{n}\right],\
 \dots
\end{gather}

この時、各$A_i$は閉集合であるが、
$\bigcup_{i=1}^{\infty}A_i$は閉集合ではなく開集合となる。


$A_n=\left[-1+\frac{1}{n},1-\frac{1}{n}\right]$
に対して補集合は次のようになる。
\begin{equation}
 A_n^{c}
  = \left( -\infty,-1+\frac{1}{n} \right)
  \cup
  \left( 1-\frac{1}{n},\infty \right)
\end{equation}


${}^{\forall}x\in A_n^{c}$とすると
$x\in \left( -\infty,-1+\frac{1}{n} \right)$
または
$x\in \left( 1-\frac{1}{n},\infty \right)$
である。

$x\in \left( -\infty,-1+\frac{1}{n} \right)$の場合を考える。
$\varepsilon = \frac{1}{2}\left( \left(-1+\frac{1}{n}\right) - x \right)$
とすると
\begin{equation}
 x\in (x-\varepsilon , x+\varepsilon) \subset A_n^c
\end{equation}
である。
$x\in \left( 1-\frac{1}{n}, \infty \right)$の場合も同様の議論により
$x$の$\varepsilon$近傍は$A_n$の補集合に含まれる。
この為、補集合$A_n^c$が開集合となり、
$A_n$は閉集合である。


閉集合$A_n$の和集合$\bigcup_{i=1}^{\infty}A_i$について考える。

$A_n$は次のような包含関係が成り立っている。
\begin{equation}
 A_1 \subset A_2 \subset A_3 \subset A_4 \subset \cdots  \subset A_n \subset \cdots
\end{equation}

${}^{\forall}x \in \bigcup_{i=1}^{\infty}A_i$とする。
この時、ある$k$が存在し$x\in A_k$であり、
$x=-1+\frac{1}{k}$
又は
$x=1-\frac{1}{k}$
又は
$-1+\frac{1}{k} < x < 1-\frac{1}{k}$
である。

$-1+\frac{1}{k} < x < 1-\frac{1}{k}$の場合
\begin{equation}
 \varepsilon = \min\left\{
                    \frac{1}{2}\left( x- \left(-1+\frac{1}{k}\right)\right),
                    \frac{1}{2}\left( \left(1-\frac{1}{k}\right)-x \right)
                   \right\}
\end{equation}
このように$\varepsilon$を定義すると$\varepsilon$近傍$U_{\varepsilon}(x)$は
$U_{\varepsilon}(x) \subset A_k$である。

$x=-1+\frac{1}{k}$
又は
$x=1-\frac{1}{k}$
の場合

閉集合$A_{k+1}$が存在し$x\in A_{k+1}$である。
\begin{equation}
 \varepsilon = \min\left\{
                    \frac{1}{2}\left( x- \left(-1+\frac{1}{k+1}\right)\right),
                    \frac{1}{2}\left( \left(1-\frac{1}{k+1}\right)-x \right)
                   \right\}
\end{equation}
このように$\varepsilon$を定義すると$\varepsilon$近傍$U_{\varepsilon}(x)$は
$U_{\varepsilon}(x) \subset A_{k+1}$である。


この為、
${}^{\forall}x \in \bigcup_{i=1}^{\infty}A_i$について
$\varepsilon$近傍が存在し
$U_{\varepsilon}(x) \subset \bigcup_{i=1}^{\infty}A_i$
であることが分かる。

つまり、
$\bigcup_{i=1}^{\infty}A_i$は開集合となる。


\hrulefill


\end{document}
