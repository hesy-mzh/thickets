\documentclass[12pt,b5paper]{ltjsarticle}

%\usepackage[margin=15truemm, top=5truemm, bottom=5truemm]{geometry}
\usepackage[margin=15truemm]{geometry}

\usepackage{amsmath,amssymb}
%\pagestyle{headings}
\pagestyle{empty}

%\usepackage{listings,url}
%\renewcommand{\theenumi}{(\arabic{enumi})}

\usepackage{graphicx}

\usepackage{tikz}
\usetikzlibrary {arrows.meta}
\usepackage{wrapfig}	% required for `\wrapfigure' (yatex added)
\usepackage{bm}	% required for `\bm' (yatex added)
\usepackage{luatexja-ruby}	% required for `\ruby'
%% 像Im を定義
%\newcommand{\Img}{\mathop{\mathrm{Im}}\nolimits}

\begin{document}

$K$を体、$V$を$K$-線形空間とする。
\begin{equation}
 \Phi : V \rightarrow \mathrm{Hom}_K(\mathrm{Hom}_K(V,K),K)
  \quad
  v \mapsto (f\mapsto f(v))
\end{equation}
写像$\Phi$を上のように定義する。
$\Phi (v) (f) = f(v)$
\begin{enumerate}
 \item $\Phi$は$K$-線形写像である。
 \item $\Phi$は単射である。
 \item $V$が有限生成の時、$\Phi$は全単射である。
\end{enumerate}

\hrulefill

\begin{equation}
 \mathrm{Hom}_K(V,W)
  \stackrel{\mathrm{def}}{=}
  \{ f \mid f:V\rightarrow W \ (K\text{-線形写像}) \}
\end{equation}

$(\lambda f + g)(v) = \lambda f(v)+g(v)$と定義することによりにより
$\mathrm{Hom}_K(V,W)$は線形空間となる。


\dotfill

$V,W$:$K$-線形空間

写像$f: V\rightarrow W$が次の2つの条件を満たす時、
$K$-線形写像という。
\begin{enumerate}
 \item $x,y\in V$について、$f(x+y)=f(x)+f(y)$
 \item $x\in V,\ k\in K$について、$f(kx)=kf(x)$
\end{enumerate}

\dotfill

写像$f:V\rightarrow W$について

$f$が単射である
$\Leftrightarrow$
$f(a)=f(b)$であるなら$a=b$

$f$が全射である
$\Leftrightarrow$
${}^{\forall}w\in W$に対し$w=f(v)$となる$v\in V$が存在


\hrulefill

\begin{enumerate}
 \item $\Phi$は$K$-線形写像である。

       $x,y\in V$とする。
       $\Phi$は写像である為、
       \begin{equation}
        \Phi (x+y) \in \mathrm{Hom}_K(\mathrm{Hom}_K(V,K),K)
       \end{equation}
       $\Phi(x+y)$は次のような線形写像である。
       \begin{equation}
        \Phi (x+y) : \mathrm{Hom}_K(V,K) \rightarrow K
       \end{equation}

       これより${}^{\forall}f\in\mathrm{Hom}_K(V,K)$に対し
       \begin{equation}
         \Phi (x+y)(f)=f(x+y)
       \end{equation}
       であり、$f$は線形写像であるので、$f(x+y)=f(x)+f(y)$。
        $\Phi (x)(f)=f(x)$、$\Phi (y)(f)=f(y)$ より
       \begin{equation}
        \Phi (x+y)(f) = \Phi (x)(f)+\Phi (y)(f)
       \end{equation}
       この為、$\Phi (x+y) = \Phi (x)+\Phi (y)$となる事がわかる。

       同様にして、$k\in K$に対し
       $\Phi (kx) = k \Phi (x)$である為、
       $\Phi$は$K$-線形写像である。

       \dotfill

 \item $\Phi$は単射である。

       $v,w\in V$に対し、$\Phi (v)=\Phi (w)$とする。
       $\Phi (v)-\Phi (w)=0$だが、
       $\Phi$の線形性から$\Phi (v)-\Phi (w)=\Phi (v-w)=0$となる。

       また、
       ${}^{\forall}f\in\mathrm{Hom}_K(V,K)$に対し
       $\Phi (v-w)(f)=f(v-w)$となる為、
       $f(v-w)=0$である。
       
       もし、$f$が零写像であれば$V=\{\bm{0}\}$であるので、
       $v,w\in V$から$v=w$である。

       $\mathrm{Hom}_K(V,K)$の元として
       $V$の基底と標準内積を取る写像$f_e$を考える。
       $i$番目が$1$でそれ以外が$0$の基底で考えると
       $f_e(v-w)$は$i$番目の成分のみが取り出される。
       $f_e(v-w)=0$より$v-w$の$i$成分は$0$である。
       すべての成分について同じように考えると
       $v-w=0$ということが分かる。
       これにより$v=w$となり、
       $\Phi$は単射である。
       
       \dotfill
 \item $V$が有限生成の時、$\Phi$は全単射である。

       単射は示されているので、全射であることを示す。

       $V$が有限生成であるので、
       次元を$n$とする。
       $\dim_{K}V =n$

       $V$の次元は写像$\Phi$のイメージとカーネルより次のようになる。
       \begin{equation}
        \dim_{K}V = \dim_{K} \mathrm{Im}\Phi + \dim_{K} \mathrm{Ker}\Phi
         = n
       \end{equation}
       $\Phi$は単射であるので、$\dim_{K} \mathrm{Ker}\Phi =0$である。

       線形空間$V$と$\mathrm{Hom}_K(V,K)$は同じ次元である為、
       $V^{*}=\mathrm{Hom}_K(V,K)$と$\mathrm{Hom}_K(V^{*},K)$も同じ次元である。
       \begin{equation}
        \dim_{K} V
         =
        \dim_{K}\mathrm{Hom}_K(V,K)
         =
        \dim_{K}\mathrm{Hom}_K(\mathrm{Hom}_K(V,K),K)
        =n
       \end{equation}

       これにより次の2つの次元が同じであることが分かる。
       \begin{equation}
        \dim_{K} \mathrm{Im}\Phi
         =
        \dim_{K}\mathrm{Hom}_K(\mathrm{Hom}_K(V,K),K)
        =n
       \end{equation}

       ここから
       $\Phi$の像は$\mathrm{Hom}_K(\mathrm{Hom}_K(V,K),K)$は一致する事がわかり、
       $\Phi$が全射であることが分かる。

       $\Phi$は全単射であるので、
       $V$と
       $\mathrm{Hom}_K(\mathrm{Hom}_K(V,K),K)$
       は同型であることが分かる。

%       ${}^{\forall}f\in\mathrm{Hom}_K(\mathrm{Hom}_K(V,K),K)$とする。
%       \begin{equation}
%        f : \mathrm{Hom}_K(V,K) \rightarrow K
%       \end{equation}
%
%       線形空間$\mathrm{Hom}_K(V,K)$の次元は$V$の次元と同じであるので、
%       $\dim_{K}\mathrm{Hom}_K(V,K) =n$である。
%       この基底を$e^{\star}_1,\dots,e^{\star}_n \in \mathrm{Hom}_K(V,K)$とする。
%       $e^{\star}_i$は$\bm{v}\in V$の$i$成分を取り出す線形写像である。


%       $V$の基底を$e_1,\dots,e_n$とする。
%       $\Phi(e_1),\dots,\Phi(e_n)$
%       は
%       \begin{equation}
%        \Phi(e_i) : \mathrm{Hom}_K(V,K) \rightarrow K ,
%         \quad f\mapsto f(e_i)
%       \end{equation}
%
%
%
%
%       $e^{\star}_i \in \mathrm{Hom}_K(V,K)$を
%       $\bm{v}\in V$の$i$成分を取り出す線形写像とする。
%       ${}^{\forall}f\in\mathrm{Hom}_K(V,K)$は
%       $e^{\star}_i$の線形結合で表せる。
%       つまり、
%       線形空間$\mathrm{Hom}_K(V,K)$の次元は$V$の次元と同じである。
%       $\dim_{K}V=\dim_{K}\mathrm{Hom}_K(V,K)=n$
%
%       同じ議論を行うことで$\mathrm{Hom}_K(\mathrm{Hom}_K(V,K),K)$の次元は
%       $\mathrm{Hom}_K(V,K)$と同じである。
%       \begin{equation}
%        \dim_{K}V=\dim_{K}\mathrm{Hom}_K(V,K)
%         =\dim_{K}\mathrm{Hom}_K(\mathrm{Hom}_K(V,K),K)
%         =n
%       \end{equation}

\end{enumerate}

\hrulefill

\textbf{双対空間}

$K$上ベクトル空間$V$に対し、
線形写像$V\rightarrow K$全体の集合$\mathrm{Hom}_{K}(V,K)$を
双対ベクトル空間という。

$V$の次元が有限次元であれば
\begin{equation}
 \dim_{K}V
  =
  \dim_{K}\mathrm{Hom}_{K}(V,K)
\end{equation}
である。
$V$の双対ベクトル空間を$V^{*}$で表す。

$V$の基底
\begin{equation}
 \begin{pmatrix}1\\0\\ \vdots\\0\end{pmatrix}
 , \
   \begin{pmatrix}0\\1\\ \vdots\\0\end{pmatrix}
   , \ \dots ,
 \begin{pmatrix}0\\ \vdots\\0\\1\end{pmatrix}
\end{equation}
に対し、
双対空間$V^{*}$の基底は
$\hat{e}_1, \dots,\hat{e}_n$とする。
$\hat{e}_i$は
$\bm{v}\in V$から$i$番目を取り出す線形写像である。

例えば写像$\hat{e}_1$は次のような写像である。
\begin{equation}
 \hat{e}_1(\begin{pmatrix}1\\0\\ \vdots\\0\end{pmatrix}) = 1, \
 \hat{e}_1(\begin{pmatrix}0\\1\\ \vdots\\0\end{pmatrix}) = 0, \ \dots ,
 \hat{e}_1(\begin{pmatrix}0\\ \vdots\\0\\1\end{pmatrix}) = 0
\end{equation}






\end{document}
