\documentclass[12pt,b5paper]{ltjsarticle}

%\usepackage[margin=15truemm, top=5truemm, bottom=5truemm]{geometry}
%\usepackage[margin=10truemm,left=15truemm]{geometry}
\usepackage[margin=10truemm]{geometry}

\usepackage{amsmath,amssymb}
%\pagestyle{headings}
\pagestyle{empty}

%\usepackage{listings,url}
%\renewcommand{\theenumi}{(\arabic{enumi})}

\usepackage{graphicx}

%\usepackage{tikz}
%\usetikzlibrary {arrows.meta}
%\usepackage{wrapfig}	% required for `\wrapfigure' (yatex added)
%\usepackage{bm}	% required for `\bm' (yatex added)

% ルビを振る
%\usepackage{luatexja-ruby}	% required for `\ruby'

%% 核Ker 像Im Hom を定義
%\newcommand{\Img}{\mathop{\mathrm{Im}}\nolimits}
%\newcommand{\Ker}{\mathop{\mathrm{Ker}}\nolimits}
%\newcommand{\Hom}{\mathop{\mathrm{Hom}}\nolimits}

%\DeclareMathOperator{\Rot}{rot}
%\DeclareMathOperator{\Div}{div}
%\DeclareMathOperator{\Grad}{grad}
%\DeclareMathOperator{\arcsinh}{arcsinh}
%\DeclareMathOperator{\arccosh}{arccosh}
%\DeclareMathOperator{\arctanh}{arctanh}



%\usepackage{listings,url}
%
%\lstset{
%%プログラム言語(複数の言語に対応,C,C++も可)
%  language = Python,
%%  language = Lisp,
%%  language = C,
%  %背景色と透過度
%  %backgroundcolor={\color[gray]{.90}},
%  %枠外に行った時の自動改行
%  breaklines = true,
%  %自動改行後のインデント量(デフォルトでは20[pt])
%  breakindent = 10pt,
%  %標準の書体
%%  basicstyle = \ttfamily\scriptsize,
%  basicstyle = \ttfamily,
%  %コメントの書体
%%  commentstyle = {\itshape \color[cmyk]{1,0.4,1,0}},
%  %関数名等の色の設定
%  classoffset = 0,
%  %キーワード(int, ifなど)の書体
%%  keywordstyle = {\bfseries \color[cmyk]{0,1,0,0}},
%  %表示する文字の書体
%  %stringstyle = {\ttfamily \color[rgb]{0,0,1}},
%  %枠 "t"は上に線を記載, "T"は上に二重線を記載
%  %他オプション:leftline,topline,bottomline,lines,single,shadowbox
%  frame = TBrl,
%  %frameまでの間隔(行番号とプログラムの間)
%  framesep = 5pt,
%  %行番号の位置
%  numbers = left,
%  %行番号の間隔
%  stepnumber = 1,
%  %行番号の書体
%%  numberstyle = \tiny,
%  %タブの大きさ
%  tabsize = 4,
%  %キャプションの場所("tb"ならば上下両方に記載)
%  captionpos = t
%}



\begin{document}

\hrulefill

$6x^5+12x^2+4$の既約性を判定せよ。

\dotfill

\textbf{整数上の多項式}

$6x^5+12x^2+4 = 3(2x^5+4x^2+2)$
であるので、可約である。


\textbf{有理数上の多項式}

$6x^5+12x^2+4 = 2(3x^5+6x^2+2)$
であるので、
$3x^5+6x^2+2$
の既約性を調べる。

アイゼンシュタインの既約判定法より、
係数$(3,0,0,6,0,2)$
を調べる。
\begin{itemize}
 \item 最高次数の係数$3$は素数$2$で割り切れない
 \item 最高次数の係数以外は素数$2$で割り切れる
 \item 定数項$2$は素数の二乗$2^2$で割り切れない
\end{itemize}

これにより既約であることがわかる。

\textbf{実数上の多項式}

$3x^5+6x^2+2$
の既約性を調べる。

$y=3x^5+6x^2+2$のグラフを考えると
左下から右上に向かうグラフである。
つまり、
$0=3x^5+6x^2+2$を満たす実数$x=\alpha$が存在する。
よって、$x-\alpha$と4次式の積に分かれるため
可約である。

\textbf{複素数上の多項式}

代数学の基本定理により
複素数係数の多項式
は1次式の積に分けられる。

$6x^5+12x^2+4$は可約である。

\hrulefill

$f\in\mathbb{C}[X]$において2次以上の式はすべて可約である。

\dotfill

代数学の基本定理より従う。

\hrulefill

$f\in\mathbb{R}[X]$において3次以上の式はすべて可約である。

\dotfill

ある虚数$\alpha$が存在し、$f(\alpha)=0$であるとする。
この時、複素共役な元$\overline{\alpha}$も
$f(\overline{\alpha})=0$である。
これは多項式$f$の係数が実数であることから言える。

これにより$f(x)$は$(x-\alpha)$と$(x-\overline{\alpha})$を
因子に持つことがわかる。
この積は次のように$\mathbb{R}[x]$の元である。
\begin{equation}
 (x-\alpha)(x-\overline{\alpha})
  =x^2-(\alpha+\overline{\alpha})x+\alpha\overline{\alpha}
  \in\mathbb{R}[x]
\end{equation}

よって、ある虚数根を持つ多項式は
2次式を因子に持つ。

代数学の基本定理により
必ず複素数根を持つので、
実数根であれば1次式、
虚数根であれば2次式
に分解できる。
この為、3次以上の多項式は可約となる。

\hrulefill

\end{document}
