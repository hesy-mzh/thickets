\documentclass[12pt,b5paper]{ltjsarticle}

%\usepackage[margin=15truemm, top=5truemm, bottom=5truemm]{geometry}
%\usepackage[margin=10truemm,left=15truemm]{geometry}
\usepackage[margin=10truemm]{geometry}

\usepackage{amsmath,amssymb}
%\pagestyle{headings}
\pagestyle{empty}

%\usepackage{listings,url}
%\renewcommand{\theenumi}{(\arabic{enumi})}

%\usepackage{graphicx}

%\usepackage{tikz}
%\usetikzlibrary {arrows.meta}
%\usepackage{wrapfig}
%\usepackage{bm}

% ルビを振る
\usepackage{luatexja-ruby}	% required for `\ruby'

%% 核Ker 像Im Hom を定義
%\newcommand{\Img}{\mathop{\mathrm{Im}}\nolimits}
%\newcommand{\Ker}{\mathop{\mathrm{Ker}}\nolimits}
%\newcommand{\Hom}{\mathop{\mathrm{Hom}}\nolimits}

%\DeclareMathOperator{\Rot}{rot}
%\DeclareMathOperator{\Div}{div}
%\DeclareMathOperator{\Grad}{grad}
%\DeclareMathOperator{\arcsinh}{arcsinh}
%\DeclareMathOperator{\arccosh}{arccosh}
%\DeclareMathOperator{\arctanh}{arctanh}

\usepackage{url}

%\usepackage{listings}
%
%\lstset{
%%プログラム言語(複数の言語に対応,C,C++も可)
%  language = Python,
%%  language = Lisp,
%%  language = C,
%  %背景色と透過度
%  %backgroundcolor={\color[gray]{.90}},
%  %枠外に行った時の自動改行
%  breaklines = true,
%  %自動改行後のインデント量(デフォルトでは20[pt])
%  breakindent = 10pt,
%  %標準の書体
%%  basicstyle = \ttfamily\scriptsize,
%  basicstyle = \ttfamily,
%  %コメントの書体
%%  commentstyle = {\itshape \color[cmyk]{1,0.4,1,0}},
%  %関数名等の色の設定
%  classoffset = 0,
%  %キーワード(int, ifなど)の書体
%%  keywordstyle = {\bfseries \color[cmyk]{0,1,0,0}},
%  %表示する文字の書体
%  %stringstyle = {\ttfamily \color[rgb]{0,0,1}},
%  %枠 "t"は上に線を記載, "T"は上に二重線を記載
%  %他オプション:leftline,topline,bottomline,lines,single,shadowbox
%  frame = TBrl,
%  %frameまでの間隔(行番号とプログラムの間)
%  framesep = 5pt,
%  %行番号の位置
%  numbers = left,
%  %行番号の間隔
%  stepnumber = 1,
%  %行番号の書体
%%  numberstyle = \tiny,
%  %タブの大きさ
%  tabsize = 4,
%  %キャプションの場所("tb"ならば上下両方に記載)
%  captionpos = t
%}

%\usepackage{cancel}
%\usepackage{bussproofs}
%\usepackage{proof}

\begin{document}

\hrulefill

\textbf{テンソル積}

\ruby{tensor}{テンソル}積 は
加群の直積に多重線形性をもたせた集合である。

\dotfill

\textbf{テンソル積の定義}

可換環$R$と$R$加群$M,N$に対し、
直積$M\times N$を考える。

この直積集合$M\times N$に同値関係$\sim$を定め、
これによる商$M\times N \slash \sim$を
$M$と$N$のテンソル積といい
$M \otimes_{R} N$と書く。

同値関係$\sim$は次の様に定義する。
$m_{1},m_{2}\in M,\; n_{1},n_{2}\in N,\; r\in R$
\begin{itemize}
 \item $(n_{1}+n_{2},m_{1})\sim (n_{1},m_{1})+(n_{2},m_{1})$
 \item $(n_{1},m_{1}+m_{2})\sim (n_{1},m_{1})+(n_{1},m_{2})$
 \item $r(n_{1},m_{1}) \sim (r n_{1},m_{1}) \sim (n_{1},r m_{1})$
\end{itemize}

同値関係$\sim$の性質から
テンソル積の元$n\otimes m$は次のような性質を持つ。
\begin{itemize}
 \item $(n_{1}+n_{2}) \otimes m_{1} = n_{1} \otimes m_{1}+n_{2} \otimes m_{1}$
 \item $n_{1} \otimes (m_{1}+m_{2}) = n_{1}\otimes m_{1} + n_{1}\otimes m_{2}$
 \item $r(n_{1}\otimes m_{1}) = (r n_{1})\otimes m_{1} = n_{1} \otimes (r m_{1})$
\end{itemize}


\dotfill

\textbf{テンソル積の定義 2}

可換環$R$に対し、
$M,N$を$R$加群とする。

このとき、$R$加群$T$と$R$双線形写像$\varphi : M\times N \to T$が存在し、
次を満たす。

任意の$R$加群$Z$と任意の$R$双線形写像$\psi : M \times N \to Z$に対して、
$\psi = f \circ \varphi$を満たす
$R$線形写像$f: T \to Z$が唯一つだけ存在する。

このとき、$R$加群$T$をテンソル積といい
$T = M \otimes_{R} N$とかく。

\hrulefill

加群のテンソル積は、
2つ目の定義(普遍性での定義)を使うことが多い。

\begin{enumerate}

 \item
      $m, n$ を正の整数とし、$d$ を $m$ と $n$ の最大公約数とする。

      このとき、
      $\mathbb{Z}/m\mathbb{Z}
       \otimes_{\mathbb{Z}}
       \mathbb{Z}/n\mathbb{Z}$ は
      $\mathbb{Z}/d\mathbb{Z}$ と同型であることを示せ。

      \dotfill


      $\mathbb{Z}/m\mathbb{Z}$の元を$[z]_{m}$というように
      右下に添え字を書くことで表すものとする。
      このとき、$z\in\mathbb{Z}$である。


      $\mathbb{Z}$準同型写像$f$を次のように定める。
      \begin{equation}
       f : \mathbb{Z} \to
        \mathbb{Z}/m\mathbb{Z}\otimes_{\mathbb{Z}}\mathbb{Z}/n\mathbb{Z},
        \qquad z \mapsto [z]_{m} \otimes [1]_{n}
      \end{equation}

      ${}^{\forall} [\alpha]_{m} \otimes [\beta]_{n} \in \mathbb{Z}/m\mathbb{Z}\otimes_{\mathbb{Z}}\mathbb{Z}/n\mathbb{Z}$とする。

      $[\alpha]_{m} \otimes [\beta]_{n}$は次のように計算できる。
      \begin{equation}
       [\alpha]_{m} \otimes [\beta]_{n}
        = \beta([\alpha]_{m} \otimes [1]_{n})
        = [\alpha\beta]_{m} \otimes [1]_{n}
      \end{equation}

      つまり、$[\alpha]_{m} \otimes [\beta]_{n} = f(\alpha\beta)$であるので、
      $f$は全射である。

      そこで
      $f$の準同型定理より次が得られる。
      \begin{equation}
       \mathbb{Z}/\mathrm{Ker}f \cong \mathbb{Z}/m\mathbb{Z}\otimes_{\mathbb{Z}}\mathbb{Z}/n\mathbb{Z}
        \label{fundatheo}
      \end{equation}

      $\mathrm{Ker}f$について調べる。

      $d$は$m,n$の最大公約数であるので、
      ある整数$s,t$を用いて$d=ms+nt$と表せる。

      ${}^{\forall}z\in\mathbb{Z}$に対して、
      $f(dz)$を計算する。
      \begin{align}
       f(dz) &= [dz]_{m} \otimes [1]_{n}
       &
       &= [(ms+nt)z]_{m} \otimes [1]_{n}\\
       &= [msz]_{m} \otimes [1]_{n} + [ntz]_{m}  \otimes [1]_{n}
       &
       &= [msz]_{m} \otimes [1]_{n} + [1]_{m} \otimes [ntz]_{n}\\
       &= [0]_{m} \otimes [1]_{n} + [1]_{m} \otimes [0]_{n}
       &
       &= 0
      \end{align}

      つまり、$d\mathbb{Z} \subset \mathrm{Ker}f$である。


      写像
      $\psi : \mathbb{Z}/m\mathbb{Z} \times \mathbb{Z}/n\mathbb{Z}
      \to \mathbb{Z}/d\mathbb{Z}$
      を$\psi([\alpha]_{m},[\beta]_{n}) = [\alpha\beta]_{d}$と定める。
      $\psi$は$\mathbb{Z}$双線形写像である。
      ${}^{\forall}[z]_{d}\in\mathbb{Z}/d\mathbb{Z}$に対し、
      $[z]_{d} = \psi([z]_{m}, [1]_{n})$となるので、
      $\psi$は全射である。

      テンソル積の普遍性より
      次の準同型写像$g$が存在する。
      \begin{equation}
       g : \mathbb{Z}/m\mathbb{Z} \otimes_{\mathbb{Z}} \mathbb{Z}/n\mathbb{Z}
        \to \mathbb{Z}/d\mathbb{Z},
        \qquad [\alpha]_{m} \otimes [\beta]_{n} \mapsto [\alpha\beta]_{d}
      \end{equation}

      $\varphi : \mathbb{Z}/m\mathbb{Z} \times \mathbb{Z}/n\mathbb{Z} \to \mathbb{Z}/m\mathbb{Z} \otimes_{\mathbb{Z}} \mathbb{Z}/n\mathbb{Z}$
      とすれば、
      $\psi = g\circ \varphi$である。
      $\psi$が全射なので、$g$も全射である。

      ${}^{\forall}x\in\mathrm{Ker}f$とする。

      $g$の全射性より
      $[x]_{d} = g([x]_{m}\otimes [1]_{n})$となるが、
      $f(x)=[x]_{m}\otimes [1]_{n}$であるので
      $[x]_{d}=g(f(x))$。
      $x\in\mathrm{Ker}f$から$f(x)=0$であるので、
      $[x]_{d}=g(f(x))=[0]_{d}$であり、
      $\mathrm{Ker}f \subset d\mathbb{Z}$ということがわかる。

      よって、
      $\mathrm{Ker}f = d\mathbb{Z}$であるから
      式\eqref{fundatheo}より次が得られる。
      \begin{equation}
       \mathbb{Z}/d\mathbb{Z} \cong \mathbb{Z}/m\mathbb{Z}\otimes_{\mathbb{Z}}\mathbb{Z}/n\mathbb{Z}
      \end{equation}


      \hrulefill

 \item
      $R$を環とするとき、
      正の整数 $m, n$ に対し $R^{n} \otimes_{R} R^{m}$ と $R^{mn}$ は
      同型であることを示せ。

      \dotfill

      $u_{i}=(0,\dots,0,1,0,\dots,0) \in R^{m}$
      を$i$成分のみ$1$とすると$R^{m}$は次のように表せる。
      \begin{equation}
       R^{m} = \left \{ \sum_{i=1}^{m} r_{i}u_{i} \:\middle|\: r_{i} \in R \right \}
      \end{equation}

      $v_{i}=(0,\dots,0,1,0,\dots,0) \in R^{n}$
      を$i$成分のみ$1$とすると$R^{n}$は次のように表せる。
      \begin{equation}
       R^{n} = \left \{ \sum_{i=1}^{n} r_{i}v_{i} \:\middle|\: r_{i} \in R \right \}
      \end{equation}

      同様に$w_{ij}\in R^{mn}$を定義し、$R^{mn}$を次のように表す。
      \begin{equation}
       R^{mn} = \left \{ \sum_{i=1}^{m} \sum_{j=1}^{n} r_{ij}w_{ij} \:\middle|\: r_{ij} \in R \right \}
      \end{equation}

      $a\in R^{n}, \; b\in R^{m}$
      を
      $a=\sum_{j=1}^{n}a_{j}v_{j}, \; b=\sum_{i=1}^{m}b_{i}u_{i}$
      として、
      テンソル積$a\otimes b$を計算する。
      \begin{align}
       a\otimes b =& \left( \sum_{j=1}^{n}a_{j}v_{j} \right) \otimes \left( \sum_{i=1}^{m}b_{i}u_{i} \right) &
       =& \sum_{j=1}^{n} a_{j} \left( v_{j} \otimes \left( \sum_{i=1}^{m}b_{i}u_{i} \right)\right)\\
       =& \sum_{j=1}^{n} a_{j} \left(  \sum_{i=1}^{m}b_{i} \left( v_{j} \otimes u_{i} \right)\right) &
       =& \sum_{j=1}^{n} \sum_{i=1}^{m} a_{j} b_{i} \left( v_{j} \otimes u_{i} \right)
      \end{align}


      準同型写像$f: R^{n}\otimes_{R} R^{m} \to R^{mn}$を
      $f(v_{j} \otimes u_{i}) = w_{ij}$と定める事により
      $f(a\otimes b) = \sum_{j=1}^{n} \sum_{i=1}^{m} a_{j} b_{i} w_{ij}$とする。

      $f$は全単射であるなら、
      $R^{n}\otimes_{R} R^{m}$
      と $R^{mn}$は同型である。


      $0\in R^{mn}$の逆像$f^{-1}(0)\in R^{n}\otimes_{R} R^{m}$
      を考える。

      \begin{equation}
       \sum_{j=1}^{n} \sum_{i=1}^{m} a_{j} b_{i} w_{ij} = 0
        \Longleftrightarrow a_{j} b_{i} = 0 \; (j=1,\dots,n,\; i=1,\dots,m)
      \end{equation}

      よって、
      $(a_{1},\dots,a_{m})=(0,\dots,0)$
      または
      $(b_{1},\dots,b_{n})=(0,\dots,0)$
      であるので、
      $\mathrm{Ker}f=\{0\}$となり$f$は単射である。
      \qquad ($R$は整域 ?)

      ${}^{\forall} c =\sum_{j=1}^{n} \sum_{i=1}^{m} c_{ij}w_{ij} \in R^{mn}$
      に対し、
      $\sum_{j=1}^{n} \sum_{i=1}^{m} c_{ij} \left( v_{j}\otimes u_{i} \right) \in R^{n} \otimes_{R} R^{m}$
      が存在し、次の式が成り立つ。
      \begin{equation}
       \sum_{j=1}^{n} \sum_{i=1}^{m} c_{ij}w_{ij}
        = f\left(
            \sum_{j=1}^{n} \sum_{i=1}^{m} c_{ij} (v_{j}\otimes u_{i})
           \right)
      \end{equation}

      よって、$f$は全射である。


      以上により、$f$は同型写像であり、
      $R^{n}\otimes R^{m} \cong R^{mn}$
      である。


      \hrulefill

 \item
      整数環$\mathbb{Z}$、
      多項式環$\mathbb{R}[x]$は
      アルチン環ではないことを示せ。

      \dotfill

      \ruby{Artin}{アルティン}環とは、
      イデアル$I_{n}$の
      無限列$I_{1}\supset I_{2} \supset \cdots $が存在するとき、
      ある$N$番目が存在し、$I_{N}=I_{N+1}=\cdots$となるときを言う。

      整数環$\mathbb{Z}$について、
      イデアル$I_{n}$を$I_{n} = (2^{n})$と定める。
      このとき、無限列$I_{1}\supset I_{2} \supset \cdots$が存在し、
      任意の$N$に対して$I_{N}\ne I_{N+1}$である。
      よって、$\mathbb{Z}$はアルティン環ではない。

      多項式環$\mathbb{R}[x]$について
      イデアル$J_{n}$を$J_{n} = (x^{n})$と定める。
      このとき、無限列$J_{1}\supset J_{2} \supset \cdots$が存在し、
      任意の$N$に対して$J_{N}\ne J_{N+1}$である。
      よって、$\mathbb{R}[x]$はアルティン環ではない。

      \hrulefill

 \item
      $K$を体とする。
      $n$を正の整数とするとき、
      左 $M_{n}(K)$ 加群 $K^{n}$ は単純であることを示せ。



      \dotfill

      部分加群$S\subset K^{n}$をとってきて
      $S\ne \{0\}$とする。

      $s\in S$を$s\ne 0$とし、
      $s$の0でない成分を$s_{i}$とする。

      ${}^{\forall}k\in K^{n}$を
      $k=(k_{1},k_{2},\dots,k_{n})$とする。

      行列$A\in M_{n}(K)$を次のように
      $i$列目以外はすべて0の行列として定める。
      \begin{equation}
       A=
        \begin{pmatrix}
         0 & \cdots & k_{1}s_{i}^{-1} & \cdots & 0\\
         0 & \cdots & \vdots & \cdots & 0\\
         0 & \cdots & k_{n}s_{i}^{-1} & \cdots & 0\\
        \end{pmatrix}
      \end{equation}

      これにより$k = As$となる。
      よって、$k\in S$となり、
      $S=K^{n}$となる。

      つまり、$K^{n}$は単純加群である。

      \hrulefill

 \item
      左 $M_{n}(K)$ 加群 $M_{n}(K)$ は組成列を持つことを示せ。

      \dotfill

      $K$は体とし、
      $M_{n}(K)$は$n$次正方行列全体の集合とする。

      $M_{n}(K)$は環としてみて、左イデアル$I$を考える。

      $I\ne\{0\}$とする。

      $A\in I$が正則行列とする。
      ${}^{\forall}B\in M_{n}(K)$に対し、
      $CA=B$となる$C\in M_{n}(K)$が存在する。($C=BA^{-1}$より)
      よって、$I=M_{n}(K)$となる。

      つまり、$I$に正則行列が含まれていれば$I=M_{n}(K)$である。

      %$I$は正則行列が含まれていないとする。

      $A\in M_{n}(K)$の$(1,1)$成分が0以外で、
      その他の成分が0である行列とする。
      ${}^{\forall}B\in M_{n}(K)$に対し、
      $BA$は第1列以外が0となる行列である。
      よって、第1列以外が0である行列全体の集合$I_{1}$は
      $M_{n}(K)$の左イデアルである。

      同様に$i$列以外が0である行列全体の集合を$I_{i}$とすると、
      これらはすべて左イデアルである。

      行列環$M_{n}(K)$の左イデアルは
      $M_{n}(K)$の他、
      正則でない行列の集合$I_{i}\;(i=1,\dots,n)$と
      各$I_{i}$の組み合わせた集合である。

      左イデアルは左加群であるので、
      組成列は有限となる

%      $M_{n}(K)$は環としてみて、Artin環である。

%      Artin環であればNoether環である。

%      そのイデアルを加群としてみれば、
%      $M_{n}(K)$はArtin加群であり、Noether加群である。

      よって、
      $M_{n}(K)$は組成列を持つことがわかる。

      \hrulefill
\end{enumerate}

\hrulefill

\end{document}
