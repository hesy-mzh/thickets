\documentclass[12pt,b5paper]{ltjsarticle}

%\usepackage[margin=15truemm, top=5truemm, bottom=5truemm]{geometry}
\usepackage[margin=15truemm]{geometry}

\usepackage{amsmath,amssymb}
%\pagestyle{headings}
\pagestyle{empty}

%\usepackage{listings,url}
\renewcommand{\theenumi}{(\arabic{enumi})}

\usepackage{graphicx}

\usepackage{tikz}
\usetikzlibrary {arrows.meta}
\usepackage{wrapfig}	% required for `\wrapfigure' (yatex added)
\usepackage{bm}	% required for `\bm' (yatex added)

%% 像Im を定義
\newcommand{\Img}{\mathop{\mathrm{Im}}\nolimits}

\begin{document}



\textbf{問1}

集合 $A, B$
\quad
写像 $f: A \rightarrow B$

集合 $\Img f$を 写像$f$の像といい、次のように定める。
\begin{equation}
 \Img f = \{ b \in B \mid {}^\exists a\in A \; s.t. \; b=f(a)\}
\end{equation}

集合 $f(C)$を 写像$f$による集合$C$の像といい、次のように定める。
\begin{equation}
 f(C) = \{ b \in B \mid {}^\exists c\in C \; s.t. \; b=f(c)\}
\end{equation}

集合$A$の任意の部分集合$C$に対し、
$f(C)\subset \Img f$を証明せよ。

\dotfill

次を示せばよい。
\begin{equation}
 {}^\forall x\in f(C) \Rightarrow x \in \Img f
\end{equation}

$x\in f(C)$ より
$f(C)$の定義から
\begin{equation}
 {}^\exists c\in C \; s.t. \; x=f(c)
\end{equation}

$C\subset A$ より 上の式は次のように$C$を$A$に書き換えることが出来る。
\begin{equation}
 {}^\exists c\in A \; s.t. \; x=f(c)
\end{equation}

この式は $\Img f$の定義式であるので、
$x\in \Img f$である。
よって、
${}^\forall x\in f(C) \Rightarrow x \in \Img f$ であり、
$f(C)\subset \Img f$となる。


\hrulefill


\textbf{問2}

点$\bm{p}=\begin{pmatrix}a\\b\end{pmatrix} \in \mathbb{R}^2$と
正の実数$r$に対して開球$B(\bm{p},r)$を次のように定義する。
\begin{equation}
 B(\bm{p},r) = \left\{ \begin{pmatrix}x\\y\end{pmatrix} \in \mathbb{R}^2 \;
               \middle| \; (x-a)^2 + (y-b)^2 < r^2 \right\}
\end{equation}
この開球$B(\bm{p},r)$は 点$\bm{p}$を中心とする半径$r$の2次元開球という。

$\mathbb{R}^2$の部分集合$D$が開集合であるとは次を満たすときをいう。
\begin{equation}
 {}^\forall \bm{p}\in D, \; {}^\exists r>0
  \; s.t. \; B(\bm{p},r) \subset D
\end{equation}


次の様に定義した集合$D$が開集合であることを示せ。
\begin{equation}
 D = (a,b) \times (c,d)
            = \left\{ \begin{pmatrix}x\\y\end{pmatrix} \in \mathbb{R}^2 \;
               \middle| \; a<x<b, \, c<y<d \right\}
\end{equation}
$(a,b)$と$(c,d)$は開区間であり、$(a,b) \times (c,d)$は直積集合

\dotfill

開集合の定義に従い
${}^\forall \bm{p}\in D$ に対し 開球$B$が$B\subset D$を示せばよい。
つまり、開球の半径が存在することを示せばよい。


${}^\forall \bm{p}=\begin{pmatrix}p_x\\p_y\end{pmatrix}\in D$ とする。
$D$の定義から$\bm{p}$の各成分は開区間$(a,b),\, (c,d)$の要素であるので、
$p_x \in (a,b),\, p_y\in (c,d)$となる。

そこで次の4つの数$p_x-a, b-p_x, p_y-c, d-p_y$ のなかで最も小さい数を$r$とする。
\begin{equation}
 r = \min \{ p_x-a,\; b-p_x, \; p_y-c, \; d-p_y \}
\end{equation}

この$r$を用いると
\begin{equation}
 (p_x-r,p_x+r) \subset (a,b), \
 (p_y-r,p_y+r) \subset (c,d)
\end{equation}
となる。

これにより
$B(\bm{p},r)\subset D$となる。



\hrulefill

\textbf{問3}

${}^\forall \bm{x}, \bm{y} \in \mathbb{R}^n$と
直交行列$A$に対して
$\langle A\bm{x}, A\bm{y} \rangle = \langle \bm{x}, \bm{y} \rangle$
であることを示せ。

\dotfill

行列$A$に対し、転置行列を${}^t\!A$と書く。

ベクトル$\bm{x}\in\mathbb{R}^n$を
$\bm{x}=\begin{pmatrix}x_1\\x_2\\ \vdots \\ x_n\end{pmatrix}$
とする時、
内積は
$\langle \bm{x}, \bm{y} \rangle = {}^t\!\bm{x}\bm{y}$
となる。(右辺は行列の積)

直交行列とは転置行列と逆行列が同じ行列である。
$A{}^t\!A = {}^t\!AA = E$

\dotfill

\begin{align}
\langle A\bm{x}, A\bm{y} \rangle
 &= {}^t\!(A\bm{x})A\bm{y}
 = {}^t\!\bm{x}{}^t\!AA\bm{y}\\
 &= {}^t\!\bm{x}({}^t\!AA)\bm{y}
 = {}^t\!\bm{x}E\bm{y}\\
 &= {}^t\!\bm{x}\bm{y}
 = \langle \bm{x}, \bm{y} \rangle
\end{align}


\end{document}
