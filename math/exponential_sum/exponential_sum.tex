\documentclass[12pt,b5paper]{ltjsarticle}

%\usepackage[margin=15truemm, top=5truemm, bottom=5truemm]{geometry}
%\usepackage[margin=10truemm,left=15truemm]{geometry}
\usepackage[margin=10truemm]{geometry}

\usepackage{amsmath,amssymb}
%\pagestyle{headings}
\pagestyle{empty}

%\usepackage{listings,url}
%\renewcommand{\theenumi}{(\arabic{enumi})}

%\usepackage{graphicx}

%\usepackage{tikz}
%\usetikzlibrary {arrows.meta}
%\usepackage{wrapfig}
%\usepackage{bm}

% ルビを振る
%\usepackage{luatexja-ruby}	% required for `\ruby'

%% 核Ker 像Im Hom を定義
%\newcommand{\Img}{\mathop{\mathrm{Im}}\nolimits}
%\newcommand{\Ker}{\mathop{\mathrm{Ker}}\nolimits}
%\newcommand{\Hom}{\mathop{\mathrm{Hom}}\nolimits}

%\DeclareMathOperator{\Rot}{rot}
%\DeclareMathOperator{\Div}{div}
%\DeclareMathOperator{\Grad}{grad}
%\DeclareMathOperator{\arcsinh}{arcsinh}
%\DeclareMathOperator{\arccosh}{arccosh}
%\DeclareMathOperator{\arctanh}{arctanh}



%\usepackage{listings,url}
%
%\lstset{
%%プログラム言語(複数の言語に対応,C,C++も可)
%  language = Python,
%%  language = Lisp,
%%  language = C,
%  %背景色と透過度
%  %backgroundcolor={\color[gray]{.90}},
%  %枠外に行った時の自動改行
%  breaklines = true,
%  %自動改行後のインデント量(デフォルトでは20[pt])
%  breakindent = 10pt,
%  %標準の書体
%%  basicstyle = \ttfamily\scriptsize,
%  basicstyle = \ttfamily,
%  %コメントの書体
%%  commentstyle = {\itshape \color[cmyk]{1,0.4,1,0}},
%  %関数名等の色の設定
%  classoffset = 0,
%  %キーワード(int, ifなど)の書体
%%  keywordstyle = {\bfseries \color[cmyk]{0,1,0,0}},
%  %表示する文字の書体
%  %stringstyle = {\ttfamily \color[rgb]{0,0,1}},
%  %枠 "t"は上に線を記載, "T"は上に二重線を記載
%  %他オプション:leftline,topline,bottomline,lines,single,shadowbox
%  frame = TBrl,
%  %frameまでの間隔(行番号とプログラムの間)
%  framesep = 5pt,
%  %行番号の位置
%  numbers = left,
%  %行番号の間隔
%  stepnumber = 1,
%  %行番号の書体
%%  numberstyle = \tiny,
%  %タブの大きさ
%  tabsize = 4,
%  %キャプションの場所("tb"ならば上下両方に記載)
%  captionpos = t
%}



\begin{document}

\hrulefill

\begin{enumerate}
 \item
      \begin{enumerate}
       \item
            関数$f:\mathbb{N}\to\mathbb{C}$
            と
            $F:\mathbb{N}\to\mathbb{R}_{\geq 0}$
            に対して、
            $f(n) \ll F(n) \quad (n\in\mathbb{N})$
            が成り立つとき、
            実数$x\geq 1$に対して、
            次が成り立つことを示せ。
            \begin{equation}
             \sum_{n\leq x} f(n) \ll \sum_{n\leq x} F(n)
            \end{equation}

       \item
            関数$f_{i}:\mathbb{N}\to\mathbb{C}$
            と
            $F_{i}:\mathbb{N}\to\mathbb{R}_{\geq 0}$
            $(i=1,\dots,K)$
            に対して、
            条件
            $\lvert f_{k}(n) \rvert \leq 1, f_{k}(n) \ll F_{k}(n)$
            $(k\in\{1,\dots,K\}, \ n\in\mathbb{N})$
            (但し、ここで implicit constant は絶対定数)
            が成立すれば、
            次が成り立つことを示せ。
            \begin{equation}
             \prod_{k=1}^{K} (1 + f_{k}(n))
              = 1 + O_{k}\left( \sum_{k=1}^{K}F_{k}(n) \right)
            \end{equation}

      \end{enumerate}

 \item
      集合$X$上の関数$f,g:X\to\mathbb{R}_{\geq 0}$に対して
      関係$\asymp$を次のように定義する。
      \begin{equation}
       F(x) \asymp G(x) \quad (x\in X)
        \overset{\mathrm{def}}{\iff}
        F(x) \ll G(x) \ \text{かつ} \ G(x) \ll F(x) \quad (x\in X)
      \end{equation}
      \begin{enumerate}
       \item
            集合$X$上の関数$f,g:X\to\mathbb{R}_{\geq 0}$に対して
            次が成り立つことを示せ。
            \begin{equation}
             f(x)+g(x) \asymp \max(f(x),g(x)) \quad (x\in X)
            \end{equation}

       \item
            集合$X$上の関数$f,g:X\to\mathbb{R}_{\geq 0}$に対して
            次が成り立つことを示せ。
            \begin{equation}
             (f(x)+g(x))^{\frac{1}{2}}
              \asymp
              f(x)^{\frac{1}{2}}+g(x)^{\frac{1}{2}} \quad (x\in X)
            \end{equation}

      \end{enumerate}


 \item
      実数$x\in\mathbb{R}$に対して、
      $\exp(ix) = 1+O(\lvert x \rvert)$
      が成立することを示せ。


 \item
      関数$\Phi:[1,+\infty)\to\mathbb{C}$と
      $F:[1,+\infty)\to\mathbb{R}_{\geq 0}$に対して
      次の式が成立するとする。
      \begin{equation}
       \Phi(x) = 1+ O(F(x)) \quad (x\geq 1)
      \end{equation}
      
      このとき、次を示せ。
      \begin{enumerate}
       \item
            もし、$\lim_{x\to\infty}F(x)=0$だったなら、
            ある$x_{0}=x_{0}(\Phi)$が存在して
            次が成立する。
            \begin{equation}
             \frac{1}{\Phi(x)}= 1+ O(F(x)) \quad (x\geq x_{0})
            \end{equation}


       \item
            もし、
            $\frac{1}{\Phi(x)} \ll 1$ $(x\geq 1)$だったなら
            次が成立する。
            \begin{equation}
             \frac{1}{\Phi(x)}= 1+ O(F(x)) \quad (x\geq 1)
            \end{equation}
            但し、ここで implicit constant は
            $\frac{1}{\Phi(x)} \ll 1$ $(x\geq 1)$
            の implicit constant に依存する。


      \end{enumerate}


 \item
      実数$x\geq 1$に対して、
      次が成立することを示せ。
      \begin{equation}
       \sum_{n\leq x} \sum_{d \mid n}(-1)^{d}
        = (-\log{2})\cdot x + O(x^{\frac{1}{2}})
      \end{equation}
      (Hint: hyperbola method を用いる)



 \item
      数論的関数
      $\chi_{4} : \mathbb{Z}\to\mathbb{R}
      ,\ 
      r:\mathbb{N}\to\mathbb{R}$
      を
      次のように定める。
      \begin{equation}
       \chi_{4}(n) =
        \begin{cases}
         +1 & ( n\equiv 1 \pmod{4})\\
         0 & ( n\equiv 0 \pmod{2})\\
         -1 & ( n\equiv 3 \pmod{4})
        \end{cases}
        ,\quad
        r(n)=4\sum_{d\mid n} \chi_{4}(d)
      \end{equation}

      このとき、
      $x\geq 1$に対して、次が成り立つことを示せ。
      \begin{equation}
       \sum_{n\leq x} r(n) = \pi x + O(x^{\frac{1}{2}})
      \end{equation}

      (Hint: hyperbola method を用いる)

      (補足 :
      実は、$n\in\mathbb{N}$に対して、
      $r(n) = \# \{ (u,v) \in\mathbb{Z}^2 \mid u^2+v^2=n \}$
      となることが知られている。
      格子点の数え上げと上記の結果を比較してみると良い
      )

\end{enumerate}

\hrulefill

\end{document}
