\documentclass[12pt,b5paper]{ltjsarticle}

%\usepackage[margin=15truemm, top=5truemm, bottom=5truemm]{geometry}
%\usepackage[margin=10truemm,left=15truemm]{geometry}
\usepackage[margin=10truemm]{geometry}

\usepackage{amsmath,amssymb}
%\pagestyle{headings}
\pagestyle{empty}

%\usepackage{listings,url}
%\renewcommand{\theenumi}{(\arabic{enumi})}

\usepackage{graphicx}

%\usepackage{tikz}
%\usetikzlibrary {arrows.meta}
%\usepackage{wrapfig}
%\usepackage{bm}

% ルビを振る
%\usepackage{luatexja-ruby}	% required for `\ruby'

%% 核Ker 像Im Hom を定義
%\newcommand{\Img}{\mathop{\mathrm{Im}}\nolimits}
%\newcommand{\Ker}{\mathop{\mathrm{Ker}}\nolimits}
%\newcommand{\Hom}{\mathop{\mathrm{Hom}}\nolimits}

%\DeclareMathOperator{\Rot}{rot}
%\DeclareMathOperator{\Div}{div}
%\DeclareMathOperator{\Grad}{grad}
%\DeclareMathOperator{\arcsinh}{arcsinh}
%\DeclareMathOperator{\arccosh}{arccosh}
%\DeclareMathOperator{\arctanh}{arctanh}



%\usepackage{listings,url}
%
%\lstset{
%%プログラム言語(複数の言語に対応,C,C++も可)
%  language = Python,
%%  language = Lisp,
%%  language = C,
%  %背景色と透過度
%  %backgroundcolor={\color[gray]{.90}},
%  %枠外に行った時の自動改行
%  breaklines = true,
%  %自動改行後のインデント量(デフォルトでは20[pt])
%  breakindent = 10pt,
%  %標準の書体
%%  basicstyle = \ttfamily\scriptsize,
%  basicstyle = \ttfamily,
%  %コメントの書体
%%  commentstyle = {\itshape \color[cmyk]{1,0.4,1,0}},
%  %関数名等の色の設定
%  classoffset = 0,
%  %キーワード(int, ifなど)の書体
%%  keywordstyle = {\bfseries \color[cmyk]{0,1,0,0}},
%  %表示する文字の書体
%  %stringstyle = {\ttfamily \color[rgb]{0,0,1}},
%  %枠 "t"は上に線を記載, "T"は上に二重線を記載
%  %他オプション:leftline,topline,bottomline,lines,single,shadowbox
%  frame = TBrl,
%  %frameまでの間隔(行番号とプログラムの間)
%  framesep = 5pt,
%  %行番号の位置
%  numbers = left,
%  %行番号の間隔
%  stepnumber = 1,
%  %行番号の書体
%%  numberstyle = \tiny,
%  %タブの大きさ
%  tabsize = 4,
%  %キャプションの場所("tb"ならば上下両方に記載)
%  captionpos = t
%}



\begin{document}


\hrulefill

\textbf{定義}
\quad
集合列
$\displaystyle \{ A_{n} \}_{n=1}^{\infty}$
に対して 次のように極限集合を定義する。
\begin{align}
 \text{上極限集合} &
 \varlimsup_{n\to\infty} A_{n}
 = \bigcap _{n=1}^{\infty} \bigcup_{k=n}^{\infty} A_{k}
 %
 &
 %
 \text{下極限集合} &
  \varliminf_{n\to\infty} A_{n}
 = \bigcup _{n=1}^{\infty} \bigcap_{k=n}^{\infty} A_{k}
\end{align}



\hrulefill

$\mathbb{R}^{2}$の
部分集合の
列$\{ A_{n} \}_{n=1}^{\infty}$で、
\begin{equation}
 \varliminf_{n\to\infty} A_{n}
  = [0,1/2] \times [0,1/2]
  \quad \text{かつ} \quad
 \varlimsup_{n\to\infty} A_{n}
  = [0,1] \times [0,1]
\end{equation}
という条件を満たす例を
証明付きで一つ挙げよ。

\dotfill

$\mathbb{R}^2$の部分集合$A_{k}$を次のように定める。
\begin{equation}
 A_k
  = \left[0,\frac{3+(-1)^k}{4}\right]
  \times \left[0,\frac{3+(-1)^k}{4}\right]
\end{equation}

これは$k$が奇数の時、$A_{k}= [0,1/2] \times [0,1/2]$であり、
$k$が偶数の時、$A_{k}= [0,1] \times [0,1]$である集合である。

\dotfill
$\displaystyle  \varliminf_{n\to\infty} A_{n} = [0,1/2] \times [0,1/2] $
\dotfill

$\alpha \in [0,1/2] \times [0,1/2]$とする。
全ての$n\in\mathbb{N}$に対して、
$[0,1/2] \times [0,1/2] \subset A_n$
であるので、
$\displaystyle \alpha\in \varliminf_{n\to\infty} A_{n}$
となる。
つまり、
$\displaystyle  \varliminf_{n\to\infty} A_{n} \supset [0,1/2] \times [0,1/2] $
である。

逆に、
$\displaystyle \alpha\in \varliminf_{n\to\infty} A_{n}$
とする。
$\displaystyle \varliminf_{n\to\infty} A_{n}
 = \bigcup _{n=1}^{\infty} \bigcap_{k=n}^{\infty} A_{k}$
であるので、
ある自然数$n\in\mathbb{N}$が存在し、
$\displaystyle \alpha\in \bigcap_{k=n}^{\infty} A_{k}$
である。
$\displaystyle \bigcap_{k=n}^{\infty} A_{k}$は次のような集合である。
\begin{equation}
 \bigcap_{k=n}^{\infty} A_{k}
  = \cdots \cap [0,1/2] \times [0,1/2] \cap [0,1] \times [0,1]
  \cap [0,1/2] \times [0,1/2] \cap \cdots
\end{equation}
つまり、$\displaystyle \bigcap_{k=n}^{\infty} A_{k} =  [0,1/2] \times [0,1/2]$
であるので、
$\displaystyle  \varliminf_{n\to\infty} A_{n} \subset [0,1/2] \times [0,1/2] $
である。


\dotfill
$\displaystyle  \varlimsup_{n\to\infty} A_{n} = [0,1] \times [0,1]$
\dotfill

$\alpha \in [0,1] \times [0,1]$とする。
$n$が偶数の時、$[0,1] \times [0,1] \subset A_{n}$である。
つまり、
すべての自然数$n\in\mathbb{N}$に対して
$\alpha\in A_{2n}$であるので、
$\displaystyle \alpha \in \bigcap _{n=1}^{\infty} A_{2n}$
である。
$\displaystyle A_{2n} \subset \bigcup_{k=n}^{\infty} A_{k}$
であるため、
\begin{equation}
 \alpha \in \bigcap _{n=1}^{\infty} A_{2n}
  \subset \bigcap _{n=1}^{\infty} \bigcup_{k=n}^{\infty} A_{k}
  = \varlimsup_{n\to\infty} A_{n}
\end{equation}
となる。

逆に、
$\displaystyle \alpha \in \varlimsup_{n\to\infty} A_{n}$とする。
$A_{k}$は$[0,1/2] \times [0,1/2]$または$[0,1] \times [0,1]$であり、
交互に現れるので、
$\displaystyle \bigcup_{k=n}^{\infty} A_{k} \subset [0,1] \times [0,1]$
である。
\begin{equation}
  \varlimsup_{n\to\infty} A_{n}
  = \bigcap _{n=1}^{\infty} \bigcup_{k=n}^{\infty} A_{k}
  \subset \bigcap _{n=1}^{\infty} [0,1] \times [0,1]
  = [0,1] \times [0,1]
\end{equation}
よって、$\alpha \in [0,1] \times [0,1]$である。



\hrulefill

$X$を集合とし、
$\{ A_{n} \}_{n=1}^{\infty}$を
$X$の部分集合の列とする。
\begin{enumerate}
 \item
      $\displaystyle \varlimsup_{n\to\infty}A_{n}
      = \{ x\in X \mid \text{無限個の } n \text{ に対して } x\in A_{n} \}$

 \item
      $\displaystyle \varliminf_{n\to\infty}A_{n}
      = \{ x\in X \mid \text{有限個を除く } n \text{ に対して } x\in A_{n} \}$

 \item
      $\displaystyle \varliminf_{n\to\infty}A_{n}
      \subset \varlimsup_{n\to\infty}A_{n}$

\end{enumerate}


\dotfill

\begin{enumerate}
 \item
      $\displaystyle \varlimsup_{n\to\infty}A_{n}
      = \{ x\in X \mid \text{無限個の } n \text{ に対して } x\in A_{n} \}$

      \dotfill \textbf{Proof} \dotfill

      集合$B_{n}$を
      $\displaystyle B_{n} = \bigcup_{k=n}^{\infty} A_{k}$
      とする。

      \begin{equation}
       \varlimsup_{n\to\infty}A_{n}
        = \bigcap _{n=1}^{\infty} \bigcup_{k=n}^{\infty} A_{k}
        = \bigcap _{n=1}^{\infty} B_{n}
      \end{equation}
      であるので、
      任意の元$\displaystyle \alpha \in \varlimsup_{n\to\infty}A_{n}$
      について調べる。
      \begin{align}
       \alpha \in \varlimsup_{n\to\infty}A_{n}
       \Leftrightarrow\ &\ \alpha \in \bigcap _{n=1}^{\infty} B_{n} &
       \Leftrightarrow\ &\ {}^{\forall}n \in \mathbb{N},\ \alpha \in B_{n}\\
       \Leftrightarrow\ &\ {}^{\forall}n \in \mathbb{N},\ \alpha \in \bigcup_{k=n}^{\infty} A_{k} &
       \Leftrightarrow\ &\ {}^{\forall}n \in \mathbb{N},\ {}^{\exists}k_n\geq n \ s.t. \ \alpha \in A_{k_n}
      \end{align}

      ここから
      自然数の部分集合$\{ k_n \} \subset \mathbb{N}$が存在する。
      集合$\{ k_n \}$の濃度は自然数と一致するので、
      無限個の$A_{k_n}$に対して$\alpha\in A_{k_n}$となる。

      \dotfill
 \item
      $\displaystyle \varliminf_{n\to\infty}A_{n}
      = \{ x\in X \mid \text{有限個を除く } n \text{ に対して } x\in A_{n} \}$

      \dotfill \textbf{Proof} \dotfill

      集合$B_{n}$を
      $\displaystyle B_{n} = \bigcap_{k=n}^{\infty} A_{k}$
      とする。

      \begin{equation}
       \varliminf_{n\to\infty}A_{n}
        = \bigcup _{n=1}^{\infty} \bigcap_{k=n}^{\infty} A_{k}
        = \bigcup _{n=1}^{\infty} B_{n}
      \end{equation}
      であるので、
      任意の元$\displaystyle \alpha \in \varliminf_{n\to\infty}A_{n}$
      について調べる。
      \begin{align}
       \alpha \in \varliminf_{n\to\infty}A_{n}
       \Leftrightarrow\ &\ \alpha \in \bigcup _{n=1}^{\infty} B_{n} &
       \Leftrightarrow\ &\ {}^{\exists}n \in \mathbb{N},\ s.t. \ \alpha \in B_{n}\\
       \Leftrightarrow\ &\ {}^{\exists}n \in \mathbb{N},\ s.t. \ \alpha \in \bigcap_{k=n}^{\infty} A_{k} &
       \Leftrightarrow\ &\ {}^{\exists}n \in \mathbb{N},\ s.t. \ k\geq n, \ \alpha \in A_{k}
      \end{align}

      これより、
      ある自然数$n\in\mathbb{N}$が存在し、
      $n$以上の自然数$k$に対し、
      $\alpha \in A_{k}$である。
      つまり、最初のいくつかの有限個を除いて残り全て含まれることになる。

      \dotfill
 \item
      $\displaystyle \varliminf_{n\to\infty}A_{n}
      \subset \varlimsup_{n\to\infty}A_{n}$

      \dotfill \textbf{Proof} \dotfill

      上の2つの内容より
      $\displaystyle \varliminf_{n\to\infty}A_{n}$
      は
      $\displaystyle \varlimsup_{n\to\infty}A_{n}$
      より条件が厳しい。

      $\displaystyle \varliminf_{n\to\infty}A_{n}$
      はある数以上の全ての$A_{n}$に含まれないといけないが、
      $\displaystyle \varlimsup_{n\to\infty}A_{n}$
      は$n$は連続である必要はなく、
      飛び飛びの数字で構わない。

      例えば、偶数番目の$A_n$にのみ含まれる元$\beta$は
      $\displaystyle \beta \in \varlimsup_{n\to\infty}A_{n}$
      であるが、
      $\displaystyle \beta \not\in \varliminf_{n\to\infty}A_{n}$
      である。

      
      \dotfill
\end{enumerate}

\hrulefill

\end{document}
