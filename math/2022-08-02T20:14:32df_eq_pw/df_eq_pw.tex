\documentclass[12pt,b5paper]{ltjsarticle}

%\usepackage[margin=15truemm, top=5truemm, bottom=5truemm]{geometry}
\usepackage[margin=10truemm]{geometry}

\usepackage{amsmath,amssymb}
%\pagestyle{headings}
\pagestyle{empty}

%\usepackage{listings,url}
%\renewcommand{\theenumi}{(\arabic{enumi})}

\usepackage{graphicx}

\usepackage{tikz}
\usetikzlibrary {arrows.meta}
\usepackage{wrapfig}	% required for `\wrapfigure' (yatex added)
\usepackage{bm}	% required for `\bm' (yatex added)

% ルビを振る
%\usepackage{luatexja-ruby}	% required for `\ruby'

%% 核Ker 像Im Hom を定義
%\newcommand{\Img}{\mathop{\mathrm{Im}}\nolimits}
%\newcommand{\Ker}{\mathop{\mathrm{Ker}}\nolimits}
%\newcommand{\Hom}{\mathop{\mathrm{Hom}}\nolimits}

\DeclareMathOperator{\Rot}{rot}
\DeclareMathOperator{\Div}{div}
\DeclareMathOperator{\Grad}{grad}
%\DeclareMathOperator{\arcsinh}{arcsinh}



\begin{document}

次の微分方程式の原点周りの級数解と、
その収束半径を求めよ。
\begin{enumerate}
 \item $y^{\prime\prime} + xy^{\prime} +y =0$
 \item $(x-1)y^{\prime\prime} - (x+1) y^{\prime} + 2y =0$
\end{enumerate}

\dotfill

級数解を次のようにおく。
\begin{equation}
 y=\sum_{n=0}^{\infty}a_n x^n
\end{equation}
これを微分する。
\begin{equation}
 y^{\prime}=\sum_{n=1}^{\infty}na_n x^{n-1}
  ,\qquad
 y^{\prime\prime}=\sum_{n=2}^{\infty}n(n-1)a_n x^{n-2}
\end{equation}

$y,y^{\prime},y^{\prime\prime}$を微分方程式に当てはめ
係数$a_n$を求める。

%%%%%
\dotfill
\textbf{$y^{\prime\prime} + xy^{\prime} +y =0$の場合}
\dotfill
%%%%%


級数解を代入し同じ次数の項をまとめると次のようになる。
\begin{gather}
 \sum_{n=2}^{\infty}n(n-1)a_n x^{n-2}
 +
 x\sum_{n=1}^{\infty}na_n x^{n-1}
 +
 \sum_{n=0}^{\infty}a_n x^n
 =0\\
 2a_2+a_0
 +
 \sum_{n=1}^{\infty}( (n+2)(n+1)a_{n+2} + (n+1)a_n  )x^{n}
 =0
\end{gather}

任意の$x$について成立する為、この式は恒等的に$0$である。
よって、次の式が得られる。
\begin{align}
 & 2a_2+a_0 = 0
 &
 &(n+2)(n+1)a_{n+2} + (n+1)a_n = 0\\
 & a_2 = -\frac{1}{2}a_0
 &
 &a_{n+2} = -\frac{1}{n+2}a_n
 \label{recurr}
\end{align}

$a_n$は次のような形で得られる。
\begin{equation}
 a_{n} = -\frac{1}{n}a_{n-2} = \left(-\frac{1}{n}\right)\left(-\frac{1}{n-2}\right)a_{n-4}
  = \left(-\frac{1}{n}\right)\left(-\frac{1}{n-2}\right)\left(-\frac{1}{n-4}\right)a_{n-6}
\end{equation}

漸化式(\ref{recurr})が一つ飛ばしているので、
$n$が偶数の場合と奇数の場合で分けて考える。

$n=2k$の場合
\begin{equation}
 a_{2k} = \left(-\frac{1}{2k}\right)\left(-\frac{1}{2(k-1)}\right)\cdots\left(-\frac{1}{2}\right)a_{0}
  =\frac{(-1)^k}{2^k k!}a_0
\end{equation}
となる。

$x=0$における$e^x$のTaylor展開は
$e^x = \sum_{k=0}^{\infty}\frac{x^k}{k!}$
であるので、
\begin{equation}
 e^{-\frac{x^2}{2}}
  = \sum_{k=0}^{\infty}\frac{\left(-\frac{x^2}{2}\right)^k}{k!}
  = \sum_{k=0}^{\infty}\frac{(-1)^k x^{2k}}{2^kk!}
\end{equation}
となる。
つまり、$y$の偶数番目の総和は
\begin{equation}
 \sum_{k=0}^{\infty}a_{2k}x^{2k}
  = \sum_{k=0}^{\infty}\frac{(-1)^k}{2^k k!}a_0x^{2k}
  = a_0 \sum_{k=0}^{\infty}\frac{(-1)^kx^{2k}}{2^k k!}
  = a_0 e^{-\frac{x^2}{2}}
\end{equation}
となる。
この級数の収束半径は
\begin{equation}
 \lim_{k\to\infty} \left. \left\lvert \frac{(-1)^k}{2^k k!}a_0\right\rvert \middle/ \left\lvert \frac{(-1)^{k+1}}{2^{k+1} (k+1)!}a_0\right\rvert \right.
  = \lim_{k\to\infty} 2(k+1) = \infty
\end{equation}
であるので、実数全体で収束する。



$n=2k+1$の場合
\begin{equation}
 a_{2k+1} = \left(-\frac{1}{2k+1}\right)\left(-\frac{1}{2(k-1)+1}\right)\cdots\left(-\frac{1}{2+1}\right)a_{1}
  =\frac{(-1)^k2^k k!}{(2k+1)!}a_1
\end{equation}
である。
収束半径は同じように考えて
\begin{equation}
 \lim_{k\to\infty} \left.
                    \left\lvert \frac{(-1)^k2^k k!}{(2k+1)!}a_1 \right\rvert
                    \middle/
                    \left\lvert \frac{(-1)^{k+1}2^{k+1} (k+1)!}{(2(k+1)+1)!}a_1 \right\rvert
                   \right.
  = \lim_{k\to\infty} \frac{(2k+3)(2k+2)}{2(k+1)} = \infty
\end{equation}
となるので、実数全体となる。

この為、級数解$y$は次のようになる。
\begin{equation}
 y=
  a_0 e^{-\frac{x^2}{2}}
  +
  a_1\sum_{k=0}^{\infty} \left( \frac{(-1)^k2^k k!}{(2k+1)!}x^{2k+1}  \right)
\end{equation}

収束半径は$\infty$(実数全体で収束)となる。



%%%%%
\dotfill
\textbf{$(x-1)y^{\prime\prime} - (x+1) y^{\prime} + 2y =0$の場合}
\dotfill
%%%%%

%  y=\sum_{n=0}^{\infty}a_n x^n

%  y^{\prime}=\sum_{n=1}^{\infty}na_n x^{n-1}

%  y^{\prime\prime}=\sum_{n=2}^{\infty}n(n-1)a_n x^{n-2}



$y,y^{\prime},y^{\prime\prime}$を微分方程式に代入する。
\begin{equation}
 (x-1)\sum_{n=2}^{\infty}n(n-1)a_n x^{n-2}
  - (x+1) \sum_{n=1}^{\infty}na_n x^{n-1}
  + 2 \sum_{n=0}^{\infty}a_n x^n =0
\end{equation}

展開をし、$x$の次数が揃うように$n$の値を振り直す。
\begin{gather}
 \sum_{n=2}^{\infty}n(n-1)a_n x^{n-1}
  -\sum_{n=2}^{\infty}n(n-1)a_n x^{n-2}
  - \sum_{n=1}^{\infty}na_n x^{n}
  - \sum_{n=1}^{\infty}na_n x^{n-1}
  +2 \sum_{n=0}^{\infty}a_n x^n =0
\end{gather}
\begin{equation}
 \begin{split}
 \sum_{n=1}^{\infty}(n+1)na_{n+1} x^{n}
  &-\sum_{n=0}^{\infty}(n+2)(n+1)a_{n+2} x^{n}\\
  &- \sum_{n=1}^{\infty}na_n x^{n}
  - \sum_{n=0}^{\infty}(n+1)a_{n+1} x^{n}
  +2 \sum_{n=0}^{\infty}a_n x^n =0
 \end{split}
\end{equation}

定数項から順に同類項をまとめる。
\begin{equation}
 \begin{split}
  & ( -2a_2 -a_1 +2a_0 )\\
  & + \sum_{n=1}^{\infty}(
  (n+1)na_{n+1}
  -(n+2)(n+1)a_{n+2}
  -na_{n}
  -(n+1)a_{n+1}
  +2a_n
  )x^n
  =0
 \end{split}
\end{equation}

これより次の式が得られる。
\begin{gather}
 -2a_2 -a_1 +2a_0 = 0\\
  -(n+2)(n+1)a_{n+2}
  +( n^2-1 )a_{n+1}
  + (-n+2)a_n
 =0
 \quad (n\geq 1)
\end{gather}

\begin{gather}
 a_2 = -\frac{1}{2}a_1 + a_0\\
 a_{n+2}
  = \frac{1}{(n+2)(n+1)}(
   (n+1)(n-1)a_{n+1} +(-n+2)a_n
 )
 \label{recurr2}
\end{gather}

この式より係数$a_i$を計算してみる。
\begin{align}
 a_3 =& \frac{1}{6}a_1
  ,&
 a_4 =& \frac{1}{12}\times 3a_3=\frac{1}{24}a_1\\
 a_5 =& \frac{1}{20}(8a_4-a_3)=\frac{1}{120}a_1
 ,&
 a_6 =& \frac{1}{30}(15a_5-2a_4) =\frac{1}{720}a_1
\end{align}

これにより一般項は$a_k=\frac{1}{k}a_{k-1}\ (k\geq 4)$を
満たしていると予測できる。
その為、数学的帰納法で証明を行う。

まず、$k=4$の時、式(\ref{recurr2})に代入することで、
$a_4=\frac{1}{4}a_3$であることがわかる。

そこで、$a_{n+1}$までは全て成立していると仮定し、
$a_{n+2}$について考える。
\begin{align}
 a_{n+2}
  =& \frac{1}{(n+2)(n+1)}(
   (n+1)(n-1)a_{n+1} +(-n+2)a_n
 )\\
 =& \frac{1}{(n+2)(n+1)}(
   (n+1)(n-1)\frac{1}{n+1}a_n +(-n+2)a_n
 )\\
 =& \frac{1}{(n+2)(n+1)}( (n-1)+(-n+2) )a_n\\
 =& \frac{1}{(n+2)(n+1)}a_n
 = \frac{1}{n+2}a_{n+1}
\end{align}

よって、
$a_k=\frac{1}{k}a_{k-1}\ (k\geq 4)$
が成立することがわかる。
上の計算で$a_3=\frac{1}{6}a_1$であることが求まるので
$n\geq 4$において$a_n=\frac{1}{n!}a_1$ということがわかる。

これより、
級数解$y$は次のような式となる。
\begin{equation}
 y=a_0+a_1x+ \left(-\frac{1}{2}a_1+a_0\right)x^2 + a_1\sum_{k=3}^{\infty}\frac{1}{k!}x^k
\end{equation}

$e^x = \sum_{k=0}^{\infty}\frac{x^k}{k!}$であるので、
級数解をこれに合わせて変形する。
\begin{equation}
 \begin{split}
 y = a_0 + a_1x + \left(-\frac{1}{2}a_1+a_0 \right)x^2
  -& a_1\left( \frac{1}{0!}x^0 + \frac{1}{1!}x^1 + \frac{1}{2!}x^2 \right)\\
  +& a_1\left( \frac{1}{0!}x^0 + \frac{1}{1!}x^1 + \frac{1}{2!}x^2 \right)
  + a_1\sum_{k=3}^{\infty}\frac{1}{k!}x^k
 \end{split}
\end{equation}

これにより級数解は
\begin{equation}
 y = (a_0-a_1) + (a_0-a_1)x^2 + a_1 e^x
\end{equation}
となり、
$y$は実数全体で収束するので
収束半径は$+\infty$となる。



\end{document}
