\documentclass[12pt,b5paper]{ltjsarticle}

\usepackage{amssymb}
\pagestyle{empty}

\usepackage{amsmath}	% required for `\align' (yatex added)
\begin{document}

$\displaystyle \frac{0}{0}$
と
$\displaystyle \frac{1}{0}$
について

\quad

割り算は掛け算から出来ている。
次の式のように$\times b$を右辺から消して$\div b$を付け加えている。
\begin{align}
 a \times b = & c \phantom{\div b}\\
 a \hspace*{20pt} = & c \div b
\end{align}

$b=0$であれば$c=0$であるので次のようになる。
\begin{align}
 x \times 0 = & 0 \phantom{\div b}\\
 x \hspace*{20pt} = & 0 \div 0
\end{align}
この場合、$x$はどんな値でもいいので、
$0\div0$は値が一つに定まらない(不定)となる。

$1\div0$は同じように考えると次の式になる
\begin{align}
 x \times 0 = & 1 \phantom{\div b}\\
 x \hspace*{20pt} = & 1 \div 0
\end{align}
$x\times0$は$1$になることはないので
$1\div0$は値を持たない(不能)となる。

$0\div0$は様々な値になり得るが、
$1\div0$はどんな値にもなり得ない。
この為、
$0$で割るのは問題があるが
分けて考える必要がある。





\end{document}
