\documentclass[12pt,b5paper]{ltjsarticle}

%\usepackage[margin=15truemm, top=5truemm, bottom=5truemm]{geometry}
\usepackage[margin=10truemm]{geometry}

\usepackage{amsmath,amssymb}
%\pagestyle{headings}
\pagestyle{empty}

%\usepackage{listings,url}
%\renewcommand{\theenumi}{(\arabic{enumi})}

%\usepackage{graphicx}

%\usepackage{tikz}
%\usetikzlibrary {arrows.meta}
%\usepackage{wrapfig}	% required for `\wrapfigure' (yatex added)
%\usepackage{bm}	% required for `\bm' (yatex added)

% ルビを振る
%\usepackage{luatexja-ruby}	% required for `\ruby'

%% 核Ker 像Im Hom を定義
%\newcommand{\Img}{\mathop{\mathrm{Im}}\nolimits}
%\newcommand{\Ker}{\mathop{\mathrm{Ker}}\nolimits}
%\newcommand{\Hom}{\mathop{\mathrm{Hom}}\nolimits}

%\DeclareMathOperator{\Rot}{rot}
%\DeclareMathOperator{\Div}{div}
%\DeclareMathOperator{\Grad}{grad}
%\DeclareMathOperator{\arcsinh}{arcsinh}
%\DeclareMathOperator{\arccosh}{arccosh}
%\DeclareMathOperator{\arctanh}{arctanh}



\begin{document}

\textbf{開集合}

集合$A$において、
任意の$a\in A$に対し
次を満たす $\varepsilon\in >0$ が存在する時
$A$を開集合という。

$d(a,b)<\varepsilon$となる全ての$b$が
$b\in A$である。
なお、$d(a,b)$は二点間の距離を表す。


つまり、$a\in A$の周りの点は必ず$A$に含まれる時に
$A$を開集合という。


\dotfill

区間
$(a,b) \subset \mathbb{R}$
について

$p_0 \in (a,b)$とするとき、
$\varepsilon$を次のように定める。
\begin{equation}
 \varepsilon = \min \left\{ \frac{p_0-a}{2}, \frac{b-p_0}{2} \right\}
\end{equation}

これにより 点$p_0$から距離$\varepsilon$未満の全ての点が
区間$(a,b)$に含まれる。

$p_0$は区間内のどの点であっても上記を満たす為、
$(a,b)$は開集合となる。


区間$(a,b]$は多くの点が開集合の定義を満たすが、
端の点$b\in (a,b]$はどれほど$\varepsilon$を小さくとっても
$b$より大きい点は区間$(a,b]$に含まれないので
開集合ではない。


\end{document}

