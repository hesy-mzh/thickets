\documentclass[12pt,b5paper]{ltjsarticle}

%\usepackage[margin=15truemm, top=5truemm, bottom=5truemm]{geometry}
%\usepackage[margin=10truemm,left=15truemm]{geometry}
\usepackage[margin=10truemm]{geometry}

\usepackage{amsmath,amssymb}
%\pagestyle{headings}
\pagestyle{empty}

%\usepackage{listings,url}
%\renewcommand{\theenumi}{(\arabic{enumi})}

\usepackage{graphicx}

%\usepackage{tikz}
%\usetikzlibrary {arrows.meta}
%\usepackage{wrapfig}	% required for `\wrapfigure' (yatex added)
%\usepackage{bm}	% required for `\bm' (yatex added)

% ルビを振る
%\usepackage{luatexja-ruby}	% required for `\ruby'

%% 核Ker 像Im Hom を定義
%\newcommand{\Img}{\mathop{\mathrm{Im}}\nolimits}
%\newcommand{\Ker}{\mathop{\mathrm{Ker}}\nolimits}
%\newcommand{\Hom}{\mathop{\mathrm{Hom}}\nolimits}

%\DeclareMathOperator{\Rot}{rot}
%\DeclareMathOperator{\Div}{div}
%\DeclareMathOperator{\Grad}{grad}
%\DeclareMathOperator{\arcsinh}{arcsinh}
%\DeclareMathOperator{\arccosh}{arccosh}
%\DeclareMathOperator{\arctanh}{arctanh}



%\usepackage{listings,url}
%
%\lstset{
%%プログラム言語(複数の言語に対応,C,C++も可)
%  language = Python,
%%  language = Lisp,
%%  language = C,
%  %背景色と透過度
%  %backgroundcolor={\color[gray]{.90}},
%  %枠外に行った時の自動改行
%  breaklines = true,
%  %自動改行後のインデント量(デフォルトでは20[pt])
%  breakindent = 10pt,
%  %標準の書体
%%  basicstyle = \ttfamily\scriptsize,
%  basicstyle = \ttfamily,
%  %コメントの書体
%%  commentstyle = {\itshape \color[cmyk]{1,0.4,1,0}},
%  %関数名等の色の設定
%  classoffset = 0,
%  %キーワード(int, ifなど)の書体
%%  keywordstyle = {\bfseries \color[cmyk]{0,1,0,0}},
%  %表示する文字の書体
%  %stringstyle = {\ttfamily \color[rgb]{0,0,1}},
%  %枠 "t"は上に線を記載, "T"は上に二重線を記載
%  %他オプション:leftline,topline,bottomline,lines,single,shadowbox
%  frame = TBrl,
%  %frameまでの間隔(行番号とプログラムの間)
%  framesep = 5pt,
%  %行番号の位置
%  numbers = left,
%  %行番号の間隔
%  stepnumber = 1,
%  %行番号の書体
%%  numberstyle = \tiny,
%  %タブの大きさ
%  tabsize = 4,
%  %キャプションの場所("tb"ならば上下両方に記載)
%  captionpos = t
%}



\begin{document}

\hrulefill

\begin{enumerate}
 \item
      $\sigma$を$0$でない定数とし、
      実数値弱定常過程
      $\varepsilon = \{ \varepsilon_{t} ; t\in\mathbb{Z}\}$
      はホワイトノイズ$WN(0,\sigma^2)$であるとする。
      この時、
      全ての$t,h \in\mathbb{Z}$に対して次が成り立つ。
      \begin{equation}
       E [ \varepsilon_{t+h}\varepsilon_{t} ]=
        \sigma^2\frac{\sin{\pi h}}{\pi h}
      \end{equation}

      \dotfill

      共分散は次の式より得られる。
      \begin{equation}
       Cov( \varepsilon_{t+h}, \varepsilon_{t} )
        = E[ \varepsilon_{t+h} \varepsilon_{t} ]
        - E[ \varepsilon_{t+h}] E[ \varepsilon_{t} ]
      \end{equation}

      $\varepsilon_{i}$はホワイトノイズであるので期待値は0、
      分散は$\sigma^2$である。
      よって、
      $Cov( \varepsilon_{t+h}, \varepsilon_{t} )
      =E[ \varepsilon_{t+h} \varepsilon_{t} ]$
      である。

      任意の$t\in\mathbb{Z}$に対し、
      $h\ne 0$である$h\in\mathbb{Z}$について
      $Cov( \varepsilon_{t+h}, \varepsilon_{t} ) =E[ \varepsilon_{t+h} \varepsilon_{t} ]=0$となる。
      また、$V[\epsilon_{t}]=\sigma^2$となるので、
      $E[ \varepsilon_{t} \varepsilon_{t} ]=\sigma^2$である。

      \begin{gather}
       E[ \varepsilon_{t} \varepsilon_{t} ]=\sigma^2
        =\sigma^2\cdot \lim_{h\rightarrow 0}\frac{\sin{(\pi h)}}{\pi h}
        =\sigma^2\frac{\sin{(\pi\cdot 0)}}{\pi\cdot 0}\\
       E[ \varepsilon_{t+h} \varepsilon_{t} ]=\sigma^2
        =0
        =\sigma^2\frac{ 0}{\pi h}
        =\sigma^2\frac{ \sin{\pi h}}{\pi h}
      \end{gather}

      上記のように$h$の値について式が成り立つので
      まとめると次を得る。
      \begin{equation}
       E [ \varepsilon_{t+h}\varepsilon_{t} ]=
        \sigma^2\frac{\sin{\pi h}}{\pi h}
        \qquad
        (t,h\in\mathbb{Z})
      \end{equation}


      \hrulefill

 \item
      
\end{enumerate}





\hrulefill

\end{document}
