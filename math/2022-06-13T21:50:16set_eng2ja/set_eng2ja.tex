\documentclass[a5paper]{ltjarticle}

\usepackage[margin=15truemm]{geometry}
\usepackage{amssymb}

\begin{document}

\hrulefill

\textbf{Theorem 1.3.5}
A set $S$ is closed if and only if no point of $S^c$ is a limit point of $S$.

\dotfill

\textbf{定理 1.3.5}
集合$S$が閉であるとは
$S$の極限点が補集合$S^c$に存在しない場合に限る。

\dotfill

「if and only if」とは
前後が必要十分である時に使う。
この場合、閉集合であることと補集合に極限点がないことを指す。


極限点(きょくげんてん、英: limit point)とは
集積点(しゅうせきてん、英: accumulation point)の事。

\hrulefill


\textbf{Proof}
Suppose that $S$ is closed and $x_0\in S^c$.
Since $S^c$ is open, there is a neighborhood of $x_0$ that is contained in $S^c$
and therefore contains no points of $S$.
Hence, $x_0$ connot be a limit point of $S$.
For the converse, if no point of $S^c$ is a limit point of $S$
then every point in $S^c$ must have a neighborhood contained in $S^c$.
Therefore, $S^c$ is open and $S$ is closed.
$\blacksquare$

\dotfill

$S$は閉であり、$x_0\in S^c$と仮定する。

$S^c$は開であることから、
$x_0$の近傍が$S^c$に含まれ、$S$には含まれない。
したがって、$x_0$は$S$の極限点ではない。

逆に、
$S^c$の点が$S$の極限点ではないのであれば、
$S^c$の全ての点において近傍が$S^c$に含まれていないといけない。
したがって、$S^c$は開であり、$S$は閉である。

\hrulefill

Theorem 1.3.5 is usually stated as follows.

\dotfill

定理1.3.5 はよく次のように述べられる。

\hrulefill

\textbf{Corollary 1.3.6}
A set is closed if and only if it contains all its limit points.

\dotfill

\textbf{系 1.3.6}
集合が閉であるということは
全ての極限点を含んでいる場合に限る。

\hrulefill

Theorem 1.3.5 and Corollary 1.3.6 are equivalent.
However, we stated the theorem as we did because students sometimes incorrectly conclude
 from the corollary that a closed set must have limit points.
The corollary does not say this.
If $S$ has no limit points,
then the set of limit points is empty and therefore contained in $S$.
Hence, a set with no limit points is closed according to the corollary,
in agreement with Theorem 1.3.5.
For example, any finite set is closed.
More generally, $S$ is closed if there is a $\delta>0$ such $| x-y | \geq \delta$
for every pair $\{x,y\}$ of distinct points in $S$.

\dotfill

定理1.3.5 と 系1.3.6 は同値である。
ただ、学生が閉集合は極限点を持ってなければならないと
誤った結論に時々至るため、
先に定理を述べた。
系はこれを示していない。
もし$S$に極限点がないなら、
極限点の集合は空であり、$S$はこれを含む。
したがって、極限点のない集合は
定理1.3.5 により閉となる。
例えば、有限集合は閉である。
より一般的に、
$S$の異なる2点のすべての組において
$\delta>0$が存在し
$|x-y|\geq 0$であるなら
$S$は閉である。


\end{document}