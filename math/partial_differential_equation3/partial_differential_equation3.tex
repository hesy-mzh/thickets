\documentclass[12pt,b5paper]{ltjsarticle}

%\usepackage[margin=15truemm, top=5truemm, bottom=5truemm]{geometry}
%\usepackage[margin=10truemm,left=15truemm]{geometry}
\usepackage[margin=10truemm]{geometry}

\usepackage{amsmath,amssymb}
%\pagestyle{headings}
\pagestyle{empty}
\usepackage{url}

%\usepackage{listings,url}
%\renewcommand{\theenumi}{(\arabic{enumi})}

%\usepackage{graphicx}

%\usepackage{tikz}
%\usetikzlibrary {arrows.meta}
%\usepackage{wrapfig}
%\usepackage{bm}

% ルビを振る
\usepackage{luatexja-ruby}	% required for `\ruby'

%% 核Ker 像Im Hom を定義
%\newcommand{\Img}{\mathop{\mathrm{Im}}\nolimits}
%\newcommand{\Ker}{\mathop{\mathrm{Ker}}\nolimits}
%\newcommand{\Hom}{\mathop{\mathrm{Hom}}\nolimits}

%\DeclareMathOperator{\Rot}{rot}
\DeclareMathOperator{\Div}{div}
%\DeclareMathOperator{\Grad}{grad}
%\DeclareMathOperator{\arcsinh}{arcsinh}
%\DeclareMathOperator{\arccosh}{arccosh}
%\DeclareMathOperator{\arctanh}{arctanh}



%\usepackage{listings,url}
%
%\lstset{
%%プログラム言語(複数の言語に対応,C,C++も可)
%  language = Python,
%%  language = Lisp,
%%  language = C,
%  %背景色と透過度
%  %backgroundcolor={\color[gray]{.90}},
%  %枠外に行った時の自動改行
%  breaklines = true,
%  %自動改行後のインデント量(デフォルトでは20[pt])
%  breakindent = 10pt,
%  %標準の書体
%%  basicstyle = \ttfamily\scriptsize,
%  basicstyle = \ttfamily,
%  %コメントの書体
%%  commentstyle = {\itshape \color[cmyk]{1,0.4,1,0}},
%  %関数名等の色の設定
%  classoffset = 0,
%  %キーワード(int, ifなど)の書体
%%  keywordstyle = {\bfseries \color[cmyk]{0,1,0,0}},
%  %表示する文字の書体
%  %stringstyle = {\ttfamily \color[rgb]{0,0,1}},
%  %枠 "t"は上に線を記載, "T"は上に二重線を記載
%  %他オプション:leftline,topline,bottomline,lines,single,shadowbox
%  frame = TBrl,
%  %frameまでの間隔(行番号とプログラムの間)
%  framesep = 5pt,
%  %行番号の位置
%  numbers = left,
%  %行番号の間隔
%  stepnumber = 1,
%  %行番号の書体
%%  numberstyle = \tiny,
%  %タブの大きさ
%  tabsize = 4,
%  %キャプションの場所("tb"ならば上下両方に記載)
%  captionpos = t
%}



\begin{document}

\textbf{Laplacian}
\begin{equation}
 \Delta u = \sum_{i=1}^{n} u_{x_{i}x_{i}}
\end{equation}


\hrulefill

\textbf{Report 1.12}

$U$は連結とする。
関数$u$は$U$上で$C^2$-級、
$\overline{U}$上で$C^1$-級であり、
次を満たしているとする。
\begin{equation}
 \Delta u = 0\ (\text{in}\ U),\ u=g \ (\text{on}\ \partial U, \ g\geq 0)
\end{equation}

$g$が$\partial U$上のどこかで正であるなら
$u$は$U$内で常に正であることを示せ。

\dotfill

$\overline{U}$上で$C^1$-級であるので、
$u$は連続である。
この為、ある点$x_{0}\in\overline{U}$が存在し、
$u(x_{0})$は最小となる。
つまり、
$u(x_{0}) \leq u(x) \ ({}^{\forall}x\in\overline{U})$
である。

もし、$x_{0}\in\partial U$であれば、$u(x_{0})=g(x_{0})\geq 0$であり、
$0 \leq u(x_{0}) \leq u(x)$となる。

もし、$x_{0}\in U$であれば、$u$は$U$で定数関数となる。
$\partial U$にて$g\leq0$なる点があるので$u\geq 0$である。



\hrulefill

\textbf{Report 1.13}

%$- \Delta u = f$

\begin{equation}
 \tilde{u}
=\int_{\mathbb{R}^n}\Phi(x-y)f(y)dy
\end{equation}

$n=2$のとき、$\tilde{u}$は有界ではないことを示せ。

\dotfill

調和関数$\Phi(x)$は$n=2$において
$\Phi(x)=-\frac{1}{2\pi}\log{\lvert x \rvert}$
である。

$\lvert x \rvert$がそれぞれ$0$と$\infty$に飛ばした場合、
$\Phi(x)\to \infty \ (\lvert x \rvert \to 0)$
と
$\Phi(x)\to -\infty \ (\lvert x \rvert \to\infty)$
であるので、
$\lvert \Phi(x) \rvert \to \infty$
である。

\begin{equation}
 \tilde{u}
  =\int_{\mathbb{R}^2}\Phi(x-y)f(y)dy
  =\int_{\mathbb{R}^2}\Phi(y)f(x-y)dy
\end{equation}



\hrulefill

\textbf{Report 1.14}

$n=2,\ N=3$のとき、
次の式を示せ。
\begin{equation}
 u(x) - \sum_{k=0}^{N-1}\sum_{\lvert \alpha \rvert =k}
  \frac{D^{\alpha}u(x_{0})(x-x_{0})^{\alpha}}{\alpha !}
  =\sum_{\lvert \alpha \rvert =N}
  \frac{D^{\alpha}u(x_{0}+t(x-x_{0}))(x-x_{0})^{\alpha}}{\alpha !}
\end{equation}


\dotfill

\hrulefill


%%%%%
\newpage
%%%%%


\hrulefill

\begin{enumerate}
 \item


      次の初期値問題の解である関数$u$を求めよ。

      \begin{equation}
       \begin{cases}
        u_{t} +b\cdot Du + cu =0 & \text{in}\: \mathbb{R}^n \times (0,\infty)\\
        u=g & \text{on}\: \mathbb{R}^n \times \{t=0\}
       \end{cases}
      \end{equation}

      $c\in\mathbb{R}$と$b\in\mathbb{R}^{n}$は定数とする。

      \dotfill


      式を次のように変形する。
      \begin{equation}
       u_{t} +b\cdot Du + cu =0
        \: \Rightarrow \:
        u_{t} +b\cdot Du = - cu
      \end{equation}

      ここから、左辺は$u$を微分すると
      $u$の定数倍になることが読み取れる。

      \begin{equation}
       u
        %=u(x_{1},\dots,x_{n},t)
        =\exp\left(
                \frac{c}{n+1}\left( t + \sum_{i=1}^{n}\frac{x_{i}}{b_{i}} \right)
              \right)
      \end{equation}
      


      \begin{equation}
       u(x,t) = g(x-tb) \quad (x\in\mathbb{R}^{n},\: t\geq 0)
      \end{equation}

      \hrulefill


 \item

      Laplace 方程式 $\Delta u =0$は
      回転不変である、
      つまり
      $n$次直交行列$O$で変換しても
      Laplace 方程式を満たすこと
      を示せ。

      \begin{equation}
       v(x) = u(Ox) \quad (x\in\mathbb{R}^{n})
        \Rightarrow
        \Delta v = 0
      \end{equation}

      \dotfill

      $O$を$n$次直交行列とし、$O=\{o_{ij}\}$とする。
      $O$は直交行列であるので、
      $O$の転置行列と逆行列が一致する。
      \begin{equation}
       O{}^{t}O = {}^{t}O O = E
        ,\quad
        \sum_{i=1}^{n}o_{ki}o_{li} = \delta_{kl}
        =
        \begin{cases}
         1 & (k=l)\\
         0 & (k\ne l)
        \end{cases}
      \end{equation}
      $x$はベクトルであり、その成分を$x=(x_{1},\dots,x_{n})$とし、
      $\bar{x} = Ox$とする。
      これにより、$v(x)=u(\bar{x})$となる。

      $x_{i}$における偏微分は合成関数の微分より次のように変形できる。
      \begin{equation}
       \frac{\partial}{\partial x_{i}}
        = \sum_{k=1}^{n}\frac{\partial \bar{x}_{k}}{\partial x_{i}}
              \frac{\partial}{\partial \bar{x}_{k}}
        = \sum_{k=1}^{n} o_{ki}\frac{\partial}{\partial \bar{x}_{k}}
      \end{equation}

      2階の偏微分は次のように求められる。
      \begin{equation}
       \frac{\partial^{2}}{\partial x_{i}^{2}}
        =\left( \sum_{k=1}^{n} o_{ki}\frac{\partial}{\partial \bar{x}_{k}} \right) \cdot
        \left( \sum_{l=1}^{n} o_{li}\frac{\partial}{\partial \bar{x}_{l}} \right)
        =
        \sum_{k,l = 1}^{n} o_{ki}o_{li}\frac{\partial^{2}}{\partial \bar{x}_{k}\partial \bar{x}_{l}}
      \end{equation}

      これらを用いて
      $\Delta v$ を計算する。
      \begin{align}
       \Delta v
         =& \Delta u(\bar{x})
         = \sum_{i=1}^{n} \frac{\partial^2}{\partial x_{i}^{2}}u(\bar{x})
         = \sum_{i=1}^{n} \sum_{k,l = 1}^{n} o_{ki}o_{li}
           \frac{\partial^{2}}{\partial \bar{x}_{k}\partial \bar{x}_{l}} u(\bar{x})\\
         =& \sum_{k,l = 1}^{n} \left( \sum_{i=1}^{n} o_{ki}o_{li} \right)
           \frac{\partial^{2}}{\partial \bar{x}_{k}\partial \bar{x}_{l}} u(\bar{x})
         = \sum_{k,l = 1}^{n} \delta_{kl}
           \frac{\partial^{2}}{\partial \bar{x}_{k}\partial \bar{x}_{l}} u(\bar{x})\\
         =& \sum_{k = 1}^{n}
           \frac{\partial^{2}}{\partial \bar{x}_{k}^{2}} u(\bar{x})
         = \Delta u
      \end{align}

      よって、$\Delta u=0$であれば、$\Delta v =0$である。

      \hrulefill


 \item

      平均値定理の証明を利用し、
      $n\geq 3$の時、次の式を証明せよ。
      \begin{equation}
       \begin{split}
       u(0) =& \frac{1}{n\alpha(n)r^{n-1}} \int_{\partial B(0,r)} g dS\\
         &+ \frac{1}{n(n-2)\alpha(n)} \int_{B(0,r)}
         \left( \frac{1}{\lvert x \rvert^{n-2}}
          -  \frac{1}{r^{n-2}}\right)  f dx
       \end{split}
      \end{equation}

      \begin{equation}
       \begin{cases}
        -\Delta u =f & \text{in}\: B^{0}(0,r)\\
        u=g & \text{on}\: \partial B^{0}(0,r)
       \end{cases}
      \end{equation}

      \dotfill



      \hrulefill

 \item

      関数$u \in C^{2}(U) \cap C(\overline{U})$が
      開集合$U$の境界上を除いて調和的である時、
      次が成り立つことを示せ。
      \begin{equation}
       \underset{\overline{U}}{\max {u}}
        = \underset{\partial{U}}{\max {u}}
      \end{equation}

      HINT :
      $\varepsilon >0$の時
      $u_{\varepsilon}=u+\varepsilon \lvert x \rvert^{2}$とおくと、
      $U$の内部では最大値を取らないことを示せばよい。

      \dotfill



      \hrulefill

 \item

      次のような$v\in C^{2}(\overline{U})$を 劣調和関数 (subharmonic) という。
      \begin{equation}
       -\Delta v \leq 0 \quad \text{in}\: U
      \end{equation}


      \begin{enumerate}
       \item

            $v$が次を満たすことを示せ。
            \begin{equation}
             v(x) \leq \frac{1}{\alpha(n)r^{n}}
              \int_{B(x,r)} v dy
              \quad \text{for all} B(x,r) \subset U
            \end{equation}

            \dotfill



            \hrulefill

       \item

            次を示せ。
            \begin{equation}
             \underset{\overline{U}}{\max {u}}
              = \underset{\partial{U}}{\max {u}}
            \end{equation}

            \dotfill



            \hrulefill

       \item

            $\phi : \mathbb{R}\to\mathbb{R}$
            は
            なめらかな凸関数とする。

            $u$は調和関数、
            $v = \phi(u)$
            とした時、
            $v$は劣調和関数であることを示せ。

            \dotfill



            \hrulefill

       \item

            $u$が調和的である時、
            $v = \lvert Du \rvert^{2}$
            は劣調和的であることを示せ。

            \dotfill



            \hrulefill

      \end{enumerate}



 \item


      開集合$U \subset \mathbb{R}^{n}$は有界であるとする。

      この時、$U$のみに依存した定数$C$が存在し、次の式が成り立つことを示せ。
      \begin{equation}
       \underset{\overline{U}}{\max {\lvert u \rvert}}
        \leq
        C \left(
        \underset{\partial{U}}{\max {\lvert g \rvert}}
        +
        \underset{\overline{U}}{\max {\lvert f \rvert}}
        \right)
      \end{equation}


      なお、$u$は滑らかな関数で、次の解である。
      \begin{equation}
       \begin{cases}
        -\Delta u =f & \text{in}\: U\\
        u=g & \text{on}\: \partial U
       \end{cases}
      \end{equation}


      HINT :
      $\lambda = \underset{\overline{U}}{\max {\lvert f \rvert}}$
      に対して、
      $-\Delta \left(u + \frac{\lvert x \rvert^{2}}{2n}\lambda \right)
      \leq 0$

      \dotfill



      \hrulefill


\end{enumerate}

\hrulefill




%%%%%
\newpage
%%%%%

\hrulefill

\textbf{参考文献}

偏微分方程式 : 講義ノート

\url{https://www.math.kyoto-u.ac.jp/~karel/files/notes_pde_2015.pdf}

非線型解析 : 講義ノート

\url{https://www.math.kyoto-u.ac.jp/~karel/files/notes_nonlinear_analysis_2019.pdf}


\hrulefill

$C^{\infty}(\Omega)$は$\Omega$上で無限に微分可能(calculus)な関数の集合

\dotfill

$\mathrm{supp}(f)$ は $f$の台(サポート)といい、
$f(x)\ne 0$となる点$x$の集合

\dotfill

\begin{equation}
 C_{c}^{\infty}(\Omega)
  =\{ \phi\in C^{\infty}(\Omega) \mid \mathrm{supp}(\phi) \: がコンパクト \}
\end{equation}
$C_{0}^{\infty}(\Omega)$も$C_{c}^{\infty}(\Omega)$と同じ意味として使われる。

\dotfill

\textbf{テスト関数}

関数$\phi$は$C^{\infty}$級でコンパクトな台を持ち、
その境界上では$0$になるとき、
テスト関数という。

$\partial U$上で$\phi=0$となる
テスト関数は
$\phi\in C^{\infty}_{c}(U)$
とかかれる。


\dotfill

可積分関数のなす線形空間($L^{p}$空間、エルピー空間)であり、
次のような集合である。$p$-ノルム空間
\begin{align}
 L^{1}(\Omega) &= \left\{ f:\Omega\to\mathbb{R} \:\middle|\: \int_{\Omega}\lvert f(x)\rvert dx < \infty \right\}\\
 L^{p}(\Omega) &= \left\{ f:\Omega\to\mathbb{R} \:\middle|\: \int_{\Omega}\lvert f(x)\rvert^{p} dx < \infty \right\}
\end{align}

$f\in L^{p}(\Omega)$の
ノルム$\| f \|_{L^{p}}$
は次で定義する。
\begin{equation}
 \| f \|_{L^{p}}
  =
  \left( \int_{\Omega} \lvert f(x) \rvert^{p} dx \right)^{\frac{1}{p}}
\end{equation}


\dotfill

\textbf{局所可積分関数}

$\Omega$:開集合、
$p\in [1,\infty)$とする。
可測関数$f:\Omega\to\mathbb{R}$が
$p$乗局所可積分関数であるとは
任意のコンパクト集合$k\subset \Omega$
に対して
$\int_{K} \lvert f(x) \rvert^{p} dx < \infty$
が成り立つことと定義する。

\begin{equation}
 L^{p}_{loc}(\Omega)
  =\left\{
    f:\Omega\to \mathbb{R} %\:\text{or}\: \mathbb{C}
    \:\middle|\:
    {}^{\forall}K \subset \Omega コンパクト,\:
    \int_{K} \lvert f(x) \rvert^{p} dx < \infty
   \right\}
\end{equation}

$L^{p}_{loc}(\Omega)$の他、
$L_{p,loc}(\Omega)$や
$L_{p}(\Omega,log)$等の記号が使われている。

次のような包含関係がある。
\begin{equation}
 L^{p}(\Omega)
  \subset L^{p}_{loc}(\Omega)
  \subset L^{1}_{loc}(\Omega)
\end{equation}


\dotfill

\textbf{弱微分}

$u\in L^{1}([a,b])$とする。
$\phi(a)=\phi(b)=0$を満たす任意の無限階可能関数
$\phi$(テスト関数 $\phi\in C_{c}^{\infty}([a,b])$)に対して
次を満たす$v\in L^{1}([a,b])$を$u$の弱微分という。
\begin{equation}
 \int_{a}^{b}u(t)\phi^{\prime}dt = -\int_{a}^{b} v(t)\phi(t) dt
\end{equation}
この式は部分積分の式の変形である。



\dotfill

\textbf{多重指数}

$n$個の数の組
$\alpha = (\alpha_{1},\dots , \alpha_{n})$
を用いて微分$D^{\alpha}u$を表す。

$\lvert \alpha \rvert = \sum_{k=1}^{n}\alpha_{k}$とする。
\begin{equation}
 D^{\alpha}u = \frac{\partial^{\lvert \alpha \rvert} u}{\partial x_{1}^{\alpha_{1}}\cdots\partial x_{n}^{\alpha_{n}}}
\end{equation}

\dotfill


\ruby{Sobolev}{ソボレフ}空間

\begin{align}
 W^{k,p}(\Omega) &= \left\{ u\in L^{p}(\Omega) \: \middle| \:
  \begin{aligned}
   & u は k 階弱微分可能\\
   & k 階までの全ての導関数が L^{p}(\Omega) に含まれる
  \end{aligned}
 \right\}\\
 W^{k,p}(\Omega) &= \left\{ u\in L^{p}(\Omega) \: \middle| \:
  \begin{aligned}
   & \alpha は 多重指数\\
   & 0 \leq \lvert \alpha \rvert \leq k \Rightarrow D^{\alpha}u \in L^{p}(\Omega)
  \end{aligned}
 \right\}
\end{align}


ノルム$\| u \|_{W^{k,p}}$は次のように定義する。
\begin{equation}
 \| u \|_{W^{k,p}} =
  \left( \sum_{0 \leq \lvert \alpha \rvert \leq k} \| D^{\alpha} u \|_{L^{p}}^{p} \right)^{\frac{1}{p}}
\end{equation}


\hrulefill

\begin{enumerate}
 \item [2.1.]

       $u:U\to\mathbb{R}$の
       第$\gamma$ \ruby{H\"{o}lder}{ヘルダー} 半ノルム
       を次で定義する。
       \begin{equation}
        [u]_{C^{0,\gamma}(\overline{U})}
          =
          \sup_{\substack{x,y\in U \\ x \ne y}}
          \left\{
           \frac{\lvert u(x)-u(y) \rvert}{\lvert x-y \rvert^{\gamma}}
          \right\}
       \end{equation}

       この定義は半ノルムであることを確認せよ。

       \dotfill

       半ノルムとは
       絶対斉次性($p(\lambda x) = \lvert \lambda \rvert p(x)$)
       と
       劣加法性($p(x+y)\leq p(x)+p(y)$)
       を満たす写像$p$のことをいう。

       ${}^{\forall}\lambda \in \mathbb{R}$とする。
       \begin{align}
        [\lambda u]_{C^{0,\gamma}(\overline{U})}
         =&
          \sup_{\substack{x,y\in U \\ x \ne y}}
          \left\{
           \frac{\lvert \lambda u(x)- \lambda u(y) \rvert}{\lvert x-y \rvert^{\gamma}}
          \right\}\\
          =&
          \lvert \lambda \rvert \sup_{\substack{x,y\in U \\ x \ne y}}
          \left\{
           \frac{\lvert u(x)- u(y) \rvert}{\lvert x-y \rvert^{\gamma}}
          \right\}
          =
          \lvert \lambda \rvert [u]_{C^{0,\gamma}(\overline{U})}
       \end{align}
       よって、絶対斉次性を満たす。

       劣加法性を確認する。
       \begin{align}
        [u+v]_{C^{0,\gamma}(\overline{U})}
         =&
          \sup_{\substack{x,y\in U \\ x \ne y}}
          \left\{
           \frac{\lvert (u(x)+v(x))- (u(y)+v(y)) \rvert}{\lvert x-y \rvert^{\gamma}}
          \right\}\\
          \leq &
          \sup_{\substack{x,y\in U \\ x \ne y}}
          \left\{
           \frac{\lvert u(x)- u(y) \rvert + \lvert v(x)-v(y) \rvert}{\lvert x-y \rvert^{\gamma}}
          \right\}\\
          \leq &
          \sup_{\substack{x,y\in U \\ x \ne y}}
          \left\{
           \frac{\lvert u(x)- u(y) \rvert}{\lvert x-y \rvert^{\gamma}}
          \right\}
        +
          \sup_{\substack{x,y\in U \\ x \ne y}}
          \left\{
           \frac{ \lvert v(x)-v(y) \rvert}{\lvert x-y \rvert^{\gamma}}
          \right\}\\
        =&
        [u]_{C^{0,\gamma}(\overline{U})}
        +
        [v]_{C^{0,\gamma}(\overline{U})}
       \end{align}

       以上により半ノルムであることが確認できる。

       \hrulefill

 \item [2.3.]

       $U$上の滑らかな関数全体を
       $C^{\infty}_{c}(U) \subset W^{k,p}(U)$とし、
       これの閉包を$W_{0}^{k,p}(U)$とする。
       $W_{0}^{k,p}(U)$は
       $\lvert \alpha \rvert \leq k-1$を満たす$\alpha$において
       $\partial U$上で
       $D^{\alpha}u =0$
       となる関数$u\in W^{k,p}(U)$の集まりである。

       トレースの定理
       を認めて
       これを示せ。


       \textbf{トレースの定理}

       $U$を有界、$\partial U$を$C^{1}$とする。
       \begin{equation}
        T : W^{1,p}(U) \to L^{p}(\partial U)
       \end{equation}
       この時、次を満たす有界線形作用素$T$が存在する。
       \begin{enumerate}
        \item
             $u\in W^{1,p}(U) \cap C(\overline{U})$に対して
             $Tu = u|_{\partial U}$

        \item
             各$u\in W^{1,p}(U)$に対し、
             $\| Tu \|_{L^{p}(\partial U)} \leq C \| u \|_{W^{1,p}(U)}$
             である。
             定数$C$は$p$と$U$のみに依存する。

       \end{enumerate}

       \dotfill


       \begin{equation}
        F : W^{k,p}(U) \to W^{1,p}(U)
         ,\quad
         u \mapsto D^{\alpha}u
       \end{equation}

       上記写像とトレースの定理の写像$T$の合成$T\circ F$を考える。

%       $\partial U$上で$D^{\alpha}u =0$となる
%       関数$u$は$(T\circ F)(u)=0$である。

       $T(u) = u|_{\partial U}$であるから
       $T(u)=0$となる$u$は
       $\partial U$上で$u=0$ということである。
       つまり、
       $(T\circ F)(u)=0$となる関数$u$は
       $\partial U$上で
       $D^{\alpha}u=0$
       を満たす。

       $f\in C^{\infty}_{c}(U)$とすれば
       $\mathrm{supp}(f) \subset U$であり、
       $\partial U$上では$f=0$となる。
       つまり、$\partial U$上
       $D^{\alpha}f=0$である。

       $f\in W^{k,p}_{0}(U)$が
       $f_{i}\in C^{\infty}_{c}(U)$
       により
       $f=\lim_{i\to\infty}f_{i}$
       とする。

       \begin{equation}
        \lim_{i\to\infty} \|f_{i}-f \|_{W^{k,p}(U)}=0
       \end{equation}

       \begin{equation}
        \lim_{i\to\infty} \|f_{i}-f \|_{W^{k,p}(U)}
         =
         \lim_{i\to\infty}
         \left( \sum_{0\leq \lvert \alpha \rvert \leq k}
          \| D^{\alpha}(f_{i}-f) \|_{L^{p}(U)}^{p}
         \right)^{\frac{1}{p}}
       \end{equation}


       $\partial U$上では
       $f_{k}=0$であるので、
       \begin{equation}
        \lim_{i\to\infty}f_{i}
       \end{equation}





       \hrulefill

 \item [2.5.]

       $\{r_{k}\}_{k=1}^{\infty}$は可算で稠密な
       $U=B^{0}(0,1)$の部分集合とする。
       \begin{equation}
        u(x) =
         \sum_{k=1}^{\infty} \frac{1}{2^{k}} \lvert x-r_{k} \rvert ^{-\alpha}
         \quad
         (x\in U)
       \end{equation}

       $0 \leq \alpha \leq \frac{n-p}{p}$であれば
       $u\in W^{1,p}(U)$である。

       この時、
       $u$は
       $U$の部分集合である開集合上で
       有界ではないことを示せ。
       

       \dotfill

       \hrulefill

 \item [2.7.]


       \textbf{弱微分の性質}
       
       $u\in W^{k,p}(U),\: \lvert \alpha \rvert \leq k$
       とする。

       このとき、$V\subset U$が開集合であるなら
       $u\in W^{k,p}(V)$となることを示せ。

       \dotfill


       $u\in W^{k,p}(U)$であるから
       $D^{\alpha}u \in L^{p}(U)$である。
       つまり、任意のテスト関数$\phi\in C^{\infty}_{c}(U)$に対して次を満たす。
       \begin{equation}
        \int_{U} D^{\alpha}u(x) \phi(x) dx
         = (-1)^{\lvert \alpha \rvert}\int_{U} u(x) D^{\alpha}\phi(x) dx
         \label{eq_weak_deriva}
       \end{equation}

       この$\phi$は$\phi:U\to\mathbb{R}$で$C^{\infty}$級な関数あり、
       関数の台$\mathrm{supp}(f)=\{x\in U \mid f(x)\ne 0\}$は
       $\mathrm{supp}(f) \subset U$である。

       \eqref{eq_weak_deriva}は任意の$\phi$について成り立つ。
       この為、$U\backslash V$上で$0$となるテスト関数$\bar{\phi}$
       についても成り立つので、
       $V\subset U$上に制限した次の式も成り立つ。
       \begin{equation}
        \int_{V} D^{\alpha}u(x) \bar{\phi}(x) dx
         = (-1)^{\lvert \alpha \rvert}\int_{V} u(x) D^{\alpha}\bar{\phi}(x) dx
       \end{equation}

       つまり、
       $D^{\alpha}u \in L^{p}(V)$である。
       よって、
       $u\in W^{k,p}(V)$である。





       \hrulefill

 \item [2.9.]

%       $W^{k,p}(U)$が完備であることを示せばよい。
%       そこで、
%       $\{u_{m}\}_{m=1}^{\infty}$が
%       $W^{k,p}(U)$の\ruby{Cauchy}{コーシー}列とする。
%
%       この事実が成り立つことを示せ。


       $\{u_{m}\}_{m=1}^{\infty}$が
       $W^{k,p}(U)$の\ruby{Cauchy}{コーシー}列とする。

       この時、
       $\lvert \alpha \rvert \leq k$となる各$\alpha$に対して
       $\{D^{\alpha}u_{m}\}_{m=1}^{\infty}$は
       $L^{p}(U)$のCauchy列となることを示せ。

       \dotfill

       $\{u_{m}\}_{m=1}^{\infty}$がコーシー列であるので、
       次の極限が$0$となる。
       \begin{equation}
        \lim_{i,j\to\infty}\| u_{i} - u_{j} \|_{W^{k,p}(U)}=0
       \end{equation}

       極限を取るノルムは次のように$L^{P}(U)$のノルムで書き換える。
       \begin{equation}
        \| u_{i} - u_{j} \|_{W^{k,p}(U)}
         =
         \left(
          \sum_{0\leq \lvert \alpha \rvert \leq k}
          \| D^{\alpha}(u_{i}-u_{j}) \|_{L^{p}(U)}^{p}
         \right)^{\frac{1}{p}}
       \end{equation}

       これが$0$に収束するので、
       $0\leq \lvert \alpha \rvert \leq k$
       において
       各$\| D^{\alpha}(u_{i}-u_{j}) \|_{L^{p}(U)}$
       は$0$に収束する。

       \begin{equation}
        \lim_{i,j\to\infty} \| D^{\alpha}(u_{i}-u_{j}) \|_{L^{p}(U)}
         =
         \lim_{i,j\to\infty} \| D^{\alpha}u_{i} - D^{\alpha}u_{j} \|_{L^{p}(U)}
         =0
       \end{equation}

       これにより
       $\{D^{\alpha}u_{m}\}_{m=1}^{\infty}$は
       コーシー列だとわかる。
       
       \hrulefill

\end{enumerate}

\hrulefill
%%%
\newpage
%%%
\hrulefill


この問では
$U\subset \mathbb{R}^{n}$を開集合とし、
$\partial U$は滑らかな境界とする。
関数は指定がない限り滑らかな関数であるとする。

\begin{enumerate}
 \item
      $U \subset \mathbb{R}^{n}$は開集合、
      $f\in L^{1}_{loc}(U)$とする。

      ${}^{\forall}\phi \in C^{\infty}_{c}(U)$に対して、
      恒等式
      $\int_{U}f\phi dx =0$
      が成り立つなら、
      ほとんど至るところで
      $f=0$である。

      \dotfill

      $f\in L^{1}_{loc}(U)$より、
      任意のコンパクトな集合$K \subset U$において
      $\int_{K}\lvert f(x) \rvert dx < \infty$
      である。

      テスト関数$\phi$と$f$の積の$U$上の積分が$0$となる。

      \begin{equation}
       \int_{U}f\phi dx
        =
        \int_{K}f\phi dx
        + \int_{U\backslash K}f\phi dx =0
      \end{equation}

      $\phi_{K}\in C^{\infty}_{c}(U)$が
      $\mathrm{supp}(\phi_{K})=K$を満たすとする。
      つまり、$\phi_{K}\in C^{\infty}_{c}(K)$とする。
      これにより$U\backslash K$上で $\phi_{K}=0$となる為、
      $\int_{U\backslash K}f\phi_{K} dx =0$となる。

      よって、
      $\int_{K}f\phi_{K} dx=0$
      となる。

      任意の関数$\phi_{K}$に対して積分が0となるので、
      $f=0$となる。

      

      \hrulefill
 \item
      $k\in \{ 0,1,\dots \},\: 0< \gamma \leq 1$
      とする。
      この時、
      $C^{k,\gamma}(\overline{U})$はバナッハ空間となることを示せ。

      \dotfill


      \hrulefill
 \item
      $0 < \beta < \gamma \leq 1$とする。
      この時、次の不等式を示せ。
      \begin{equation}
       \left\| u \right\|_{C^{0,\gamma}(U)}
        \leq
        \left\| u \right\|_{C^{0,\beta}(U)}^{\frac{1-\gamma}{1-\beta}}
        \left\| u \right\|_{C^{0,1}(U)}^{\frac{\gamma-\beta}{1-\beta}}
      \end{equation}
      
      \dotfill


      \hrulefill
 \item
      開集合$U$を
      $U=\{ x \in \mathbb{R}^{2} \mid \lvert x_{i} \rvert <1 \: (i=1,2) \}$
      とする。

      $u(x)$を次のように定義する。
      \begin{equation}
       u(x)=
        \begin{cases}
         1-x_{1} & x_{1}>0 ,\: \lvert x_{2} \rvert < x_{1}\\
         1+x_{1} & x_{1}<0 ,\: \lvert x_{2} \rvert < -x_{1}\\
         1-x_{2} & x_{2}>0 ,\: \lvert x_{1} \rvert < x_{2}\\
         1+x_{2} & x_{2}<0 ,\: \lvert x_{1} \rvert < -x_{2}
        \end{cases}
      \end{equation}

      $1\leq p \leq \infty$に対して
      $u$は$W^{1,p}(U)$に属するのは
      $p$がいくつのときか調べよ。

      \dotfill


      \begin{align}
       W^{1,p}(U) &= \{ u\in L^{p}(U) \mid 0\leq \lvert \alpha \rvert \leq 1 \Rightarrow D^{\alpha}u \in L^{p}(U) \}\\
       L^{p}(U) &= \left\{ f:U\to \mathbb{R} \:\middle|\: \int_{U}\lvert f(x) \rvert^{p}dx < \infty  \right\}
      \end{align}

      \hrulefill
 \item
      $n=1$と
      ある$p \: (1\leq p < \infty)$について
      $u\in W^{1,p}(0,1)$
      を仮定する。

      \begin{enumerate}
       \item
            $u$はほとんど至るところで絶対連続関数であり、
            $u^{\prime}\in L^{p}(0,1)$である。

            \dotfill


            \hrulefill

       \item
            $1< p < \infty$とする。
            $x,y\in [0,1]$に対し、
            ほとんど至るところで次が成り立つ。
            \begin{equation}
             \lvert u(x) - u(y) \rvert
              \leq
              \lvert x-y \rvert^{1-\frac{1}{p}}
              \left( \int_{0}^{1}\lvert u^{\prime}\rvert^{p}dt \right)^{\frac{1}{p}}
            \end{equation}


            \dotfill


            \hrulefill

      \end{enumerate}
 \item
      $U,V$を開集合とし、
      $V\subset\subset U$とする。

      この時、
      滑らかな関数$\zeta$が存在し、
      $V$上で$\zeta\equiv 1$、
      $\partial U$の近くで$\zeta = 0$となるものが存在する。

      (HINT: $V \subset\subset W \subset\subset U$を取ってきて、
      molify $\chi_{W}$)
      
      \dotfill


      \hrulefill
 \item
      $U$を有界とし、
      $U \subset\subset \bigcup_{i=1}^{N}V_{i}$とする。

      この時、
      $C^{\infty}$級関数
      $\zeta_{i} \: (i=1,\dots,N)$が存在し
      次を満たす。
      \begin{equation}
       \begin{cases}
        0\leq \zeta_{i} \leq 1
        , \quad
        \mathrm{supp}(\zeta_{i}) \subset V_{i} \: (i=1,\ldots,N)\\
        \sum_{i=1}^{N}\zeta_{i} = 1 \quad U\text{上}
       \end{cases}
      \end{equation}

      関数$\{\zeta_{i}\}_{i=1}^{N}$は
      $1$の分割となっている。

      \dotfill


      \hrulefill
 \item
      $U$を有界とし、
      $C^{1}$な境界を持つとする

      この時、
      一般的な関数$u \in L^{p}(U) \: (1\leq p \leq \infty)$
      は$\partial U$上で trace を持たないことを示せ。

      より正確には、
      $u\in C(\overline{U})\cap L^{p}(\partial U)$について
      $Tu = u|_{\partial U}$を満たす
      有界線形作用素$T:L^{p}(U) \to L^{p}(\partial U)$
      は存在しないことを示せ。。

      \dotfill


      \hrulefill
 \item
      部分積分を行い、
      任意の$u\in C^{\infty}_{c}(U)$
      に対し、次の補間不等式を示せ。
      \begin{equation}
       \left\| Du \right\|_{L^{2}}
        \leq
        C \left\| u \right\|_{L^{2}}^{1/2}
        \left\| D^{2}u \right\|_{L^{2}}^{1/2}
      \end{equation}

      $U$を有界、
      $\partial U$は滑らかであるとする。
      $u\in H^{2}(U)\cap H^{1}_{0}(U)$であるとき、
      上記不等式を示せ。

      (HINT:
      関数列$\{ v_{k }\}_{k=1}^{\infty} \subset C^{\infty}_{c}(U)$を
      $H^{1}_{0}(U)$内で$u$、
      $\{ w_{k }\}_{k=1}^{\infty} \subset C^{\infty}(\overline{U})$を
      $H^{2}(U)$内で$u$
      に収束するように取れる。
      )

      \dotfill


      \hrulefill
 \item
      



\end{enumerate}


\hrulefill

\end{document}
