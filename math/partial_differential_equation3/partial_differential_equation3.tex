\documentclass[12pt,b5paper]{ltjsarticle}

%\usepackage[margin=15truemm, top=5truemm, bottom=5truemm]{geometry}
%\usepackage[margin=10truemm,left=15truemm]{geometry}
\usepackage[margin=10truemm]{geometry}

\usepackage{amsmath,amssymb}
%\pagestyle{headings}
\pagestyle{empty}

%\usepackage{listings,url}
%\renewcommand{\theenumi}{(\arabic{enumi})}

%\usepackage{graphicx}

%\usepackage{tikz}
%\usetikzlibrary {arrows.meta}
%\usepackage{wrapfig}
%\usepackage{bm}

% ルビを振る
\usepackage{luatexja-ruby}	% required for `\ruby'

%% 核Ker 像Im Hom を定義
%\newcommand{\Img}{\mathop{\mathrm{Im}}\nolimits}
%\newcommand{\Ker}{\mathop{\mathrm{Ker}}\nolimits}
%\newcommand{\Hom}{\mathop{\mathrm{Hom}}\nolimits}

%\DeclareMathOperator{\Rot}{rot}
\DeclareMathOperator{\Div}{div}
%\DeclareMathOperator{\Grad}{grad}
%\DeclareMathOperator{\arcsinh}{arcsinh}
%\DeclareMathOperator{\arccosh}{arccosh}
%\DeclareMathOperator{\arctanh}{arctanh}



%\usepackage{listings,url}
%
%\lstset{
%%プログラム言語(複数の言語に対応,C,C++も可)
%  language = Python,
%%  language = Lisp,
%%  language = C,
%  %背景色と透過度
%  %backgroundcolor={\color[gray]{.90}},
%  %枠外に行った時の自動改行
%  breaklines = true,
%  %自動改行後のインデント量(デフォルトでは20[pt])
%  breakindent = 10pt,
%  %標準の書体
%%  basicstyle = \ttfamily\scriptsize,
%  basicstyle = \ttfamily,
%  %コメントの書体
%%  commentstyle = {\itshape \color[cmyk]{1,0.4,1,0}},
%  %関数名等の色の設定
%  classoffset = 0,
%  %キーワード(int, ifなど)の書体
%%  keywordstyle = {\bfseries \color[cmyk]{0,1,0,0}},
%  %表示する文字の書体
%  %stringstyle = {\ttfamily \color[rgb]{0,0,1}},
%  %枠 "t"は上に線を記載, "T"は上に二重線を記載
%  %他オプション:leftline,topline,bottomline,lines,single,shadowbox
%  frame = TBrl,
%  %frameまでの間隔(行番号とプログラムの間)
%  framesep = 5pt,
%  %行番号の位置
%  numbers = left,
%  %行番号の間隔
%  stepnumber = 1,
%  %行番号の書体
%%  numberstyle = \tiny,
%  %タブの大きさ
%  tabsize = 4,
%  %キャプションの場所("tb"ならば上下両方に記載)
%  captionpos = t
%}



\begin{document}

\textbf{Laplacian}
\begin{equation}
 \Delta u = \sum_{i=1}^{n} u_{x_{i}x_{i}}
\end{equation}


\hrulefill

\textbf{Report 1.12}

$U$は連結とする。
関数$u$は$U$上で$C^2$-級、
$\overline{U}$上で$C^1$-級であり、
次を満たしているとする。
\begin{equation}
 \Delta u = 0\ (\text{in}\ U),\ u=g \ (\text{on}\ \partial U, \ g\geq 0)
\end{equation}

$g$が$\partial U$上のどこかで正であるなら
$u$は$U$内で常に正であることを示せ。

\dotfill

$\overline{U}$上で$C^1$-級であるので、
$u$は連続である。
この為、ある点$x_{0}\in\overline{U}$が存在し、
$u(x_{0})$は最小となる。
つまり、
$u(x_{0}) \leq u(x) \ ({}^{\forall}x\in\overline{U})$
である。

もし、$x_{0}\in\partial U$であれば、$u(x_{0})=g(x_{0})\geq 0$であり、
$0 \leq u(x_{0}) \leq u(x)$となる。

もし、$x_{0}\in U$であれば、$u$は$U$で定数関数となる。
$\partial U$にて$g\leq0$なる点があるので$u\geq 0$である。



\hrulefill

\textbf{Report 1.13}

%$- \Delta u = f$

\begin{equation}
 \tilde{u}
=\int_{\mathbb{R}^n}\Phi(x-y)f(y)dy
\end{equation}

$n=2$のとき、$\tilde{u}$は有界ではないことを示せ。

\dotfill

調和関数$\Phi(x)$は$n=2$において
$\Phi(x)=-\frac{1}{2\pi}\log{\lvert x \rvert}$
である。

$\lvert x \rvert$がそれぞれ$0$と$\infty$に飛ばした場合、
$\Phi(x)\to \infty \ (\lvert x \rvert \to 0)$
と
$\Phi(x)\to -\infty \ (\lvert x \rvert \to\infty)$
であるので、
$\lvert \Phi(x) \rvert \to \infty$
である。

\begin{equation}
 \tilde{u}
  =\int_{\mathbb{R}^2}\Phi(x-y)f(y)dy
  =\int_{\mathbb{R}^2}\Phi(y)f(x-y)dy
\end{equation}



\hrulefill

\textbf{Report 1.14}

$n=2,\ N=3$のとき、
次の式を示せ。
\begin{equation}
 u(x) - \sum_{k=0}^{N-1}\sum_{\lvert \alpha \rvert =k}
  \frac{D^{\alpha}u(x_{0})(x-x_{0})^{\alpha}}{\alpha !}
  =\sum_{\lvert \alpha \rvert =N}
  \frac{D^{\alpha}u(x_{0}+t(x-x_{0}))(x-x_{0})^{\alpha}}{\alpha !}
\end{equation}


\dotfill





\hrulefill

\end{document}
