\documentclass[12pt,b5paper]{ltjsarticle}

%\usepackage[margin=15truemm, top=5truemm, bottom=5truemm]{geometry}
%\usepackage[margin=10truemm,left=15truemm]{geometry}
\usepackage[margin=10truemm]{geometry}

\usepackage{amsmath,amssymb}
%\pagestyle{headings}
\pagestyle{empty}

%\usepackage{listings,url}
%\renewcommand{\theenumi}{(\arabic{enumi})}

%\usepackage{graphicx}

%\usepackage{tikz}
%\usetikzlibrary {arrows.meta}
%\usepackage{wrapfig}
%\usepackage{bm}

% ルビを振る
\usepackage{luatexja-ruby}	% required for `\ruby'

%% 核Ker 像Im Hom を定義
%\newcommand{\Img}{\mathop{\mathrm{Im}}\nolimits}
%\newcommand{\Ker}{\mathop{\mathrm{Ker}}\nolimits}
%\newcommand{\Hom}{\mathop{\mathrm{Hom}}\nolimits}

%\DeclareMathOperator{\Rot}{rot}
\DeclareMathOperator{\Div}{div}
%\DeclareMathOperator{\Grad}{grad}
%\DeclareMathOperator{\arcsinh}{arcsinh}
%\DeclareMathOperator{\arccosh}{arccosh}
%\DeclareMathOperator{\arctanh}{arctanh}



%\usepackage{listings,url}
%
%\lstset{
%%プログラム言語(複数の言語に対応,C,C++も可)
%  language = Python,
%%  language = Lisp,
%%  language = C,
%  %背景色と透過度
%  %backgroundcolor={\color[gray]{.90}},
%  %枠外に行った時の自動改行
%  breaklines = true,
%  %自動改行後のインデント量(デフォルトでは20[pt])
%  breakindent = 10pt,
%  %標準の書体
%%  basicstyle = \ttfamily\scriptsize,
%  basicstyle = \ttfamily,
%  %コメントの書体
%%  commentstyle = {\itshape \color[cmyk]{1,0.4,1,0}},
%  %関数名等の色の設定
%  classoffset = 0,
%  %キーワード(int, ifなど)の書体
%%  keywordstyle = {\bfseries \color[cmyk]{0,1,0,0}},
%  %表示する文字の書体
%  %stringstyle = {\ttfamily \color[rgb]{0,0,1}},
%  %枠 "t"は上に線を記載, "T"は上に二重線を記載
%  %他オプション:leftline,topline,bottomline,lines,single,shadowbox
%  frame = TBrl,
%  %frameまでの間隔(行番号とプログラムの間)
%  framesep = 5pt,
%  %行番号の位置
%  numbers = left,
%  %行番号の間隔
%  stepnumber = 1,
%  %行番号の書体
%%  numberstyle = \tiny,
%  %タブの大きさ
%  tabsize = 4,
%  %キャプションの場所("tb"ならば上下両方に記載)
%  captionpos = t
%}



\begin{document}

\textbf{Laplacian}
\begin{equation}
 \Delta u = \sum_{i=1}^{n} u_{x_{i}x_{i}}
\end{equation}


\hrulefill

\textbf{Report 1.12}

$U$は連結とする。
関数$u$は$U$上で$C^2$-級、
$\overline{U}$上で$C^1$-級であり、
次を満たしているとする。
\begin{equation}
 \Delta u = 0\ (\text{in}\ U),\ u=g \ (\text{on}\ \partial U, \ g\geq 0)
\end{equation}

$g$が$\partial U$上のどこかで正であるなら
$u$は$U$内で常に正であることを示せ。

\dotfill

$\overline{U}$上で$C^1$-級であるので、
$u$は連続である。
この為、ある点$x_{0}\in\overline{U}$が存在し、
$u(x_{0})$は最小となる。
つまり、
$u(x_{0}) \leq u(x) \ ({}^{\forall}x\in\overline{U})$
である。

もし、$x_{0}\in\partial U$であれば、$u(x_{0})=g(x_{0})\geq 0$であり、
$0 \leq u(x_{0}) \leq u(x)$となる。

もし、$x_{0}\in U$であれば、$u$は$U$で定数関数となる。
$\partial U$にて$g\leq0$なる点があるので$u\geq 0$である。



\hrulefill

\textbf{Report 1.13}

%$- \Delta u = f$

\begin{equation}
 \tilde{u}
=\int_{\mathbb{R}^n}\Phi(x-y)f(y)dy
\end{equation}

$n=2$のとき、$\tilde{u}$は有界ではないことを示せ。

\dotfill

調和関数$\Phi(x)$は$n=2$において
$\Phi(x)=-\frac{1}{2\pi}\log{\lvert x \rvert}$
である。

$\lvert x \rvert$がそれぞれ$0$と$\infty$に飛ばした場合、
$\Phi(x)\to \infty \ (\lvert x \rvert \to 0)$
と
$\Phi(x)\to -\infty \ (\lvert x \rvert \to\infty)$
であるので、
$\lvert \Phi(x) \rvert \to \infty$
である。

\begin{equation}
 \tilde{u}
  =\int_{\mathbb{R}^2}\Phi(x-y)f(y)dy
  =\int_{\mathbb{R}^2}\Phi(y)f(x-y)dy
\end{equation}



\hrulefill

\textbf{Report 1.14}

$n=2,\ N=3$のとき、
次の式を示せ。
\begin{equation}
 u(x) - \sum_{k=0}^{N-1}\sum_{\lvert \alpha \rvert =k}
  \frac{D^{\alpha}u(x_{0})(x-x_{0})^{\alpha}}{\alpha !}
  =\sum_{\lvert \alpha \rvert =N}
  \frac{D^{\alpha}u(x_{0}+t(x-x_{0}))(x-x_{0})^{\alpha}}{\alpha !}
\end{equation}


\dotfill

\hrulefill


%%%%%
\newpage
%%%%%


\hrulefill

\begin{enumerate}
 \item


      次の初期値問題の解である関数$u$を求めよ。

      \begin{equation}
       \begin{cases}
        u_{t} +b\cdot Du + cu =0 & \text{in}\: \mathbb{R}^n \times (0,\infty)\\
        u=g & \text{on}\: \mathbb{R}^n \times \{t=0\}
       \end{cases}
      \end{equation}

      $c\in\mathbb{R}$と$b\in\mathbb{R}^{n}$は定数とする。

      \dotfill


      式を次のように変形する。
      \begin{equation}
       u_{t} +b\cdot Du + cu =0
        \: \Rightarrow \:
        u_{t} +b\cdot Du = - cu
      \end{equation}

      ここから、左辺は$u$を微分すると
      $u$の定数倍になることが読み取れる。

      \begin{equation}
       u
        %=u(x_{1},\dots,x_{n},t)
        =\exp\left(
                \frac{c}{n+1}\left( t + \sum_{i=1}^{n}\frac{x_{i}}{b_{i}} \right)
              \right)
      \end{equation}
      


      \begin{equation}
       u(x,t) = g(x-tb) \quad (x\in\mathbb{R}^{n},\: t\geq 0)
      \end{equation}

      \hrulefill


 \item

      Laplace 方程式 $\Delta u =0$は
      回転不変である、
      つまり
      $n$次直交行列$O$で変換しても
      Laplace 方程式を満たすこと
      を示せ。

      \begin{equation}
       v(x) = u(Ox) \quad (x\in\mathbb{R}^{n})
        \Rightarrow
        \Delta v = 0
      \end{equation}

      \dotfill

      $O$を$n$次直交行列とし、$O=\{o_{ij}\}$とする。
      $O$は直交行列であるので、
      $O$の転置行列と逆行列が一致する。
      \begin{equation}
       O{}^{t}O = {}^{t}O O = E
        ,\quad
        \sum_{i=1}^{n}o_{ki}o_{li} = \delta_{kl}
        =
        \begin{cases}
         1 & (k=l)\\
         0 & (k\ne l)
        \end{cases}
      \end{equation}
      $x$はベクトルであり、その成分を$x=(x_{1},\dots,x_{n})$とし、
      $\bar{x} = Ox$とする。
      これにより、$v(x)=u(\bar{x})$となる。

      $x_{i}$における偏微分は合成関数の微分より次のように変形できる。
      \begin{equation}
       \frac{\partial}{\partial x_{i}}
        = \sum_{k=1}^{n}\frac{\partial \bar{x}_{k}}{\partial x_{i}}
              \frac{\partial}{\partial \bar{x}_{k}}
        = \sum_{k=1}^{n} o_{ki}\frac{\partial}{\partial \bar{x}_{k}}
      \end{equation}

      2階の偏微分は次のように求められる。
      \begin{equation}
       \frac{\partial^{2}}{\partial x_{i}^{2}}
        =\left( \sum_{k=1}^{n} o_{ki}\frac{\partial}{\partial \bar{x}_{k}} \right) \cdot
        \left( \sum_{l=1}^{n} o_{li}\frac{\partial}{\partial \bar{x}_{l}} \right)
        =
        \sum_{k,l = 1}^{n} o_{ki}o_{li}\frac{\partial^{2}}{\partial \bar{x}_{k}\partial \bar{x}_{l}}
      \end{equation}

      これらを用いて
      $\Delta v$ を計算する。
      \begin{align}
       \Delta v
         =& \Delta u(\bar{x})
         = \sum_{i=1}^{n} \frac{\partial^2}{\partial x_{i}^{2}}u(\bar{x})
         = \sum_{i=1}^{n} \sum_{k,l = 1}^{n} o_{ki}o_{li}
           \frac{\partial^{2}}{\partial \bar{x}_{k}\partial \bar{x}_{l}} u(\bar{x})\\
         =& \sum_{k,l = 1}^{n} \left( \sum_{i=1}^{n} o_{ki}o_{li} \right)
           \frac{\partial^{2}}{\partial \bar{x}_{k}\partial \bar{x}_{l}} u(\bar{x})
         = \sum_{k,l = 1}^{n} \delta_{kl}
           \frac{\partial^{2}}{\partial \bar{x}_{k}\partial \bar{x}_{l}} u(\bar{x})\\
         =& \sum_{k = 1}^{n}
           \frac{\partial^{2}}{\partial \bar{x}_{k}^{2}} u(\bar{x})
         = \Delta u
      \end{align}

      よって、$\Delta u=0$であれば、$\Delta v =0$である。

      \hrulefill


 \item

      平均値定理の証明を利用し、
      $n\geq 3$の時、次の式を証明せよ。
      \begin{equation}
       \begin{split}
       u(0) =& \frac{1}{n\alpha(n)r^{n-1}} \int_{\partial B(0,r)} g dS\\
         &+ \frac{1}{n(n-2)\alpha(n)} \int_{B(0,r)}
         \left( \frac{1}{\lvert x \rvert^{n-2}}
          -  \frac{1}{r^{n-2}}\right)  f dx
       \end{split}
      \end{equation}

      \begin{equation}
       \begin{cases}
        -\Delta u =f & \text{in}\: B^{0}(0,r)\\
        u=g & \text{on}\: \partial B^{0}(0,r)
       \end{cases}
      \end{equation}

      \dotfill



      \hrulefill

 \item

      関数$u \in C^{2}(U) \cap C(\overline{U})$が
      開集合$U$の境界上を除いて調和的である時、
      次が成り立つことを示せ。
      \begin{equation}
       \underset{\overline{U}}{\max {u}}
        = \underset{\partial{U}}{\max {u}}
      \end{equation}

      HINT :
      $\varepsilon >0$の時
      $u_{\varepsilon}=u+\varepsilon \lvert x \rvert^{2}$とおくと、
      $U$の内部では最大値を取らないことを示せばよい。

      \dotfill



      \hrulefill

 \item

      次のような$v\in C^{2}(\overline{U})$を 劣調和関数 (subharmonic) という。
      \begin{equation}
       -\Delta v \leq 0 \quad \text{in}\: U
      \end{equation}


      \begin{enumerate}
       \item

            $v$が次を満たすことを示せ。
            \begin{equation}
             v(x) \leq \frac{1}{\alpha(n)r^{n}}
              \int_{B(x,r)} v dy
              \quad \text{for all} B(x,r) \subset U
            \end{equation}

            \dotfill



            \hrulefill

       \item

            次を示せ。
            \begin{equation}
             \underset{\overline{U}}{\max {u}}
              = \underset{\partial{U}}{\max {u}}
            \end{equation}

            \dotfill



            \hrulefill

       \item

            $\phi : \mathbb{R}\to\mathbb{R}$
            は
            なめらかな凸関数とする。

            $u$は調和関数、
            $v = \phi(u)$
            とした時、
            $v$は劣調和関数であることを示せ。

            \dotfill



            \hrulefill

       \item

            $u$が調和的である時、
            $v = \lvert Du \rvert^{2}$
            は劣調和的であることを示せ。

            \dotfill



            \hrulefill

      \end{enumerate}



 \item


      開集合$U \subset \mathbb{R}^{n}$は有界であるとする。

      この時、$U$のみに依存した定数$C$が存在し、次の式が成り立つことを示せ。
      \begin{equation}
       \underset{\overline{U}}{\max {\lvert u \rvert}}
        \leq
        C \left(
        \underset{\partial{U}}{\max {\lvert g \rvert}}
        +
        \underset{\overline{U}}{\max {\lvert f \rvert}}
        \right)
      \end{equation}


      なお、$u$は滑らかな関数で、次の解である。
      \begin{equation}
       \begin{cases}
        -\Delta u =f & \text{in}\: U\\
        u=g & \text{on}\: \partial U
       \end{cases}
      \end{equation}


      HINT :
      $\lambda = \underset{\overline{U}}{\max {\lvert f \rvert}}$
      に対して、
      $-\Delta \left(u + \frac{\lvert x \rvert^{2}}{2n}\lambda \right)
      \leq 0$

      \dotfill



      \hrulefill


\end{enumerate}

\hrulefill




%%%%%
\newpage
%%%%%


\hrulefill


\begin{enumerate}
 \item [2.1.]

       $u:U\to\mathbb{R}$の
       第$\gamma$ \ruby{H\"{o}lder}{ヘルダー} 半ノルム
       を次で定義する。
       \begin{equation}
        [u]_{C^{0,\gamma}(\overline{U})}
          =
          \sup_{\substack{x,y\in U \\ x \ne y}}
          \left\{
           \frac{\lvert u(x)-u(y) \rvert}{\lvert x-y \rvert^{\gamma}}
          \right\}
       \end{equation}

       この定義は半ノルムであることを確認せよ。

       \dotfill

       半ノルムとは
       絶対斉次性($p(\lambda x) = \lvert \lambda \rvert p(x)$)
       と
       劣加法性($p(x+y)\leq p(x)+p(y)$)
       を満たす写像$p$のことをいう。

       ${}^{\forall}\lambda \in \mathbb{R}$とする。
       \begin{align}
        [\lambda u]_{C^{0,\gamma}(\overline{U})}
         =&
          \sup_{\substack{x,y\in U \\ x \ne y}}
          \left\{
           \frac{\lvert \lambda u(x)- \lambda u(y) \rvert}{\lvert x-y \rvert^{\gamma}}
          \right\}\\
          =&
          \lvert \lambda \rvert \sup_{\substack{x,y\in U \\ x \ne y}}
          \left\{
           \frac{\lvert u(x)- u(y) \rvert}{\lvert x-y \rvert^{\gamma}}
          \right\}
          =
          \lvert \lambda \rvert [u]_{C^{0,\gamma}(\overline{U})}
       \end{align}
       よって、絶対斉次性を満たす。

       劣加法性を確認する。
       \begin{align}
        [u+v]_{C^{0,\gamma}(\overline{U})}
         =&
          \sup_{\substack{x,y\in U \\ x \ne y}}
          \left\{
           \frac{\lvert (u(x)+v(x))- (u(y)+v(y)) \rvert}{\lvert x-y \rvert^{\gamma}}
          \right\}\\
          \leq &
          \sup_{\substack{x,y\in U \\ x \ne y}}
          \left\{
           \frac{\lvert u(x)- u(y) \rvert + \lvert v(x)-v(y) \rvert}{\lvert x-y \rvert^{\gamma}}
          \right\}\\
          \leq &
          \sup_{\substack{x,y\in U \\ x \ne y}}
          \left\{
           \frac{\lvert u(x)- u(y) \rvert}{\lvert x-y \rvert^{\gamma}}
          \right\}
        +
          \sup_{\substack{x,y\in U \\ x \ne y}}
          \left\{
           \frac{ \lvert v(x)-v(y) \rvert}{\lvert x-y \rvert^{\gamma}}
          \right\}\\
        =&
        [u]_{C^{0,\gamma}(\overline{U})}
        +
        [v]_{C^{0,\gamma}(\overline{U})}
       \end{align}

       以上により半ノルムであることが確認できる。

       \hrulefill

 \item [2.3.]

       $U$上の滑らかな関数全体を
       $C^{\infty}_{c}(U) \subset W^{k,p}(U)$とし、
       これの閉包を$W_{0}^{k,p}(U)$とする。
       $W_{0}^{k,p}(U)$は
       $\lvert \alpha \rvert \leq k-1$を満たす$\alpha$において
       $\partial U$上で
       $D^{\alpha}u =0$
       となる関数$u\in W^{k,p}(U)$の集まりである。

       定理17
       を認めて
       これを示せ。

       \dotfill



       \hrulefill

 \item [2.5.]

       $\{r_{k}\}_{k=1}^{\infty}$は可算で稠密な
       $U=B^{0}(0,1)$の部分集合とする。
       \begin{equation}
        u(x) =
         \sum_{k=1}^{\infty} \frac{1}{2^{k}} \lvert x-r_{k} \rvert ^{-\alpha}
         \quad
         (x\in U)
       \end{equation}

       $0 \leq \alpha \leq \frac{n-p}{p}$であれば
       $u\in W^{1,p}(U)$である。

       この時、
       $u$は
       $U$の部分集合である開集合上で
       有界ではないことを示せ。
       

       \dotfill

       \hrulefill

 \item [2.7.]


       \textbf{弱微分の性質}
       
       $u\in W^{k,p}(U),\: \lvert \alpha \rvert \leq k$
       とする。

       このとき、$V\subset U$が開集合であるなら
       $u\in W^{k,p}(V)$となることを示せ。

       \dotfill

       \hrulefill

 \item [2.9.]

       $W^{k,p}(U)$が完備であることを示せばよい。
       そこで、
       $\{u_{m}\}_{m=1}^{\infty}$が
       $W^{k,p}(U)$の\ruby{Cauchy}{コーシー}列とする。

       この事実が成り立つことを示せ。

       \dotfill

       \hrulefill

\end{enumerate}

\hrulefill



\end{document}
