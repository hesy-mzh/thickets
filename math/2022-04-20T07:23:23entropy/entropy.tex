\documentclass[12pt,b5paper]{ltjsarticle}

%\usepackage[margin=15truemm, top=5truemm, bottom=5truemm]{geometry}
\usepackage[margin=15truemm]{geometry}

\usepackage{amsmath,amssymb}
%\pagestyle{headings}
\pagestyle{empty}

%\usepackage{listings,url}
\renewcommand{\theenumi}{(\arabic{enumi})}

\usepackage{graphicx}

\usepackage{tikz}
\usetikzlibrary {arrows.meta}
\usepackage{wrapfig}	% required for `\wrapfigure' (yatex added)

%% 像Im を定義
%\newcommand{\Img}{\mathop{\mathrm{Im}}\nolimits}

\begin{document}

\textbf{エントロピー}

\begin{enumerate}
 \item $H(A) \geq 0$ ( $p(a_j)=1$となる$a_j$が存在する時、等号が成立)
 \item $H(A) \leq \log_{2}{m}$ ($p(a_1)=p(a_2)=\cdots =p(a_m)=\frac{1}{m}$の時、等号が成立)
\end{enumerate}

\dotfill

$A=\{a_1,a_2,\dots,a_m\}$ : 事象の集合

$\Omega$ : 標本空間

$p : \Omega \rightarrow [0,1] \subset \mathbb{R}$ : 確率関数
\begin{itemize}
 \item $0 \leq p(a_i) \leq 1 \; (i=1,2,\dots,m)$ : 各確率は0から1の間をとる
 \item $\log_{2}{p(a_i)} \leq 0 \; (i=1,2,\dots,m)$ : 1でない確率の対数は負の数
 \item $0 \leq \sum_{i=1}^{m} p(a_i) \leq 1$ : 全ての確率の和は1を超えない
\end{itemize}

$H : 2^{\Omega} \rightarrow \mathbb{R}$ : エントロピー関数


$H(A) = \sum_{i=1}^{m} p(a_i) \left( -\log_{2}{p(a_i)} \right)$
: $-\log_{2}{p(a_i)}$の期待値

\hrulefill

\textbf{証明}

$H(A) \geq 0$ ( $p(a_j)=1$となる$a_j$が存在する時、等号が成立)

\textbf{等号の場合}

$p(a_j)=1$となる$j$が存在する場合、$i\ne j$であるとき$p(a_i)=0$である。
これを$H(A)$に当てはめると次の式を得る。
\begin{align}
 H(A) &= \sum_{i=1}^{m} p(a_i) \left( -\log_{2}{p(a_i)} \right)\\
 &= 0(-\log_{2}{0}) + \cdots + 1(-\log_{2}{1}) + \cdots + 0(-\log_{2}{0})
\end{align}
$\log_{2}{1}=0$より$j$番目の項は$0$である。
それ以外の項については次の式から$0$とする。
\begin{equation}
 \lim_{x\rightarrow 0} x\log x =0
\end{equation}
これにより$H(A)=0$である。

\textbf{等号でない場合}

$0< p(a_j) <1$の確率がある場合、
$\log_{2}{p(a_j)}<0$であるから、$p(a_j)(-\log_{2}{p(a_j)})>0$となる。

$p(a_j)=0$または$p(a_j)=1$ のときは
$p(a_j)(-\log_{2}{p(a_j)})=0$である。

よって
$H(A) = \sum_{i=1}^{m} p(a_i) \left( -\log_{2}{p(a_i)} \right)$
の全ての項は0以上となるので
$H(A)>0$


\dotfill

$H(A) \leq \log_{2}{m}$ ($p(a_1)=p(a_2)=\cdots =p(a_m)=\frac{1}{m}$の時、等号が成立)

\textbf{等号の場合}

$p(a_i)=1/m \;(i=1,\dots,m)$
を$H(A)$に代入する
\begin{align}
 H(A) &= \sum_{i=1}^{m} p(a_i) \left( -\log_{2}{p(a_i)} \right)\\
 &= \sum_{i=1}^{m} \frac{1}{m} \left( -\log_{2}{\frac{1}{m}} \right)
 = \log_{2}{m}
\end{align}

\textbf{等号でない場合}

$H(A)$の最大値を求め、この最大値が$\log_{2}{m}$であることを示す。

\dotfill
[ラグランジュの未定係数法]
\dotfill

$\lambda$を定数、$\sum_{i=1}^{m}x_i =1$とし次の関数$F$の最大値について考える。
\begin{equation}
 F(x_1,x_2,\dots,x_m) = -\sum_{i=1}^{m}x_i\log_{2}{x_i} + \lambda \left(1- \sum_{i=1}^{m}x_i\right)
\end{equation}

$F(x_1,x_2,\dots,x_m)$を各$x_i$について偏微分する。
\begin{equation}
 \frac{\partial}{\partial x_i} F(x_1,x_2,\dots,x_m) = -\log_2{x_i} -\frac{1}{\log{2}} -\lambda
  \quad (i=1,\dots,m)
\end{equation}

最大値を考えるので偏導関数が0となる時を考える。
\begin{equation}
 -\log_2{x_i} -\frac{1}{\log{2}} -\lambda = 0
  \quad (i=1,\dots,m)
\end{equation}
これを$x_i$について解くと次が得られる。
\begin{equation}
 x_i = \exp (-1-\lambda \log{2})
  \quad (i=1,\dots,m)\label{064805_20Apr22}
\end{equation}
$\lambda$は定数であるので右辺は$i$によらずある値を指す。
つまり各$i=1,\dots,m$において偏導関数が0となるときはいつも同じ値となる。
\begin{equation}
 x_1=\cdots = x_m = \exp (-1-\lambda \log{2})
\end{equation}

これを$\sum_{i=1}^{m}x_i=1$に代入し$\lambda$について解くと次のようになる。
\begin{gather}
 \sum_{i=1}^{m} \exp (-1-\lambda \log{2})=1\\
 m \cdot \exp (-1-\lambda \log{2})=1\\
 m = \exp (1+\lambda \log{2})\\
 \log{m} = 1+\lambda \log{2}\\
 \lambda = \frac{1}{\log{2}} \left( \log{m}-1 \right)\label{064741_20Apr22}
\end{gather}

この式(\ref{064741_20Apr22})
を
式(\ref{064805_20Apr22})に代入すると
$x_i=m^{-1}$が得られる。

また、次の2式からこの極値の候補は最大値であることがわかる。
\begin{gather}
 \frac{\partial^2}{\partial x_i\partial x_j} F(x_1,x_2,\dots,x_m) = 0 \quad (i\ne j) \\
 \frac{\partial^2}{\partial x_i^2} F(x_1,x_2,\dots,x_m)
 = -\frac{1}{x_i\log{2}} 
 = -\frac{m}{\log{2}} <0
\end{gather}

よって、$x_1=\cdots=x_m=m^{-1}$の時最大値を取る。

\dotfill
[ラグランジュの未定係数法 ここまで]
\dotfill

上記の 未定係数法から
$p(a_1)=\cdots=p(x_m)=1/m$である時、$H(A)$が最大値を取ることがわかる。

これにより$H(A)$は最大値$\log_{2}{m}$より小さくなるため
$H(A)<\log_{2}{m}$である。


\textbf{等号の場合}
と
\textbf{等号でない場合}
を合わせ
\begin{equation}
 H(A) \leq \log_{2}{m}
\end{equation}
がわかる。
\end{document}
