\documentclass[12pt,b5paper]{ltjsarticle}

%\usepackage[margin=15truemm, top=5truemm, bottom=5truemm]{geometry}
\usepackage[margin=15truemm]{geometry}

\usepackage{amsmath,amssymb}
%\pagestyle{headings}
\pagestyle{empty}

%\usepackage{listings,url}
%\renewcommand{\theenumi}{(\arabic{enumi})}

\usepackage{graphicx}

\usepackage{tikz}
\usetikzlibrary {arrows.meta}
\usepackage{wrapfig}	% required for `\wrapfigure' (yatex added)
\usepackage{bm}	% required for `\bm' (yatex added)

% ルビを振る
\usepackage{luatexja-ruby}	% required for `\ruby'

%% 核Ker 像Im Hom を定義
%\newcommand{\Img}{\mathop{\mathrm{Im}}\nolimits}
%\newcommand{\Ker}{\mathop{\mathrm{Ker}}\nolimits}
%\newcommand{\Hom}{\mathop{\mathrm{Hom}}\nolimits}
\newcommand{\Rot}{\mathop{\mathrm{rot}}\nolimits}
\newcommand{\Div}{\mathop{\mathrm{div}}\nolimits}

\begin{document}

\hrulefill

\textbf{発散}

\begin{equation}
 \bm{f} : \mathbb{R}^2 \to \mathbb{R}^2, \quad (x,y) \mapsto (f_1(x,y),f_2(x,y))
\end{equation}


\begin{equation}
 \Div \bm{f} = \frac{\partial f_1}{\partial x} + \frac{\partial f_2}{\partial y}
\end{equation}

\dotfill

$r>0$、$\bm{x}\in\mathbb{R}^2$について
開球$B(\bm{x},r)$を次のように定義する。
\begin{equation}
 B(\bm{x},r) = \{ \bm{p}\in\mathbb{R}^2 \mid \lvert \bm{p}-\bm{x} \rvert < r \}
\end{equation}

また、開球の閉包$\bar{B}(\bm{x},r)$は次のようになる。
\begin{equation}
 \bar{B}(\bm{x},r) = \{ \bm{p}\in\mathbb{R}^2 \mid \lvert \bm{p}-\bm{x} \rvert \leq r \}
\end{equation}

この場合、$B(\bm{x},r)$は点$\bm{x}$を中心とした半径$r$の円の内部で境界を含めない集合であり、
$\bar{B}(\bm{x},r)$は境界を含めた集合である。

\dotfill
\textbf{\ruby{Lebesgue}{ルベーグ}の微分定理}
\dotfill

$U \subset \mathbb{R}^2$を開集合とし、
$f:U\to\mathbb{R}$を$C^1$-級関数とする。

$\bar{B}(\bm{x},r)\subset U$である時、
\begin{equation}
 \lim_{r\rightarrow 0}
  \left(
   \frac{1}{\lvert\bar{B}(\bm{x},r)\rvert} \int_{\bar{B}(\bm{x},r)}f(x,y)\mathrm{d}x\mathrm{d}y
  \right)
  =
  f(\bm{x})
  \label{lebesgue}
\end{equation}
となる。

ただし、
$\lvert\bar{B}(\bm{x},r)\rvert$は$\bar{B}(\bm{x},r)$の面積を指す。
つまり、$\lvert\bar{B}(\bm{x},r)\rvert=\pi r^2$


%%%%%%%%%%
\hrulefill
%%%%%%%%%%



$D\subset\mathbb{R}^2$を開集合とする。
$T>0$とし、区間$I_T\subset\mathbb{R}$を$I_T=[0,T]$とする。

$D_T\subset\mathbb{R}^3$を$D_T=D\times I_T$とし、
$C^1$-級関数$\rho , X$を次のように定める。
\begin{equation}
 \rho : D_T \to \mathbb{R},\quad X : D_T \to \mathbb{R}^2
\end{equation}

各$t\in I_T$に対し、
関数$\rho_t,\dot{\rho}_t,X_t$を次のように定める。
\begin{align}
 \rho_t :& D\to\mathbb{R} , \quad (x,y) \mapsto \rho(x,y,t)\\
 \dot{\rho}_t :& D\to\mathbb{R} ,
 \quad (x,y) \mapsto \lim_{h\rightarrow 0}\frac{1}{h}(\rho(x,y,t+h)-\rho(x,y,t))\\
 X_t :& D\to\mathbb{R}^2 , \quad (x,y) \mapsto X(x,y,t)
\end{align}


ここで、次を仮定する。
\begin{center}
 \fbox{
 \begin{minipage}{350pt}
  各$t\in I_T$と
  $\bar{B}(\bm{x},r) \subset D$となるような
  $\bm{x}\in D,\, r>0$に対して
  次が成り立つ。
  \begin{equation}
   \frac{\mathrm{d}}{\mathrm{d}t}\int_{\bar{B}(\bm{x},r)}\rho_t \mathrm{d}x\mathrm{d}y
    =- \int_{\partial\bar{B}(\bm{x},r)}\rho_t \langle X_t, \bm{n} \rangle \mathrm{d}s
    \label{joken}
  \end{equation}
 \end{minipage}
 }
\end{center}

この時、次が成り立つことを示せ。

\begin{center}
 \fbox{
 \begin{minipage}{350pt}
  各$t\in I_T$に対して、
  $D$上の関数$\rho_t, \dot{\rho}_t$及びベクトル場$X_t$は次の式を満たす。
  \begin{equation}
   \dot{\rho}_t =- \Div(\rho_t X_t)
  \end{equation}
 \end{minipage}
 }
\end{center}

\dotfill

式(\ref{joken})の両辺を$\lvert\bar{B}(\bm{x},r)\rvert$で割る。
\begin{equation}
  \frac{1}{\lvert\bar{B}(\bm{x},r)\rvert}\frac{\mathrm{d}}{\mathrm{d}t}\int_{\bar{B}(\bm{x},r)}\rho_t \mathrm{d}x\mathrm{d}y
  =- \frac{1}{\lvert\bar{B}(\bm{x},r)\rvert}\int_{\partial\bar{B}(\bm{x},r)}\rho_t \langle X_t, \bm{n} \rangle \mathrm{d}s
  \label{eq01}
\end{equation}

この時、左辺はルベーグの微分定理(式(\ref{lebesgue}))により
次のように変形できる。
\begin{equation}
 \lim_{r\rightarrow 0} \frac{1}{\lvert\bar{B}(\bm{x},r)\rvert}\frac{\mathrm{d}}{\mathrm{d}t}
  \int_{\bar{B}(\bm{x},r)}\rho_t \mathrm{d}x\mathrm{d}y
  = \frac{\mathrm{d}}{\mathrm{d}t}\rho_t
  = \dot{\rho}_t
\end{equation}

この為、
式(\ref{eq01})の右辺が
次のようになれば証明終了である。
\begin{equation}
 \lim_{r\rightarrow 0}
 \frac{1}{\lvert\bar{B}(\bm{x},r)\rvert}
 \int_{\partial\bar{B}(\bm{x},r)}\rho_t \langle X_t, \bm{n} \rangle \mathrm{d}s
 =
 \Div(\rho_t X_t)
 \label{eq02}
\end{equation}


\hrulefill

式(\ref{eq02})を示す為にどのような方針を取るか?
\begin{enumerate}
 \item
      ルベーグの微分定理(\ref{lebesgue})を用いる

 \item
      $\partial\bar{B}(\bm{x},r)$が中心$\bm{x}$半径$r$の円であるので、
      これを極座標表示し積分を計算する

\end{enumerate}

\dotfill

微分定理を用いる場合、次のような変形を行いたい。
\begin{equation}
 \int_{\partial\bar{B}(\bm{x},r)}\rho_t \langle X_t, \bm{n} \rangle \mathrm{d}s
  = \int_{\bar{B}(\bm{x},r)} \fbox{\phantom{\phantom{I}\hspace{20pt}}}\ \mathrm{d}x\mathrm{d}y
\end{equation}

その為に次のように$X_t,\bm{n}$を置く。
\begin{equation}
 X_t(x,y)=\begin{pmatrix}X_{t1}(x,y)\\X_{t2}(x,y)\end{pmatrix}
 \quad
 \bm{n}=\begin{pmatrix}\frac{\mathrm{d}x}{\mathrm{d}s}\\\frac{\mathrm{d}y}{\mathrm{d}s}\end{pmatrix}
\end{equation}

これにより
\begin{align}
  \int_{\partial\bar{B}(\bm{x},r)}\rho_t \langle X_t, \bm{n} \rangle \mathrm{d}s
 =& \int_{\partial\bar{B}(\bm{x},r)}\rho_t \left( X_{t1}(x,y)\frac{\mathrm{d}x}{\mathrm{d}s} + X_{t2}(x,y)\frac{\mathrm{d}y}{\mathrm{d}s} \right) \mathrm{d}s\\
 =& \int_{\partial\bar{B}(\bm{x},r)} \left( \rho_t(x,y)X_{t1}(x,y)\mathrm{d}x + \rho_t(x,y)X_{t2}(x,y)\mathrm{d}y \right)
\end{align}

ここにグリーンの定理を用いると次のような式が得られる。
\begin{equation}
 \int_{\bar{B}(\bm{x},r)} \left( \frac{\partial}{\partial x}\rho_t(x,y)X_{t2}(x,y) - \frac{\partial}{\partial y}\rho_t(x,y)X_{t1}(x,y) \right) \mathrm{d}x  \mathrm{d}y
\end{equation}
よって、(\ref{eq01})の右辺は微分定理より次のようになる。
\begin{equation}
 \lim_{r\rightarrow 0}
 \frac{1}{\lvert\bar{B}(\bm{x},r)\rvert}
 \int_{\partial\bar{B}(\bm{x},r)}\rho_t \langle X_t, \bm{n} \rangle \mathrm{d}s
 =\frac{\partial}{\partial x}\rho_t(\bm{x})X_{t2}(\bm{x}) - \frac{\partial}{\partial y}\rho_t(\bm{x})X_{t1}(\bm{x})
\end{equation}

この式の右辺が次の式と一致すればいい。

\begin{equation}
 \Div(\rho_t X_t) = \nabla \rho_t \cdot X_t + \rho_t \Div X_t
\end{equation}



\dotfill



\hrulefill






\begin{align}
 \Div(\rho_t X_t) =& \left( \frac{\partial}{\partial x}\rho_t, \frac{\partial}{\partial y}\rho_t  \right) \cdot X_t + \rho_t \Div X_t\\
 =& \frac{\partial\rho_t}{\partial x} X_{t1} + \frac{\partial\rho_t}{\partial y}X_{t2} +
 \rho_t\frac{\partial X_{t1}}{\partial x} + \rho_t\frac{\partial X_{t2}}{\partial y}
\end{align}


\begin{equation}
 \lim_{r\rightarrow 0}
 \frac{1}{\lvert\bar{B}(\bm{x},r)\rvert}
 \int_{\bar{B}(\bm{x},r)} \Div(\rho_t X_t) \mathrm{d}x\mathrm{d}y
 = \Div(\rho_t X_t)
\end{equation}


\begin{equation}
 X_t = \begin{pmatrix}X_{t1}(x,y)\\X_{t2}(x,y)\end{pmatrix}
 \qquad
 \bm{n} =
 \begin{pmatrix}\frac{\mathrm{d}x}{\mathrm{d}s}\\ \frac{\mathrm{d}y}{\mathrm{d}s}\end{pmatrix}
\end{equation}

\begin{equation}
 \mathrm{d}s = \sqrt{x^\prime(t)^2 + y^\prime(t)^2}\mathrm{d}t
\end{equation}


\begin{align}
 \int_{\partial\bar{B}(\bm{x},r)}\rho_t \langle X_t, \bm{n} \rangle \mathrm{d}s
  =&
 \int_{\partial\bar{B}(\bm{x},r)}\left(
  \rho_t X_{t1} \frac{\mathrm{d}x}{\mathrm{d}s} + \rho_t X_{t2} \frac{\mathrm{d}y}{\mathrm{d}s}
 \right) \mathrm{d}s\\
 =&
 \int_{\partial\bar{B}(\bm{x},r)}\left(
  \rho_t X_{t1} \mathrm{d}x + \rho_t X_{t2} \mathrm{d}y
 \right)\\
 =&
 \int_{\bar{B}(\bm{x},r)}
 \left(
  \frac{\partial}{\partial x}\rho_t X_{t2}
   -
  \frac{\partial}{\partial y}\rho_t X_{t1}
 \right)
 \mathrm{d}x \mathrm{d}y
\end{align}

\ruby{Green}{グリーン}の定理
\begin{equation}
 \int_{\partial B} \left( P\mathrm{d}x + Q\mathrm{d}y \right)
  =\int_{B} \left( \frac{\partial Q}{\partial x} - \frac{\partial P}{\partial y}\right)\mathrm{d}x\,\mathrm{d}y
\end{equation}







\end{document}
