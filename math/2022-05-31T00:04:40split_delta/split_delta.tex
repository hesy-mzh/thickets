\documentclass[12pt,b5paper]{ltjsarticle}

%\usepackage[margin=15truemm, top=5truemm, bottom=5truemm]{geometry}
\usepackage[margin=15truemm]{geometry}

\usepackage{amsmath,amssymb}
%\pagestyle{headings}
\pagestyle{empty}

%\usepackage{listings,url}
%\renewcommand{\theenumi}{(\arabic{enumi})}

\usepackage{graphicx}

\usepackage{tikz}
\usetikzlibrary {arrows.meta}
\usepackage{wrapfig}	% required for `\wrapfigure' (yatex added)
\usepackage{bm}	% required for `\bm' (yatex added)
\usepackage{luatexja-ruby}	% required for `\ruby'
%% 像Im を定義
%\newcommand{\Img}{\mathop{\mathrm{Im}}\nolimits}

\begin{document}

%\begin{equation}
% \bar{s}(\tilde{f}, \Delta_{ij}^{\prime\prime})
%  \leq
% \bar{s}(\tilde{f}, \Delta_{ij}^{\prime})
%% \underbar{s}
%\end{equation}

\begin{equation}
 \bar{s}(\tilde{f}, \Delta^{\prime\prime})
  \leq
 \bar{s}(\tilde{f}, \Delta^{\prime})
\end{equation}

$\Delta^{\prime}, \ \Delta^{\prime\prime}$は分割であり、
$\Delta^{\prime\prime}$は$\Delta^{\prime}$を更に分割したものである。

\hrulefill

\begin{equation}
 \bar{s}(\tilde{f}, \Delta_{ij})
  =
  \sum_{i=1}^{m}
  \sum_{j=1}^{n}
  (\sup_{\Delta_{ij}}\tilde{f})\times \lvert \Delta_{ij} \rvert
\end{equation}

\hrulefill

分割$\Delta^{\prime}$の中から一つ取り出し、これを$\Delta_{ij}^{\prime}$とする。

$\Delta_{ij}^{\prime}$は$\Delta^{\prime\prime}$では1つ以上に分割されている。
この時、2つの分割の関係は次のようになる。
\begin{equation}
 \Delta_{ij}^{\prime} = \bigcup_{k} \Delta_{k}^{\prime\prime}
  ,\
  \lvert \Delta_{ij}^{\prime} \rvert = \sum_{k} \lvert\Delta_{k}^{\prime\prime}\rvert
\end{equation}


%$\displaystyle \sup_{\Delta_{ij}^{\prime}}\tilde{f}$が表す値は
%各$\displaystyle \sup_{\Delta_{k}^{\prime\prime}}\tilde{f}$
%の値と一致するものが一つは存在する。

$\Delta_{ij}^{\prime}$の
分割の一つと比較すると次のような包含関係がある。
\begin{equation}
 \Delta_{k}^{\prime\prime}
  \subset
  \Delta_{ij}^{\prime}
\end{equation}
狭い領域での上限は広い領域での上限より小さくなる為、
包含関係から次の式が得られる。
\begin{equation}
 \sup_{\Delta_{k}^{\prime\prime}}\tilde{f}
  \leq
  \sup_{\Delta_{ij}^{\prime}}\tilde{f}
\end{equation}

この不等式から次の不等式が得られる。
\begin{align}
 \sup_{\Delta_{ij}^{\prime}}\tilde{f} \times \lvert \Delta_{ij}^{\prime} \rvert
 =& \sup_{\Delta_{ij}^{\prime}}\tilde{f} \times \sum_{k} \lvert\Delta_{k}^{\prime\prime}\rvert\\
 =& \sum_{k} \left(
 \sup_{\Delta_{ij}^{\prime}}\tilde{f} \times \lvert\Delta_{k}^{\prime\prime}\rvert
 \right)\\
 \geq & \sum_{k} \left(
 \sup_{\Delta_{k}^{\prime\prime}}\tilde{f} \times \lvert\Delta_{k}^{\prime\prime}\rvert
 \right)
\end{align}

任意の$\Delta_{ij}^{\prime}$に対してこの不等式が成り立つので、
分割$\Delta^{\prime}$とより細かい分割$\Delta^{\prime\prime}$についても
次の式が成り立つ。

\begin{equation}
 \bar{s}(\tilde{f}, \Delta^{\prime\prime})
  \leq
 \bar{s}(\tilde{f}, \Delta^{\prime})
\end{equation}





\end{document}
