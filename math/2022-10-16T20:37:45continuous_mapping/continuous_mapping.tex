\documentclass[12pt,b5paper]{ltjsarticle}

%\usepackage[margin=15truemm, top=5truemm, bottom=5truemm]{geometry}
\usepackage[margin=10truemm]{geometry}

\usepackage{amsmath,amssymb}
%\pagestyle{headings}
\pagestyle{empty}

%\usepackage{listings,url}
%\renewcommand{\theenumi}{(\arabic{enumi})}

%\usepackage{graphicx}

%\usepackage{tikz}
%\usetikzlibrary {arrows.meta}
%\usepackage{wrapfig}	% required for `\wrapfigure' (yatex added)
%\usepackage{bm}	% required for `\bm' (yatex added)

% ルビを振る
%\usepackage{luatexja-ruby}	% required for `\ruby'

%% 核Ker 像Im Hom を定義
%\newcommand{\Img}{\mathop{\mathrm{Im}}\nolimits}
%\newcommand{\Ker}{\mathop{\mathrm{Ker}}\nolimits}
%\newcommand{\Hom}{\mathop{\mathrm{Hom}}\nolimits}

%\DeclareMathOperator{\Rot}{rot}
%\DeclareMathOperator{\Div}{div}
%\DeclareMathOperator{\Grad}{grad}
%\DeclareMathOperator{\arcsinh}{arcsinh}
%\DeclareMathOperator{\arccosh}{arccosh}
%\DeclareMathOperator{\arctanh}{arctanh}



\begin{document}

\hrulefill
\textbf{定義}
\hrulefill

\textbf{同相}

位相空間$X,Y$の間に全単射$f$があるとする。
全単射なので、逆写像$f^{-1}$も存在する。
\begin{equation}
 f: X \rightarrow Y
  \quad
  f^{-1}: Y \rightarrow X
\end{equation}

写像$f,f^{-1}$が共に連続である時、
$f$を同相写像といい、
$X$と$Y$は同相であるという。


\dotfill

$X,Y$を位相空間、$f:X\to Y$を写像とする。
この時、次の2つは同値である。
\begin{itemize}
 \item $f$が連続
 \item
      $X$の点列$\{x_i\}_{i\in\mathbb{N}}$が$x\in X$に収束するなら
      $Y$の点列$\{f(x_i)\}_{i\in\mathbb{N}}$が$f(x)\in Y$に収束する。
\end{itemize}

ここから [A]と[B] の条件を考えた。
$[A^{\prime}]と[B^{\prime}]$はその対偶である。
\begin{align}
& [A] & f\text{:連続} \Rightarrow& \lim_{i\to\infty}f(x_i)=f(x)\\
& [B] & \lim_{i\to\infty}f(x_i)=f(x) \Rightarrow& f\text{:連続}\\
& [A^\prime] & \lim_{i\to\infty}x_i \ne x \text{又は} \lim_{i\to\infty}f(x_i)\ne f(x) \Rightarrow&  f\text{:連続でない}\\
& [B^\prime] & f\text{:連続でない} \Rightarrow& \lim_{i\to\infty}x_i \ne x \text{又は} \lim_{i\to\infty}f(x_i)\ne f(x) \\
\end{align}



\hrulefill



距離空間$X,Y$に対し写像$f:X \to Y$を連続全単射写像とする。

$X$の点列$\{x_n\}_{n\in\mathbb{N}}$で収束しないものを用いて
集合$L(f)$を次のように定める。
\begin{equation}
 L(f)= \left\{ \lim_{n\to\infty}f(x_n) \in Y \mid
  \text{ 点列 }\{x_n\}_{n\in\mathbb{N}}\text{ は }X\text{ 上で収束しない} \right\}
\end{equation}

このとき、
$f$が同相写像であるための必要十分条件は
$L(f)=\emptyset$となることを示せ。

\dotfill

$(X,d),(Y,d)$を距離空間、$f:X\to Y$を写像とするとき、
次が同値である。
\begin{itemize}
 \item $f$が連続
 \item
      $X$の点列$\{x_i\}_{i\in\mathbb{N}}$が$x\in X$に収束するなら
      $Y$の点列$\{f(x_i)\}_{i\in\mathbb{N}}$が$f(x)\in Y$に収束する。
\end{itemize}

\dotfill

示したいことは
\begin{equation}
 f:\text{同相写像} \Leftrightarrow L(f)=\emptyset
\end{equation}
だが、
$L(f)=\emptyset$は
$X$ の点列 $\{x_n\}_{i\in\mathbb{N}}$ が収束しないとき
$Y$ の点列 $\{f(x_n)\}_{i\in\mathbb{N}}$ も収束しない
ことを意味している。
%だが、前提条件を考えると示すべきことは
%\begin{equation}
% f^{-1}:\text{連続写像} \Leftrightarrow
%  X \text{の点列} \{x_n\}_{i\in\mathbb{N}} \text{が収束しないとき}
%  Y \text{の点列} \{f(x_n)\}_{i\in\mathbb{N}} \text{も収束しない}
%\end{equation}

\dotfill
$f:\text{同相写像} \Rightarrow L(f)=\emptyset$
\dotfill

$f$が同相写像であるので
$f$は全単射、$f,f^{-1}$は連続写像である。

$f:X\to Y$が連続であることから「ヒントの定理」より
$X$の任意の点に収束する点列 $\{x_n\}_{i\in\mathbb{N}}$は
$Y$の点列$\{f(x_n)\}_{i\in\mathbb{N}}$ も収束する
ことが言える。

同様に
$f^{-1}:Y\to X$が連続であることから「ヒントの定理」より
$Y$の任意の点に収束する点列 $\{y_n\}_{i\in\mathbb{N}}$は
$X$の点列$\{f^{-1}(y_n)\}_{i\in\mathbb{N}}$ も収束する
ことが言える。

これにより
$X$と$Y$の点列は収束する点列が対応し、
収束しない点列が対応する。
つまり、$X$で収束しない点列の$Y$での収束値は存在しないので
$L(f)=\emptyset$である。


\dotfill
$L(f)=\emptyset
\Rightarrow
f:\text{同相写像}$
\dotfill

$L(f)=\emptyset$より
$X$ の点列 $\{x_n\}_{i\in\mathbb{N}}$ が収束しないとき
$Y$ の点列 $\{f(x_n)\}_{i\in\mathbb{N}}$ も収束しない。

これの対偶を考えれば
$Y$ の点列 $\{f(x_n)\}_{i\in\mathbb{N}}$ が収束するとき、
$X$ の点列 $\{x_n\}_{i\in\mathbb{N}}$ も収束する。

$f$は全単射であるので逆写像$f^{-1}$が存在する。
これにより
$Y$ の点列 $\{(f(x_n))\}_{i\in\mathbb{N}}$ が収束するとき、
$X$ の点列 $\{f^{-1}(f(x_n))\}_{i\in\mathbb{N}}$ も収束する。

「ヒントの定理」より
$Y$で収束する点列は写像$f^{-1}$で移した点列も収束するため
$f^{-1}$は連続写像である。
$f$は連続全単射であるので$f$は同相写像であることが分かる。


\dotfill

これらにより
$f$が同相写像であることと
$L(f)=\emptyset$が同値であることが分かる。



\end{document}
