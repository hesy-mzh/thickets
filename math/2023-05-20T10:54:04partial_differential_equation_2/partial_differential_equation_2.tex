\documentclass[12pt,b5paper]{ltjsarticle}

%\usepackage[margin=15truemm, top=5truemm, bottom=5truemm]{geometry}
%\usepackage[margin=10truemm,left=15truemm]{geometry}
\usepackage[margin=10truemm]{geometry}

\usepackage{amsmath,amssymb}
%\pagestyle{headings}
\pagestyle{empty}

%\usepackage{listings,url}
%\renewcommand{\theenumi}{(\arabic{enumi})}

%\usepackage{graphicx}

%\usepackage{tikz}
%\usetikzlibrary {arrows.meta}
%\usepackage{wrapfig}
%\usepackage{bm}

% ルビを振る
\usepackage{luatexja-ruby}	% required for `\ruby'

%% 核Ker 像Im Hom を定義
%\newcommand{\Img}{\mathop{\mathrm{Im}}\nolimits}
%\newcommand{\Ker}{\mathop{\mathrm{Ker}}\nolimits}
%\newcommand{\Hom}{\mathop{\mathrm{Hom}}\nolimits}

%\DeclareMathOperator{\Rot}{rot}
\DeclareMathOperator{\Div}{div}
%\DeclareMathOperator{\Grad}{grad}
%\DeclareMathOperator{\arcsinh}{arcsinh}
%\DeclareMathOperator{\arccosh}{arccosh}
%\DeclareMathOperator{\arctanh}{arctanh}



%\usepackage{listings,url}
%
%\lstset{
%%プログラム言語(複数の言語に対応,C,C++も可)
%  language = Python,
%%  language = Lisp,
%%  language = C,
%  %背景色と透過度
%  %backgroundcolor={\color[gray]{.90}},
%  %枠外に行った時の自動改行
%  breaklines = true,
%  %自動改行後のインデント量(デフォルトでは20[pt])
%  breakindent = 10pt,
%  %標準の書体
%%  basicstyle = \ttfamily\scriptsize,
%  basicstyle = \ttfamily,
%  %コメントの書体
%%  commentstyle = {\itshape \color[cmyk]{1,0.4,1,0}},
%  %関数名等の色の設定
%  classoffset = 0,
%  %キーワード(int, ifなど)の書体
%%  keywordstyle = {\bfseries \color[cmyk]{0,1,0,0}},
%  %表示する文字の書体
%  %stringstyle = {\ttfamily \color[rgb]{0,0,1}},
%  %枠 "t"は上に線を記載, "T"は上に二重線を記載
%  %他オプション:leftline,topline,bottomline,lines,single,shadowbox
%  frame = TBrl,
%  %frameまでの間隔(行番号とプログラムの間)
%  framesep = 5pt,
%  %行番号の位置
%  numbers = left,
%  %行番号の間隔
%  stepnumber = 1,
%  %行番号の書体
%%  numberstyle = \tiny,
%  %タブの大きさ
%  tabsize = 4,
%  %キャプションの場所("tb"ならば上下両方に記載)
%  captionpos = t
%}



\begin{document}



\hrulefill

\textbf{Report 1.5}

$D^2 \Phi(x-y)$は
$y=x$近くで積分不可能である。

\dotfill

$C$を定数とし、$x\ne 0$において次が成り立つ。
\begin{equation}
 \lvert D^2 \Phi(x) \rvert \leq \frac{C}{\lvert x \rvert^n}
\end{equation}
このとき、$x=0$において $D^2 \Phi(x)$が発散する。

つまり、$x=y$において $D^2 \Phi(x-y)$が発散するため、
$D^2 \Phi(x-y)$は$x=y$の付近で積分できない。


\hrulefill

\textbf{Report 1.6}

$f:\mathbb{R}^{n}\to\mathbb{R}$であり、
$f\in C^{2}_{c}(\mathbb{R}^{n})$な連続関数とする。
つまり、$f$は2回微分可能でコンパクトなサポートを持つ連続関数である。

次の式の右辺は変数$x$について連続であることを示せ。
\begin{equation}
 u_{x_{i}x_{j}}(x) =
  \int_{\mathbb{R}^n} \Phi(y) f_{x_{i}x_{j}}(x-y)dy
  \quad (i,j = 1,2,\dots,n)
\end{equation}


\dotfill

$f$はコンパクトなサポートを持つので、
$f_{x_{i}x_{j}}$もコンパクトなサポートを持つ。

また、$f_{x_{i}x_{j}}$が連続であることから
有界関数である。

${}^{\forall}\alpha\in\mathbb{R}^{n}$とする。

\begin{align}
 \lim_{x\to\alpha}\int_{\mathbb{R}^n} \Phi(y) f_{x_{i}x_{j}}(x-y)dy
  =&
 \int_{\mathbb{R}^n} \lim_{x\to\alpha}\Phi(y) f_{x_{i}x_{j}}(x-y)dy\\
 =&
 \int_{\mathbb{R}^n} \Phi(y) f_{x_{i}x_{j}}(\alpha-y)dy
\end{align}
となるので、
$x$について連続であることがわかる。



\hrulefill

\textbf{Report 1.7}


\begin{equation}
 \lvert I_{\varepsilon} \rvert
  \leq
  C \| D^{2}f \|_{L^{\infty}(\mathbb{R}^{n})} \int_{B(0,\varepsilon)} \lvert \Phi(y)\rvert dy
  \leq
\begin{cases}
 C\varepsilon^{2}\lvert \log{\varepsilon} \rvert & (n=2)\\
 C\varepsilon^{2} & (n\geq 3)
\end{cases}
\end{equation}

上記式の$n=3$における右側の不等式
\begin{equation}
   C \| D^{2}f \|_{L^{\infty}(\mathbb{R}^{n})} \int_{B(0,\varepsilon)} \lvert \Phi(y)\rvert dy
      \leq
       C\varepsilon^{2}
\end{equation}
を示せ。

\dotfill

$n = 3$において
\begin{equation}
 \Phi(x)
  = \frac{1}{n(n-2)\alpha(n)}\frac{1}{\lvert x \rvert^{n-2}}
  = \frac{1}{3\alpha(3)\lvert x \rvert}
  = \frac{1}{4\pi \lvert x \rvert}
\end{equation}

$\alpha(3)$は$\mathbb{R}^{3}$における単位球の体積を表す。
$\alpha(3)=\frac{4}{3}\pi$

$\lvert x \rvert$は$\mathbb{R}^{3}$におけるベクトルの大きさを表す。
$\lvert x \rvert=\sqrt{x_{1}^{2}+x_{2}^{2}+x_{3}^{3}}$


\begin{equation}
 u(x)
  = \int_{\mathbb{R}^{n}} \Phi(x-y) f(y) dy
  = \frac{1}{3\alpha(3)}\int_{\mathbb{R}^{3}} \frac{1}{\lvert x-y \rvert}f(y) dy
\end{equation}

\begin{equation}
 \Delta u = \sum_{i=1}^{n} u_{x_{i}x_{i}}
\end{equation}



\hrulefill

\textbf{Report 1.8}


\begin{equation}
 \lvert L_{\varepsilon} \rvert
  \leq
  \| D f \|_{L^{\infty}(\mathbb{R}^{n})} \int_{\partial B(0,\varepsilon)} \lvert \Phi(y)\rvert dS(y)
  \leq
\begin{cases}
 C\varepsilon\lvert \log{\varepsilon} \rvert & (n=2)\\
 C\varepsilon & (n\geq 3)
\end{cases}
\end{equation}

上記式の$n=3$における右側の不等式
\begin{equation}
   \| D f \|_{L^{\infty}(\mathbb{R}^{n})}
    \int_{\partial B(0,\varepsilon)} \lvert \Phi(y)\rvert dS(y)
      \leq
       C\varepsilon
\end{equation}
を示せ。

\dotfill

\hrulefill

\textbf{Report 1.9}

\begin{equation}
 -\frac{1}{n\alpha(n) \varepsilon^{n-1}}
  \int_{\partial B(x,\varepsilon)} f(y)dS(y)
  \to
  -f(x)
  \quad
  \text{as}\
  \varepsilon \to 0
\end{equation}

この式を示せ。

\dotfill


\begin{equation}
 \lim_{\varepsilon\to 0}
  \left\lvert
 -\frac{1}{n\alpha(n) \varepsilon^{n-1}}
  \int_{\partial B(x,\varepsilon)} f(y)dS(y)
  -
  (-f(x))
  \right\rvert
  =0
\end{equation}

上記式を満たすことを示せばよい。


\begin{align}
  & \left\lvert
 -\frac{1}{n\alpha(n) \varepsilon^{n-1}}
  \int_{\partial B(x,\varepsilon)} f(y)dS(y)
  -
  (-f(x))
  \right\rvert\\
 = &
 \left\lvert
 \frac{1}{n\alpha(n) \varepsilon^{n-1}}
  \int_{\partial B(x,\varepsilon)} f(y)dS(y) -f(x)
 \right\rvert\\
 = &
 \left\lvert
  \frac{1}{n\alpha(n) \varepsilon^{n-1}}
   \int_{\partial B(x,\varepsilon)} f(y)dS(y)
 -
 \frac{1}{n\alpha(n) \varepsilon^{n-1}}
 \int_{\partial B(x,\varepsilon)} f(x)dS(y)
  \right\rvert\\
 = &
 \frac{1}{n\alpha(n) \varepsilon^{n-1}}
 \left\lvert
 \int_{\partial B(x,\varepsilon)} (f(y)-f(x))dS(y)
 \right\rvert
\end{align}

$y=x+\varepsilon z$とおくと、
$\lvert x-y \rvert = \varepsilon \Leftrightarrow \lvert z \rvert =1$
であり、
$dy=\varepsilon^{n-1}dz$
である。

\begin{align}
 & \frac{1}{n\alpha(n) \varepsilon^{n-1}}
 \left\lvert
 \int_{\partial B(0,1)} (f(x+\varepsilon z)-f(x)) \varepsilon^{n-1}dz
 \right\rvert\\
 = &
 \frac{1}{n\alpha(n)}
 \left\lvert
 \int_{\partial B(0,1)} (f(x+\varepsilon z)-f(x)) dz
 \right\rvert
\end{align}

これにより
$\displaystyle
\lim_{\varepsilon\to 0}
\int_{\partial B(0,1)} (f(x+\varepsilon z)-f(x)) dz
=0$
であることを示せればよい。

$f\in C_{c}^{2}(\mathbb{R}^{n})$である。
コンパクトなサポートを持つ連続関数は有界関数であるので、
積分と極限を入れ替えることができる。
\begin{equation}
 \lim_{\varepsilon\to 0}
 \int_{\partial B(0,1)} (f(x+\varepsilon z)-f(x)) dz
 =
 \int_{\partial B(0,1)}
 \lim_{\varepsilon\to 0}
 (f(x+\varepsilon z)-f(x)) dz
 = 0
\end{equation}




よって、次の式が成り立つことがわかる。
\begin{equation}
 \lim_{\varepsilon\to 0}
  \left\lvert
 -\frac{1}{n\alpha(n) \varepsilon^{n-1}}
  \int_{\partial B(x,\varepsilon)} f(y)dS(y)
  -
  (-f(x))
  \right\rvert
  =0
\end{equation}




\hrulefill

\end{document}
