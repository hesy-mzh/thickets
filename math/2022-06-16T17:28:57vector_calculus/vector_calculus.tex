\documentclass[12pt,b5paper]{ltjsarticle}

%\usepackage[margin=15truemm, top=5truemm, bottom=5truemm]{geometry}
\usepackage[margin=15truemm]{geometry}

\usepackage{amsmath,amssymb}
%\pagestyle{headings}
\pagestyle{empty}

%\usepackage{listings,url}
%\renewcommand{\theenumi}{(\arabic{enumi})}

\usepackage{graphicx}

\usepackage{tikz}
\usetikzlibrary {arrows.meta}
\usepackage{wrapfig}	% required for `\wrapfigure' (yatex added)
\usepackage{bm}	% required for `\bm' (yatex added)
\usepackage{luatexja-ruby}	% required for `\ruby'
%% 像Im を定義
%\newcommand{\Img}{\mathop{\mathrm{Im}}\nolimits}

\begin{document}

\hrulefill

\textbf{合成関数の微分(連鎖公式)}

$y=f(x),x=u(t)$の時
\begin{equation}
 \frac{\mathrm{d}y}{\mathrm{d}t}
  = \frac{\mathrm{d}y}{\mathrm{d}x}\frac{\mathrm{d}x}{\mathrm{d}t}
  = \frac{\mathrm{d}f(x)}{\mathrm{d}x}\frac{\mathrm{d}u(t)}{\mathrm{d}t}
\end{equation}

$z=f(x,y),\, x=u(t),\,y=v(t)$の時
\begin{align}
 \frac{\mathrm{d}z}{\mathrm{d}t}
   =& \frac{\partial z}{\partial x}\frac{\mathrm{d}x}{\mathrm{d}t}
     + \frac{\partial z}{\partial y}\frac{\mathrm{d}y}{\mathrm{d}t}\\
   =& \frac{\partial f(x,y)}{\partial x}\frac{\mathrm{d}u(t)}{\mathrm{d}t}
     + \frac{\partial f(x,y)}{\partial y}\frac{\mathrm{d}v(t)}{\mathrm{d}t}
\end{align}

\dotfill


\textbf{面積}

$D\subset\mathbb{R}^2$を有界集合とする。
関数$f:D\rightarrow\mathbb{R},\, (x,y)\mapsto 1$を定義し
$f$が$D$上積分可能である時、$D$は面積確定である
といい、$\int_{D}f(x,y)\mathrm{d}x\mathrm{d}y$を
$D$の面積といい、$\mathrm{Area}(D)$とかく。

\hrulefill

\begin{enumerate}
 \setcounter{enumi}{1}
 \item
      \begin{align}
       f:& \mathbb{R}^2 \rightarrow \mathbb{R}, \quad (x,y) \mapsto f(x,y)\\
       g:& \mathbb{R}^2 \rightarrow \mathbb{R}, \quad (u,v) \mapsto g(u,v)\\
       \varphi:& \mathbb{R} \rightarrow \mathbb{R}, \quad t \mapsto \varphi(t)
      \end{align}

      $f,\,g,\,\varphi$を$C^1$-級関数とする。

      この時、${}^{\forall}t\in\mathbb{R}$において
      \begin{equation}
       \frac{\mathrm{d}}{\mathrm{d}t}f(t,g(t,\varphi(t)))
      \end{equation}
      を$f,g$の偏微分、$\varphi$の微分で表し、
      その式を証明せよ。

      \dotfill

      合成関数の微分より
      \begin{align}
       \frac{\mathrm{d}}{\mathrm{d}t}f(x,y)
        =& \frac{\partial f(x,y)}{\partial x}\frac{\mathrm{d}x}{\mathrm{d}t}
        + \frac{\partial f(x,y)}{\partial y}\frac{\mathrm{d}y}{\mathrm{d}t}\\
       \frac{\mathrm{d}g(u,v)}{\mathrm{d}t}
        =& \frac{\partial g(u,v)}{\partial u}\frac{\mathrm{d}u}{\mathrm{d}t}
        + \frac{\partial g(u,v)}{\partial v}\frac{\mathrm{d}v}{\mathrm{d}t}
      \end{align}

      $f(t,g(t,\varphi(t)))$
      は$f(x,y)$に次を代入した式である。
      \begin{equation}
       x=t,\ y=g(u,v),\ u=t,\ v=\varphi (t)
      \end{equation}

      この為、$x,y$を置き換えると
      \begin{align}
       \frac{\mathrm{d}}{\mathrm{d}t}f(t,g(u,v))
        =& \frac{\partial f(x,y)}{\partial x}(t,g(u,v))\frac{\mathrm{d}t}{\mathrm{d}t}
        + \frac{\partial f(x,y)}{\partial y}(t,g(u,v))\frac{\mathrm{d}g(u,v)}{\mathrm{d}t}\\
       =& \frac{\partial f(x,y)}{\partial x}(t,g(u,v))
        + \frac{\partial f(x,y)}{\partial y}(t,g(u,v))\frac{\mathrm{d}g(u,v)}{\mathrm{d}t}
       \label{a}
      \end{align}
      同様に$u,v$を置き換える。
      \begin{align}
       \frac{\mathrm{d}g(t,\varphi(t))}{\mathrm{d}t}
        =& \frac{\partial g(u,v)}{\partial u}(t,\varphi(t))\frac{\mathrm{d}t}{\mathrm{d}t}
        + \frac{\partial g(u,v)}{\partial v}(t,\varphi(t))\frac{\mathrm{d}\varphi (t)}{\mathrm{d}t}\\
        =& \frac{\partial g(u,v)}{\partial u}(t,\varphi(t))
        + \frac{\partial g(u,v)}{\partial v}(t,\varphi(t))\frac{\mathrm{d}\varphi (t)}{\mathrm{d}t}
       \label{b}
      \end{align}

      式(\ref{a})に式(\ref{b})を代入すると
      \begin{multline}
       \frac{\mathrm{d}}{\mathrm{d}t}f(t,g(u,v))\\
       =
        \frac{\partial f(x,y)}{\partial x}(t,g(u,v))
        + \frac{\partial f(x,y)}{\partial y}(t,g(u,v))
       \left(
       \frac{\partial g(u,v)}{\partial u}(t,\varphi(t)) \right.\\
        + \left.\frac{\partial g(u,v)}{\partial v}(t,\varphi(t))
        \frac{\mathrm{d}\varphi (t)}{\mathrm{d}t}
       \right)
      \end{multline}

      これにより次のような式を得る。
      \begin{align}
       & \frac{\mathrm{d}}{\mathrm{d}t}f(t,g(t,\varphi(t)))\\
%       =&
%        \frac{\partial f}{\partial x}(t,g(u,v))
%        + \frac{\partial f}{\partial y}(t,g(u,v))
%       \left(
%       \frac{\partial g}{\partial u}(t,\varphi(t))
%        + \frac{\partial g}{\partial v}(t,\varphi(t))
%        \frac{\mathrm{d}\varphi}{\mathrm{d}t}(t)
%       \right)\\
       =&
       \frac{\partial f}{\partial x}(t,g(t,\varphi(t)))
        + \frac{\partial f}{\partial y}(t,g(t,\varphi(t)))
       \left(
       \frac{\partial g}{\partial u}(t,\varphi(t))
        + \frac{\partial g}{\partial v}(t,\varphi(t))
        \frac{\mathrm{d}\varphi}{\mathrm{d}t}(t)
       \right)
      \end{align}

      \hrulefill

 \item
      領域$D$は次のような集合とする。
      \begin{equation}
       D = \{ (x,y)\in\mathbb{R}^2 \mid 1\leq x^2 + y^2 \leq 4 \}
      \end{equation}

      開区間$I$を$I=(-3,3)$とし、写像$\bm{f}$を次のように定義する。
      \begin{gather}
       \bm{f}: I^2 \rightarrow \mathbb{R}^2, \quad (x,y)\mapsto
        \begin{pmatrix}f_1(x,y)\\f_2(x,y)\end{pmatrix}\\
       f_1(x,y) = y-2xy^2 , \quad f_2(x,y) = 2x-2x^2y
      \end{gather}

      閉区間上の写像$C_1,\,C_2$を次のように定める。
      \begin{align}
       C_1 :& [1,e] \rightarrow \mathbb{R}^2, \quad
       t \mapsto (2\cos (2\pi \log (t)) , \, 2\sin (2\pi \log (t)))\\
       C_2 :& [0,1] \rightarrow \mathbb{R}^2, \quad
       t\mapsto \left( \cos\left(\frac{2\pi e}{e-1}(e^{-t}-e^{-1})\right),
       \sin\left(\frac{2\pi e}{e-1}(e^{-t}-e^{-1})\right) \right)
      \end{align}

      \begin{enumerate}
       \item $\bm{f}$が連続写像であることを示せ。
       \item $D$が面積確定であることを示し、$\mathrm{Area}(D)$を求めよ。
       \item $C_1$と$C_2$が$C^1$-級曲線であることを示せ。
       \item 次の線積分の値を求めよ。
             \begin{equation}
              \int_{C_1}\bm{f} + \int_{C_2}\bm{f}
             \end{equation}
      \end{enumerate}

      \dotfill

      \begin{enumerate}
       \item
            写像$f_1,\,f_2$は多項式で表される関数であるので連続写像である。
            写像$\bm{f}$は連速写像$f_1,\,f_2$で定義されるためこれもまた連続写像である。

       \item
            領域$D$は半径2と半径1の同心円の間の領域である。

            領域$D$は極座標で表示すると次のようになる。
            \begin{align}
             D =& \{ (x,y)\in\mathbb{R}^2 \mid 1\leq x^2 + y^2 \leq 4 \}\\
             =& \{ (r, \theta)\in\mathbb{R}^2 \mid 1\leq r \leq 2,\, 0\leq \theta <2\pi \}
            \end{align}
            変数変換は次のように行う。
            \begin{equation}
             x=r\cos\theta,\, y=r\sin\theta
            \end{equation}
            ヤコビ行列式
            \begin{equation}
             \begin{vmatrix}
              \frac{\partial x}{\partial r} & \frac{\partial x}{\partial \theta}\\
              \frac{\partial y}{\partial r} & \frac{\partial y}{\partial \theta}
             \end{vmatrix}
             =
             \begin{vmatrix}
              \cos\theta & -r\sin\theta\\
              \sin\theta & r\cos\theta
             \end{vmatrix}
             =r
            \end{equation}

            これらを用いて$\mathrm{Area}(D)$を計算する。
            \begin{align}
             \mathrm{Area}(D) =& \int_{D}\mathrm{d}x\mathrm{d}y
             &=& \int_{D}r\mathrm{d}r\mathrm{d}\theta\\
             =& \int_{0}^{2\pi}\!\!\!\int_{1}^{2} r\mathrm{d}r\mathrm{d}\theta
             &=& \int_{0}^{2\pi}\!\!\!\int_{1}^{2} r\mathrm{d}r\mathrm{d}\theta
             = 3\pi
            \end{align}


       \item
            $t$が$1 \to e$に変化するのに対して
            $\log (t)$は単調増加で$0\to 1$に変化する。
            これにより$C_1(t) = (2\cos (2\pi \log (t)) , \, 2\sin (2\pi \log (t)))$は
            原点中心の半径2の円周となる。
            \begin{equation}
             C_1(\theta) = (2\cos\theta , \, 2\sin\theta)
            \end{equation}

            $t$が$0 \to 1$に変化するのに対して
            $e^{-t}-e^{-1}$は単調減少で$1-e^{-1} \to 0$に変化する。
            つまり、$\frac{2\pi e}{e-1}(e^{-t}-e^{-1})$は
            $2\pi \to 0$へと変化する。
            これにより
            $C_2(t)=\left( \cos\left(\frac{2\pi e}{e-1}(e^{-t}-e^{-1})\right),
            \sin\left(\frac{2\pi e}{e-1}(e^{-t}-e^{-1})\right) \right)$は
            原点中心の半径1の円周となる。
            \begin{equation}
             C_2(\theta) = (\cos\theta , \, \sin\theta)
            \end{equation}

            ただし、$C_1$は正の向き、$C_2$は負の向きとなっている。

            $\sin\theta,\,\cos\theta$は$C^{\infty}$-級関数であるので、
            $C_1,\,C_2$は$C^1$-級である。

       \item
            $C_1.C_2$の示す図形は$\bm{f}$の定義域に含まれている。

            $\bm{f}$を極座標($x=r\cos\theta,\,y=r\sin\theta$)に変数変換する。
            \begin{align}
             \bm{f}(x,y) =& \begin{pmatrix}y-2xy^2\\2x-2x^2y\end{pmatrix}\\
             =& \begin{pmatrix}y-2xy^2\\2x-2x^2y\end{pmatrix}
            \end{align}


            $C_1$を$(2\cos\theta,\,2\sin\theta)$とし
            $0$から$2\pi$まで曲線として積分を考える。

            $x=2\cos\theta,\,y=2\sin\theta$とすると、
            \begin{equation}
             \frac{\mathrm{d} x}{\mathrm{d} \theta} = -2\sin\theta = -y,\quad
              \frac{\mathrm{d} y}{\mathrm{d} \theta} = 2\cos\theta = x
            \end{equation}
            であるので、$C_1$上の積分は次のようになる。
            \begin{align}
             \int_{C_1}\bm{f}
             = & \int_{0}^{2\pi}
             \begin{pmatrix}y-2xy^2\\2x-2x^2y\end{pmatrix}
                  \cdot\begin{pmatrix}-y\\x\end{pmatrix}\mathrm{d}\theta\\
             = & \int_{0}^{2\pi}(-y^2+2xy^3+2x^2-2x^3y)\mathrm{d}\theta\\
%             = & \int_{0}^{2\pi}(-3y^2 +2(y^2+x^2)+2xy(y^2-x^2))\mathrm{d}\theta\\
             = & \int_{0}^{2\pi}(-4\sin^2\theta+32\cos\theta\sin^3\theta+8\cos^2\theta-32\cos^3\theta\sin\theta)\mathrm{d}\theta\\
             = & \int_{0}^{2\pi}(-2(1-\cos2\theta)+4(1+\cos2\theta)-16\sin2\theta\cos2\theta)\mathrm{d}\theta\\
             = & \int_{0}^{2\pi}(2+6\cos2\theta-8\sin4\theta)\mathrm{d}\theta
            \end{align}

            $C_2$を$(\cos\theta,\,\sin\theta)$とし
            $2\pi$から$0$まで曲線として積分を考える。

            $x=\cos\theta,\,y=\sin\theta$とすると、
            \begin{equation}
             \frac{\mathrm{d} x}{\mathrm{d} \theta} = -\sin\theta = -y,\quad
              \frac{\mathrm{d} y}{\mathrm{d} \theta} = \cos\theta = x
            \end{equation}
            であるので、$C_2$上の積分は次のようになる。
            \begin{align}
             \int_{C_2}\bm{f}
             = & \int_{2\pi}^{0}
             \begin{pmatrix}y-2xy^2\\2x-2x^2y\end{pmatrix}
                  \cdot\begin{pmatrix}-y\\x\end{pmatrix}\mathrm{d}\theta\\
             = & \int_{2\pi}^{0}(-y^2+2xy^3+2x^2-2x^3y)\mathrm{d}\theta\\
%             = & \int_{0}^{2\pi}(-3y^2 +2(y^2+x^2)+2xy(y^2-x^2))\mathrm{d}\theta\\
             = & \int_{2\pi}^{0}(-\sin^2\theta+2\cos\theta\sin^3\theta+2\cos^2\theta-2\cos^3\theta\sin\theta)\mathrm{d}\theta\\
             =& \int_{2\pi}^{0}\left(-\frac{1-\cos2\theta}{2}+(1+\cos2\theta)-\sin2\theta\cos2\theta\right)\mathrm{d}\theta\\
             =& \int_{2\pi}^{0}\left(\frac{1+3\cos2\theta}{2}-\frac{1}{2}\sin4\theta\right)\mathrm{d}\theta
            \end{align}

            2つの積分は区間が等しく向きが逆になっているので
            次のように合わせて計算をする。
            \begin{align}
             & \int_{C_1}\bm{f} + \int_{C_2}\bm{f}\\
             =& \int_{0}^{2\pi}(2+6\cos2\theta-8\sin4\theta)\mathrm{d}\theta
             + \int_{2\pi}^{0}\left(\frac{1+3\cos2\theta}{2}-\frac{1}{2}\sin4\theta\right)\mathrm{d}\theta\\
             = & \int_{0}^{2\pi}\left(\frac{3}{2}+\frac{9}{2}\cos2\theta-\frac{15}{2}\sin4\theta\right)\mathrm{d}\theta\\
             = & \left[ \frac{3}{2}\theta+\frac{9}{4}\sin2\theta +\frac{15}{8}\cos4\theta \right]_{0}^{2\pi}\\
             = & 3\pi
            \end{align}
      \end{enumerate}


\end{enumerate}





\end{document}
