\documentclass[12pt,b5paper]{ltjsarticle}

%\usepackage[margin=15truemm, top=5truemm, bottom=5truemm]{geometry}
\usepackage[margin=10truemm]{geometry}

\usepackage{amsmath,amssymb}
%\pagestyle{headings}
\pagestyle{empty}

%\usepackage{listings,url}
%\renewcommand{\theenumi}{(\arabic{enumi})}

%\usepackage{graphicx}

%\usepackage{tikz}
%\usetikzlibrary {arrows.meta}
%\usepackage{wrapfig}	% required for `\wrapfigure' (yatex added)
\usepackage{bm}	% required for `\bm' (yatex added)

% ルビを振る
\usepackage{luatexja-ruby}	% required for `\ruby'

%% 核Ker 像Im Hom を定義
%\newcommand{\Img}{\mathop{\mathrm{Im}}\nolimits}
%\newcommand{\Ker}{\mathop{\mathrm{Ker}}\nolimits}
%\newcommand{\Hom}{\mathop{\mathrm{Hom}}\nolimits}

%\DeclareMathOperator{\Rot}{rot}
%\DeclareMathOperator{\Div}{div}
%\DeclareMathOperator{\Grad}{grad}
%\DeclareMathOperator{\arcsinh}{arcsinh}
%\DeclareMathOperator{\arccosh}{arccosh}
%\DeclareMathOperator{\arctanh}{arctanh}



\usepackage{listings,url}

\lstset{
%プログラム言語(複数の言語に対応,C,C++も可)
%  language = Python,
%  language = Lisp,
  language = C,
  %背景色と透過度
  %backgroundcolor={\color[gray]{.90}},
  %枠外に行った時の自動改行
  breaklines = true,
  %自動改行後のインデント量(デフォルトでは20[pt])
  breakindent = 10pt,
  %標準の書体
%  basicstyle = \ttfamily\scriptsize,
  basicstyle = \ttfamily,
  %コメントの書体
%  commentstyle = {\itshape \color[cmyk]{1,0.4,1,0}},
  %関数名等の色の設定
  classoffset = 0,
  %キーワード(int, ifなど)の書体
%  keywordstyle = {\bfseries \color[cmyk]{0,1,0,0}},
  %表示する文字の書体
  %stringstyle = {\ttfamily \color[rgb]{0,0,1}},
  %枠 "t"は上に線を記載, "T"は上に二重線を記載
  %他オプション:leftline,topline,bottomline,lines,single,shadowbox
  frame = TBrl,
  %frameまでの間隔(行番号とプログラムの間)
  framesep = 5pt,
  %行番号の位置
  numbers = left,
  %行番号の間隔
  stepnumber = 1,
  %行番号の書体
%  numberstyle = \tiny,
  %タブの大きさ
  tabsize = 4,
  %キャプションの場所("tb"ならば上下両方に記載)
  captionpos = t
}



\begin{document}

\hrulefill
\textbf{
  \ruby{Runge}{ルンゲ}
  \ -
  \ruby{Kutta}{クッタ}
  法}
\hrulefill

\begin{equation}
 y^{\prime} = f(t,y), \quad y(t_0)=y_0
\end{equation}

\begin{equation}
 \begin{cases}
  y_{n+1} =& y_{n} + \frac{h}{6}(k_{1}+2k_{2}+2k_{3}+k_{4})\\
  t_{n+1} =& t_{n}+h
 \end{cases}
\end{equation}
但し、
\begin{equation}
 \begin{cases}
  k_{1} =& f(t_{n},y_{n})\\
  k_{2} =& f(t_{n}+\frac{h}{2},y_{n}+\frac{h}{2}k_{1})\\
  k_{3} =& f(t_{n}+\frac{h}{2},y_{n}+\frac{h}{2}k_{2})\\
  k_{4} =& f(t_{n}+h,y_{n}+hk_{3})\\
 \end{cases}
\end{equation}

\dotfill



\textbf{2階微分方程式}

\begin{equation}
 y^{\prime\prime} = f(t,y,y^{\prime})
\end{equation}
2階微分方程式の場合、
$z=y^{\prime}$と置いて、
1階微分方程式を2つ作る。
\begin{equation}
 \begin{cases}
  y^{\prime} = z = f(t,y,z)\\
  y^{\prime\prime} = z^{\prime} = g(t,y,z)
 \end{cases}
\end{equation}

ここから
次のような式を作り、
右辺を計算することで順次値を計算していく。
\begin{align}
 t_{n+1} =& t_{n}+h\\
 y_{n+1} =& y_{n} + \frac{h}{6}(k_{1}+2k_{2}+2k_{3}+k_{4})\\
 z_{n+1} =& z_{n} + \frac{h}{6}(l_{1}+2l_{2}+2l_{3}+l_{4})
\end{align}
但し、係数$k_{i},l_{i}$は次の通り。
 \begin{align}
  k_{1} =& f(t_{n},y_{n},z_{n})&
  l_{1} =& g(t_{n},y_{n},z_{n})\\
  k_{2} =& f\left(t_{n}+\frac{h}{2},y_{n}+\frac{h}{2}k_{1},z_{n}+\frac{h}{2}l_{1}\right)&
  l_{2} =& g\left(t_{n}+\frac{h}{2},y_{n}+\frac{h}{2}k_{1},z_{n}+\frac{h}{2}l_{1}\right)\\
  k_{3} =& f\left(t_{n}+\frac{h}{2},y_{n}+\frac{h}{2}k_{2},z_{n}+\frac{h}{2}l_{2}\right)&
  l_{3} =& g\left(t_{n}+\frac{h}{2},y_{n}+\frac{h}{2}k_{2},z_{n}+\frac{h}{2}l_{2}\right)\\
  k_{4} =& f(t_{n}+h,y_{n}+hk_{3},z_{n}+hl_{3})&
  l_{4} =& g(t_{n}+h,y_{n}+hk_{3},z_{n}+hl_{3})
 \end{align}


ベクトルを用いて表現すると、
次のようになる。

$\bm{x}_n=(t_n,y_n,z_n),\bm{v}_{i}=(1,k_i,l_i)$
\begin{align}
 k_{1} =& f(\bm{x}_n) &
 l_{1} =& g(\bm{x}_n)\\
 k_{2} =& f\left(\bm{x}_n+\frac{h}{2}\bm{v}_1\right) &
 l_{2} =& g\left(\bm{x}_n+\frac{h}{2}\bm{v}_1\right)\\
 k_{3} =& f\left(\bm{x}_n+\frac{h}{2}\bm{v}_2\right) &
 l_{3} =& g\left(\bm{x}_n+\frac{h}{2}\bm{v}_2\right)\\
 k_{4} =& f\left(\bm{x}_n+h\bm{v}_3\right) &
 l_{4} =& g\left(\bm{x}_n+h\bm{v}_3\right)
\end{align}


プログラムで表す場合も上記順序で計算を繰り返すことにより
近似解が計算される。

上記の場合、
$z$は$y^{\prime}$の計算をしている。


\hrulefill
\textbf{問題}
\hrulefill

空気抵抗のある振り子の運動方程式の初期値問題
\begin{equation}
 \begin{cases}
  \ddot{u}= -\sin u - \mu \dot{u} \quad (0\leq t \leq T)\\
  u(0) = A,\ \dot{u}(0) = B
 \end{cases}
\end{equation}

\begin{enumerate}
 \item
      空気抵抗係数$\mu$と初期条件は次のように選んだ時、
      4次ルンゲクッタ法で数値解を求めよ。
      \begin{enumerate}
       \item $(\mu,\ A,\ B,\ T)=(0.0,\ 2.0,\ 0,\ 50)$
       \item $(\mu,\ A,\ B,\ T)=(0.2,\ 2.0,\ 0,\ 50)$
       \item $(\mu,\ A,\ B,\ T)=(0.2,\ 2.0,\ 1.5,\ 50)$
      \end{enumerate}

      それらを1つのグラフに載せたもの
      (縦軸は$u$、横軸は$t$)を作成せよ。
      (刻み幅$\Delta t$は自由だが、
      $\Delta t \leq 0.5$程度に抑えて作成)

 \item
      求めた数値解に対し、
      $E(t) = \dot{u}^2/2- \cos u$を計算し、
      それらを一つのグラフに載せたものを作成せよ。
\end{enumerate}

\dotfill


$v=\dot{u}$と置く。
\begin{equation}
 \begin{cases}
  \frac{\mathrm{d}}{\mathrm{d}t}u=v\\
  \frac{\mathrm{d}}{\mathrm{d}t}v= -\sin u - \mu v\\
  u(0) = A,\ v(0)=\dot{u}(0) = B
 \end{cases}
\end{equation}

\begin{equation}
 f(t,u,v) = v
  ,\quad
 g(t,u,v) = -\sin u - \mu v
 \label{function_def}
\end{equation}

\begin{align}
 t_{n+1} =& t_{n}+\Delta t\\
 u_{n+1} =& u_{n}+(k_1+2k_2+2k_3+k_4)\frac{\Delta t}{6}\\
 v_{n+1} =& v_{n}+(m_1+2m_2+2m_3+m_4)\frac{\Delta t}{6}
\end{align}
\begin{align}
 \bm{x}_{n} =& (t_n,u_n,v_n)&
 \bm{c}_{i} =& (1,k_i,m_i)\\
 k_{1} =& f(\bm{x}_n) &
 m_{1} =& g(\bm{x}_n)\\
 k_{2} =& f\left(\bm{x}_n+\frac{\Delta t}{2}\bm{c}_1\right) &
 m_{2} =& g\left(\bm{x}_n+\frac{\Delta t}{2}\bm{c}_1\right)\\
 k_{3} =& f\left(\bm{x}_n+\frac{\Delta t}{2}\bm{c}_2\right) &
 m_{3} =& g\left(\bm{x}_n+\frac{\Delta t}{2}\bm{c}_2\right)\\
 k_{4} =& f\left(\bm{x}_n+\Delta t\bm{c}_3\right) &
 m_{4} =& g\left(\bm{x}_n+\Delta t\bm{c}_3\right)
\end{align}

\hrulefill

下記サイトにて同様の計算は可能。

ルンゲクッタ法(4次,2階常微分方程式)
Runge-Kutta method

\url{https://keisan.casio.jp/exec/system/1548123555}

\dotfill

式(\ref{function_def})の内容を
\texttt{fn\_main}と
\texttt{fn\_sub}に反映。
関数$f$を\texttt{fn\_main}、
関数$g$を\texttt{fn\_sub}に記述し、
初期条件は\texttt{\#define}から始まる行に記述。

\begin{lstlisting}
#include <stdio.h>
#include <stdlib.h>
#include <math.h>

#define A 2.0 /* u の初期値 */
#define B 0.0 /* v の初期値 */
#define M 0.0 /* 空気抵抗 */
#define T 50.0 /* 終了時間 */
#define DT 0.1 /* 刻み時間 */

// 関数 メイン
double fn_main (double x, double y, double z) {
    (void) x;
    (void) y;
    return z;
}
// 関数 サブ
double fn_sub (double x, double y, double z) {
    (void) x;
    return - sin(y) - M * z;
}


int main(void) {
    int i;
    double u=A, v=B; /* 実数型の変数宣言、初期値で初期化 */
    double k_1, k_2, k_3, k_4, m_1, m_2, m_3, m_4;

    int N = (int) ceil(T / DT);
    double t=0;

    printf("%f %f\n", 0.0, A);
    for (i=1; i<=N; i++) {

        t=DT*i;
        
        k_1 = fn_main(t,u,v);
        m_1 = fn_sub(t,u,v);
        
        k_2 = fn_main(t+DT/2, u+k_1*DT/2, v+m_1*DT/2);
        m_2 = fn_sub(t+DT/2, u+k_1*DT/2, v+m_1*DT/2);
        
        k_3 = fn_main(t+DT/2, u+k_2*DT/2, v+m_2*DT/2);
        m_3 = fn_sub(t+DT/2, u+k_2*DT/2, v+m_2*DT/2);
        
        k_4 = fn_main(t+DT/2, u+k_3*DT, v+m_3*DT);
        m_4 = fn_sub(t+DT/2, u+k_3*DT, v+m_3*DT);
        
        u = u + (k_1 + 2*k_2 + 2*k_3 + k_4)*DT/6; /* u の差分式 */
        v = v + (m_1 + 2*m_2 + 2*m_3 + m_4)*DT/6; /* v の差分式 */
        printf("%f %f\n", i*DT, u);
        
    }
    
    return EXIT_SUCCESS;
}
\end{lstlisting}

コンパイル時、
オプション''-lm''が必要な場合あり。

\dotfill

$\frac{v^2}{2}-\cos u$を求めるのは
上記コードの終りにある\texttt{printf}行にて
計算を行う。

%%% ¥n はバックスラッシュではなく円記号 %%%
\texttt{printf(\"\%f \%f¥n\", i*DT, u);}
を
\texttt{printf(\"\%f \%f¥n\", i*DT, v*v/2-cos(u));}
に
変更。

\begin{lstlisting}
#include <stdio.h>
#include <stdlib.h>
#include <math.h>

#define A 2.0 /* u の初期値 */
#define B 0.0 /* v の初期値 */
#define M 0.0 /* 空気抵抗 */
#define T 50.0 /* 終了時間 */
#define DT 0.1 /* 刻み時間 */

// 関数 メイン
double fn_main (double x, double y, double z) {
    (void) x;
    (void) y;
    return z;
}
// 関数 サブ
double fn_sub (double x, double y, double z) {
    (void) x;
    return - sin(y) - M * z;
}


int main(void) {
    int i;
    double u=A, v=B; /* 実数型の変数宣言、初期値で初期化 */
    double k_1, k_2, k_3, k_4, m_1, m_2, m_3, m_4;

    int N = (int) ceil(T / DT);
    double t=0;

    printf("%f %f\n", 0.0, A);
    for (i=1; i<=N; i++) {

        t=DT*i;
        
        k_1 = fn_main(t,u,v);
        m_1 = fn_sub(t,u,v);
        
        k_2 = fn_main(t+DT/2, u+k_1*DT/2, v+m_1*DT/2);
        m_2 = fn_sub(t+DT/2, u+k_1*DT/2, v+m_1*DT/2);
        
        k_3 = fn_main(t+DT/2, u+k_2*DT/2, v+m_2*DT/2);
        m_3 = fn_sub(t+DT/2, u+k_2*DT/2, v+m_2*DT/2);
        
        k_4 = fn_main(t+DT/2, u+k_3*DT, v+m_3*DT);
        m_4 = fn_sub(t+DT/2, u+k_3*DT, v+m_3*DT);
        
        u = u + (k_1 + 2*k_2 + 2*k_3 + k_4)*DT/6; /* u の差分式 */
        v = v + (m_1 + 2*m_2 + 2*m_3 + m_4)*DT/6; /* v の差分式 */
        printf("%f %f\n", i*DT, v*v/2-cos(u));
        
    }
    
    return EXIT_SUCCESS;
}
\end{lstlisting}



\end{document}
