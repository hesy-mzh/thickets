\documentclass[12pt,b5paper]{ltjsarticle}

%\usepackage[margin=15truemm, top=5truemm, bottom=5truemm]{geometry}
\usepackage[margin=15truemm]{geometry}

\usepackage{amsmath,amssymb}
%\pagestyle{headings}
\pagestyle{empty}

%\usepackage{listings,url}
\renewcommand{\theenumi}{(\arabic{enumi})}

\begin{document}

%1次元複体$X, Y$について位相同型であるかどうかを判定せよ。
以下の$X$と$Y$が位相同型であるかないかを示せ。
\begin{enumerate}
 \item $X=\{ x \in\mathbb{R}^1 \mid -2<x<5\},\
       Y=\{y\in\mathbb{R}^1 \mid 0<y\}$
 \item $X=\bigcirc\llap{\small ---},\ Y=\bigcirc\llap{\large +}$
\end{enumerate}
\hrulefill
\begin{enumerate}
 \item $X=\{ x \in\mathbb{R}^1 \mid -2<x<5\},\
       Y=\{y\in\mathbb{R}^1 \mid 0<y\}$

       \dotfill

       開区間$(-2,5)$から開区間$(0,\infty)$への写像$f$を次のように定義します。
       \begin{equation}
        f(x)= -\frac{x+2}{x-5}
       \end{equation}
       この$f$は連続写像であり全単射です。
       \begin{equation}
        \lim_{x\rightarrow -2}f(x)=0 \qquad \lim_{x\rightarrow 5}f(x)=\infty
       \end{equation}
       この時、逆写像が存在し次の式で表せます。
       \begin{equation}
        f^{-1}(y)= \frac{5y-2}{y+1}
       \end{equation}
%       この写像$f^{-1}$は
%       \begin{equation}
%        \lim_{y\rightarrow 0}f^{-1}(y)=-2 \qquad \lim_{y\rightarrow \infty}f(y)=5
%       \end{equation}
%       であり、全単射な連続写像です。

       写像$f$は全単射な連続写像でその逆写像$f^{-1}$も連続写像であるので、
       位相同型写像となり、$X$と$Y$は位相同型であることがわかります。

       \hrulefill
 \item $X=\bigcirc\llap{\small ---},\ Y=\bigcirc\llap{\large +}$

       \dotfill

       円周$S^1$、線分$l_x$、$l_y$を次のようにおきます。
       \begin{gather}
       S^1=\{ (x,y)\in\mathbb{R}^2 \mid x^2+y^2=1)\}\\
       l_x = \{ (x,0)\in\mathbb{R}^2 \mid -1\leq x \leq 1)\}\\
       l_y = \{ (0,y)\in\mathbb{R}^2 \mid -1\leq y \leq 1)\}
       \end{gather}

       これらを使うと 概ね
       $X= S^1 \cup l_x$ と $Y= S^1 \cup l_x \cup l_y$
       になります。
%       \begin{equation}
%        X  \qquad Y
%       \end{equation}

       この時、$X$から$Y$への全単射かつ連続な写像は存在しません。
       これは、$X$は一筆書き出来るが$Y$は一筆書きできないので、
%       $Y$の任意の点に対応する$X$の点が一つになるように写像があると
       全単射な写像は連続写像ではなくなるからです。
       一筆書きについては$X$は半オイラーグラフですが、
       $Y$は奇数分岐が3以上ある為オイラーグラフではないことからわかります。

       これにより $X$ と $Y$ は同相ではないことになります。

\end{enumerate}
%\end{enumerate}

\end{document}
