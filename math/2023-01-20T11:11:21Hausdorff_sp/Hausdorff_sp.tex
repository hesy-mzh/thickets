\documentclass[12pt,b5paper]{ltjsarticle}

%\usepackage[margin=15truemm, top=5truemm, bottom=5truemm]{geometry}
%\usepackage[margin=10truemm,left=15truemm]{geometry}
\usepackage[margin=10truemm]{geometry}

\usepackage{amsmath,amssymb}
%\pagestyle{headings}
\pagestyle{empty}

%\usepackage{listings,url}
%\renewcommand{\theenumi}{(\arabic{enumi})}

%\usepackage{graphicx}

%\usepackage{tikz}
%\usetikzlibrary {arrows.meta}
%\usepackage{wrapfig}	% required for `\wrapfigure' (yatex added)
%\usepackage{bm}	% required for `\bm' (yatex added)

% ルビを振る
%\usepackage{luatexja-ruby}	% required for `\ruby'

%% 核Ker 像Im Hom を定義
%\newcommand{\Img}{\mathop{\mathrm{Im}}\nolimits}
%\newcommand{\Ker}{\mathop{\mathrm{Ker}}\nolimits}
%\newcommand{\Hom}{\mathop{\mathrm{Hom}}\nolimits}

%\DeclareMathOperator{\Rot}{rot}
%\DeclareMathOperator{\Div}{div}
%\DeclareMathOperator{\Grad}{grad}
%\DeclareMathOperator{\arcsinh}{arcsinh}
%\DeclareMathOperator{\arccosh}{arccosh}
%\DeclareMathOperator{\arctanh}{arctanh}



%\usepackage{listings,url}
%
%\lstset{
%%プログラム言語(複数の言語に対応,C,C++も可)
%  language = Python,
%%  language = Lisp,
%%  language = C,
%  %背景色と透過度
%  %backgroundcolor={\color[gray]{.90}},
%  %枠外に行った時の自動改行
%  breaklines = true,
%  %自動改行後のインデント量(デフォルトでは20[pt])
%  breakindent = 10pt,
%  %標準の書体
%%  basicstyle = \ttfamily\scriptsize,
%  basicstyle = \ttfamily,
%  %コメントの書体
%%  commentstyle = {\itshape \color[cmyk]{1,0.4,1,0}},
%  %関数名等の色の設定
%  classoffset = 0,
%  %キーワード(int, ifなど)の書体
%%  keywordstyle = {\bfseries \color[cmyk]{0,1,0,0}},
%  %表示する文字の書体
%  %stringstyle = {\ttfamily \color[rgb]{0,0,1}},
%  %枠 "t"は上に線を記載, "T"は上に二重線を記載
%  %他オプション:leftline,topline,bottomline,lines,single,shadowbox
%  frame = TBrl,
%  %frameまでの間隔(行番号とプログラムの間)
%  framesep = 5pt,
%  %行番号の位置
%  numbers = left,
%  %行番号の間隔
%  stepnumber = 1,
%  %行番号の書体
%%  numberstyle = \tiny,
%  %タブの大きさ
%  tabsize = 4,
%  %キャプションの場所("tb"ならば上下両方に記載)
%  captionpos = t
%}



\begin{document}

\hrulefill

\textbf{ハウスドルフ空間、$T_{2}$-空間}

位相空間$X$において、
任意の異なる2点$a,b\in X ,\ a\ne b$をとってくる。
それぞれの近傍$U_{a}(\ni a),\ U_{b}\ni b$が存在し
$U_{a}\cap U_{b} =\emptyset$ である時、
$X$をハウスドルフ空間 または $T_2$-空間という。

\textbf{$T_{1}$-空間}

位相空間$X$において、
任意の点$a\in X$の近傍$U_{a}$が次を満たすように存在するとき
$X$を$T_{1}$-空間という。
\begin{equation}
 b\in X, b\ne a \text{ に対し }
  b\not\in U_{a}
\end{equation}



\hrulefill

\textbf{距離空間の分離公理}

ユークリッド空間$\mathbb{R}^{n}$は
ハウスドルフ空間($T_2$-空間)
であることを示せ。

\dotfill

$a,b\in\mathbb{R}^{n},\ a\ne b$とし、
2点間の距離$d(a,b)=l>0$とする。

ここで、$a,b$のそれぞれの開近傍を
$U(a,\frac{l}{2}),\ U(b,\frac{l}{2})$
とすると、
$U(a,\frac{l}{2})\cap U(b,\frac{l}{2}) = \emptyset$
であるので、
$\mathbb{R}^{n}$はハウスドルフ空間である。


\hrulefill

\textbf{3点集合の分離公理}

3点集合の位相のうち、
$T_1$-空間となるものをすべて答えよ。

\dotfill

$X=\{a,b,c\}$とする。
この時、部分集合は次の8個である。
\begin{equation}
  \emptyset,\
  \{a\},\ \{b\},\ \{c\},\
  \{a,b\},\ \{b,c\},\ \{c,a\},\
  X
\end{equation}

$T_{1}$-空間は任意の点に対してその他の点を含まない近傍が存在する。

例えば、$a\in X$の近傍で$b\in X$を含まないものとなると
$\{a\}$または$\{a,c\}$が開集合でないといけない。
$a\in X$の近傍で$c\in X$を含まないものとなると
$\{a\}$または$\{a,b\}$が開集合でないといけない。
ここから
$\{a\}$が開集合であるか、
$\{a\}$が開集合でないなら
$\{a,b\},\{a,c\}$が開集合でないといけない。
もし、$\{a,b\},\{a,c\}$が開集合であるなら、
開集合の公理から$\{a,b\}\cap\{a,c\}=\{a\}$となるので、
$\{a\}$が開集合であることになる。
よって、$a\in X$の近傍を考えると$\{a\}$は開集合となる。

同様に$b\in X$の近傍と$c\in X$の近傍を考えると
$\{b\},\{c\}$が開集合となる。

$X=\{a,b,c\}$に対し、
$\{a\},\{b\},\{c\}$が開集合であれば、
$\{a\}\cup\{b\}=\{a,b\},\{b\}\cup\{c\}=\{b,c\},\{c\}\cup\{a\}=\{c,a\}$
も開集合となり、
位相$\mathcal{O}$の要素はすべての部分集合となる。

つまり、$X=\{a,b,c\}$の$T_{1}$-空間となる位相は
離散位相のみとなる。


\hrulefill

\textbf{ハウスドルフ空間の商空間}

一般にハウスドルフ空間の商空間は
ハウスドルフ空間にならないことを
例を用いて示せ。

\dotfill

$S=\{a,b\}$とし、
実数$\mathbb{R}$から$S$への全射を次のように定義する。
\begin{equation}
 f:\mathbb{R}\to S,\quad
  x\mapsto
  \begin{cases}
   a & (x=0)\\
   b & (x\ne 0)
  \end{cases}
\end{equation}

$\mathbb{R}$
の位相は距離空間から定義される位相が入っているとする。
この場合、$\mathbb{R}$はハウスドルフ空間である。

$S$には商位相を導入する。
つまり、$S$の部分集合$U$は$f^{-1}(U)$が$\mathbb{R}$の開集合である時に
開集合である。

今、$S$の部分集合は次の4つ。
\begin{equation}
 \emptyset,\{a\},\{b\},S
\end{equation}

これらの逆像はつぎのようになる。
\begin{equation}
  f^{-1}(\emptyset)=\emptyset ,\
  f^{-1}(\{a\})=\{0\} ,\
  f^{-1}(\{b\})=\mathbb{R}\backslash\{0\} ,\
  f^{-1}(S)=\mathbb{R}
\end{equation}

この時、
$\emptyset,\mathbb{R}\backslash\{0\},\mathbb{R}$が
$\mathbb{R}$の開集合であるので、
$\emptyset,\{b\},S$が$S$の開集合である。
しかし、$\{a\}$は$S$の開集合ではないので、
$S$はハウスドルフ空間ではない。

\hrulefill

\end{document}
