\documentclass[12pt,b5paper]{ltjsarticle}

%\usepackage[margin=15truemm, top=5truemm, bottom=5truemm]{geometry}
\usepackage[margin=15truemm]{geometry}

\usepackage{amsmath,amssymb}
\pagestyle{empty}

\usepackage{listings,url}

\begin{document}


\begin{enumerate}
 \item $f(z)=\frac{1}{2-3z}$とする。
       \begin{enumerate}\renewcommand{\theenumii}{\arabic{enumii}}
        \item マクローリン展開とその収束円を求めよ。

              \dotfill

              $\frac{1}{1-z}=\sum_{k=0}^{\infty}z^k\ (\lvert z\rvert  <1)$を利用します。

              $\frac{1}{2-3z}=\frac{1}{2}\frac{1}{1-\frac{3}{2}z}$です。
              $\lvert \frac{3}{2}z\rvert<1$において上の式を利用すると
              \begin{equation}
               \frac{1}{1-\frac{3}{2}z}
                = 1+\left(\frac{3}{2}z\right)+\left(\frac{3}{2}z\right)^2
                +\left(\frac{3}{2}z\right)^3+\left(\frac{3}{2}z\right)^4+\cdots
              \end{equation}
              となるので、$f(z)$は
              \begin{align}
               f(z) =& \frac{1}{2-3z}=\frac{1}{2}\frac{1}{1-\frac{3}{2}z}\\
               =& \frac{1}{2} \left(1+\left(\frac{3}{2}z\right)+\left(\frac{3}{2}z\right)^2
                +\left(\frac{3}{2}z\right)^3+\left(\frac{3}{2}z\right)^4+\cdots \right)\\
               =& \sum_{k=0}^{\infty}\frac{1}{2}\left(\frac{3}{2}z\right)^k
               = \sum_{k=0}^{\infty}\frac{3^k}{2^{k+1}}z^k
              \end{align}

              収束円は$\lvert \frac{3}{2}z\rvert <1$より $\lvert z \rvert < \frac{2}{3}$

              \hrulefill
        \item $z=2$におけるテイラー展開とその収束円を求めよ。

              \dotfill

              $v=z-2$とおくと$z=v+2$なので、$f(z)$は
              \begin{equation}
               f(z) = \frac{1}{2-3z} = \frac{1}{-4-3v}
               = \frac{-\frac{1}{4}}{1-(-\frac{3}{4}v)}
               = -\frac{1}{4}\left(\frac{1}{1-(-\frac{3}{4}v)}\right)
               \end{equation}
              となります。
              先ほどと同じように変形を行うと
              \begin{align}
               -\frac{1}{4}\left(\frac{1}{1-(-\frac{3}{4}v)}\right)
               =& -\frac{1}{4} \left(1+\left(-\frac{3}{4}v\right)+\left(-\frac{3}{4}v\right)^2
               +\cdots \right)\\
               =& \sum_{k=0}^{\infty}\left(-\frac{1}{4}\right)\left(-\frac{3}{4}v\right)^k\\
               =& \sum_{k=0}^{\infty}(-1)^{k+1}\frac{3^k}{4^{k+1}}v^k\\
               =& \sum_{k=0}^{\infty}(-1)^{k+1}\frac{3^k}{4^{k+1}}(z-2)^k
              \end{align}

              収束円は$\lvert -\frac{3}{4}v\rvert <1$より
              $\lvert v \rvert < \frac{4}{3}$であるので、
              $\lvert z-2 \rvert < \frac{4}{3}$です。
       \end{enumerate}
       \hrulefill
 \item $g(z)=z^2(z^4+z^2+1)(z^3-1)$の全ての零点とその位数を求めよ。

       \dotfill

%       $g(z)$を実数の範囲で因数分解すると
%       \begin{equation}
%        g(z) = z^2(z^4+z^2+1)(z^3-1) = z^2(z^4+z^2+1)(z-1)(z^2+z+1)
%       \end{equation}
%       です。
       $g(z)$の零点なので、まず$z^3-1=0$を考えます。
       $z^3-1=0$は$1=\cos(2\pi n) +i\sin(2\pi n) ,\ (n=0,1,2,\dots)$の$3$乗根なので、
       零点は次の3つ。
       \begin{align}
        \cos\left(\frac{2\pi\cdot 0}{3}\right) +i\sin\left(\frac{2\pi\cdot 0}{3}\right) =& 1\\
        \cos\left(\frac{2\pi\cdot 1}{3}\right) +i\sin\left(\frac{2\pi\cdot 1}{3}\right)
        =& \frac{-1+\sqrt{3}i}{2}\\
        \cos\left(\frac{2\pi\cdot 2}{3}\right) +i\sin\left(\frac{2\pi\cdot 2}{3}\right)
        =& \frac{-1-\sqrt{3}i}{2}
       \end{align}
       %$z^4+z^2+1$と
%       $\omega = \frac{-1+\sqrt{3}i}{2}$とすると虚数のもう一つの3乗根は共役な複素数なので、
%       $\overline{\omega}$です。
%       これらを用いて$z^3-1=(z-1)(z-\omega)(z-\overline{\omega})$と書けます。
%       $z^2+z+1$を複素数の範囲で分解すると
%       %$z^2+z+1$は
%       \begin{equation}
%        z^2+z+1 = \left(z-\frac{-1+\sqrt{3}i}{2}\right)\left(z-\frac{-1-\sqrt{3}i}{2}\right)
%       \end{equation}

       次に$z^4+z^2+1$の零点ですが、$(z^2-1)(z^4+z^2+1)=z^6-1$なので、
       1の6乗根のうち$\pm1$を除いたものとなります。
       6乗根を$\omega_i$とすると
       次の4つが零点となります。
       \begin{align}
        \omega_1 =
        \cos\left(\frac{2\pi\cdot 1}{6}\right) +i\sin\left(\frac{2\pi\cdot 1}{6}\right)
        =& \frac{1+\sqrt{3}i}{2}\\
        \omega_2 =
        \cos\left(\frac{2\pi\cdot 2}{6}\right) +i\sin\left(\frac{2\pi\cdot 2}{6}\right)
        =& \frac{-1+\sqrt{3}i}{2}\\
        \omega_4 =
        \cos\left(\frac{2\pi\cdot 4}{6}\right) +i\sin\left(\frac{2\pi\cdot 5}{6}\right)
        =& \frac{-1-\sqrt{3}i}{2}\\
        \omega_5 =
        \cos\left(\frac{2\pi\cdot 5}{6}\right) +i\sin\left(\frac{2\pi\cdot 4}{6}\right)
        =& \frac{1-\sqrt{3}i}{2}
       \end{align}

       $\omega_2, \omega_4$は1の3乗根でもあるので、
       $g(z)$は次のような式になります。
       \begin{equation}
       g(z) = z^2(z-1)(z-\omega_1)(z-\omega_2)^2(z-\omega_4)^2(z-\omega_5)
       \end{equation}

       位数はそれぞれの指数からわかるため、
       零点とその位数は次のようになります。
       \begin{center}
        \begin{tabular}{|c|c|c|c|c|c|c|}
        \hline
        零点 & 0 & 1 & $\omega_1$ & $\omega_2$ & $\omega_4$ & $\omega_5$\\
        \hline
        位数 & 2 & 1 & 1 & 2 & 2 & 1\\ \hline
        \end{tabular}
       \end{center}

       \hrulefill
 \item $h(z)=e^{2\pi iz^2}-1$の$\lvert z \rvert < \frac{3}{2}$における
       全ての零点とその位数を求めよ。

       \dotfill

       
\end{enumerate}




\end{document}
