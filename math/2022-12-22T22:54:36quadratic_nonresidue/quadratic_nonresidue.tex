\documentclass[12pt,b5paper]{ltjsarticle}

%\usepackage[margin=15truemm, top=5truemm, bottom=5truemm]{geometry}
%\usepackage[margin=10truemm,left=15truemm]{geometry}
\usepackage[margin=10truemm]{geometry}

\usepackage{amsmath,amssymb}
%\pagestyle{headings}
\pagestyle{empty}

%\usepackage{listings,url}
%\renewcommand{\theenumi}{(\arabic{enumi})}

%\usepackage{graphicx}

%\usepackage{tikz}
%\usetikzlibrary {arrows.meta}
%\usepackage{wrapfig}	% required for `\wrapfigure' (yatex added)
%\usepackage{bm}	% required for `\bm' (yatex added)

% ルビを振る
%\usepackage{luatexja-ruby}	% required for `\ruby'

%% 核Ker 像Im Hom を定義
%\newcommand{\Img}{\mathop{\mathrm{Im}}\nolimits}
%\newcommand{\Ker}{\mathop{\mathrm{Ker}}\nolimits}
%\newcommand{\Hom}{\mathop{\mathrm{Hom}}\nolimits}

%\DeclareMathOperator{\Rot}{rot}
%\DeclareMathOperator{\Div}{div}
%\DeclareMathOperator{\Grad}{grad}
%\DeclareMathOperator{\arcsinh}{arcsinh}
%\DeclareMathOperator{\arccosh}{arccosh}
%\DeclareMathOperator{\arctanh}{arctanh}



%\usepackage{listings,url}
%
%\lstset{
%%プログラム言語(複数の言語に対応,C,C++も可)
%  language = Python,
%%  language = Lisp,
%%  language = C,
%  %背景色と透過度
%  %backgroundcolor={\color[gray]{.90}},
%  %枠外に行った時の自動改行
%  breaklines = true,
%  %自動改行後のインデント量(デフォルトでは20[pt])
%  breakindent = 10pt,
%  %標準の書体
%%  basicstyle = \ttfamily\scriptsize,
%  basicstyle = \ttfamily,
%  %コメントの書体
%%  commentstyle = {\itshape \color[cmyk]{1,0.4,1,0}},
%  %関数名等の色の設定
%  classoffset = 0,
%  %キーワード(int, ifなど)の書体
%%  keywordstyle = {\bfseries \color[cmyk]{0,1,0,0}},
%  %表示する文字の書体
%  %stringstyle = {\ttfamily \color[rgb]{0,0,1}},
%  %枠 "t"は上に線を記載, "T"は上に二重線を記載
%  %他オプション:leftline,topline,bottomline,lines,single,shadowbox
%  frame = TBrl,
%  %frameまでの間隔(行番号とプログラムの間)
%  framesep = 5pt,
%  %行番号の位置
%  numbers = left,
%  %行番号の間隔
%  stepnumber = 1,
%  %行番号の書体
%%  numberstyle = \tiny,
%  %タブの大きさ
%  tabsize = 4,
%  %キャプションの場所("tb"ならば上下両方に記載)
%  captionpos = t
%}



\begin{document}

\hrulefill

$p,q\in\mathbb{Z}$
について次を満たすとする。
\begin{equation}
 {}^{\exists}x\in\mathbb{Z}
  \text{ s.t. }
  x^2\equiv q \pmod{p}
\end{equation}

この時、$q$を$p$を法とする\textbf{平方剰余}という。
平方剰余でない数を\textbf{平方非剰余}という。


\hrulefill


$p$:奇素数、
$a$:$p$と互いに素な整数

$a$が$p$を法として\textbf{平方非剰余}であれば
次の式が成り立つ。(オイラーの規準)
\begin{equation}
 a^{\frac{p-1}{2}}\equiv -1 \pmod{p}
\end{equation}

\dotfill

この性質を利用し整数の素因子を探すことが出来る。

$n$を正の奇数とし、
$\frac{p-1}{2}$の奇数倍であるとする。
この時、整数$a$が$p$を法として平方非剰余であれば
$a^n+1\equiv 0 \pmod{p}$である。

%\begin{equation}
%  n,p,k\in\mathbb{Z},\
%  n>0,\
%  p:prime,\
%  n\equiv 1 \pmod{2},\
%  p\equiv 1 \pmod{2},\
%  k\equiv 1 \pmod{2},\
%  n=k\times \frac{p-1}{2}
%  \qquad
%  a:p\text{ を法として平方非剰余}
%  \Rightarrow
%  a^n+1\equiv 0 \pmod{p}
%\end{equation}

\begin{equation}
 a^{k\times \frac{p-1}{2}}\equiv -1 \pmod{p}
\end{equation}


\hrulefill

\textbf{問}

$5^{4851}+1$の素因子をこの方法でできるだけ多く見つけよ。
但し、$4851=3^2\cdot7^2\cdot11$である。

\dotfill

$4851=k\times\frac{p-1}{2}$となるような$p$を探す。
つまり、$4851$の因数となる$\frac{p-1}{2}$を探す。

$4851$の因数は次の通りである。
\begin{align}
 % 0個
 & 1\\
 % 1個
 & 3,\ 7,\ 11\\
 % 2個
 & 3^2,\ 3\cdot7,\ 3\cdot11,\ 7^2,\ 7\cdot11\\
 % 3個
 & 3^2\cdot7,\ 3^2\cdot11,\ 3\cdot7^2,\ 3\cdot7\cdot11,\ 7^2\cdot11\\
 % 4個
 & 3^2\cdot7^2,\ 3^2\cdot7\cdot11,\ 3\cdot7^2\cdot11\\
 % 5個
 & 3^2\cdot7^2\cdot11
\end{align}
因数は18個ある。
素因数は全て奇数$3,7,11$なので、因数も全て奇数である。

$\frac{p-1}{2}=1$の時、
つまり$p=3$を考える。
$3$と$5$は互いに素である。
そこで次のように式を変形できる。
\begin{equation}
 5^{4851}+1
  = 5^{4851\times \frac{3-1}{2}}+1
  \equiv 0\pmod{3}
\end{equation}
よって、素因数$3$を持つことがわかる。


これを残り17個の因数に対して考える。
つまり、$\frac{p-1}{2}$が因数と等しくなるような
奇素数$p(\ne5)$を探せばよい。

\begin{align}
 \frac{p-1}{2}=&3\ \Rightarrow &\ p=&7  \text{ [p]} \\
 \frac{p-1}{2}=&7 \ \Rightarrow &\ p=&15=3\cdot5  \\
 \frac{p-1}{2}=&11\ \Rightarrow &\ p=&23  \text{ [p]} \\
 %
 \frac{p-1}{2}=&3^2\ \Rightarrow &\ p=&19  \text{ [p]} \\
 \frac{p-1}{2}=&3\cdot7\ \Rightarrow &\ p=&43 \text{ [p]} \\
 \frac{p-1}{2}=&3\cdot11\ \Rightarrow &\ p=& 67  \text{ [p]} \\
 \frac{p-1}{2}=&7^2\ \Rightarrow &\ p=& 99 =3\cdot11\\
 \frac{p-1}{2}=&7\cdot11\ \Rightarrow &\ p=& 155 = 5\cdot31\\
 %
 \frac{p-1}{2}=&3^2\cdot7\ \Rightarrow &\ p=& 127 \text{ [p]} \\
 \frac{p-1}{2}=&3^2\cdot11\ \Rightarrow &\ p=& 199 \text{ [p]} \\
 \frac{p-1}{2}=&3\cdot7^2\ \Rightarrow &\ p=& 295 = 5\cdot59 \\
 \frac{p-1}{2}=&3\cdot7\cdot11\ \Rightarrow &\ p=& 463 \text{ [p]} \\
 \frac{p-1}{2}=&7^2\cdot11\ \Rightarrow &\ p=& 1079 =13\cdot83\\
 %
 \frac{p-1}{2}=&3^2\cdot7^2\ \Rightarrow &\ p=& 883  \text{ [p]} \\
 \frac{p-1}{2}=&3^2\cdot7\cdot11\ \Rightarrow &\ p=& 1387 = 19\cdot73\\
 \frac{p-1}{2}=&3\cdot7^2\cdot11\ \Rightarrow &\ p=& 3235=5\cdot647\\
 %
 \frac{p-1}{2}=&3^2\cdot7^2\cdot11\ \Rightarrow &\ p=& 9703 = 31\cdot313
\end{align}

18個の因数から10個の
素数$p=3,7,23,19,43,67,127,199,463,883$
が見つかる。
$p$がこれらの素数の時、
$\frac{p-1}{2}$は4851の因数であり、
$5^{4851}+1=5^{k\times\frac{p-1}{2}}+1$
となる。
このとき$k$は奇数である。

これらの素数は奇素数であり、
$5$と互いに素である。
$5^{k\times\frac{p-1}{2}}+1$は
$p$を法として$0$であるので、
次の素数が$5^{4851}+1$の素因数として見つかる。

\begin{equation}
 p=3,7,23,19,43,67,127,199,463,883
\end{equation}

\hrulefill

\end{document}
