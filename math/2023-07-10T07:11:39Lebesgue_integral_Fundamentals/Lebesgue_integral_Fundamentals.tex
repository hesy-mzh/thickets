\documentclass[12pt,b5paper]{ltjsarticle}

%\usepackage[margin=15truemm, top=5truemm, bottom=5truemm]{geometry}
%\usepackage[margin=10truemm,left=15truemm]{geometry}
\usepackage[margin=10truemm]{geometry}

\usepackage{amsmath,amssymb}
%\pagestyle{headings}
\pagestyle{empty}

%\usepackage{listings,url}
%\renewcommand{\theenumi}{(\arabic{enumi})}

%\usepackage{graphicx}

%\usepackage{tikz}
%\usetikzlibrary {arrows.meta}
%\usepackage{wrapfig}
%\usepackage{bm}

% ルビを振る
\usepackage{luatexja-ruby}	% required for `\ruby'

%% 核Ker 像Im Hom を定義
%\newcommand{\Img}{\mathop{\mathrm{Im}}\nolimits}
%\newcommand{\Ker}{\mathop{\mathrm{Ker}}\nolimits}
%\newcommand{\Hom}{\mathop{\mathrm{Hom}}\nolimits}

%\DeclareMathOperator{\Rot}{rot}
%\DeclareMathOperator{\Div}{div}
%\DeclareMathOperator{\Grad}{grad}
%\DeclareMathOperator{\arcsinh}{arcsinh}
%\DeclareMathOperator{\arccosh}{arccosh}
%\DeclareMathOperator{\arctanh}{arctanh}

\usepackage{url}

%\usepackage{listings}
%
%\lstset{
%%プログラム言語(複数の言語に対応,C,C++も可)
%  language = Python,
%%  language = Lisp,
%%  language = C,
%  %背景色と透過度
%  %backgroundcolor={\color[gray]{.90}},
%  %枠外に行った時の自動改行
%  breaklines = true,
%  %自動改行後のインデント量(デフォルトでは20[pt])
%  breakindent = 10pt,
%  %標準の書体
%%  basicstyle = \ttfamily\scriptsize,
%  basicstyle = \ttfamily,
%  %コメントの書体
%%  commentstyle = {\itshape \color[cmyk]{1,0.4,1,0}},
%  %関数名等の色の設定
%  classoffset = 0,
%  %キーワード(int, ifなど)の書体
%%  keywordstyle = {\bfseries \color[cmyk]{0,1,0,0}},
%  %表示する文字の書体
%  %stringstyle = {\ttfamily \color[rgb]{0,0,1}},
%  %枠 "t"は上に線を記載, "T"は上に二重線を記載
%  %他オプション:leftline,topline,bottomline,lines,single,shadowbox
%  frame = TBrl,
%  %frameまでの間隔(行番号とプログラムの間)
%  framesep = 5pt,
%  %行番号の位置
%  numbers = left,
%  %行番号の間隔
%  stepnumber = 1,
%  %行番号の書体
%%  numberstyle = \tiny,
%  %タブの大きさ
%  tabsize = 4,
%  %キャプションの場所("tb"ならば上下両方に記載)
%  captionpos = t
%}



\begin{document}

\hrulefill

\begin{enumerate}
 \item
      $(X,\mathcal{M})$を可測空間とし、
      $f:X\to [0,\infty]$は$\mathcal{M}$-可測とする。
      $X$上の$[0,\infty]$-値関数の列$\{ f_{n} \}_{n=1}^{\infty}$で
      次の条件を満たすものを構成せよ。
      \begin{enumerate}
       \item
            ${}^{\forall}n\in\mathbb{N}$に対し、
            $f_{n}$は$\mathcal{M}$-可測
       \item
            ${}^{\forall}n\in\mathbb{N}$と
            ${}^{\forall}x\in X$に対し、
            $f_{n}(x)\leq f_{n+1}(x)$
       \item
            ${}^{\forall}x\in X$に対し、
            $f(x)= \lim_{n\to\infty} f_{n}(x)$
      \end{enumerate}

      \dotfill

      $f_{n}(x) = f(x)-\frac{1}{n}$と定義すると、
      $f$は$\mathcal{M}$-可測関数なので、$f_{n}$も可測関数。

      $f_{n}(x) \leq f_{n+1}(x)$であり、
      $f(x) = \lim_{n\to\infty}f_{n}(x)$
      である。

      \hrulefill
 \item
      $(X,\mathcal{M})$を可測空間とする。
      $z\in X$を固定し、
      $\delta_{z}:\mathcal{M}\to [0,\infty]$
      を次で定義する。
      \begin{equation}
       \delta_{z}(A)=
        \begin{cases}
         1 & (z\in A)\\
         0 & (z\not\in A)
        \end{cases}
      \end{equation}
      この定義により
      $\delta_{z}$は
      $(X,\mathcal{M})$上の測度となる。

      \begin{enumerate}
       \item
            $A\in\mathcal{M}$に対し、
            $\delta_{z}(A) = \mathbf{1}_{A}(z)$
            が成り立つことを示せ。

            \dotfill

            指示関数
            $\mathbf{1}_{A}(z)$は
            次のような定義である。
             \begin{equation}
              \mathbf{1}_{A}(z)=
                \begin{cases}
                 1 & (z\in A)\\
                 0 & (z\not\in A)
                \end{cases}
             \end{equation}

            $z\in A$であれば、
            $\delta_{z}(A) = \mathbf{1}_{A}(z)=1$
            であり、
            $z\not\in A$であれば、
            $\delta_{z}(A) = \mathbf{1}_{A}(z)=0$
            である。

            つまり、
            $A\in\mathcal{M}$に対し、
            $\delta_{z}(A) = \mathbf{1}_{A}(z)$
            である。


            \hrulefill

       \item
            $f: X \to [0,\infty)$
            を$\mathcal{M}$-可測な単関数とする。
            この時、次が成り立つことを示せ。
            \begin{equation}
             \int_{X} f d\delta_{z} =f(z)
            \end{equation}

            \dotfill


            $\{\alpha_{j}\}_{j=1}^{k} \subset \mathbb{R}$
            と
            $\{A_{j}\}_{j=1}^{k} \subset \mathcal{M}$
            に対して、
            $f=\sum_{j=1}^{k}\alpha_{j}\mathbf{1}_{A_{j}}$
            とすると、
            ルベーグ積分の定義より左辺は次のようになる。
            \begin{equation}
             \int_{X} f d\delta_{z} = \sum_{j=1}^{k}\alpha_{j}\delta_{z}(A_{j})
            \end{equation}

            前問より、
            $\delta_{z}(A) = \mathbf{1}_{A}(z)$
            であるので、
            次が成り立つ。
            \begin{equation}
             \int_{X} f d\delta_{z}
              = \sum_{j=1}^{k}\alpha_{j}\delta_{z}(A_{j})
              = \sum_{j=1}^{k}\alpha_{j}\mathbf{1}_{A_{j}}(z)
              = f(z)
            \end{equation}

            \hrulefill

       \item
            $\mathcal{M}$-可測な関数
            $f: X \to [0,\infty]$に対し、
            (a) % 1問目
            と同じ等式が成り立つことを示せ。

            \dotfill


            $f(A) = \mathbf{1}_{A}(z)$



            \hrulefill

      \end{enumerate}



\end{enumerate}

\hrulefill

\end{document}
