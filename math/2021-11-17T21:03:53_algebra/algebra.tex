\documentclass[12pt,b5paper]{ltjsarticle}

\usepackage{amsmath}
\usepackage{amssymb}
\pagestyle{empty}

\begin{document}

\begin{enumerate}\setcounter{enumi}{4}
 \item
      $\vec{u} = \begin{pmatrix}1\\1\end{pmatrix},
      \quad \vec{v}=\begin{pmatrix}2\\3\end{pmatrix}$
      \begin{enumerate}\renewcommand{\theenumii}{\roman{enumii}}
       \item
            任意のベクトル$\vec{x}$に対し、
            $\vec{x}=a\vec{u}+b\vec{v}$を満たす
            整数の組$(a,b)\in\mathbb{Z}^2$が一意に存在するということは
            次の式の変形で得られる行列が正則である時である。
            \begin{align}
             \begin{pmatrix}x\\y\end{pmatrix}
                &= a\begin{pmatrix}1\\1\end{pmatrix}
               +b\begin{pmatrix}2\\3\end{pmatrix}\\
             &= \begin{pmatrix}1 & 2\\1 & 3\end{pmatrix}\begin{pmatrix}a\\b\end{pmatrix}
            \end{align}
            実際の逆行列
            $\begin{pmatrix}1 & 2\\1 & 3\end{pmatrix}^{-1}
             =\begin{pmatrix}3&-2\\-1&1\end{pmatrix}$
            が存在し、これを用いて
            \begin{align}
             \begin{pmatrix}3 & -2\\-1 & 1\end{pmatrix}\begin{pmatrix}x\\y\end{pmatrix}
             = \begin{pmatrix}3&-2\\-1&1\end{pmatrix}
             \begin{pmatrix}1 & 2\\1 & 3\end{pmatrix}
             \begin{pmatrix}a\\b\end{pmatrix}
            \end{align}
            となり、$\vec{x}$が定まると$(a,b)$が一意に定まる。
      \end{enumerate}
\end{enumerate}





\end{document}
