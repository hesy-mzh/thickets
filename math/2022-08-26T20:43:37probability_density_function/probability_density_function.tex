\documentclass[12pt,b5paper]{ltjsarticle}

%\usepackage[margin=15truemm, top=5truemm, bottom=5truemm]{geometry}
\usepackage[margin=10truemm]{geometry}

\usepackage{amsmath,amssymb}
%\pagestyle{headings}
\pagestyle{empty}

%\usepackage{listings,url}
%\renewcommand{\theenumi}{(\arabic{enumi})}

%\usepackage{graphicx}

%\usepackage{tikz}
%\usetikzlibrary {arrows.meta}
%\usepackage{wrapfig}	% required for `\wrapfigure' (yatex added)
%\usepackage{bm}	% required for `\bm' (yatex added)

% ルビを振る
%\usepackage{luatexja-ruby}	% required for `\ruby'

%% 核Ker 像Im Hom を定義
%\newcommand{\Img}{\mathop{\mathrm{Im}}\nolimits}
%\newcommand{\Ker}{\mathop{\mathrm{Ker}}\nolimits}
%\newcommand{\Hom}{\mathop{\mathrm{Hom}}\nolimits}

%\DeclareMathOperator{\Rot}{rot}
%\DeclareMathOperator{\Div}{div}
%\DeclareMathOperator{\Grad}{grad}
%\DeclareMathOperator{\arcsinh}{arcsinh}
%\DeclareMathOperator{\arccosh}{arccosh}
%\DeclareMathOperator{\arctanh}{arctanh}



\begin{document}

確率密度関数

\hrulefill

確率変数$X,Y$の密度関数をそれぞれ$f_X,f_Y$とする
また、$X,Y$の同時確率密度関数を$f_{X,Y}$とする。

この時、$f_X \times f_Y = f_{X,Y}$である。

\hrulefill

確率変数$X$は一様分布$U(0,1)$に従うとする。
この時、密度関数は次のようになる。
\begin{equation}
 f_X(x) = \begin{cases}1 & (0< x < 1)\\ 0 & (\text{other})\end{cases}
\end{equation}

確率変数$X,Y,Z$は独立で$U(0,1)$に従うとする。
この時、同時確率密度関数$f_{X,Y,Z}(x,y,z)$は次のようになる。
\begin{align}
 f_{X,Y,Z}(x,y,z) &= f_X(x)f_Y(y)f_Z(z)\\
 &=
 \begin{cases}
 1 & (0< x < 1 \ \text{かつ}\ 0< y < 1 \ \text{かつ}\ 0< z < 1)\\
 0 & (\text{other})
 \end{cases}
\end{align}

\end{document}

