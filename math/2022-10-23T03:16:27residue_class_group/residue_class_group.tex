\documentclass[12pt,b5paper]{ltjsarticle}

%\usepackage[margin=15truemm, top=5truemm, bottom=5truemm]{geometry}
\usepackage[margin=10truemm]{geometry}

\usepackage{amsmath,amssymb}
%\pagestyle{headings}
\pagestyle{empty}

%\usepackage{listings,url}
%\renewcommand{\theenumi}{(\arabic{enumi})}

%\usepackage{graphicx}

%\usepackage{tikz}
%\usetikzlibrary {arrows.meta}
%\usepackage{wrapfig}	% required for `\wrapfigure' (yatex added)
%\usepackage{bm}	% required for `\bm' (yatex added)

% ルビを振る
%\usepackage{luatexja-ruby}	% required for `\ruby'

%% 核Ker 像Im Hom を定義
%\newcommand{\Img}{\mathop{\mathrm{Im}}\nolimits}
%\newcommand{\Ker}{\mathop{\mathrm{Ker}}\nolimits}
%\newcommand{\Hom}{\mathop{\mathrm{Hom}}\nolimits}

%\DeclareMathOperator{\Rot}{rot}
%\DeclareMathOperator{\Div}{div}
%\DeclareMathOperator{\Grad}{grad}
%\DeclareMathOperator{\arcsinh}{arcsinh}
%\DeclareMathOperator{\arccosh}{arccosh}
%\DeclareMathOperator{\arctanh}{arctanh}



\begin{document}

\hrulefill
\textbf{定義}
\hrulefill

\textbf{剰余類群}

整数を$n$で割った余りの等しい整数を同じもの(剰余類)として
集めた集合を剰余類群といい、
$\mathbb{Z}/n\mathbb{Z}$と書く。

整数の加法($+$)をそのままこの剰余類群の演算として定義できる。

\textbf{準同型写像(群)}

$G,H$を群とする。
\begin{equation}
 f:G\to H
\end{equation}
写像$f$が準同型であるとは次を満たすときをいう。
\begin{equation}
 [{}^{\forall}a,b\in G] \qquad
  f(ab)=f(a)f(b)
\end{equation}

この時、
単位元$e_G\in G, e_H\in H$について、
$f(e_G)=e_H$が成り立つ。




\hrulefill
問題
\hrulefill

$\mathbb{Z}/6\mathbb{Z}$
と
$\mathbb{Z}/9\mathbb{Z}$
を加法群とみなす。

\begin{enumerate}
 \item
      $1 \in \mathbb{Z}/9\mathbb{Z}$
      は
      $\mathbb{Z}/9\mathbb{Z}$の生成元となること
      が正しいか否かを答えよ。

\dotfill

      $\mathbb{Z}/9\mathbb{Z}$の要素は次の通り。
      \begin{equation}
       \mathbb{Z}/9\mathbb{Z}=
        \{ \bar{0},\bar{1},\bar{2},\bar{3},\bar{4},\bar{5},
        \bar{6},\bar{7},\bar{8}\}
      \end{equation}

      これらは以下の演算により計算できる。
      \begin{align}
       \bar{0} =& \bar{9}
       = \bar{1}+_{9} \bar{1}+_{9} \bar{1}+_{9} \bar{1}+_{9}
       \bar{1}+_{9} \bar{1}+_{9} \bar{1}+_{9} \bar{1}+_{9} \bar{1}\\
       \bar{1}=& \bar{1}\\
       \bar{2} =& \bar{1}+_{9}\bar{1}\\
       \bar{3} =& \bar{1}+_{9}\bar{1}+_{9}\bar{1}\\
       \bar{4} =& \bar{1}+_{9}\bar{1}+_{9}\bar{1}+_{9}\bar{1}\\
       \bar{5} =& \bar{1}+_{9}\bar{1}+_{9}\bar{1}+_{9}\bar{1}+_{9}\bar{1}\\
       \bar{6} =& \bar{1}+_{9}\bar{1}+_{9}\bar{1}+_{9}\bar{1}+_{9}\bar{1}+_{9}\bar{1}\\
       \bar{7} =& \bar{1}+_{9}\bar{1}+_{9}\bar{1}+_{9}\bar{1}+_{9}\bar{1}+_{9}\bar{1}+_{9}\bar{1}\\
       \bar{8} =& \bar{1}+_{9}\bar{1}+_{9}\bar{1}+_{9}\bar{1}+_{9}\bar{1}+_{9}\bar{1}+_{9}\bar{1}+_{9}\bar{1}\\
      \end{align}

      よって、$1$を生成元とする部分群$\langle 1 \rangle$は
      $\langle 1 \rangle = \mathbb{Z}/9\mathbb{Z}$
      であることが分かる。

\hrulefill
 \item
      ${}^{\forall} a\in\mathbb{Z}/6\mathbb{Z}$に対し、
      $a +_{6} a +_{6} a +_{6} a +_{6} a +_{6} a =0$
      であること
      が正しいか否かを答えよ。

\dotfill

      $a\in\mathbb{Z}/6\mathbb{Z}$とする。

      $a=\bar{0}$の時、
      次のように$\bar{0}$となる。
      \begin{equation}
       a +_{6} a +_{6} a +_{6} a +_{6} a +_{6} a
        =
       \bar{0} +_{6} \bar{0} +_{6} \bar{0} +_{6} \bar{0} +_{6} \bar{0} +_{6} \bar{0} = \bar{0}
      \end{equation}

      $a\ne\bar{0}$の時を考える。
      
      $a$は$\bar{1}$のいくつかの和で表される。
      そこで、$\bar{1}$の$n$個の和として$a=\sum_{k=1}^{n}\bar{1}$とする。
      \begin{equation}
       a +_{6} a +_{6} a +_{6} a +_{6} a +_{6} a
        =
       \sum_{k=1}^{n}\bar{1} +_{6} \sum_{k=1}^{n}\bar{1} +_{6}
       \sum_{k=1}^{n}\bar{1} +_{6} \sum_{k=1}^{n}\bar{1} +_{6}
       \sum_{k=1}^{n}\bar{1} +_{6} \sum_{k=1}^{n}\bar{1}
      \end{equation}
      これは$k$番目の$\bar{1}$の和をまとめると次のようになる。
      \begin{align}
       a +_{6} a +_{6} a +_{6} a +_{6} a +_{6} a
        =&
       \sum_{k=1}^{n}( \bar{1} +_{6} \bar{1} +_{6}
       \bar{1} +_{6} \bar{1} +_{6} \bar{1} +_{6} \bar{1})\\
        =& \sum_{k=1}^{n} \bar{6}
        = \sum_{k=1}^{n} \bar{0}
        = \bar{0}
      \end{align}

      よって、
      ${}^{\forall} a\in\mathbb{Z}/6\mathbb{Z}$に対し、
      $a +_{6} a +_{6} a +_{6} a +_{6} a +_{6} a =0$
      である。

\hrulefill
 \item
      任意の群準同型
      $f:\mathbb{Z}/9\mathbb{Z}\to\mathbb{Z}/6\mathbb{Z}$
      に対し、
      $f(a) +_{6} f(a) +_{6} f(a) =0$
      であること
      が正しいか否かを答えよ。

\dotfill


      準同型写像$f$は単位元同士対応する。
      $f(\bar{0})=\bar{0}$

      単位元$\bar{0}$は$\bar{1}$の和として表される。
      \begin{align}
       \bar{0}=& \bar{1}+_{9}\bar{1}+_{9}\bar{1}+_{9}\bar{1}
       +_{9}\bar{1}+_{9}\bar{1}+_{9}\bar{1}+_{9}\bar{1}+_{9}\bar{1}
       & (\text{ in } \mathbb{Z}/9\mathbb{Z})\\
       \bar{0}=& \bar{1}+_{6}\bar{1}+_{6}\bar{1}+_{6}\bar{1}
       +_{6}\bar{1}+_{6}\bar{1}
       & (\text{ in } \mathbb{Z}/6\mathbb{Z})
      \end{align}

      これらにより
      単位元$\bar{0}$を$\bar{1}$の和として考える。
      \begin{align}
       \bar{0} =& f(\bar{0})\\
       =& f(\bar{1}+_{9}\bar{1}+_{9}\bar{1}+_{9}\bar{1}
       +_{9}\bar{1}+_{9}\bar{1}+_{9}\bar{1}+_{9}\bar{1}+_{9}\bar{1})\\
       =& f(\bar{1}) +_{6} f(\bar{1}) +_{6} f(\bar{1})+_{6}
       f(\bar{1}) +_{6} f(\bar{1}) +_{6} f(\bar{1}) +_{6}
       f(\bar{1}) +_{6} f(\bar{1}) +_{6} f(\bar{1})
      \end{align}

      $\mathbb{Z}/6\mathbb{Z}$の元は6個足すと$\bar{0}$になるので、
      上の式は次のようになる。
      \begin{equation}
       \bar{0} = f(\bar{1}) +_{6} f(\bar{1}) +_{6} f(\bar{1})+_{6} \bar{0}
      \end{equation}

      よって次の式が得られる。
      \begin{equation}
       f(\bar{1}) +_{6} f(\bar{1}) +_{6} f(\bar{1}) = \bar{0}
      \end{equation}

      ${}^{\forall}a\in\mathbb{Z}/9\mathbb{Z}$とする。

      $a$が$n$個の$\bar{1}$の和として表されるとする。
      \begin{equation}
       a=\sum_{k=1}^{n}\bar{1}
      \end{equation}

      これを用いて$f(a)$の和を考える。
      \begin{align}
       f(a) +_{6} f(a) +_{6} f(a)
       =& f\left( \sum_{k=1}^{n}\bar{1} \right)
       +_{6} f\left( \sum_{k=1}^{n}\bar{1} \right)
       +_{6} f\left( \sum_{k=1}^{n}\bar{1} \right)\\
       =& \sum_{k=1}^{n}f(\bar{1})
       +_{6} \sum_{k=1}^{n}f(\bar{1})
       +_{6} \sum_{k=1}^{n}f(\bar{1})\\
       =& \sum_{k=1}^{n}(f(\bar{1}) +_{6} f(\bar{1}) +_{6} f(\bar{1}))\\
       =& \sum_{k=1}^{n}\bar{0} = \bar{0}
      \end{align}

      以上により
      $f(a) +_{6} f(a) +_{6} f(a) = \bar{0}$
      である。

\hrulefill
 \item
      群準同型
      $\mathbb{Z}/9\mathbb{Z}\to\mathbb{Z}/6\mathbb{Z}$
      の総数を答えよ。

\dotfill

      準同型写像は$f(\bar{1})$が何になるかで具体的に定まる。

      $f(\bar{1})=\bar{1}$、
      $f(\bar{1})=\bar{3}$、
      $f(\bar{1})=\bar{5}$の場合、
      $f(\bar{1})+_{6}f(\bar{1})+_{6}f(\bar{1})\ne\bar{0}$となる為、
      準同型は存在しない。

      $f(\bar{1})=\bar{2}$の場合、
      $f(\bar{2})=f(\bar{1}+_{9}\bar{1})=f(\bar{1})+_{6}f(\bar{1})=\bar{4}$となる。
      これを$\mathbb{Z}/9\mathbb{Z}$の他の元についても行うと次が得られる。
      \begin{gather}
       f(\bar{0}) = f(\bar{3}) = f(\bar{6}) = \bar{0}\\
       f(\bar{1}) = f(\bar{4}) = f(\bar{7}) = \bar{2}\\
       f(\bar{2}) = f(\bar{5}) = f(\bar{8}) = \bar{4}
      \end{gather}

      $f(\bar{1})=\bar{4}$の場合、
      同様に考えると次のようになる。
      \begin{gather}
       f(\bar{0}) = f(\bar{3}) = f(\bar{6}) = \bar{0}\\
       f(\bar{1}) = f(\bar{4}) = f(\bar{7}) = \bar{4}\\
       f(\bar{2}) = f(\bar{5}) = f(\bar{8}) = \bar{2}
      \end{gather}

      $f(\bar{1})=\bar{0}$の場合、
      全ての要素を$\bar{0}$と対応づけるため
      $f$は零写像となる。

      以上により$f$の総数は3である。

\hrulefill
\end{enumerate}




\hrulefill



\end{document}

