\documentclass[12pt,b5paper]{ltjsarticle}

%\usepackage[margin=15truemm, top=5truemm, bottom=5truemm]{geometry}
%\usepackage[margin=10truemm,left=15truemm]{geometry}
\usepackage[margin=10truemm]{geometry}

\usepackage{amsmath,amssymb}
%\pagestyle{headings}
\pagestyle{empty}

%\usepackage{listings,url}
%\renewcommand{\theenumi}{(\arabic{enumi})}

%\usepackage{graphicx}

%\usepackage{tikz}
%\usetikzlibrary {arrows.meta}
%\usepackage{wrapfig}
%\usepackage{bm}

% ルビを振る
\usepackage{luatexja-ruby}	% required for `\ruby'

%% 核Ker 像Im Hom を定義
%\newcommand{\Img}{\mathop{\mathrm{Im}}\nolimits}
%\newcommand{\Ker}{\mathop{\mathrm{Ker}}\nolimits}
%\newcommand{\Hom}{\mathop{\mathrm{Hom}}\nolimits}

%\DeclareMathOperator{\Rot}{rot}
%\DeclareMathOperator{\Div}{div}
%\DeclareMathOperator{\Grad}{grad}
%\DeclareMathOperator{\arcsinh}{arcsinh}
%\DeclareMathOperator{\arccosh}{arccosh}
%\DeclareMathOperator{\arctanh}{arctanh}

%\usepackage{url}

%\usepackage{listings}
%
%\lstset{
%%プログラム言語(複数の言語に対応,C,C++も可)
%  language = Python,
%%  language = Lisp,
%%  language = C,
%  %背景色と透過度
%  %backgroundcolor={\color[gray]{.90}},
%  %枠外に行った時の自動改行
%  breaklines = true,
%  %自動改行後のインデント量(デフォルトでは20[pt])
%  breakindent = 10pt,
%  %標準の書体
%%  basicstyle = \ttfamily\scriptsize,
%  basicstyle = \ttfamily,
%  %コメントの書体
%%  commentstyle = {\itshape \color[cmyk]{1,0.4,1,0}},
%  %関数名等の色の設定
%  classoffset = 0,
%  %キーワード(int, ifなど)の書体
%%  keywordstyle = {\bfseries \color[cmyk]{0,1,0,0}},
%  %表示する文字の書体
%  %stringstyle = {\ttfamily \color[rgb]{0,0,1}},
%  %枠 "t"は上に線を記載, "T"は上に二重線を記載
%  %他オプション:leftline,topline,bottomline,lines,single,shadowbox
%  frame = TBrl,
%  %frameまでの間隔(行番号とプログラムの間)
%  framesep = 5pt,
%  %行番号の位置
%  numbers = left,
%  %行番号の間隔
%  stepnumber = 1,
%  %行番号の書体
%%  numberstyle = \tiny,
%  %タブの大きさ
%  tabsize = 4,
%  %キャプションの場所("tb"ならば上下両方に記載)
%  captionpos = t
%}

%\usepackage{cancel}
%\usepackage{bussproofs}
%\usepackage{proof}

\begin{document}

\hrulefill

\begin{gather}
    \mathcal{L}(\mathbb{R})
     : \mathbb{R} の ルベーグ可測集合全体\\
    C_{per}^{m}([-\pi,\pi])
     =\left\{ f:[-\pi,\pi]\to \mathbb{C} \mid f:C^{m}級、\;
     f^{(k)}(-\pi)=f^{(k)}(\pi) \; (0\leq k \leq m) \right\}\\
    \mathcal{L}^{p}(\mu)=\{ f:X\to\mathbb{C}
     \mid f:\mathcal{F}可測、\; \int_{X} \lvert f(x)\rvert^{p} \mu(dx) < \infty \}\\
     \|f\|_{p} = \left( \int_{X} \lvert f(x)\rvert^{p} \mu(dx)\right)^{\frac{1}{p}}\\
     L^{p}(\mu)=\mathcal{L}^{p}(\mu)/\sim,\quad
      f\sim g \Leftrightarrow f=g \; \mu - a.e.
\end{gather}

\begin{gather}
    \mathcal{L}^{p}(\mathbb{R}^{d})=
    \left\{ f:\mathbb{R}^{d}\to\mathbb{C}
     \;\middle|\; f:\mathcal{F}可測、\; \int_{\mathbb{R}^{d}} \lvert f(x)\rvert^{p} dx < \infty \right\}
\end{gather}

\hrulefill

\begin{description}
 \item [第6回]
      任意の$g\in \mathcal{L}^{p}(\mathbb{R}^{d})$に対して、
      次を証明せよ。
      \begin{equation}
       \int_{\mathbb{R}^{d}}
        \lvert g(x) - g(x)1{B(0,m)}(x) \rvert^{p}
        \mathrm{d}x
        \overset{m \to \infty}{\longrightarrow}
        0
      \end{equation}
      (HINT: \ruby{Lebesgue}{ルベーグ} の収束定理を使う。
      仮定の殆どは$\mathcal{L}^{p}(\mathbb{R}^{d})$
      の定義からわかる。)

    \dotfill

    $g\in \mathcal{L}^{p}(\mathbb{R}^{d})$より
    $g$は可測関数で、
    $\int_{\mathbb{R}^{d}} \lvert g(x)\rvert^{p} dx < \infty$である。

    任意の$m \in \mathbb{R}$に対して
    $\lvert g(x)1_{B(0,m)}(x) \rvert \leq \lvert g(x) \rvert$
    であり、
    $\displaystyle g(x) = \lim_{m\to\infty} g(x)1_{B(0,m)}(x)$
    である。
    
    これより次の式が成り立つ。
    \begin{gather}
        0\leq
        \lvert g(x) - g(x)1_{B(0,m)}(x) \rvert
        \leq \lvert g(x) \rvert\\
        \lim_{m\to\infty}\lvert g(x) - g(x)1_{B(0,m)}(x) \rvert =0
    \end{gather}

    $g(x) - g(x)1_{B(0,m)}(x)$も可測関数であるから、
    \ruby{Lebesgue}{ルベーグ} の収束定理より
    \begin{equation}
        \lim_{m\to\infty}\int_{\mathbb{R}^{d}}
        \lvert g(x) - g(x)1_{B(0,m)}(x) \rvert^{p}
        \mathrm{d}x
        =
        \int_{\mathbb{R}^{d}} \lim_{m\to\infty}
        \lvert g(x) - g(x)1_{B(0,m)}(x) \rvert^{p}
        \mathrm{d}x
    \end{equation}
    となる。

    よって、
    \begin{equation}
        \lim_{m\to\infty}\int_{\mathbb{R}^{d}}
        \lvert g(x) - g(x)1_{B(0,m)}(x) \rvert^{p}
        \mathrm{d}x
        =0
    \end{equation}

    \ruby{Lebesgue}{ルベーグ} の収束定理より
    \begin{equation}
        \int_{\mathbb{R}^{d}} \lvert g(x) \rvert \mathrm{d}x
        =
        \lim_{m\to\infty}
        \int_{\mathbb{R}^{d}} \lvert g(x)1_{B(0,m)}(x) \rvert \mathrm{d}x
    \end{equation}
    であるから
    \begin{gather}
        \int_{\mathbb{R}^{d}} \lvert g(x) \rvert \mathrm{d}x
        -
        \lim_{m\to\infty}
        \int_{\mathbb{R}^{d}} \lvert g(x)1_{B(0,m)}(x) \rvert \mathrm{d}x =0\\
        \lim_{m\to\infty}\int_{\mathbb{R}^{d}}
        \left( \lvert g(x) \rvert - \lvert g(x)1_{B(0,m)}(x) \rvert \right)\mathrm{d}x =0
    \end{gather}

    $|x-y| \leq |x|+|-y|$より
    $|x-y| - |-y| \leq |x|$
    \begin{equation}
        -|x-y| \leq |x|-|y| \leq |x-y| 
    \end{equation}

    $\lvert g(x)1_{B(0,m)}(x) \rvert \leq \lvert g(x) \rvert$より
    \begin{equation}
        \lvert g(x) \rvert - \lvert g(x)1_{B(0,m)}(x) \rvert
        \geq \lvert g(x) - g(x)1_{B(0,m)}(x) \rvert
    \end{equation}


    \begin{equation}
        \lvert g(x) - g(x)1_{B(0,m)}(x) \rvert
        \leq
        \lvert g(x) \rvert
    \end{equation}
    
    \begin{equation}
        \lim_{m\to\infty}\int_{\mathbb{R}^{d}}
        \lvert g(x) - g(x)1_{B(0,m)}(x) \rvert^{p}
        \mathrm{d}x
        =
        \int_{\mathbb{R}^{d}} \lim_{m\to\infty}
        \lvert g(x) - g(x)1_{B(0,m)}(x) \rvert^{p}
        \mathrm{d}x
    \end{equation}
    

    \hrulefill


 \item [第7回]
      $a = (a_{n})_{n\in\mathbb{Z}} \in \ell^{2}(\mathbb{Z})$
      に対して
      $f=\sum_{n\in\mathbb{Z}} a_{n}e_{n}
             \in L^{2}((-\pi,\pi),\frac{1}{2\pi}\mathrm{d}x)$
      とおく。
      このとき、
      次の式を示せ。
      \begin{equation}
       \hat{f}(m) = \lim_{N\to\infty}
        \left\langle \sum_{n=-N}^{N}a_{n}e_{n},e_{m} \right\rangle
      \end{equation}
      (HINT:内積が片方の変数について連続であることをまず示す。
      それは$\langle u_{n},v\rangle - \langle u,v\rangle
             = \langle u_{n}-u,v \rangle$ に
      \ruby{Schwarz}{シュワルツ}の不等式を使って分かる。)

             \dotfill




            \hrulefill

\newpage

 \item [第10回]
      内積空間$(V,\; \langle \cdot , \cdot \rangle)$において、
      以下の集合は閉部分集合であることを示せ。
      \begin{enumerate}
       \item
            $u\in V$と$c\in \mathbb{C}$に対して、
            $\{ v\in V : \langle u,v\rangle =c \}$

            \dotfill

            $V$ の部分集合$S_{(u,c)}$ を次のようにおく。
            \begin{equation}
             S_{(u,c)} = \{ v\in V : \langle u,v\rangle =c \}
            \end{equation}

            $p \in S_{(u,c)}^{c}$とし、
            $\varepsilon=\min \{ \lvert \langle p,s \rangle \rvert/2 : s \in S_{(u,c)} \}$
            とおき、
            $p$の$\varepsilon$近傍を
            $U_{(p,\varepsilon)}= \{ v\in V : \lvert \langle v,p \rangle \rvert < \varepsilon \}$
            とする。
            このとき、
            $S_{(u,c)} \cap U_{(p,\varepsilon)} = \emptyset$
            である。

            任意の$S_{(u,c)}^{c}$に対して同様の
            $\varepsilon$近傍が存在するため、
            $S_{(u,c)}^{c}$は開集合である。

            よって、
            $S_{(u,c)}$は閉集合である。

            \hrulefill

       \item
            一般の$A \subset V$に対して
            $A^{\perp} = \{ v\in V : {}^{\forall}u\in A,\;
            \langle u,v \rangle =0 \}$

            (HINT: 内積は片方の変数について連続、
            連続写像による閉集合の逆像は閉集合、
            閉集合族の共通部分は閉集合、などを思い出す。)

            \dotfill

            $v\in V$に対して
            連続写像$f_{v}$を次のように定義する。
            \begin{equation}
             f_{v} : V \rightarrow \mathbb{C} ,\quad u \mapsto \langle u, v \rangle
            \end{equation}

            逆像 $f_{v}^{-1}( \{ 0 \} )$ は
            閉集合$\{0\} \subset V$の逆像であるので、
            閉集合である。

            $A \subset V$の任意の元$a\in A$ に対して逆像が考えられ、
            $A^{\perp}$は次のような閉集合の共通部分である。
            \begin{equation}
             A^{\perp} = \bigcap_{a\in A} f_{a}^{-1}(\{0\})
            \end{equation}

            閉集合の共通部分は閉集合となるので、
            $A^{\perp}$は閉集合である。


            \hrulefill

      \end{enumerate}

\end{description}

\hrulefill

\end{document}
