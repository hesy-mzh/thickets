\documentclass[12pt,b5paper]{ltjsarticle}

%\usepackage[margin=15truemm, top=5truemm, bottom=5truemm]{geometry}
\usepackage[margin=10truemm]{geometry}

\usepackage{amsmath,amssymb}
%\pagestyle{headings}
\pagestyle{empty}

%\usepackage{listings,url}
%\renewcommand{\theenumi}{(\arabic{enumi})}

%\usepackage{graphicx}

%\usepackage{tikz}
%\usetikzlibrary {arrows.meta}
%\usepackage{wrapfig}	% required for `\wrapfigure' (yatex added)
%\usepackage{bm}	% required for `\bm' (yatex added)

% ルビを振る
%\usepackage{luatexja-ruby}	% required for `\ruby'

%% 核Ker 像Im Hom を定義
%\newcommand{\Img}{\mathop{\mathrm{Im}}\nolimits}
%\newcommand{\Ker}{\mathop{\mathrm{Ker}}\nolimits}
%\newcommand{\Hom}{\mathop{\mathrm{Hom}}\nolimits}

%\DeclareMathOperator{\Rot}{rot}
%\DeclareMathOperator{\Div}{div}
%\DeclareMathOperator{\Grad}{grad}
%\DeclareMathOperator{\arcsinh}{arcsinh}
%\DeclareMathOperator{\arccosh}{arccosh}
%\DeclareMathOperator{\arctanh}{arctanh}



\begin{document}

\textbf{Laplace 変換}

0よりも大きい実数上で定義された関数$f(t)$に対して、
複素数$s$を用いて次の式で定義される$F(s)$が存在する時、
$F(s)$を関数$f(t)$のラプラス変換という。
\begin{equation}
 F(s) = \int_{0}^{\infty} f(t)e^{-st} \mathrm{d}t
\end{equation}

実変数関数$f(t)$に対して複素変数関数$F(s)$を対応させる為、
演算子$\mathcal{L}$を用いて
$F(s)=\mathcal{L}[f(t)]$と表す。

\hrulefill

\textbf{性質}

\dotfill

複素数$s$ が $\mathrm{Re}(s)>0$ を満たす時
\begin{align}
 \mathcal{L}[1] =& \frac{1}{s} \label{const} &
 \mathcal{L}[t] =& \frac{1}{s^2} &
 \mathcal{L}[t^n] =& \frac{n!}{s^{n+1}}\\
 \mathcal{L}[\sin \omega t] =& \frac{\omega}{s^{2}+\omega^2} &
 \mathcal{L}[\cos \omega t] =& \frac{s}{s^{2}+\omega^2}
 \label{triangle}
\end{align}



複素数$s$ が $\mathrm{Re}(s)>a$ を満たす時
\begin{equation}
 \mathcal{L}[e^{at}] = \frac{1}{s-a}
\end{equation}

\dotfill

線形性

\begin{equation}
 \mathcal{L}[af(t)+bg(t)] = a\mathcal{L}[f(t)] + b\mathcal{L}[g(t)]
  \label{linear}
\end{equation}



定数$a$が$a>0$、$F(s)=\mathcal{L}[f(t)]$とする。
\begin{equation}
 \mathcal{L}[f(at)] = \frac{1}{a}F(\frac{s}{a})
  \qquad
 \mathcal{L}[e^{at}f(t)] = F(s-a)
 \label{exp}
\end{equation}



\dotfill

微分

\begin{gather}
 \mathcal{L}\left[ \frac{\mathrm{d}f(t)}{\mathrm{d}t} \right]
 = sF(s)-f(0) \label{diff1}\\
 \mathcal{L}\left[ \frac{\mathrm{d}^2f(t)}{\mathrm{d}t^2} \right]
 = s^2F(s)-sf(0)-\frac{\mathrm{d}f}{\mathrm{d}t}(0)\\
 \mathcal{L}\left[ \frac{\mathrm{d}^{n}f(t)}{\mathrm{d}t^n} \right]
 = s^2F(s)-s^{n-1}f(0)-s^{n-2}\frac{\mathrm{d}f}{\mathrm{d}t}(0)
 \cdots - s\frac{\mathrm{d}^{n-2}f}{\mathrm{d}t^{n-2}}(0)
 - \frac{\mathrm{d}^{n-1}f}{\mathrm{d}t^{n-1}}(0)
\end{gather}

3つ目の式は次のようになる。
ただし、$f^{(i)}(x)$ は連続 $(i=0,\dots, n-1)$
\begin{equation}
 \mathcal{L}[ f^{(n)}(t) ]
  = s^n F(s) - \sum_{k=1}^{n} s^{n-k}f^{(k-1)}(0)
\end{equation}





\dotfill

関数$f,g$の合成積$f*g$を次のように定義する。
\begin{equation}
 f*g =
  \int_{0}^{t} f(\tau) g(t-\tau) \mathrm{d}\tau
\end{equation}
この合成積についてラプラス変換は次を満たす。
\begin{equation}
 \mathcal{L}[(f*g)(t)] = \mathcal{L}[f(t)] \mathcal{L}[g(t)]
\end{equation}









\hrulefill

\begin{enumerate}
 \item
      $\mathcal{L}[e^{\alpha t}\sin \omega t]$
      及び
      $\mathcal{L}[e^{\alpha t}\cos \omega t]$
      を推移定理を用いてラプラス変換を求めよ。

      \dotfill

      式(\ref{triangle})より
      $\mathrm{Re}(s)>0$の時次の式が成り立つ。
      \begin{equation}
       \mathcal{L}[\sin \omega t] = \frac{\omega}{s^{2}+\omega^2}
        \qquad
       \mathcal{L}[\cos \omega t] = \frac{s}{s^{2}+\omega^2}
      \end{equation}

      式(\ref{exp})より
      $a>0$の時次が成り立つ。
      \begin{equation}
       \mathcal{L}[e^{at}f(t)] = F(s-a)
      \end{equation}

      \begin{align}
       \mathcal{L}[e^{\alpha t}\sin \omega t]
        =& \mathcal{L}[\sin \omega t](s-\alpha)
       & (\leftarrow\text{ラプラス変換後の変数}s\text{に}s-\alpha\text{を代入})\\
        =& \frac{\omega}{(s-\alpha)^{2}+\omega^2} 
      \end{align}
      同様に
      \begin{equation}
       \mathcal{L}[e^{\alpha t}\cos \omega t]
        = \mathcal{L}[\cos \omega t](s-\alpha)
        = \frac{s-\alpha}{(s-\alpha)^{2}+\omega^2} 
      \end{equation}

      よって、
      $\mathrm{Re}(s)>0,\ \alpha > 0$において
      \begin{align}
       \mathcal{L}[e^{\alpha t}\sin \omega t] = \frac{\omega}{(s-\alpha)^{2}+\omega^2}\\
       \mathcal{L}[e^{\alpha t}\cos \omega t] = \frac{s-\alpha}{(s-\alpha)^{2}+\omega^2}
      \end{align}

      \dotfill

 \item
      以下の微分方程式及び初期値が与えられた時、
      $\mathcal{L}[y(t)]$を$s$を用いて表せ。
      \begin{equation}
       \frac{\mathrm{d}y(t)}{\mathrm{d}t} + 2y(t) = 2
        ,\
        y(0) = 3
        \label{diff_eq}
      \end{equation}

      (HINT)
      微分方程式を両辺ラプラス変換する。

      \dotfill


      式(\ref{diff1})より
      \begin{equation}
       \mathcal{L}\left[ \frac{\mathrm{d}f(t)}{\mathrm{d}t} \right]
        = sF(s)-f(0)
      \end{equation}

      式(\ref{const})より
      \begin{equation}
        \mathcal{L}[1] = \frac{1}{s}
      \end{equation}

      線形性 式(\ref{linear}) より
      \begin{equation}
        \mathcal{L}[af(t)+bg(t)] = a\mathcal{L}[f(t)] + b\mathcal{L}[g(t)]
      \end{equation}

      これらを使い、微分方程式(\ref{diff_eq})をラプラス変換する。
      $Y(s) = \mathcal{L}[y(t)]$とする。

      左辺
      \begin{equation}
       \mathcal{L}\left[\frac{\mathrm{d}y(t)}{\mathrm{d}t} + 2y(t)\right]
        =
        \mathcal{L}\left[ \frac{\mathrm{d}y(t)}{\mathrm{d}t} \right]
        +2
        \mathcal{L}\left[ y(t) \right]
        = sY(s)-y(0) +2Y(s)
      \end{equation}

      右辺
      \begin{equation}
       \mathcal{L}[ 2 ] = \frac{2}{s}
      \end{equation}

      初期値は$y(0) = 3$であるので、次の式が得られる。
      \begin{equation}
       sY(s)-3 +2Y(s) = \frac{2}{s}
      \end{equation}

      これを$Y(s)$でまとめると
      \begin{equation}
       Y(s) = \frac{1}{s+2}\left(\frac{2}{s}+3\right)
        = \frac{2+3s}{s(s+2)}
      \end{equation}

      よって、$\mathrm{Re}(s)>0$において
      \begin{equation}
       \mathcal{L}[y(t)] = \frac{2+3s}{s(s+2)}
      \end{equation}


\end{enumerate}




\end{document}

