\documentclass[12pt,b5paper]{ltjsarticle}

%\usepackage[margin=15truemm, top=5truemm, bottom=5truemm]{geometry}
%\usepackage[margin=10truemm,left=15truemm]{geometry}
\usepackage[margin=10truemm]{geometry}

\usepackage{amsmath,amssymb}
%\pagestyle{headings}
\pagestyle{empty}

%\usepackage{listings,url}
%\renewcommand{\theenumi}{(\arabic{enumi})}

%\usepackage{graphicx}

%\usepackage{tikz}
%\usetikzlibrary {arrows.meta}
%\usepackage{wrapfig}
%\usepackage{bm}

% ルビを振る
%\usepackage{luatexja-ruby}	% required for `\ruby'

%% 核Ker 像Im Hom を定義
%\newcommand{\Img}{\mathop{\mathrm{Im}}\nolimits}
%\newcommand{\Ker}{\mathop{\mathrm{Ker}}\nolimits}
%\newcommand{\Hom}{\mathop{\mathrm{Hom}}\nolimits}

%\DeclareMathOperator{\Rot}{rot}
%\DeclareMathOperator{\Div}{div}
%\DeclareMathOperator{\Grad}{grad}
%\DeclareMathOperator{\arcsinh}{arcsinh}
%\DeclareMathOperator{\arccosh}{arccosh}
%\DeclareMathOperator{\arctanh}{arctanh}



%\usepackage{listings,url}
%
%\lstset{
%%プログラム言語(複数の言語に対応,C,C++も可)
%  language = Python,
%%  language = Lisp,
%%  language = C,
%  %背景色と透過度
%  %backgroundcolor={\color[gray]{.90}},
%  %枠外に行った時の自動改行
%  breaklines = true,
%  %自動改行後のインデント量(デフォルトでは20[pt])
%  breakindent = 10pt,
%  %標準の書体
%%  basicstyle = \ttfamily\scriptsize,
%  basicstyle = \ttfamily,
%  %コメントの書体
%%  commentstyle = {\itshape \color[cmyk]{1,0.4,1,0}},
%  %関数名等の色の設定
%  classoffset = 0,
%  %キーワード(int, ifなど)の書体
%%  keywordstyle = {\bfseries \color[cmyk]{0,1,0,0}},
%  %表示する文字の書体
%  %stringstyle = {\ttfamily \color[rgb]{0,0,1}},
%  %枠 "t"は上に線を記載, "T"は上に二重線を記載
%  %他オプション:leftline,topline,bottomline,lines,single,shadowbox
%  frame = TBrl,
%  %frameまでの間隔(行番号とプログラムの間)
%  framesep = 5pt,
%  %行番号の位置
%  numbers = left,
%  %行番号の間隔
%  stepnumber = 1,
%  %行番号の書体
%%  numberstyle = \tiny,
%  %タブの大きさ
%  tabsize = 4,
%  %キャプションの場所("tb"ならば上下両方に記載)
%  captionpos = t
%}



\begin{document}

\hrulefill

実数列$\{ u_{2n} \}_{n=0}^{\infty},\ \{ f_{2n} \}_{n=0}^{\infty}$
を次で定義する。
\begin{equation}
 u_{0}=1,\quad
  u_{2n} = \frac{1}{2^{2n}}\begin{pmatrix}2n\\n\end{pmatrix},\quad
  f_{0}=0,\quad
   f_{2n}=\frac{1}{2n}u_{2n-2}
\end{equation}

任意の自然数$n$に対し、
$u_{2n}$と$f_{2n}$は次で与えられることを示せ。
\begin{equation}
 u_{2n}=(-1)^{n}\begin{pmatrix}-1/2\\n\end{pmatrix},\quad
 f_{2n}=(-1)^{n-1}\begin{pmatrix}1/2\\n\end{pmatrix}
\end{equation}

ここで、
$\alpha\in\mathbb{R}\backslash\mathbb{N},\ n\in\mathbb{N}$
に対し
\begin{equation}
 \begin{pmatrix}\alpha \\ n\end{pmatrix}
 =
  \frac{1}{n!}\prod_{k=0}^{n-1}(\alpha-k)
  =
  \frac{\alpha(\alpha-1)(\alpha-2)\cdots(\alpha-n+1)}{n!}
\end{equation}
と定義する。


\dotfill

$u_{2n}$について、
それぞれの式を変形する。
\begin{align}
 \frac{1}{2^{2n}}\begin{pmatrix}2n\\n\end{pmatrix}
 =&
  \frac{1}{2^{2n}}
  \frac{1}{n!}\prod_{k=0}^{n-1}(2n-k)
=  \frac{1}{2^{2n}}
  \frac{1}{n!}\prod_{k=1}^{n}(n+k)\\
 (-1)^{n}\begin{pmatrix}-1/2\\n\end{pmatrix}
  = &
   (-1)^{n}
   \frac{1}{n!}\prod_{k=0}^{n-1} \left( -\frac{1}{2}-k \right)
   =
%   \frac{1}{n!}\prod_{k=0}^{n-1} \left( \frac{1}{2}+k \right)\\
%   =&
   \frac{1}{2^{2n}}\frac{1}{n!}\prod_{k=0}^{n-1} \left( 2+4k \right)
\end{align}


そこで
$\prod_{k=1}^{n}(n+k) = \prod_{k=0}^{n-1} \left( 2+4k \right)$
が成り立てばよい。

$n=1$の時、両辺は共に$2$となる。

$n=N$の時、等号が成り立つと仮定し、
$n=N+1$の場合を考える。
\begin{align}
 & \prod_{k=1}^{N+1}((N+1)+k)
  = \prod_{k=1}^{N+1}(N+(k+1))
  = \prod_{k=2}^{N+2}(N+k)\\
  =& \left( \prod_{k=2}^{N}(N+k) \right) (N+N+1)(N+N+2)
  = \left( \prod_{k=1}^{N}(N+k) \right) \cdot 2(2N+1)\\
  =& \left( \prod_{k=0}^{N-1}(2+4k) \right) \cdot (4N+2)
  = \prod_{k=0}^{N}(2+4k)
\end{align}

これにより次の式が成り立つことがわかる。
\begin{equation}
 \frac{1}{2^{2n}}\begin{pmatrix}2n\\n\end{pmatrix}
 =(-1)^{n}\begin{pmatrix}-1/2\\n\end{pmatrix}
\end{equation}

$f_{2n}$について式を変形する。
\begin{align}
 f_{2n}
  =& \frac{1}{2n} u_{2n-2}
  =\frac{1}{2n} (-1)^{n-1}\begin{pmatrix}-1/2\\ n-1 \end{pmatrix}\\
  =& \frac{1}{2n} (-1)^{n-1}
   \frac{1}{(n-1)!}\prod_{k=0}^{n-2} \left( -\frac{1}{2}-k \right)
 = \frac{1}{2}\frac{1}{n!}\prod_{k=0}^{n-2} \left( \frac{1}{2}+k \right)
\end{align}

$(-1)^{n-1}\begin{pmatrix}1/2\\n\end{pmatrix}$
を変形すると次のようになる。
\begin{align}
 (-1)^{n-1}\begin{pmatrix}1/2\\n\end{pmatrix}
 =& (-1)^{n-1} \frac{1}{n!}
   \prod_{k=0}^{n-1} \left( \frac{1}{2}-k \right)
 = \frac{-1}{n!} \prod_{k=0}^{n-1} \left( -\frac{1}{2}+k \right)\\
 =& \frac{-1}{n!} \left( -\frac{1}{2} +0 \right)
    \prod_{k=1}^{n-1} \left( -\frac{1}{2}+k \right)
 = \frac{-1}{n!} \left( -\frac{1}{2} \right)
    \prod_{k=0}^{n-2} \left( \frac{1}{2}+k \right)
\end{align}

これにより
$f_{2n}=(-1)^{n-1}\begin{pmatrix}1/2\\n\end{pmatrix}$
となる。


\hrulefill

$1$次元の単純ランダムウォークを
$S=\{ S_{n} \}_{n=0}^{\infty}$で表し、
$S_{0}=0$と仮定する。
自然数$n$に対し
\begin{equation}
 L_{2n} = \max\{m\in\mathbb{Z} \mid 0 \leq m \leq 2n \ かつ \ S_{m}=0 \}
\end{equation}
とおく。
任意の実数$\alpha\in[0,1]$に対して、
次が成り立つことを示せ。
\begin{equation}
 \lim_{n\to\infty} P(L_{2n} \leq 2\alpha n)
  = \frac{2}{\pi} \arcsin{ \sqrt{\alpha} }
\end{equation}

\dotfill

$S$は
$1$次元の単純ランダムウォーク
であるので、
時刻$t$における位置$S_{t}$に対して
時刻$t+1$における位置$S_{t+1}$が
$S_{t}+1$または$S_{t}-1$となる事をいう。
どちらかに移動する確率はそれぞれ$1/2$である。
\begin{equation}
 P(S_{t+1}-S_{t}=1) = \frac{1}{2},\quad
 P(S_{t+1}-S_{t}=-1) = \frac{1}{2}
\end{equation}




\hrulefill

\end{document}
