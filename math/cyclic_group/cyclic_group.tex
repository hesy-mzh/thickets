\documentclass[12pt,b5paper]{ltjsarticle}

%\usepackage[margin=15truemm, top=5truemm, bottom=5truemm]{geometry}
\usepackage[margin=10truemm]{geometry}

\usepackage{amsmath,amssymb}
%\pagestyle{headings}
\pagestyle{empty}

%\usepackage{listings,url}
%\renewcommand{\theenumi}{(\arabic{enumi})}

%\usepackage{graphicx}

%\usepackage{tikz}
%\usetikzlibrary {arrows.meta}
%\usepackage{wrapfig}	% required for `\wrapfigure' (yatex added)
%\usepackage{bm}	% required for `\bm' (yatex added)

% ルビを振る
%\usepackage{luatexja-ruby}	% required for `\ruby'

%% 核Ker 像Im Hom を定義
%\newcommand{\Img}{\mathop{\mathrm{Im}}\nolimits}
%\newcommand{\Ker}{\mathop{\mathrm{Ker}}\nolimits}
%\newcommand{\Hom}{\mathop{\mathrm{Hom}}\nolimits}

%\DeclareMathOperator{\Rot}{rot}
%\DeclareMathOperator{\Div}{div}
%\DeclareMathOperator{\Grad}{grad}
%\DeclareMathOperator{\arcsinh}{arcsinh}
%\DeclareMathOperator{\arccosh}{arccosh}
%\DeclareMathOperator{\arctanh}{arctanh}



%\usepackage{listings,url}
%
%\lstset{
%%プログラム言語(複数の言語に対応,C,C++も可)
%%  language = Python,
%%  language = Lisp,
%  language = C,
%  %背景色と透過度
%  %backgroundcolor={\color[gray]{.90}},
%  %枠外に行った時の自動改行
%  breaklines = true,
%  %自動改行後のインデント量(デフォルトでは20[pt])
%  breakindent = 10pt,
%  %標準の書体
%%  basicstyle = \ttfamily\scriptsize,
%  basicstyle = \ttfamily,
%  %コメントの書体
%%  commentstyle = {\itshape \color[cmyk]{1,0.4,1,0}},
%  %関数名等の色の設定
%  classoffset = 0,
%  %キーワード(int, ifなど)の書体
%%  keywordstyle = {\bfseries \color[cmyk]{0,1,0,0}},
%  %表示する文字の書体
%  %stringstyle = {\ttfamily \color[rgb]{0,0,1}},
%  %枠 "t"は上に線を記載, "T"は上に二重線を記載
%  %他オプション:leftline,topline,bottomline,lines,single,shadowbox
%  frame = TBrl,
%  %frameまでの間隔(行番号とプログラムの間)
%  framesep = 5pt,
%  %行番号の位置
%  numbers = left,
%  %行番号の間隔
%  stepnumber = 1,
%  %行番号の書体
%%  numberstyle = \tiny,
%  %タブの大きさ
%  tabsize = 4,
%  %キャプションの場所("tb"ならば上下両方に記載)
%  captionpos = t
%}



\begin{document}

\hrulefill
\textbf{問題}
\hrulefill



$\mathbb{C}^{\times}$の部分群$H$を次のように定める。
\begin{equation}
 H=\{z\in\mathbb{C}^{\times} \mid z^{6}=1\}
  ,\quad
  \mathbb{C}^{\times}=\mathbb{C}\backslash \{0\}
\end{equation}
\begin{enumerate}
 \item
      次の中で$H$に属する複素数を全て選べ。
      \begin{gather}
        1,\ -1,\
        i,\ -i,\
        \sqrt{3}i,\ -\sqrt{3}i\\
        \frac{1}{2}-\frac{1}{2}\sqrt{3}i,\
        -\frac{1}{2}+\frac{1}{2}\sqrt{3}i,\
        \frac{1}{2}+\frac{1}{2}\sqrt{3}i,\
        -\frac{1}{2}-\frac{1}{2}\sqrt{3}i\\
        \frac{1}{2}\sqrt{2}-\frac{1}{2}\sqrt{2}i,\
        \frac{1}{2}\sqrt{2}+\frac{1}{2}\sqrt{2}i,\
        -\frac{1}{2}\sqrt{2}+\frac{1}{2}\sqrt{2}i,\
        -\frac{1}{2}\sqrt{2}-\frac{1}{2}\sqrt{2}i
      \end{gather}
\dotfill

      $z^6=1$を満たす複素数を探す。
      
      全ての複素数はオイラーの公式から$re^{i\theta}$と表せる。
      $(re^{i\theta})^6=r^{6}e^{6i\theta}$であり、
      $e^{2\pi n i}=1$であるので、
      $r= 1,\ 6\theta = 2\pi n \ (n\in\mathbb{Z})$
      を満たせばよい。
      \begin{gather}
        1=e^{i0},\ -1=e^{i\pi},\
        i=e^{\frac{i\pi}{2}},\ -i=e^{\frac{3i\pi}{2}},\
        \sqrt{3}i=\sqrt{3}e^{\frac{i\pi}{2}},\ -\sqrt{3}i=\sqrt{3}e^{\frac{3i\pi}{2}}\\
        \frac{1}{2}-\frac{1}{2}\sqrt{3}i=e^{\frac{5i\pi}{3}},\
        -\frac{1}{2}+\frac{1}{2}\sqrt{3}i,\
        \frac{1}{2}+\frac{1}{2}\sqrt{3}i,\
        -\frac{1}{2}-\frac{1}{2}\sqrt{3}i\\
        \frac{1}{2}\sqrt{2}-\frac{1}{2}\sqrt{2}i,\
        \frac{1}{2}\sqrt{2}+\frac{1}{2}\sqrt{2}i,\
        -\frac{1}{2}\sqrt{2}+\frac{1}{2}\sqrt{2}i,\
        -\frac{1}{2}\sqrt{2}-\frac{1}{2}\sqrt{2}i
      \end{gather}


\hrulefill
 \item
      $H$の要素数を求めよ。

 \item
      $H$が巡回群であるか否かを答えよ。
\end{enumerate}


\hrulefill
\textbf{問題}
\hrulefill

\begin{enumerate}
 \item
      $n\in\mathbb{Z}_{\geq 1}$とする。
      任意の整数$k$に対して、
      $\bar{k}\in\{0,1,\dots,n-1\}$を
      $k$の$n$によるじょうよとする。
      このとき、$\phi:\mathbb{Z}\to\mathbb{Z}/n\mathbb{Z}$を
      $k\mapsto \bar{k}$とすると、
      $\phi$は準同型写像であることを示せ。
      
 \item
      $\phi:\mathbb{Z}/2\mathbb{Z}\times\mathbb{Z}/2\mathbb{Z}\to D_{4}$
      を $(i,j)\mapsto r^is^j$とすると
      $\phi$は同型写像であることを示せ。
\end{enumerate}



\end{document}
