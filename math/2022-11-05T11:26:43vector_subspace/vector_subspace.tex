\documentclass[12pt,b5paper]{ltjsarticle}

%\usepackage[margin=15truemm, top=5truemm, bottom=5truemm]{geometry}
\usepackage[margin=10truemm]{geometry}

\usepackage{amsmath,amssymb}
%\pagestyle{headings}
\pagestyle{empty}

%\usepackage{listings,url}
%\renewcommand{\theenumi}{(\arabic{enumi})}

%\usepackage{graphicx}

%\usepackage{tikz}
%\usetikzlibrary {arrows.meta}
%\usepackage{wrapfig}	% required for `\wrapfigure' (yatex added)
\usepackage{bm}	% required for `\bm' (yatex added)

% ルビを振る
%\usepackage{luatexja-ruby}	% required for `\ruby'

%% 核Ker 像Im Hom を定義
%\newcommand{\Img}{\mathop{\mathrm{Im}}\nolimits}
%\newcommand{\Ker}{\mathop{\mathrm{Ker}}\nolimits}
%\newcommand{\Hom}{\mathop{\mathrm{Hom}}\nolimits}

%\DeclareMathOperator{\Rot}{rot}
%\DeclareMathOperator{\Div}{div}
%\DeclareMathOperator{\Grad}{grad}
%\DeclareMathOperator{\arcsinh}{arcsinh}
%\DeclareMathOperator{\arccosh}{arccosh}
%\DeclareMathOperator{\arctanh}{arctanh}



%\usepackage{listings,url}
%
%\lstset{
%%プログラム言語(複数の言語に対応,C,C++も可)
%%  language = Python,
%%  language = Lisp,
%  language = C,
%  %背景色と透過度
%  %backgroundcolor={\color[gray]{.90}},
%  %枠外に行った時の自動改行
%  breaklines = true,
%  %自動改行後のインデント量(デフォルトでは20[pt])
%  breakindent = 10pt,
%  %標準の書体
%%  basicstyle = \ttfamily\scriptsize,
%  basicstyle = \ttfamily,
%  %コメントの書体
%%  commentstyle = {\itshape \color[cmyk]{1,0.4,1,0}},
%  %関数名等の色の設定
%  classoffset = 0,
%  %キーワード(int, ifなど)の書体
%%  keywordstyle = {\bfseries \color[cmyk]{0,1,0,0}},
%  %表示する文字の書体
%  %stringstyle = {\ttfamily \color[rgb]{0,0,1}},
%  %枠 "t"は上に線を記載, "T"は上に二重線を記載
%  %他オプション:leftline,topline,bottomline,lines,single,shadowbox
%  frame = TBrl,
%  %frameまでの間隔(行番号とプログラムの間)
%  framesep = 5pt,
%  %行番号の位置
%  numbers = left,
%  %行番号の間隔
%  stepnumber = 1,
%  %行番号の書体
%%  numberstyle = \tiny,
%  %タブの大きさ
%  tabsize = 4,
%  %キャプションの場所("tb"ならば上下両方に記載)
%  captionpos = t
%}



\begin{document}

\hrulefill
\textbf{定義}
\hrulefill

\textbf{ベクトル空間}

集合$V$に対し、
\begin{align}
 \text{和} \qquad \qquad \bm{a},\bm{b}\in W & \Rightarrow \bm{a}+\bm{b}\in W\\
 \text{スカラー倍} \quad \bm{a}\in W, c\in K & \Rightarrow c\bm{a}\in W
\end{align}
が定義されていて、
次の8個の公理を満たすとき
$V$を$K$上のベクトル空間という。

\textbf{公理}
\begin{enumerate}
 \item
      ${}^{\forall}\bm{a},\bm{b}\in V$に対して
      $\bm{a}+\bm{b}=\bm{b}+\bm{a}$
 \item
      ${}^{\forall}\bm{a},\bm{b},\bm{c}\in V$に対して
      $(\bm{a}+\bm{b})+\bm{c}=\bm{a}+(\bm{b}+\bm{c})$
 \item
      ${}^{\forall}\bm{a}\in V$に対して
      $\bm{a}+\bm{0}=\bm{a}$となる元$\bm{0}$がある
 \item
      ${}^{\forall}\bm{a}\in V$に対して
      $\bm{a}+\bm{a^\prime}=\bm{0}$となる$V$の元$\bm{a^\prime}$がある
 \item
      ${}^{\forall}c\in K$と${}^{\forall}\bm{a},\bm{b}\in V$に対して
      $c(\bm{a}+\bm{b})=c\bm{a}+c\bm{b}$
 \item
      ${}^{\forall}c,c^{\prime}\in K$と${}^{\forall}\bm{a}\in V$に対して
      $(c+c^\prime)\bm{a}=c\bm{a}+c^\prime\bm{a}$
 \item
      ${}^{\forall}c,c^{\prime}\in K$と${}^{\forall}\bm{a}\in V$に対して
      $c(c^\prime\bm{a})=(cc^\prime)\bm{a}$
 \item
      ${}^{\forall}\bm{a}\in V$に対して
      $1\bm{a}=\bm{a}$
\end{enumerate}


\hrulefill
\textbf{問題}
\hrulefill

ベクトル空間$V$の部分集合$W$が
次の条件を満たすとする。
\begin{align}
 \bm{a},\bm{b}\in W & \Rightarrow \bm{a}+\bm{b}\in W \label{1st}\\
 \bm{a}\in W, c\in K & \Rightarrow c\bm{a}\in W \label{2nd}
\end{align}

このとき、
$W$はベクトル空間の公理を満たすことを示せ。

\dotfill

$V$はベクトル空間であるので公理をすべて満たしている。
部分集合$W\subset V$の元について
$W$の中で公理を満たしていることを確認する。

\begin{enumerate}
 \item
      ${}^{\forall}\bm{a},\bm{b}\in W$とする。

      $\bm{a},\bm{b}\in W \subset V$であるので、
      $V$の公理から$\bm{a}+\bm{b}=\bm{b}+\bm{a}$が満たされる。

      式(\ref{1st})より、
      $\bm{a}+\bm{b}\in W$であり、
      $\bm{b}+\bm{a}\in W$であるから、
      $W$の中で$\bm{a}+\bm{b}=\bm{b}+\bm{a}$である。

 \item
      ${}^{\forall}\bm{a},\bm{b},\bm{c}\in W$とする。

      $\bm{a},\bm{b},\bm{c}\in W \subset V$より
      $V$上で
      $(\bm{a}+\bm{b})+\bm{c}=\bm{a}+(\bm{b}+\bm{c})$
      を満たす。

      式(\ref{1st})より、
      $\bm{a}+\bm{b}\in W$であり、
      $\bm{b}+\bm{c}\in W$である。
      よって、
      $(\bm{a}+\bm{b})+\bm{c}\in W$であり、
      $\bm{a}+(\bm{b}+\bm{c})\in W$であるから
      $W$上でも
      $(\bm{a}+\bm{b})+\bm{c}=\bm{a}+(\bm{b}+\bm{c})$
      である。

 \item \label{ax}
      ${}^{\forall}\bm{a}\in W$とする。

      $\bm{a}\in W \subset V$である為、
      $\bm{a}+\bm{0}=\bm{a}$となる元$\bm{0}$が$V$に存在する。

      $0\in K , \bm{a}\in V$より
      $0\bm{a}=\bm{0}\in V$だが
      $\bm{a}\in W$であるので
      式(\ref{2nd})より
      $0\bm{a}=\bm{0}\in W$となる。

 \item
      ${}^{\forall}\bm{a}\in W$とする。

      $\bm{a}\in W \subset V$である為、
      $\bm{a}+\bm{a^\prime}=\bm{0}$となる元$\bm{a^\prime}$が$V$に存在する。

      上の条件\ref{ax}より
      $0\bm{a}=\bm{0}$である。
      $0\in K$であるが、$K$上で$1+\alpha =0$なる$\alpha$が存在する。

      また、次にある条件\ref{axa}
      [$(1+\alpha)\bm{a}=1\bm{a}+\alpha\bm{a}$]と
      条件\ref{axb} [$1\bm{a}=\bm{a}$] により
      $W$上で次の式が成り立つ。
      \begin{equation}
       0 = 0\bm{a}=(1+\alpha)\bm{a}
        =1\bm{a}+\alpha\bm{a}
        =\bm{a}+\alpha\bm{a}
      \end{equation}

      この$\alpha\bm{a}\in W$は$\bm{a^\prime} \in V$と一致する。

 \item
      ${}^{\forall}c\in K,\ {}^{\forall}\bm{a},\bm{b}\in W$とする。

      $\bm{a},\bm{b}\in W \subset V$であるので、
      $V$上で
      $c(\bm{a}+\bm{b})=c\bm{a}+c\bm{b}$
      である。

      式(\ref{1st})より$\bm{a}+\bm{b}\in W$であり、
      式(\ref{2nd})より$c(\bm{a}+\bm{b})\in W$となる。

      $c\in K,\ \bm{a},\bm{b}\in W$であるから
      式(\ref{2nd})より$c\bm{a},c\bm{b}\in W$であり、
      式(\ref{1st})より$c\bm{a}+c\bm{b}\in W$である。

      つまり、
      $W$上でも$c(\bm{a}+\bm{b})=c\bm{a}+c\bm{b}$である。

 \item \label{axa}
      ${}^{\forall}c,c^{\prime}\in K,\ {}^{\forall}\bm{a}\in W$とする。

      $\bm{a}\in W \subset V$である為、
      $V$上で$(c+c^\prime)\bm{a}=c\bm{a}+c^\prime\bm{a}$である。
      
      $c,c^{\prime}\in K$より
      $c+c^{\prime}\in K$である為、
      式(\ref{2nd})より$(c+c^\prime)\bm{a}\in W$である。

      また、
      式(\ref{2nd})より$c\bm{a}, c^\prime\bm{a}\in W$であり、
      式(\ref{1st})より$c\bm{a}+c^\prime\bm{a}\in W$である。
      
      つまり、$W$上で
      $(c+c^\prime)\bm{a}=c\bm{a}+c^\prime\bm{a}$
      となる。

 \item
      ${}^{\forall}c,c^\prime\in K,\ {}^{\forall}\bm{a}\in W$とする。

      $\bm{a}\in W \subset V$である為、
      $V$上で$c(c^\prime\bm{a})=(cc^\prime)\bm{a}$であるが、
      式(\ref{2nd})より$c^\prime\bm{a}\in W$であり
      $c(c^\prime\bm{a})\in W$である。
      $cc^\prime \in K$であるので$(cc^\prime)\bm{a}\in W$である。
      つまり、
      $W$の元としても$c(c^\prime\bm{a})=(cc^\prime)\bm{a}$である。

 \item\label{axb}
      ${}^{\forall}\bm{a}\in W$とする。

      $\bm{a}\in W \subset V$である為、
      $V$上で$1\bm{a}=\bm{a}$であるが、
      $W$の元としても$1\bm{a}=\bm{a}$である。

\end{enumerate}


\end{document}
