\documentclass[12pt,b5paper]{ltjsarticle}

%\usepackage[margin=15truemm, top=5truemm, bottom=5truemm]{geometry}
\usepackage[margin=10truemm]{geometry}

\usepackage{amsmath,amssymb}
%\pagestyle{headings}
\pagestyle{empty}

%\usepackage{listings,url}
%\renewcommand{\theenumi}{(\arabic{enumi})}

%\usepackage{graphicx}

%\usepackage{tikz}
%\usetikzlibrary {arrows.meta}
%\usepackage{wrapfig}	% required for `\wrapfigure' (yatex added)
%\usepackage{bm}	% required for `\bm' (yatex added)

% ルビを振る
%\usepackage{luatexja-ruby}	% required for `\ruby'

%% 核Ker 像Im Hom を定義
%\newcommand{\Img}{\mathop{\mathrm{Im}}\nolimits}
%\newcommand{\Ker}{\mathop{\mathrm{Ker}}\nolimits}
%\newcommand{\Hom}{\mathop{\mathrm{Hom}}\nolimits}

%\DeclareMathOperator{\Rot}{rot}
%\DeclareMathOperator{\Div}{div}
%\DeclareMathOperator{\Grad}{grad}
%\DeclareMathOperator{\arcsinh}{arcsinh}
%\DeclareMathOperator{\arccosh}{arccosh}
%\DeclareMathOperator{\arctanh}{arctanh}



%\usepackage{listings,url}
%
%\lstset{
%%プログラム言語(複数の言語に対応,C,C++も可)
%%  language = Python,
%%  language = Lisp,
%  language = C,
%  %背景色と透過度
%  %backgroundcolor={\color[gray]{.90}},
%  %枠外に行った時の自動改行
%  breaklines = true,
%  %自動改行後のインデント量(デフォルトでは20[pt])
%  breakindent = 10pt,
%  %標準の書体
%%  basicstyle = \ttfamily\scriptsize,
%  basicstyle = \ttfamily,
%  %コメントの書体
%%  commentstyle = {\itshape \color[cmyk]{1,0.4,1,0}},
%  %関数名等の色の設定
%  classoffset = 0,
%  %キーワード(int, ifなど)の書体
%%  keywordstyle = {\bfseries \color[cmyk]{0,1,0,0}},
%  %表示する文字の書体
%  %stringstyle = {\ttfamily \color[rgb]{0,0,1}},
%  %枠 "t"は上に線を記載, "T"は上に二重線を記載
%  %他オプション:leftline,topline,bottomline,lines,single,shadowbox
%  frame = TBrl,
%  %frameまでの間隔(行番号とプログラムの間)
%  framesep = 5pt,
%  %行番号の位置
%  numbers = left,
%  %行番号の間隔
%  stepnumber = 1,
%  %行番号の書体
%%  numberstyle = \tiny,
%  %タブの大きさ
%  tabsize = 4,
%  %キャプションの場所("tb"ならば上下両方に記載)
%  captionpos = t
%}



\begin{document}

\hrulefill
\textbf{問題}
\hrulefill



$\mathbb{C}^{\times}$の部分群$H$を次のように定める。
\begin{equation}
 H=\{z\in\mathbb{C}^{\times} \mid z^{6}=1\}
  ,\quad
  \mathbb{C}^{\times}=\mathbb{C}\backslash \{0\}
\end{equation}
\begin{enumerate}
 \item
      次の中で$H$に属する複素数を全て選べ。
      \begin{gather}
        1,\ -1,\
        i,\ -i,\
        \sqrt{3}i,\ -\sqrt{3}i\\
        \frac{1}{2}-\frac{1}{2}\sqrt{3}i,\
        -\frac{1}{2}+\frac{1}{2}\sqrt{3}i,\
        \frac{1}{2}+\frac{1}{2}\sqrt{3}i,\
        -\frac{1}{2}-\frac{1}{2}\sqrt{3}i\\
        \frac{1}{2}\sqrt{2}-\frac{1}{2}\sqrt{2}i,\
        \frac{1}{2}\sqrt{2}+\frac{1}{2}\sqrt{2}i,\
        -\frac{1}{2}\sqrt{2}+\frac{1}{2}\sqrt{2}i,\
        -\frac{1}{2}\sqrt{2}-\frac{1}{2}\sqrt{2}i
      \end{gather}
\dotfill

      $z^6=1$を満たす複素数を探す。
      
      全ての複素数はオイラーの公式から$re^{i\theta}$と表せる。
      $(re^{i\theta})^6=r^{6}e^{6i\theta}$であり、
      $e^{2\pi n i}=1$であるので、
      $r= 1,\ 6\theta = 2\pi n \ (n\in\mathbb{Z})$
      を満たせばよい。
      \begin{gather}
        1=e^{i0},\ -1=e^{i\pi},\
        i=e^{\frac{\pi}{2}i},\ -i=e^{\frac{3\pi}{2}i},\
        \sqrt{3}i=\sqrt{3}e^{\frac{\pi}{2}i},\ -\sqrt{3}i=\sqrt{3}e^{\frac{3\pi}{2}i}\\
        \frac{1}{2}-\frac{1}{2}\sqrt{3}i=e^{\frac{5\pi}{3}i},\
        -\frac{1}{2}+\frac{1}{2}\sqrt{3}i=e^{\frac{2\pi}{3}i},\\
        \frac{1}{2}+\frac{1}{2}\sqrt{3}i=e^{\frac{\pi}{3}i},\
        -\frac{1}{2}-\frac{1}{2}\sqrt{3}i=e^{\frac{4\pi}{3}i}\\
        \frac{1}{2}\sqrt{2}-\frac{1}{2}\sqrt{2}i=e^{\frac{7\pi}{4}i},\
        \frac{1}{2}\sqrt{2}+\frac{1}{2}\sqrt{2}i=e^{\frac{\pi}{4}i},\\
        -\frac{1}{2}\sqrt{2}+\frac{1}{2}\sqrt{2}i=e^{\frac{3\pi}{4}i},\
        -\frac{1}{2}\sqrt{2}-\frac{1}{2}\sqrt{2}i=e^{\frac{5\pi}{4}i}
      \end{gather}

      これにより$r=1$でない数($\pm\sqrt{3}i$)や
      $6\theta = 2\pi n \ (n\in\mathbb{Z})$を満たさない数
      ($\pm i, \pm\frac{\sqrt{2}}{2}\pm\frac{\sqrt{2}}{2}i$)
      は$H$に含まれない。

      よって、$H$に含まれる数は次の通り。
      \begin{equation}
       1,\ -1,\ 
        \frac{1}{2}-\frac{1}{2}\sqrt{3}i,\
        -\frac{1}{2}+\frac{1}{2}\sqrt{3}i,\
        \frac{1}{2}+\frac{1}{2}\sqrt{3}i,\
        -\frac{1}{2}-\frac{1}{2}\sqrt{3}i
      \end{equation}

\hrulefill
 \item
      $H$の要素数を求めよ。

\dotfill

      代数学の基本定理により
      $z^6=1$を満たす複素数は最大で6つだが、
      先程の答えより次の数は式を満たす。
      \begin{equation}
       z=
        e^{0},e^{\frac{\pi}{3}i},e^{\frac{2\pi}{3}i},
        e^{\pi i}e^{\frac{4\pi}{3}i},e^{\frac{5\pi}{3}i}
      \end{equation}

      よって、$H$の位数は6である。

\hrulefill
 \item
      $H$が巡回群であるか否かを答えよ。

      \dotfill

      $H$のすべての元は
      $(e^{\frac{\pi}{3}i})^k,\ (k=1,\dots,6)$
      と表せる。

      つまり、$H=\langle e^{\frac{\pi}{3}i} \rangle$
      であるので巡回群である。

      \hrulefill
\end{enumerate}


\hrulefill
\textbf{問題}
\hrulefill

\begin{enumerate}
 \item
      $n\in\mathbb{Z}_{\geq 1}$とする。
      任意の整数$k$に対して、
      $\bar{k}\in\{0,1,\dots,n-1\}$を
      $k$の$n$による剰余とする。
      このとき、$\phi:\mathbb{Z}\to\mathbb{Z}/n\mathbb{Z}$を
      $k\mapsto \bar{k}$とすると、
      $\phi$は準同型写像であることを示せ。

\dotfill


      準同型写像である為には
      $f(a+b)=f(a)+f(b)$
      を示せばよい。

      $a,b\in\mathbb{Z}$に対して、
      $\bar{a}=a+n\mathbb{Z},\ \bar{b}=b+n\mathbb{Z}$
      である。
      これにより次の式が成り立つ。
      \begin{equation}
       f(a)+f(b)
        =\bar{a}+\bar{b}
        =a+n\mathbb{Z}+b+n\mathbb{Z}
        =a+b+n\mathbb{Z}
        =\overline{a+b}
        =f(a+b)
      \end{equation}

      よって、写像$f$は準同型写像である。

\hrulefill
 \item
      $\phi:\mathbb{Z}/2\mathbb{Z}\times\mathbb{Z}/2\mathbb{Z}\to D_{4}$
      を $(i,j)\mapsto r^is^j$とすると
      $\phi$は同型写像であることを示せ。

\dotfill

      $D_{4}$は4次二面体群
      \begin{equation}
       \langle r,s \mid r^4=s^2=e, rs=sr^{-1}\rangle
        =\{e,r,r^2,r^3,s,rs,r^2s,r^3s\}
      \end{equation}
      ではなく、
      位数4の二面体群
      \begin{equation}
       \langle r,s \mid r^2=s^2=e, rs=sr^{-1}\rangle
        = \{e,r,s,rs\}
      \end{equation}
      として扱う。

\dotfill

%      4次の二面体群$D_{4}$は次のように生成される。
      二面体群$D_{4}$は次のような元を持つ可換群である。
      \begin{align}
       D_{4} =& \langle r,s \mid r^2=s^2=e, rs=sr^{-1}\rangle\\
       =& \{e,r,s,rs\}
      \end{align}

%      \begin{equation}
%       D_{4}=\{e,r,r^2,r^3,s,rs,r^2s,r^3s\}
%      \end{equation}

      $\mathbb{Z}/2\mathbb{Z}$は
      偶数を$\bar{0}$、
      奇数を$\bar{1}$とする群で
      直積は次のような集合である。
      \begin{equation}
       \mathbb{Z}/2\mathbb{Z}\times\mathbb{Z}/2\mathbb{Z}
        =\{ (\bar{0},\bar{0}),\ (\bar{1},\bar{0}),\ (\bar{0},\bar{1}),\ (\bar{1},\bar{1})\}
      \end{equation}
      成分ごとの和を演算として群を成す。
      \begin{equation}
       (\bar{a}_1,\bar{b}_1) + (\bar{a}_2,\bar{b}_2)
        =(\overline{a_1+a_2},\overline{b_1+b_2})
      \end{equation}


      写像$\phi$は次のような対応を取るので全単射である。
      \begin{equation}
       (\bar{0},\bar{0}) \mapsto e,\
       (\bar{1},\bar{0}) \mapsto r,\
       (\bar{0},\bar{1}) \mapsto s,\
       (\bar{1},\bar{1}) \mapsto rs
      \end{equation}

      また、次により準同型写像でもある。
      \begin{equation}
       f(\bar{a}_1,\bar{b}_1)+f(\bar{a}_2,\bar{b}_2)
        =r^{a_1}s^{b_1}r^{a_2}s^{b_2}
        =r^{a_1+a_2}s^{b_1+b_2}
        =f(\overline{a_1+a_2},\overline{b_1+b_2})
      \end{equation}


      以上より、
      $\phi$は同型写像である。
      
\hrulefill
\end{enumerate}


\end{document}
