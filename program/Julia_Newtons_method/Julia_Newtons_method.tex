\documentclass[12pt,b5paper]{ltjsarticle}

%\usepackage[margin=15truemm, top=5truemm, bottom=5truemm]{geometry}
%\usepackage[margin=10truemm,left=15truemm]{geometry}
\usepackage[margin=10truemm]{geometry}

\usepackage{amsmath,amssymb}
%\pagestyle{headings}
\pagestyle{empty}

%\usepackage{listings,url}
%\renewcommand{\theenumi}{(\arabic{enumi})}

\usepackage{graphicx}

%\usepackage{tikz}
%\usetikzlibrary {arrows.meta}
%\usepackage{wrapfig}	% required for `\wrapfigure' (yatex added)
%\usepackage{bm}	% required for `\bm' (yatex added)

% ルビを振る
%\usepackage{luatexja-ruby}	% required for `\ruby'

%% 核Ker 像Im Hom を定義
%\newcommand{\Img}{\mathop{\mathrm{Im}}\nolimits}
%\newcommand{\Ker}{\mathop{\mathrm{Ker}}\nolimits}
%\newcommand{\Hom}{\mathop{\mathrm{Hom}}\nolimits}

%\DeclareMathOperator{\Rot}{rot}
%\DeclareMathOperator{\Div}{div}
%\DeclareMathOperator{\Grad}{grad}
%\DeclareMathOperator{\arcsinh}{arcsinh}
%\DeclareMathOperator{\arccosh}{arccosh}
%\DeclareMathOperator{\arctanh}{arctanh}



\usepackage{listings,url}

\lstset{
%プログラム言語(複数の言語に対応,C,C++も可)
  language = Python,
%  language = Lisp,
%  language = C,
  %背景色と透過度
  %backgroundcolor={\color[gray]{.90}},
  %枠外に行った時の自動改行
  breaklines = true,
  %自動改行後のインデント量(デフォルトでは20[pt])
  breakindent = 10pt,
  %標準の書体
%  basicstyle = \ttfamily\scriptsize,
  basicstyle = \ttfamily,
  %コメントの書体
%  commentstyle = {\itshape \color[cmyk]{1,0.4,1,0}},
  %関数名等の色の設定
  classoffset = 0,
  %キーワード(int, ifなど)の書体
%  keywordstyle = {\bfseries \color[cmyk]{0,1,0,0}},
  %表示する文字の書体
  %stringstyle = {\ttfamily \color[rgb]{0,0,1}},
  %枠 "t"は上に線を記載, "T"は上に二重線を記載
  %他オプション:leftline,topline,bottomline,lines,single,shadowbox
  frame = TBrl,
  %frameまでの間隔(行番号とプログラムの間)
  framesep = 5pt,
  %行番号の位置
  numbers = left,
  %行番号の間隔
  stepnumber = 1,
  %行番号の書体
%  numberstyle = \tiny,
  %タブの大きさ
  tabsize = 4,
  %キャプションの場所("tb"ならば上下両方に記載)
  captionpos = t
}



\begin{document}

\textbf{Julia コード}

\hrulefill

\begin{equation}
 y^{2} - x^{3} + 4x =0
  ,\qquad
  \left( \frac{x}{2} \right)^{4} + y^{4} =1
\end{equation}

連立非線形方程式の数値解を Newton 法で計算する。

上記方程式を関数値ベクトルとして\textrm{f(x)}とし、
ヤコビ行列を \textbf{Df(x)} として定義する。

\begin{lstlisting}

using LinearAlgebra

# 関数値ベクトル
function f(x)
   return [ x[2]^2-x[1]^3+4*x[1],
            (x[1]/2)^4+x[2]^4-1 ]
end

# ヤコビアン 偏導関数行列
function Df(x)
   return [ -3*x[1]^2+4 2*x[2];
            x[1]^3/4    4*x[2]^3 ]
end

function newton(x0, f, Df)
    maxiter = 100  # 最大反復回数.適宜調整する.
    tol = 1e-6     # 停止条件の tolerance.適宜調整.
    x2 = x1 = x0 # 初期値

    for i in 1:maxiter

        # 行列Df が 退化している場合
        if det(Df(x1))==0
            return "degenerate"
        end

        x2 = x1 - Df(x1)\f(x1)

        # 途中経過の表示
        @show x2

        # 停止条件の判定
        if norm(x2 - x1) < tol
            println("Converged in $i iterations.")
            break
        end

        x1 = x2
    end
    return x2    # 数値解を返却
end

newton([0,1.0],f,Df)

\end{lstlisting}



\hrulefill

\end{document}
