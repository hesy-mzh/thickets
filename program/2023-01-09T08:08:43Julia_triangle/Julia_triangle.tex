\documentclass[12pt,b5paper]{ltjsarticle}

%\usepackage[margin=15truemm, top=5truemm, bottom=5truemm]{geometry}
%\usepackage[margin=10truemm,left=15truemm]{geometry}
\usepackage[margin=10truemm]{geometry}

\usepackage{amsmath,amssymb}
%\pagestyle{headings}
\pagestyle{empty}

%\usepackage{listings,url}
%\renewcommand{\theenumi}{(\arabic{enumi})}

%\usepackage{graphicx}

%\usepackage{tikz}
%\usetikzlibrary {arrows.meta}
%\usepackage{wrapfig}	% required for `\wrapfigure' (yatex added)
%\usepackage{bm}	% required for `\bm' (yatex added)

% ルビを振る
%\usepackage{luatexja-ruby}	% required for `\ruby'

%% 核Ker 像Im Hom を定義
%\newcommand{\Img}{\mathop{\mathrm{Im}}\nolimits}
%\newcommand{\Ker}{\mathop{\mathrm{Ker}}\nolimits}
%\newcommand{\Hom}{\mathop{\mathrm{Hom}}\nolimits}

%\DeclareMathOperator{\Rot}{rot}
%\DeclareMathOperator{\Div}{div}
%\DeclareMathOperator{\Grad}{grad}
%\DeclareMathOperator{\arcsinh}{arcsinh}
%\DeclareMathOperator{\arccosh}{arccosh}
%\DeclareMathOperator{\arctanh}{arctanh}



\usepackage{listings,url}

\lstset{
%プログラム言語(複数の言語に対応,C,C++も可)
%  language = Python,
%  language = Lisp,
%  language = C,
  %背景色と透過度
  %backgroundcolor={\color[gray]{.90}},
  %枠外に行った時の自動改行
  breaklines = true,
  %自動改行後のインデント量(デフォルトでは20[pt])
  breakindent = 10pt,
  %標準の書体
%  basicstyle = \ttfamily\scriptsize,
  basicstyle = \ttfamily,
  %コメントの書体
%  commentstyle = {\itshape \color[cmyk]{1,0.4,1,0}},
  %関数名等の色の設定
  classoffset = 0,
  %キーワード(int, ifなど)の書体
%  keywordstyle = {\bfseries \color[cmyk]{0,1,0,0}},
  %表示する文字の書体
  %stringstyle = {\ttfamily \color[rgb]{0,0,1}},
  %枠 "t"は上に線を記載, "T"は上に二重線を記載
  %他オプション:leftline,topline,bottomline,lines,single,shadowbox
  frame = TBrl,
  %frameまでの間隔(行番号とプログラムの間)
  framesep = 5pt,
  %行番号の位置
  numbers = left,
  %行番号の間隔
  stepnumber = 1,
  %行番号の書体
%  numberstyle = \tiny,
  %タブの大きさ
  tabsize = 4,
  %キャプションの場所("tb"ならば上下両方に記載)
  captionpos = t
}



\begin{document}

\hrulefill

\textbf{Julia (プログラミング言語)}

\dotfill

\begin{enumerate}
 \item
      3つの正の実数$a,b,c$が与えられたとき、
      それらを辺の長さに持つような三角形$T$の
      面積を返す関数
      \texttt{area\_triangle(a,b,c)}
      を
      Julia言語で実装せよ。
      ただし、以下の要件をすべて満たすように実装すること。
      \begin{enumerate}
       \item
            三角形の成立条件が成り立っている場合、
            三角形$T$の面積を計算し、その値を返却する。
       \item
            三角形の成立条件が成り立っていない場合は、
            \texttt{nothing}を返却する。
      \end{enumerate}
      三角形の成立条件
      \begin{equation}
       a+b > c
        \text{ かつ }
       b+c > a
        \text{ かつ }
       c+a > b
      \end{equation}

      \dotfill

      次のコードは
      関数\texttt{area\_triangle(a,b,c)}
      を定義するコードであり、
      面積の計算はヘロンの公式を利用している。

      Julia ソースコード
      \begin{lstlisting}
# 面積を求める関数(ヘロンの公式)
function area_triangle(a,b,c)
    if a+b>c && b+c>a && c+a>b
        s=(a+b+c)/2
        return sqrt(s*(s-a)*(s-b)*(s-c))
    else
        return "nothing"
    end
end
      \end{lstlisting}

      \hrulefill

 \item
      \texttt{area\_triangle(1,2,3)}
      および
      \texttt{area\_triangle(3,4,5)}
      の実行結果をそれぞれ記載し、
      結果が正しいことを検証せよ。

      \dotfill

      実行コードは次の通り。

      \begin{lstlisting}
# 面積を求める関数(ヘロンの公式)
function area_triangle(a,b,c)
    if a+b>c && b+c>a && c+a>b
        s=(a+b+c)/2
        return sqrt(s*(s-a)*(s-b)*(s-c))
    else
        return "nothing"
    end
end

println("1,2,3 : ", area_triangle(1,2,3))
println("3,4,5 : ", area_triangle(3,4,5))
      \end{lstlisting}

      実行結果
      \begin{lstlisting}
1,2,3 : nothing
3,4,5 : 6.0
      \end{lstlisting}

      \hrulefill

\end{enumerate}

\hrulefill

\textbf{フィボナッチ数列}

\begin{enumerate}
 \item
      次の漸化式で定義される数列を$\{a_n\}$とする。
      \begin{equation}
       a_0=0,
        \quad
        a_1=1,
        \quad
        a_{n+2}=a_{n}+a_{n+1}
      \end{equation}
      与えられた2以上の整数$n$に対して、
      $a_{2},\dots,a_{n}$を計算する
      関数\texttt{fib(n)}
      を
      次の要件を満たすように作成せよ。
      \begin{enumerate}
       \item
            \texttt{for}...\texttt{end}
            構文を用いて、
            漸化式により
            $a_{2},\dots,a_{n}$を計算すること
       \item
            $a_{2},\dots,a_{n}$の値をすべて表示すること
      \end{enumerate}

      \dotfill

      \begin{lstlisting}
# フィボナッチ数列
function fib(n)
    # 初期値
    a=0; b=1
    for i in 2:n
        # 漸化式
        c=a+b
        # 出力
        println("a_",i," = ",c)
        # 初期値変更
        a=b; b=c
    end
end
      \end{lstlisting}



      \hrulefill

 \item
      \texttt{fib(10)}の実行結果を記載せよ。
      この問いに関しては説明は書かなくともよい。

      \dotfill

      実行するコード
      \begin{lstlisting}
# フィボナッチ数列
function fib(n)
    # 初期値
    a=0; b=1
    for i in 2:n
        # 漸化式
        c=a+b
        # 出力
        println("a_",i," = ",c)
        # 初期値変更
        a=b; b=c
    end
end

fib(10)
      \end{lstlisting}

      実行結果
      \begin{lstlisting}
a_2 = 1
a_3 = 2
a_4 = 3
a_5 = 5
a_6 = 8
a_7 = 13
a_8 = 21
a_9 = 34
a_10 = 55
      \end{lstlisting}


      \hrulefill

\end{enumerate}

\hrulefill

\end{document}
